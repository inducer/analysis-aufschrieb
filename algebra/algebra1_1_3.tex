% -----------------------------------------------------------------------------
% Simon: Verbesserungen:
% - auch einzelne Buchstaben (``die Gruppe $G$'') in inline math einfassen.
%   d.h. nicht nur ``die Gruppe G''.
% - Mein Einr"uckungsschema verwenden (besser lesbar)
% - Zeilen bei h"ochstens 79 Zeichen umbrechen
% - \motivation {}:{bla} -> {} nicht erforderlich. Trotzdem lieber Titel 
%   ausdenken
% - Leerzeilen weghauen (ergeben "uberfl"ussige Umbrueche)
% - Display Math verwenden, statt ewig langes inline
% - := bei Definitionen verwenden
% - \index nicht vergessen
% - bei simplen Konstruktionen bleiben, \stack oder \parbox nur in 
%   Ausnahmef"allen verwenden
% - \thepara.\thenodefinition sind Implementierungsdetails meiner Makros.
%   Sie sollten nicht au"serhalb verwendet werden. Statt dessen lieber
%   \label und \ref verwenden.
% - Nie \\ zum Absatz-Beenden verwenden. Statt dessen Leerzeile lassen.
% - Insbesondere im Mathemodus hat \\ nichts zu suchen. Verwende statt
%   dessen immer align*. eqnarray* ist auch schlecht.
% - Den zu definierenden Term mit \emph hervorheben. St"ort den Lesefluss
%   weniger als Fettdruck. (ausserdem sagt \emph nur ``Emphasis'' und 
%   nichts "uber das endg"ultige Aussehen)
% - Abs"atze mit Gro"sbuchstaben beginnen
% - Zum Zwecke der Einheitlichkeit ``\iff'' statt \Leftrightarrow
% - In Formeln so wenige Spaces wie m"oglich. (h"alt den Kram kurz)
% - \setminus ist ``Ohne''-Zeichen f"ur Mengen. Verwende bei Bahnenmengen statt
%   dessen \backslash.
% - \indexthis{bla}, nicht \indexthisbla
% - wo noch was fehlt, schreib bitte ``\missing''.
% - (i),(ii) selbst zu schreiben ist doof. Nimm nur enumerate.
% - Versuche, dich nicht um das Aussehen zu kuemmern. Verwende so wenige
%   Formatierungsanweisungen wie moeglich. Versuche, nur Inhalt 
%   aufzuschreiben. Ansprechendes Aussehen ist Aufgabe des Satzsystems,
%   dein Job ist der Inhalt. (Insbesondere \hfill ist vom Teufel!)
%   (auch wenn du von Word genau das Gegenteil gewohnt bist) [Ich kenn
%   das zu gut von mir selbst!]
% - Wenn du Tabulatorzeichen verwendest, stelle diese bitte auf eine
%   Breite von 8 Zeichen ein.
% - Kein \\ am Ende von align* oder sonstigen Environments
% -----------------------------------------------------------------------------
\para{Operierende Gruppen}

\motivation{}:{Die meisten Gruppen in der Mathematik kommen nicht abstrakt vor, 
                      sondern als Gruppen von Operatoren auf einer Menge M.}
	
% -----------------------------------------------------------------------------

\example{}:{
		    }	
% -----------------------------------------------------------------------------

\definition{Operation einer Gruppe}:
  {\index{Gruppe>opposite}
   
   Sei $G$ eine Gruppe, 
   $M$ eine Menge. Eine \emph{\indexthis{Operation}} von $G$ auf $M$ ist eine Abbildung $a$
   (genannt ``"aussere`` Verkn"upfung) $a: G \times M \longrightarrow M $ mit
   $(g,m) \mapsto a(g,m) := g*m $ so dass die folgenden Eigenschaften erf"ullt sind:
   \begin{enumerate}
   \item[(i)]  $ \forall  m \in M: 1_G * m = m$
   \item[(ii)]  $ \forall  g, h \in G, m \in M: h*(g*m) = (hg)*m$
   \end{enumerate}		
 
   Operiert nun $G$ ebenso (von links) auf $M$, so heisse $M$ eine G-Gruppe.	
								     }		    
% -----------------------------------------------------------------------------
\annotation{\thepara.\thenodefinition}:{
  (ii) kann auch durch eine "aquivalente Bedingung f"ur Operation
  von rechts ersetzt werden.\\
  Oft wird $G$ multiplikativ geschrieben und `*` weggelassen.} 
% -----------------------------------------------------------------------------
\remark{opposite Gruppe}:{
  zu (G,v) Gruppe erh"alt man die {\emph \indexthis{opposite Gruppe}} $G^{op} = (G,v^{op})$ folgender\-ma"sen: \\
  $\forall g,h \in G: g*h = v^{op}(g,h) = v(h,g) = hg $\\
  klar: G operiert von links auf M $\Leftrightarrow G^{op}$ operiert von rechts auf M. \\
  D.h. f"uer die Theorie ist es ausreichend, die Operationen von links zu studieren. 
  Es gilt dann immer eine analoge Aussage "uber die Operationen von rechts.}
% -----------------------------------------------------------------------------
\convention{ schreibe ab jetzt $G$ multiplikativ, die Operation als $gm$ \\
  und O.B.d.A von links.}
% -----------------------------------------------------------------------------
\definition{Fixpunkt, Fixgruppe, Bahn, Quotient}:{
  $G$ operiere auf $M$, es sei $g \in G, m \in M, H \leq G$
  \begin{enumerate}
  \item[(i)] $m$ heisst \emph{\indexthis {Fixpunkt}} von g $:\iff$ $gm = m $
  \item[(ii)] m heisst \emph{\indexthis Fixpunkt} von H $:\iff $ $\forall h \in H: hm = m$
  \item[(iii)] $G_m = \{g \in G \mid gm = m \}$ heisst {\emph \indexthis{Fixgruppe}} von m
       (in manchen B"uchern auch \emph{\indexthis{Stabilisator}},$Stab(m)$ oder auch
		\indexthis{Isotropiegruppe})
  \item[(iv)] $Gm = \{ gm \mid g \in G\} \subseteq $ heisst {\emph \indexthis{Bahn}} von m (unter G)
  \item[(v)] Die Menge der { Bahnen} heisst Quotient von M nach G und wird mit $G \backslash M$ 
	         (bei Operation von rechts $M/G$) bezeichnet.
  \end{enumerate}}

% -----------------------------------------------------------------------------
\theorem{}:{G operiere auf M. Dann gelten}=>
  { \begin{enumerate}
     \item[(i)] {\emph \indexthis{Bahnzerlegung}}: \[ M = { \biguplus_{\alpha \in G \setminus M}} \alpha \]  (disjunkt)  		 
	 \item[(ii)] {\emph \indexthis{Bahnbeschreibung}}:\\
     Man erh"alt die bijektive Abbildung \\
     $ G/G_m \longrightarrow Gm$ \\
     $g G_m \mapsto gm $\\
	 insbesondere $ |Gm| = (G:G_m) $       
     \item[(iii)] {\emph \indexthis {Bahnbilanz}}: \\
	 sei $ \mathcal{V}$ ein Vertretersystem der Bahnen, dann gilt:\\
	 $|M| = \sum_{v \in \mathcal{V}} (G:G_v)$
	\end{enumerate}} 
% -----------------------------------------------------------------------------
\proof{}:{(i) betrachte die "Aquivalenzrelation $ m \sim n \Leftrightarrow Gm = Gn $. Die Klasse von m ist dann
  $\overline{m} = Gm \Rightarrow $ disjunkter Zerfall.\\
  (ii) wende den Homomorphiesatz f"ur Mengen an auf $f: G \longrightarrow M, g \mapsto f(g)=gm $ \\
 \insertfig{algebra1_bahnbilanz}
 bleibt zu zeigen: $ G/\sim_f = G/G_m. \\
 g \sim_f h \Leftrightarrow f(g) = f(h)
 \Leftrightarrow gm = hm \Leftrightarrow h^{-1}(gm) = m \Leftrightarrow (h^{-1}g)m = m 
 \Leftrightarrow h^{-1}g \in G_m \Leftrightarrow hG_m =gG_m.$ Also: ${G/\sim_f} = G/G_m $ \\
 (iii) nach (i) \[ M = \biguplus_{\alpha \in G\setminus M} \alpha = \biguplus_{v \in \mathcal{V}}G_v \]
 \[\Rightarrow  |M| = \sum_{v\in\mathcal{V}}|G_v| \stackrel{(ii)}{=} \sum_{v\in\mathcal{V}}(G:G_v) \] 
		   }
% -----------------------------------------------------------------------------

\example{}:{
  \begin{enumerate}
    \item[(a)]$H \leq G \Rightarrow $ H operiert (von links) auf G durch gruppentheoretische Translation. \\
     $H \times G \longrightarrow G $,
     $(h,g) \mapsto hg$  (Produkt in G)\\
     Bahn: $Hg = \{ hg \mid h\in H \}$  (linke Nebenklasse)\\
	 Quotient: $H \backslash G$ wie fr"uher\\
     Bahnbilanz: $\mathcal{V}$ Vertretersystem, $|G| =\sum_{v\in\mathcal{V}} (H:H_v) = 
     \sum_{v\in \mathcal{V}} |H| = |H| |H\backslash G|$, da\\
     $H_v = \{h\in H \mid hv=v \} = \{ 1_G \}$ und somit $(H:H_v) =|H|$\\
     Bahnbilanz liefert also den Satz von Lagrange.
    \item[(b)] Operiert $G$ auf $M$ und ist $H\leq G$, so operiert $H$ auf $M$.
	\item[(c)] Operiert $G$ auf $M$, so operiert $G$ auf der Potenzmenge $2^M$ von $M$ wie folgt: $ F \subseteq M$, so:\\
	   $gF = \{ gu \mid u\in F \}$\\
	   wichtig in der Geometrie: $M = \SetR^n$, G eine geometrisch interessante Gruppe, z.B. 
	   Gruppe der Bewegungen ${\emph (Aut_{dist}(\SetR^n))}$.\\
	   Ist F eine geometrische Figur,
	   (d.h. $F\subseteq \SetR^n$), so heisst ${G_F =\{g\in G \mid g(F) = F \}}$ Symmetriegruppe von F.\\
	  $Gl_n(K) = \{A\in K^{n \times n} \mid det(A) \neq 0 \}$ operiert auf $K^{n \times n}, 
	  (U,A) \mapsto UA $\\
	  Die Gaussnormalformen bilden ein Vertretersystem bei dieser Operation.
    \item[(d)]  $S_M$ operiert auf M \\
	  Operiert endlich zyklische Gruppe $Z=Z_d=\{1,g, \ldots, g^{d-1} \}$ auf M, d=ord(g)\\
	  Bahn $Zm =\{ m, gm, g^2m, \ldots, g^{l_m - 1}m \}$, $|Z_m| = l_m \mid d$ \\
	  $g^{l_m}m = m$, also
	  $Z_m = \langle g^{l_m} \rangle $\\
	  speziell: $Z = \langle \pi \rangle$	\\
	  Bahn: $Z_{i_k} = \{i_k, \pi i_k, \ldots, \pi^{l_{\pi}}i_k \}, i_k \in \{1,
	  \ldots, n\}, {k=1, \ldots, s}$ \\
	  $\pi_k = \pi_{|Z_{i_k}}$  ist die folgende Abbildung: \\
	  $i_k \stackrel{\pi}{\mapsto} \pi i_k \stackrel{\pi}{\mapsto} \ldots \stackrel{\pi}{\mapsto}
	  \pi^{l_{\pi} - 1} i_k \stackrel{\pi}{\mapsto} i_k $, also gerade der Zyklus\\
	  $(i_k, \pi i_k, \ldots, \pi^{l_{\pi} - 1}i_k)$\\
	  $\pi = \pi_1 \cdot \ldots \cdot \pi_s$, au"serdem gilt $\pi_i \pi_j = \pi_j \pi_i$\\
	  hier: ord($\pi$)=kgV(ord$\pi_k$		  
 \end{enumerate}			  
          }
% -----------------------------------------------------------------------------
\definition{p-Gruppe}:{
  G heisst {\emph p-Gruppe} $\Leftrightarrow |G| = p^n$, wobei p Primzahl, n$\in \SetN_{>0}$}
% -----------------------------------------------------------------------------
\definition Fixpunktmenge:{ 
  $M^G :=\{m\in M \mid \forall g\in G: gm = m \}$ \hfill Menge aller Fixpunkte
  }

% -----------------------------------------------------------------------------
\theorem{p-Fixpunktsatz}:{
  Die p-Gruppe G operiere auf einer endlichen Menge M}=>{
  $p \mid (|M| - |M^G|)$ \\
  insbes.: $p \nmid |M|$, so hat M mindestens einen Fixpunkt.
							}
% -----------------------------------------------------------------------------	   				 
\proof{}:{ $ v \in M^G \Leftrightarrow \forall g \in G: gv = v \Leftrightarrow G = G_v \Leftrightarrow
  |Gv| = (G:G_v) = 1$ insbes. $v \in \mathcal{V}$\\
  Bahnbilanz: \\ 
  \begin{eqnarray*}
    |M| & = &  \sum_{v \in M^G} 1 +  \sum_{v \in \mathcal{V}, v \not\in M^G}
	\underbrace{(G:G_v)}_{\operatorname{p-Potenz \enspace} \neq 1} \nonumber \\
	& = & |M^G| +               p(\ldots) \\
  \end{eqnarray*}
  \hfill \qed		  
	  }

% -----------------------------------------------------------------------------
% Operation durch Konjugation
%------------------------------------------------------------------------------

\subsection {Die Operation durch Konjugation}
\convention{Sei $G$ Gruppe (O.B.d.A multiplikativ geschrieben)}

\lemma{bzw Definition: Operation durch Konjugation}:{}=>{
  \begin{enumerate}
  \item[(i)] Man hat die folgende Operation (durch Konjugation bzw. 
           durch innere Automorphismen) von $G$ auf $G=M$: \\
		   $G\times G \longrightarrow G$, $(x,y) \mapsto x*y := xgx^{-1}=: in_x(g)$ (innerer
		   Automorphismus) \\
		   Grund: $(xy)*g=in_{xy}(g)=(in_x\circ in_y)(g)=in_x(in_y(g))=x*(y*g)$ \\
		   demnach zerf"allt $G$ in Klassen konjugierter Elemente (Bahnen).
 \item[(ii)]$G$ operiert auch auf der Potenzmenge $2^G$ von $G$	 und auf 
           dem Untergruppenverband durch Konjugation (denn $U\in U(G) \implies x*U
		   =xUx^{-1}=in_x(U) \leq G$). Bahn ist die Konjugiertenklasse der Untergruppe $U$.   
 \item[(iii)]Die Fixgruppe von $g\in G$ bei Konjugation: \\
           $C_G(g)=\{x\in G\mid xgx^{-1}=g\} =\{x\in G \mid xg=gx\}$ heisst {\indexthis{Zentralisator}} 
		   von $g$ in $G$. \\
		   F"ur $A \subseteq G$ heisst $C_G(g) = \{x\in G\mid \forall a\in A: xa=ax\}=$
		   \[\bigcap_{g\in A} C_G(g) \] Zentralisator von $A$.	\\
		   Die Fixgruppe von $U\in U(G)$ (bzw. $U\subseteq G$) \\ 
		   $N_G(U)=\{x\in G\mid xUx^{-1}=U\}
		   =\{x\in G\mid xU=Ux\}$ heisst {\emph \indexthis{Normalisator}} von $U$ in $G$.\\
		   Bahnbeschreibung: Die Konjugierten $xgx^{-1}$ von $g$ entsprechen
		   bijektiv den Elementen von $G/C_G(g)$, insbesondere
		   \[|\{xgx^{-1}\mid x\in G\}|=|\operatorname{cl}(g)|=G:C_G(g) \mid \frac{|G|}{ord(g)} \] falls $G$ endlich. \\
		   \insertfig{algebra1_konj1}
		   ebenso f"ur $U\leq G$.
		   \[|\{xUx^{-1}\mid x\in G\}\|=|\operatorname{cl}(U)|=G:NG(U) \mid (G:U)\]
		   falls $G$ endlich.\\
		   \insertfig{algebra1_konj2}
\end{enumerate}
  }
%------------------------------------------------------------------------------------------------------
\remark{}:{
\begin{enumerate}
  \item[(i)]$Z(G)=C_G(G)=\{g\in G\mid \forall x\in G: xg=gx\}$ heisst \indexthis{Zentrum} von G.
       ($Z(G)$ ist ein abelscher Normalteiler von $G$.) Es gilt: $g\in Z(C_G(g))$ und
	   $C_G(g)$ ist die gr"osste Untergruppe, in deren Zentrum $g$ liegt.
  \item[(ii)]$N_G(U)=\{x\in G\mid xU=Ux\}$("$x$ normalisiert $U$") ist die gr"osste Untergruppe
       von $G$, in der $U$ normal ist.
\end{enumerate}
  }  
%------------------------------------------------------------------------------------------------------
\theorem{p-Gruppe hat nichttriviales Zentrum}:{sei $G$ eine p-Gruppe}=>{Dann hat $G$ ein nichttriviales Zentrum}
%------------------------------------------------------------------------------------------------------
\proof{}:{$G$ operiere auf $G$ durch innere Automorphismen. Laut p-Fixpunktsatz: \\
          $p\mid |G|-|G^G|$, wobei $G^G$ die Menge aller Fixpunkte in $G$ ist. \\
		  (Also $\forall g\in G^G: \forall x\in G: xgx^{-1}=g$)\\
		  D.h. $G^G=Z(G)$.\\
		  $\Rightarrow p \mid (p^r-|Z(G)|)\Rightarrow p \mid |Z(G)|$ \\
		  wegen $|Z(G)|\geq 1, p \mid |Z(G)|$ 
		  folgt $|Z(G)|\geq p$ \qed
         }
%--------------------------------------------------------------------------------------------------------
\subsection{Bestimmung aller p-Gruppen mit $|G|\leq p^2$}

\begin{enumerate}
\item $|G|=p \Rightarrow G=Z_p$ (zyklische Gruppe der Ordnung $p$)
\item $|G|=p^2 \Rightarrow |Z(G)|\neq 1 \Rightarrow |G/Z(G)=1,p|
      \Rightarrow$ $G/Z(G)$ zyklisch 
	  $\stackrel{\operatorname{Blatt 3,Aufgabe 9}}{\Rightarrow} $ G abelsch. 
	  Falls $Z(G)=G$, so ist dies klar, ansonsten:
	  $G/Z(G)\cong Z(G)\cong Z_p$ \\
	  $G/Z(G)=\product{g}$, $Z(G)=\product{u} \Rightarrow$
	  $G=\product{g,u}$, aber $gu=ug \Rightarrow$ $G$ abelsch
	  $\Rightarrow G=Z(G)$.
  \begin{enumerate}
	 \item[Fall 1:] $G$ zyklisch, $G\cong Z_{p^2}\Leftrightarrow$
	  $\exists g\in G:\operatorname{ord}g>p$
	 \item[Fall 2:] $G$ nicht zyklisch $\Rightarrow \forall$
	  $g\neq 1, g\in G \operatorname{ord}g=p$ \\
	  $U=\product{g}$, $h\in G\setminus U \Rightarrow$
	  $V=\product{v}$ hat $|V|=p, V\neq U \Rightarrow$
	  $V\cap U \stack(\neq;\leq) U$ also
	  $|V\cap U|=1 \Rightarrow V\cap U =\{1\}$\\
	  $\Rightarrow G\{g^ih^i\mid 0\leq i <p,0\leq j<p\}$ ("Ubung)\\
	  Man hat $Z_p\times Z_p \longrightarrow G$,
	  $\product{w}\times \product{w}\longrightarrow G$,\\
	  $(w^i,w^j)\mapsto g^ih^j$. 
  \end{enumerate}
 \item Ergebnis: Die p-Gruppen mit Ordnung $\leq p^2$ sind bis auf
       Isomorphie in der Liste $Z_p,Z_{p^2},Z_p\times Z_p$.
\end{enumerate}		
%--------------------------------------------------------------------------------
\remark{}:{"Ahnliche Listen gibt es f"ur
       $|G|=p^3$(Ph.Hall, Grouptheory).\\
	   Eine sehr detaillierte Theorie der p-Gruppen ist vorhanden.
	   Siehe z.B. B.Huppert, Endliche Gruppen I S.252-403}
   

% -----------------------------------------------------------------------------
\para{Gruppen (Grundbegriffe)}
% -----------------------------------------------------------------------------
\subsection{Definition, Beispiele}
% -----------------------------------------------------------------------------
\definition Elementeigenschaften:{
  \index{Rechtsinvers}
  \index{Linksinvers}
  Sei $M$ Monoid, $e$ neutrales Element, O.B.d.A. multiplikativ geschrieben, $a\in M$.
  \begin{itemize}
  \item $x\in M$ hei"st \emph{linksinvers} zu $a$ $:\iff$ $xa=e$
  \item $x\in M$ hei"st \emph{rechtsinvers} zu $a$ $:\iff$ $ax=e$
  \item $a$ hei"st \emph{invertierbar} $:\iff \exists$ Linksinverses $x$, 
    Rechtsinverses $y$ zu $a$
  \item Ein Monoid $(G,\circ)$, in dem alle $a\in G$ invertierbar sind, hei"st
    eine \emph{Gruppe} $G$.
  \end{itemize}
 }
% -----------------------------------------------------------------------------
\lemma Links- und Rechtsinverse:
  $G$ Gruppe, $a\in G$ => {
  \label{lem:linksrechtsinv}
  Alle Links- und Rechtsinversen von $a$ sind gleich.
  Bezeichnung: $x=:a^{-1}$ bei fast allen Verkn"upfungen; $x=:-a$ bei $\circ =+$
  }
% -----------------------------------------------------------------------------  
\proof \ref{lem:linksrechtsinv}:{
  $e=ya=ax \implies x=ex = (ya)x = y(ax) = ye = y$.
  Sind $y_1, y_2$ zwei Linksinverse  mit $y_1=x$ und $y_2=x$, so gilt
  $y_1=y_2$. Analog f"ur Rechtsinverse.\qed
  }
% -----------------------------------------------------------------------------  
\subsubsection*{Konstruktion von Gruppen}
% -----------------------------------------------------------------------------
\theorem Einheitengruppe von Monoiden:
  $M$ Monoid=>{
  \label{the:einheitengruppe-monoid}
  $M^* := \{ x\in M \mid x$ invertierbar $ \}$ ist ein Untermonoid, das eine 
  Gruppe ist. $M^*$ hei"st \emph{\indexthis{Einheitengruppe}} von $M$.
  }
% -----------------------------------------------------------------------------
\proof \ref{the:einheitengruppe-monoid}:{
  $\surd$. Grund: $(x_1 \cdots x_n)^{-1} = x_n^{-1} \cdots x_1^{-1}$ f"ur
  invertierbare $x_i$. $x\in M^* \implies x^{-1} \in M^*$ wegen 
  $\left( x^{-1} \right) ^{-1}=x$.\qed
  }
% ----------------------------------------------------------------------------- 
\example Einheitengruppen von Monoiden:{
  \begin{itemize}
  \item $FrMon(A)^* = \{ e=() \}$
  \item Ist $A$ Menge, so ist
    \begin{align*}
      (A^A)^* 
      &= \{ f: A \to A \mid 
      \exists g,h \in A^A \mid g\circ f = id_A, f\circ g=id_A\}\\
      &=\{f \mid f\text{ ist bijektiv}\}
      \end{align*}
    Inverse von $f$ (im Sinn des Monoids) ist die Umkehrabbildung $f^{-1}$.
  \end{itemize}
  }
% -----------------------------------------------------------------------------
\definition Symmetrische Gruppe eines Monoids:{
  \index{Gruppe>symmetrische}
  $(A^A)^* =: S_A$ hei"st \emph{symmetrische Gruppe} auf $A$.
  } 
% -----------------------------------------------------------------------------  
\definition Einheitengruppe eines Rings:{
  Ist $R$ ein Ring (mit $1$), so ist $(R, \circ)$ ein Monoid 
  und man hat die Einheitengruppe von $R$:
  \[R^\times := \{ a \in R \mid \exists x,y \in R : ax = ya = 1 \}
    \]
  }
% -----------------------------------------------------------------------------
\example Einheitengruppe eines Rings:{
  $K^{n\times n}$ ist der Ring der $n\times n$-Matrizen 
  "uber dem K"orper (oder Ring) K.
  \[GL_n(K):=\left( K^{n\times n} \right)^\times = \{ A \subset K^{n\times n} \mid A \text{ invertierbar}\}
    \]
  F"ur K"orper (s. Vorlesung Lineare Algebra) gilt
  \[\left( K^{n\times n} \right)^\times = \{ A \subset K^{n\times n} \mid \det A\neq 0\}
    \]
  Zu einer Halbgruppe $M$ ist $End(M)$ ein Monoid, man hat also
  $Aut(M)=(End(M))^*$ eine Gruppe.
  }
% -----------------------------------------------------------------------------
\remark Was ist $GL_n(\SetZ)$?:{
  \index{Matrix>adjunkte}
  Nach LA: Sei $A^\#$ die adjunkte Matrix zu $A\in GL_n(\SetZ)$, f"ur die gilt
  $AA^\#=\det A E=A^\#A$. Falls $\det A\neq 0$: $A^{-1}=(\det A)^{-1} A^\#$.
  Daher $GL_n(\SetZ)=\{A\in\SetZ^{n\times n} \mid \det A\in\SetZ^\times=\{1,-1\} \}$.
  
  F"ur kommutative Ringe $R$ gilt allgemein:
  \[GL_n(R)=\{A\in\SetZ^{n\times n} \mid \det A\in R^\times\}
    \]
  }
% -----------------------------------------------------------------------------
\annotation Adjunkte Matrix:{
  Wie erh"alt man $A^\#$ allgemein?
  \missing
  }
% -----------------------------------------------------------------------------
\subsection{Erzeugung und Untergruppenverband}
% -----------------------------------------------------------------------------
\definition Untergruppe:{
  Sei $G$ eine Gruppe, o.E. multiplikativ geschrieben.
  $U$ hei"st \emph{Untergruppe} der Gruppe $G$ $:\iff$ $U$ ist
  eine Unterhalbgruppe von $G$, die selbst eine Gruppe ist.
  
  Bezeichnung: $U\leq G$.
  }
% -----------------------------------------------------------------------------
\lessertheorem Untergruppenkriterium:=>{
  $U\leq G$ ist "aquivalent damit, dass alle drei folgenden
  Eigenschaften gelten:
  \begin{enumerate}
    \item $\forall g,h\in U: gh\in U$ (Abgeschlossenheit)
    \item $1_G\in U$ (Existenz des Neutralelements)
    \item $\forall g\in U: g^{-1}\in U$ (Existenz des Inversen)
    \end{enumerate}
  }
% -----------------------------------------------------------------------------
\definition Vollst"andiger Verband:{
  \index{Verband>vollst"andiger}
  Eine geordnete Menge $A$, bei der jede Teilmenge eine Supremum und
  ein Infimum besitzt, hei"st \emph{vollst"andiger Verband}.
  }
% -----------------------------------------------------------------------------
\definition Untergruppenverband:{
  $U(G):=\{U\subset G\mid U\leq G\}$ hei"st \emph{Untergruppenverband} von $G$. $U(G)$ ist
  vollst"andiger Verband, denn:
  \begin{enumerate}
    \item $U(G)$ ist durch ``$\leq$'' geordnet.
    \item Ist $M\subset U(G)$, so ist klar, dass gilt
      \[\bigcap_{U\in M} U \in U(G)
        \]
    \item 
      \index{Erzeugnis>gruppentheoretisches}
      Ist $A\subset G$, so hei"st
      \[\product A:=\product A_{Grp}:=\bigcap_{U\leq G,A\subset U} U\in U(G)
        \]
      das \emph{\indexthis{gruppentheoretische Erzeugnis}}
      von $A$. Mit $A^{-1}:=\{a^{-1}\mid a\in A\}$ gilt $\product A=\product{A\cup A^{-1}}_{Hgr}$,
      das hei"st, $\product A$ besteht aus allen endlichen Produkten von
      Elementen in $A$ und $A^{-1}$. (Insbesondere ist $\product{A\cup A^{-1}}_{Hgr}$
      eine Gruppe!)
      
      Man erh"alt, dass 
      \[\sup M=\product{\bigcup_{U\in M} U}
        \]
    \item In $U(G)$ hat jede Teilmenge $M\subset U(G)$ ein Infimum, n"amlich
      \[\inf M=\bigcap_{U\in M} U
        \]
    \end{enumerate}
  }
% -----------------------------------------------------------------------------
\example Elemente von $U(G)$:{
  Stets sind $G\in U(G)$ und $\{1_G\}\in U(G)$. Diese beiden nennt man die
  \emph{trivialen Untergruppen} von $G$. Weiterhin erh"alt man f"ur jedes
  $g\in G$
  \[\product g:=\product{\{g\}}=\{g^z\mid z\in\SetZ\}\in U(G)
    \]
  eine Untergruppe von $G$.
  }
% -----------------------------------------------------------------------------
\definition Ordnung,Zyklizit"at:{
  \index{Gruppe>zyklische}
  Sei $G$ eine Gruppe.
  \begin{enumerate}
    \item $\ord G:=|G|$ hei"st die \emph{Ordnung} von $G$.
    \item $\ord g:=|\product g|$ hei"st die \emph{Ordnung} des Elements
      $g\in G$.
    \item $G$ hei"st \emph{zyklisch} $:\iff$ $\exists g\in G:G=\product g$.
    \end{enumerate}
  }
% -----------------------------------------------------------------------------
\subsubsection*{Bestimmung von $U((\SetZ,+))$}
% -----------------------------------------------------------------------------
\remark Zyklizit"at von $\SetZ$:{
  Es gilt $\SetZ=\product 1$, d.h. $\SetZ$ ist selbst zyklisch.
  }
% -----------------------------------------------------------------------------
\lemma Untergruppen der ganzen Zahlen:$U\leq (\SetZ,+)$, $U\neq\{0\}$=>{
  \label{lem:untergruppen-z}
  Sei $d(U):=\min\{|u|\mid u\in U\setminus\{0\} \}>0$. Dann ist
  $U=\product d=\SetZ d$.
  
  Ferner gilt f"ur $d_1,d_2\in\SetN_0$: $\product{d_1}\leq \product{d_2}\iff d_1\mid d_2$.
  }
% -----------------------------------------------------------------------------
\proof \ref{lem:untergruppen-z}:{
  Sei $u\in U$. Dann ergibt sich $-u\in U$, d.h. $|u|\in U$. Daher
  existiert ein eindeutig bestimmtes Minimum $d:=d(U):=\min\{ |u|\mid u\in U\}$
  (Wohlordnung der nat"urlichen Zahlen).
  
  Es gilt $d\in U$, also insbesondere $\product d\leq U$.
  
  Weiterhin existiert zu jedem $u\in U$ ein $k\in \SetZ$ mit $(k-1)d\leq u< kd$,
  es gilt also insbesondere $0\leq u-(k-1)d<d$, daher auch $u-(k-1)d=|u-(k-1)d|$.
  Nun ist $d\in U$, d.h. es gilt $u-(k-1)d\in U$. Da $d$ aber unter den
  Elementen von $U$ minimalen Betrag haben sollte, gilt $u-(k-1)d=0$,
  also $u=(k-1)d\in \product d$. Daher gilt $U=\product d$.
  
  Es gilt f"ur ``ferner'': $d_1\in\product{d_1}\leq \product{d_2} = \product{zd_2\mid z\in\SetZ}$
  $\iff$ $\exists z\in\SetZ:d_1=zd_2$ $\iff$ $d_2\mid d_1$.\qed
  }
% -----------------------------------------------------------------------------
\remark Verbandstheoretische Deutung:{
  $U(\SetZ)$ ist ein Verband, der zum Verband $\SetN$ mit
  umgekehrter Teilbarkeitsordnung isomorph ist. (d.h. $\exists$
  bijektive monotone Abbildung $U(\SetZ)\to \SetN$ mit $U\mapsto d(U)$.
  
  Verb"ande lassen sich grafisch veranschaulichen, so z.B.
  $\SetN$ mit umgekehrter Teilbarkeitsordnung:
  \insertfig{algebra1_teilbar_n}
  }
% -----------------------------------------------------------------------------
\subsubsection*{Bestimmung von $U(S_3)$}
% -----------------------------------------------------------------------------
\example Zyklische Permutationen:{
  \index{Permutation>zyklische}
  Seien $a_1,\ldots,a_k\subset\{1,\ldots,n\}$.  Ein $\pi$ mit 
  \[a_1\overset \pi\to a_2 \overset \pi\to a_3\overset \pi\to \cdots \overset \pi\to a_k \overset \pi\to a_1
    \]
  und $\pi(j)=j$ f"ur $j\neq a_1,\ldots,a_k$ wird \emph{zyklische Permutation} genannt.
  F"ur solche $\pi$ wird die Bezeichnung $(a_1,\ldots,a_k):=\pi$ eingef"uhrt. Offenbar
  gilt $\ord \pi =k$, und au"serdem ist klar, dass 
  $(a_1,\ldots,a_k)=(a_2,\ldots,a_k,a_1)=\cdots$.
  
  Eine zyklische Permutation $(i,j)$ hei"st \emph{\indexthis{Transposition}}.
  Nach LA gilt: Die Transpositionen erzeugen $S_n$ (Jede Transposition
  ist Produkt von Vertauschungen). Es gen"ugen zur Erzeugung der $S_n$ 
  allerdings auch die Transpositionen der Form $(i,i+1)$. Bekanntlich 
  ist $|S_n|=n!$. Wie gro"s k"onnten dann Untergruppen von $S_3$ sein?
  }
% -----------------------------------------------------------------------------
\theorem Satz von Lagrange:
  $G$ endliche Gruppe, $H\leq G$=>{
  Dann gilt $\ord H\mid \ord G$. (Siehe \ref{the:lagrange})\qed
  }
% -----------------------------------------------------------------------------
\remark:{
  Ist $\ord G=p$, wobei $p$ eine Primzahl ist, so ist $G$ zyklisch.
  }
% -----------------------------------------------------------------------------
\remark:{
  Damit: $S_3$ kann nur Untergruppen der Ordnungen $1,2,3\mid 3!=6$ enthalten.
  Dies sind gerade $\{id\},\{(12),(23),(31)\},\{(123),(123)^2\}$.
  }
% -----------------------------------------------------------------------------
\subsection{Faktorgruppen und Homomorphies"atze}
% -----------------------------------------------------------------------------
\remark Teilmengenmonoid einer Gruppe:{
  Sei $G$ Gruppe. Dann bildet die Menge $2^G$ mit der Verkn"upfung
  $A\cdot B:=\{ab\mid a\in A,b\in B\}$ f"ur $A,B\subset G$ ein Monoid. 
  (Die Assoziativit"at ist leicht nachzurechnen.)
  
  H"aufig schreibt man $aB:=\{a\}B$ f"ur derartige Verkn"upfungen mit
  einelementigen Mengen.
  
  Es gilt $(2^G)^*=\{\{g\}\mid g\in G\}$. 
  
  [Dies folgert man so: Es gilt $|Ab|=|A|$, da die Gruppenverkn"upfung
  injektiv ist f"ur jedes $b\in G$. Sucht man nun zu einer Menge $A$ das
  Inverse $B$, so muss gelten $AB=\{e\}$, also insbesondere f"ur
  jedes Element $b\in B$ $|Ab|=1$, also $|A|=1$, $|B|=1$ analog.]
  }
% -----------------------------------------------------------------------------
\example Beispiel zu Schreibweisen in $2^G$:{
  Das Untergruppenkriterium l"asst sich umformulieren. So ist $U\leq G$
  "aquivalent zu den folgenden drei Aussagen:
  \begin{enumerate}
    \item $U\cdot U=U^2\subset U$
    \item $U^{-1}\subset U$
    \item $U\neq\emptyset$.
    \end{enumerate}
  In diesem Fall gilt $U^2 =U$ ($U=1_GU\subset U^2$ gilt eh immer!).
  }
% -----------------------------------------------------------------------------
\definition Homomorphismus:{
   Seien $G,H$ Gruppen. Ein \emph{(Gruppen-)Homomorphismus} von $G$ nach $H$
   ist ein Halbgruppen-Homomorphismus $\phi:G\to H$.
   }
% -----------------------------------------------------------------------------
\remark Einfache Folgerungen "uber Homomorphismen:{
  In der Situation von obiger Definition gilt $\phi(1_G)=1_H$, $\phi(g^{-1})=\phi(g)^{-1}$.
  Insbesondere folgt $\phi(G)\leq H$.
  }
% -----------------------------------------------------------------------------
\example Homomorphismen:{
  Es ist $\exp:(\SetC,+)\to (\SetC^\times ,\cdot)$ ein Homomorphismus.
  }
% -----------------------------------------------------------------------------
\definition Nebenklassen,Quotienten,Index:{
  Sei $G$ eine Gruppe, $U\leq G$.
  \begin{enumerate}
    \item Zu $g\in G$ hei"st $gU$ die \emph{rechte} und $Ug$ die
      \emph{linke} \emph{\indexthis{Nebenklasse}}.
      (Vorsicht! Diese Definition ist in vielen B"uchern, insbesondere
      im Bosch, gerade andersherum)
    \item 
      \index{Quotient}
      $G/U:=\{gU\mid g\in G\}$ (lies: $G$ von rechts durch $U$) hei"st
      \emph{rechter Quotient} von $G$ durch/nach $U$. 
      Den \emph{linken Quotienten} $U\backslash G$ definiert man analog.
    \item $(G:U):=|G/U|$ hei"st der \emph{\indexthis{Index}} von $U$ in $G$.
    \end{enumerate}
  }
% -----------------------------------------------------------------------------
\remark Bijektion von $G/U$ nach $U\backslash G$:{
  Die Abbildung $\phi:G/U\to U\backslash G$ mit $xU\mapsto f(xU):=Ux^{-1}$ ist bijektiv.
  Es gibt also gleichviele rechte und linke Nebenklassen, d.h.
  $(G:U):=|G/U|=|U\backslash G|$.
  }
% -----------------------------------------------------------------------------
\theorem Satz von Cayley:
  $G$ Gruppe=>{
  \label{the:cayley}
  Dann ist $\phi:G\to S_G$ mit $x\mapsto \phi(x):=l_x$ mit $l_x(y):=xy$ ist ein
  injektiver Homomorphismus.
  }
% -----------------------------------------------------------------------------
\remark Folgerungen:{
  Es ist $G\cong \phi(G)$. Jede Gruppe ist isomorph zu einer Permutationsgruppe.
  (d.h. Untergruppe einer $S_M$). Folge: $|xU|=|l_x(U)|=|U|$, d.h.
  alle Nebenklassen sind gleichm"achtig.
  }
% -----------------------------------------------------------------------------
\proof \ref{the:cayley}:{
  Die Homomorphie folgt direkt durch Einsetzen, die Injektivit"at so:
  Sei $\phi(x)=\phi(y)$. Dann ist $x=l_x(1_G)=l_y(1_G)=y$, also $x=y$.\qed
  }
% -----------------------------------------------------------------------------
\theorem Satz von Lagrange:
  $G$ Gruppe, $U\leq G$=>{
  \label{the:lagrange}
  Dann gilt $|G|=|U|\cdot (G:U)$.
  }
% -----------------------------------------------------------------------------
\proof \ref{the:lagrange}:{
  F"ur $x,y\in G$ definiere $x\sim_U y:\iff xU=yU$. Das ist eine "Aquivalenzrelation.
  Die Klasse $\overline x$ von $x$ ist gerade $xU$.
  Demnach 
  \[G=\biguplus_{\alpha \in G/U} \alpha
    \]
  Da aber $|\alpha|=|U|$ und es genau $(G:U)$ $\alpha$'s gibt, folgt die Behauptung.\qed
  }
% -----------------------------------------------------------------------------
\remark Folgerungen:{
  Ist $U\leq H\leq G$, so gilt $(G:U)=(G:H)(H:U)$. Sind $U,V\leq G$, so gilt
  $UV\leq G$ $\iff $ $UV=VU$ und es gilt $(UV:V)=(U:U\cap V)$.
  }
% -----------------------------------------------------------------------------
\motivation Normalteiler:{
  Wann ist $G/U$ selbst eine Gruppe? Ein Neutralelement haben wir immer,
  wegen $U^2=U$. Die Inversenbildung ist z.B. mit $(xU)^{-1}:=x^{-1}U$ auch
  kein Problem. Aber die Abgeschlossenheit? Ist immer $xUyU\in G/U$?
  Es sollte dann $xUyU=xyU$ sein. Das tritt z.B. dann ein, wenn 
  $\forall y\in G:Uy=yU$ ist.
  }
% -----------------------------------------------------------------------------
\definition Normalteiler:{
  \index{Untergruppe>normale}
  \index{Homomorphismus>kanonischer}
  Sei $G$ Gruppe.
  Einer Untergruppe $N\leq G$ hei"st \emph{normal} oder ein \emph{Normalteiler}
  $:\iff $ $\forall x\in G:Nx=xN$.
  
  Bezeichnung: $N\unlhd G$. Vorsicht: ``$\unlhd$'' ist nicht transitiv!
  
  $G/N$ ist dann eine Gruppe (genannt die \emph{\indexthis{Faktorgruppe}}
  von $G$ nach $N$ und $\phi:G\to G/N$ mit $x\mapsto \phi(x):=xN$ ist ein
  Homomorphismus, genannt \emph{\indexthis{kanonischer Homomorphismus}}.
  }
% -----------------------------------------------------------------------------
\lessertheorem Folgerungen:$G$ Gruppe=>{
  \label{the:nt-folgerungen}
  Dann gelten:
  \begin{enumerate}
    \item Es gilt ist $U\unlhd G\iff \forall x\in G:xUx^{-1}\subset U$. (Normalteilerkriterium)
    \item Es gilt $G/U$ Gruppe $\iff$ $U\unlhd G$.
    \end{enumerate}
  }
% -----------------------------------------------------------------------------
\proof \ref{the:nt-folgerungen}:{
  \begin{enumerate}
    \item ``$\Rightarrow$'': $\surd$. ``$\Leftarrow$'': Offenbar gilt: $xU=Ux\iff xUx^{-1}=U$.
      Kann nun $xUx^{-1}\subsetneq U\iff U\not\subset xUx^{-1}$ vorkommen?
      Sei $xUx^{-1}\subset U$. Dann gilt $U=x^{-1}(xUx^{-1})x\subset U$, aber auch
      $U=x^{-1}(xUx^{-1})x\subset xUx^{-1}$.
    \item ``$\Leftarrow$'': $\surd$. ``$\Rightarrow$'': Sei $G/U$ Gruppe. Dann existiert
      f"ur alle $x\in G$ ein $y\in G$, so dass $xUyU=U$, also auch
      $x1y1\in U$, d.h. $U=xyU$, also $x^{-1}U=yU$
      Das ergibt $xUx^{-1}U=U$ und bedeutet $xUx^{-1}1\subset U$.
    \end{enumerate}
  }
% -----------------------------------------------------------------------------
\remark Wichtige andere Darstellung von $G/N$:{
  $\overline x=Nx$ ist "Aquivalenzklasse, $x$ ein Vertreter von
  $\overline x=\alpha$. Das Rechnen mit diesen Klassen geschieht vertreterweise.
  $\alpha,\beta\in G/N$. W"ahle Vertreter $x\in \alpha$ und $y\in\beta$. Dann ist $xy$ Vertreter
  von $\alpha\beta$.
  
  Wir betrachten nun ein Hilfsmittel, mit dessen Hilfe sich feststellen
  l"asst, ob $H\unlhd  G$:
  }
% -----------------------------------------------------------------------------
\definition Innere Automorphismen:{
  \index{Automorphismus>innerer}
  \label{def:innerer-automorphismus}
  Sei $G$ Gruppe.
  F"ur alle $x\in G$ ist $in_x:G\to G$ mit $y\mapsto in_x(y):=xyx^{-1}$
  ein Automorphismus von $G$, genannt \emph{innerer Automorphismus}.
  }
% -----------------------------------------------------------------------------
\proof \ref{def:innerer-automorphismus}:{
  Endomorphismus: $in_x(gh)=xghx^{-1}=xgx^{-1}xhx^{-1}=in_x(g)in_x(h)$.
  
  Bijektivit"at: $in_x\circ in_{x^{-1}}=in_{x^{-1}}\circ in_x=\id$.\qed
  }
% -----------------------------------------------------------------------------
\definition Konjugation:{
  \index{Untergruppe>konjugierte}
  \index{Konjugiertenklasse}
  Sei $G$ Gruppe, $U,V\leq G$. $U\sim V:\iff \exists x\in G:in_x(U)=V$ erkl"art
  eine "Aquivalenzrelation, falls $U\sim V$ gilt, sagt man,
  $U$ und $V$ seien \emph{zueinander konjugiert}.
  
  Es gilt $in_x(U)\leq G=in_x(U)$, weiter $|in_x(U)|=|U|$, deshalb sind z.B.
  alle zueinander konjugierten Untergruppen gleichm"achtig.
  
  Die Menge $\{in_x(H)\mid x\in G\}$ hei"st \emph{Konjugiertenklasse} von $H$.
  }
% -----------------------------------------------------------------------------
\remark Folgerung:{
  Es ist klar, dass $H\unlhd G$ $\iff$ Die Konjugiertenklasse besteht nur aus $H$.
  Damit erh"alt man: Ist $H$ einzige Untergruppe einer bestimmten Ordnung
  in $G$, so ist $H\unlhd G$. Andersherum: Gibt es mehrere konjugierte 
  Untergruppen, so ist keine von ihnen ein Normalteiler.
  }
% -----------------------------------------------------------------------------
\example Normalteiler und $\SetZ$:{
  Wegen Abelizit"at von $\SetZ$ ist zun"achst jede Untergruppe
  ein Normalteiler. Die Nebenklassen $\overline z$ haben folgende
  Gestalt: $\overline z=z+d(U)\SetZ$. 
  
  Damit: $\SetZ/U=\{0+d(U)\SetZ,\ldots,d(U)-1+d(U)\SetZ\}$.
  }
% -----------------------------------------------------------------------------
\subsection{Der Homomorphiesatz f"ur Gruppen}
% -----------------------------------------------------------------------------
\remark{Erinnerung: Klassenbildung}:{
  Sei $M$ eine Menge und ``$\sim$'' eine "Aquivalenzrelation. Sei $m\in M$.
  Dann ist $d(m):=\overline m:=\{x\in M\mid x\sim m\}$  die \emph{Klasse von $m$},
  $M/\sim:=\{d(m)\mid m\in M\}$ der \emph{Quotient} von $M$ und $\sim$ sowie
  $k:M\to M/\sim$ mit $m\mapsto \overline m$ die \emph{kanonische Abbildung}.
  
  Klassenbildung hat zwei Aspekte:
  \begin{itemize}
    \item Klassen sind Teilmengen von $M$, deren disjunkte Vereinigung 
      $M$ ist.
    \item Klassen sind Elemente (``Punkte'') des Quotienten $M/ \sim$.
      Vorstellung: Die $x\in \overline m$ ``verschmelzen'' zu einem
      neuen Ding, n"amlich zu $\overline m$.
    \end{itemize}
  \insertfig{algebra1_quotient}
  $g$ existiert $\iff$ $f$ konstant auf Klassen. Eine beliebte Schreibweise
  ist $g=\overline f$.
  }
% -----------------------------------------------------------------------------
\definition Bild-"Aquivalenzrelation:{
  \index{Faser}
  Hat man eine surjektive Abbildung $f:M\to A$ (z.B. $A=M/ \sim $, $f=k$), so
  ist $x\sim_f y :\iff f(x)=f(y)$ eine "Aquivalenzrelation. 
  (s. auch \ref{the:homomorphie-mengen})
  
  Die Klassen sind $\overline x=f^{-1}(\{f(x)\})$. Sie werden 
  \emph{Fasern} genannt.
  }
% -----------------------------------------------------------------------------
\definition Schnitt:{
  Sei $f:M\to A$ eine Abbildung.
  Ein \emph{Schnitt} ist eine Abbildung $s:A\to M$ mit $f\circ s=\id_A$.
  Beachte, dass die Forderung impliziert, dass $f$ surjektiv sein muss.
  }
% -----------------------------------------------------------------------------
\remark Folgerung:{
  In der Situation von oben ist $\{s(a)\mid a\in A\}$ ein Vertretersystem der Klassen.
  
  Merke: Schnitte entsprechen gerade den Vertretersystemen bei $\sim_f$.
  }
% -----------------------------------------------------------------------------
\example Schnitte:{
  Sei $G$ eine Gruppe, $U\leq Z$, $G/U=G/ \sim$, d.h. $x\sim y= xU=yU$.
  Betrachte speziell $G=\SetZ$, $U=d\SetZ$. 
  Dann ist $\overline z=z+d\SetZ$, und der Schnitt ist $s=r_d$, wobei
  $r_d(z):= z\bmod d$.
  }
% -----------------------------------------------------------------------------
\theorem Homomorphiesatz f"ur Mengen:
  $f:M\to N$ eine Abbildung=>{
  \label{the:homomorphie-mengen}
  Dann ist $x \sim_f:\iff f(x)=f(y)$ eine "Aquivalenzrelation,
  $\overline f(\overline x):=f(x)$ ist wohldefiniert. Weiterhin kommutiert
  das folgende Diagramm:
  \insertfig{algebra1_homsatz_mengen}
  und es gelten au"serdem
  \begin{enumerate}
    \item $k$ ist surjektiv
    \item $\overline f$ ist bijektiv
    \item $i$ (Inklusion) ist injektiv.\qed
    \end{enumerate}
  }
% -----------------------------------------------------------------------------
\theorem Homomorphiesatz f"ur Gruppen:
  $\phi :G\to H$ sei Gruppenhomomorphismus, $N:=\ker\phi=\phi^{-1}(\{1_H\})$=>{
  \label{the:homomorphie-gruppen}
  Dann gilt $N\unlhd G$ und $\phi$ hat die kanonische Faktorisierung
  $\phi=i\circ \overline \phi\circ k$, wobei
  \begin{enumerate}
    \item $i$, die Inklusion, ein Monomorphismus ist.
    \item $k$, die kanonische Abbildung $k(g):=gN=\overline g$, ein
      Epimorphismus ist.
    \item $\overline\phi$ mit $\overline \phi(\overline g)=\phi(g)$ ein
      Isomorphismus ist.
    \end{enumerate}
  Betrachte auch folgendes kommutative Diagramm:
  \insertfig{algebra1_homsatz_gruppen}
  }
% -----------------------------------------------------------------------------
\proof \ref{the:homomorphie-gruppen}:{
  Wende \ref{the:homomorphie-mengen} an auf $f=\phi$. 
  Leicht nachzurechnen: $g\sim_f h\iff gh^{-1}\in \ker\phi=N\iff gN=hN\iff g\sim h$. 
  Demnach $G/ \sim_f = G/N$. 
  
  Klar: $i,k$ Homomorphismen, Bijektivit"at von $\overline \phi$.
  
  $\overline \phi$ Homomorphismus: 
  $\overline\phi(\overline g\overline h)
    =\overline\phi(\overline{gh})
    =\phi(gh)
    =\phi(g)(h)
    =\overline\phi(\overline g)\overline\phi(\overline h)$.
  \qed
  }
% -----------------------------------------------------------------------------
\remark Folgerung:{
  Es gilt $G/N=G/ \ker\phi \cong \phi(G)$.
  
  Kennt man alle Normalteiler von $G$, so kennt man alle homomorphen
  Bilder von $G$ (bis auf Isomorphie).
  }
% -----------------------------------------------------------------------------
\definition Einfache Gruppe:{
  \index{Gruppe>einfache}
  Sei $G$ eine Gruppe. $G$ hei"st \emph{einfach} $:\iff $ $|G|>1$ und $G$
  hat nur den trivialen Normalteiler $\{1_G\}$.
  }
% -----------------------------------------------------------------------------
\remark Anwendung des Homomorphiesatzes:{
  $U(\SetZ)$ ist bekannt, also auch alle homomorphen Bilder von $\SetZ$:
  Dies sind gerade die zyklischen Gruppen. Insbesondere ist eine zyklische
  Gruppe $G$ mit erzeugendem Element $g$ isomorph zu $\SetZ/d\SetZ$ mit
  \[d:=\begin{cases}
      \ord G & \ord G<\infty \\
      0 & \ord G=\infty
      \end{cases}
    \]
  unter dem Isomorphismus $\phi:\SetZ/d\SetZ\to G$ mit $z\mapsto \phi(z):=g^z$.
  Speziell gilt $g^i=g^j\iff \overline i=i+d\SetZ=j+d\SetZ=\overline j\iff
  i-j\in d\SetZ\iff d\mid i-j$.
  }
% -----------------------------------------------------------------------------
\definition Zwischengruppenverband:{
  Sei $N$ ein Normalteiler.
  Dann nennt man
  \[U(G,N):=\{H\mid N\leq H\leq G\}
    \]
  den \emph{Zwischengruppenverband} von $G$ und $N$.
  }
% -----------------------------------------------------------------------------
\theorem Untergruppen-Entsprechungssatz:
  $\phi:G\to\tilde G$ surjektiver Gruppenhomomorphismus, $N=\ker \phi$=>{
  \label{the:untergruppen-entsprechung}
  Dann gelten die folgenden Aussagen:
  \begin{enumerate}
    \item $\phi$ induziert eine bijektive anordnungstreue streng monotone
      Abbildung $\phi:U(G,N)\to U(\tilde G)$. Man nennt dann f"ur ein
      $H\in U(G,N)$ die Gruppe $\phi(H)$ die \emph{Bildgruppe} von $H$ unter
      $\phi$.
    \item Die Konjugiertenklasse von $H\in U(G,N)$ wird auf die 
      Konjugiertenklasse von $\phi(H)$ abgebildet. Insbesondere
      gilt $H \unlhd G\iff \phi(H) \unlhd \phi(G)=\tilde G$.
    \item Ist $H\unlhd G$, so gilt $G/H\cong \phi(G)/ \phi(H)$.
    \item Ist $N\leq H\leq H'\leq G$, so ist $(H':H)=(\phi(H'):\phi(H))$.
    \end{enumerate}
  }
% -----------------------------------------------------------------------------
\proof \ref{the:untergruppen-entsprechung}:{
  Zu 1.: Wegen der Surjektivit"at von $G$ folgt $\phi(\phi^{-1}(K))=K$ f"ur alle
  Elemente $K\in U(\tilde G)$. F"ur eine Untergruppe $H\leq G$ gilt weiterhin
  auch $\phi^{-1}(\phi(H))=H$, oder
  
  $x\in\phi^{-1}(\phi(H))\iff \phi(x)\in \phi(H)\overset !\iff x\in H$. Dabei ist ``$\Rightarrow$'' die 
  interessante Richtung, die wir jetzt zeigen m"ochten. Existiere also
  nach Voraussetzung ein $y\in H$, so dass $\phi(x)=\phi(y)\iff \phi(xy^{-1})=1$.
  Dann ist $xy^{-1}\in\ker \phi=N\leq  H\leq G$. Wegen $y\in H$ folgt dann auch
  $x\in H$.
  
  Damit wissen wir, dass $U(G,N)$ bijektiv auf $U(\tilde G)$ abgebildet 
  wird.
  
  Zu 2.: Sei $H'=xHx^{-1}$. Dann ist $\phi(H')=\phi(x)\phi(H)\phi(x)^{-1}$, und damit
  gezeigt, dass die Bild-Konjugiertenklasse in der ganzen 
  Konjugiertenklasse in $\tilde G$ liegt. 
  Sei andersherum jetzt $K'=yKy^{-1}$ mit $y\in G'$ und $K\leq \tilde G$.
  Nach 1. existiert dann genau jeweils ein $H',H\in U(G,N)$ mit
  $\phi(H')=K'$ und $\phi(H)=H'$ und wegen der Surjektivit"at von $\phi$
  auch ein $x\in G$ mit $\phi(x)=y$. Dann ist
  $K'=yKy^{-1}=\phi(x)\phi(H)\phi(x)^{-1}=\phi(xHx^{-1})$.
  Wegen der Injektivit"at von $\phi$ folgt jetzt aus $\phi(H')=\phi(xHx^{-1})$ auch
  $H'=xHx^{-1}$.
  
  Zu 3.,4.: Zu festem $H$ betrachte die Abbildung $\psi:G\to\tilde G\land phi(G)$, die
  als Verkettung von $\tilde k\circ \phi$ entsteht, wobei $\tilde k$ die kanonische Abbildung
  $\tilde k:\tilde G\to \tilde G/ \phi(G)$ ist.
  $\psi$ ist offenbar surjektiv.
  
  Wende nun den Homomorphiesatz f"ur Mengen auf $\psi$ an.
  \insertfig{algebra1_ug_entspr_proof}
  Es gilt 
  \begin{align*}
    x\sim_\psi y 
    &\iff \tilde k(\phi(x))=\tilde k(\phi(y))
    \iff \phi(x)\phi(H)=\phi(y)\phi(H)\\
    &\iff \phi(xy^{-1})\in\phi(H)
    \iff xy^{-1}\in H
    \iff xH=yH
    \end{align*}
  Damit haben wir gesehen, dass gilt $G/ \sim_\psi =G/H$, woraus insbesondere
  $(G:H)=|\tilde G/ \phi(H)|$ folgt. 
  
  F"ur 4. darf o.B.d.A. angenommen werden,
  dass $H'=G$. Falls $H\unlhd G$, so ist $\overline \phi$ ein Gruppenisomorphismus,
  womit auch 3. erschlagen ist.\qed
  }
% -----------------------------------------------------------------------------
\theorem Hauptsatz "uber zyklische Gruppen:=>{
  \label{the:zyklische-gruppen-hauptsatz}
  \index{Gruppe>zyklische>Hauptsatz "uber}
  \index{Zyklische Gruppen>Hauptsatz "uber}
  Es gelten:
  \begin{enumerate}
    \item Zu jedem $d\in\SetN_{>0}$ gibt es bis auf Isomorphie genau eine 
      zyklische Gruppe $Z_d$ der Ordnung $d$ und genau eine unendliche
      zyklische Gruppe.
    \item Alle Untergruppen von $Z_d$ sind auch wieder zyklisch und entsprechen
      bijektiv den Teilern von $d$ wobei $Z_a \leq Z_b \iff a\mid b$.
    \end{enumerate}
  }
% -----------------------------------------------------------------------------
\proof \ref{the:zyklische-gruppen-hauptsatz}:{
  Es gilt $Z_d\cong Z/d\SetZ$ (siehe oben), 
  Rest folgt aus \ref{the:untergruppen-entsprechung}.\qed
  }
% -----------------------------------------------------------------------------
\theorem Noether's K"urzungsregel, Erster Isomorphiesatz f"ur Gruppen:
  $G$ Gruppe, $N\unlhd G$, $N\leq H\unlhd G$=>{
  \index{1. Isomorphiesatz f"ur Gruppen}
  \index{Erster Isomorphiesatz f"ur Gruppen}
  \index{K"urzungsregel>Noether's}
  \label{the:1-gruppen-iso}
  Dann ist $(G/N)/(H/N)\cong G/H$.
  }
% -----------------------------------------------------------------------------
\proof \ref{the:1-gruppen-iso}:{
  Sei $\phi:=k$ die kanonische Abbildung von $G$ nach $G/N$. Dann folgt
  die geforderte Isomorphie nach dem Untergruppenentsprechungssatz.
  \qed
  }
% -----------------------------------------------------------------------------
\theorem Zweiter Isomorphiesatz f"ur Gruppen:
  $G$ Gruppe, $U\leq G$, $N\unlhd G$=>{
  \index{2. Isomorphiesatz f"ur Gruppen}
  \label{the:2-gruppen-iso}
  Dann ist $UN\leq G$, weiter $N\leq UN$ und $UN/N\cong U/(U\cap N)$, genauso wieder
  $U(UN,N)\cong U(U,U\cap N)$.\qed
  }

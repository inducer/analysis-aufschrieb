\def\Ob#1{\operatorname{Ob}_{\mathcal{#1}}}
\def\Mor(#1;#2;#3){\operatorname{Mor}_{\mathcal{#1}}(#2,#3)}
%----------------------------------------------------------------
\para{Zwischenspiel: Kategorien, eine Sprechweise}
%----------------------------------------------------------------
\subsection{Einf"uhrung und Definitionen}
%----------------------------------------------------------------
\motivation{}:{(Eventuell kommutative) Diagramme mit ``Pfeilen'' ist
  man gewohnt in folgenden Situationen:\\
  \begin{tabular}{c|c|c}
  Objekte  & Pfeile & Name (in dieser Vorlesung) der Kategorie \\
  \hline
  Mengen  & Abbildungen  & \underline{Set} \\
  Gruppen & Homomorphismen & \underline{Gp} \\
  Monoide & Homomorphismen & \underline{Mon} \\
  K-Vektror"aume & lineare Abbildungen & \underline{K-Mod} \\
  Topologische R"aume & stetige Abbildungen & \underline{Top} \\
  offene Teilmengen U  & stetige Abbildungen & C \\
  in geeigneten $\SetR^n$ & & \\
  ---''--- & $\infty$ oft db. Funktionen & $C^{\infty}$ \\
  Elemente einer  & $a\geq b$, '$\geq$' als Pfeil & \\
  geord. Menge & & \\  
  \end{tabular}	
 Gesucht ist nun eine algebraische Begriffsbildung, die alle diese
 Beispiele als Spezialfall enth"alt.$\Rightarrow$ {\emph \indexthis{Kategorie}}.		
			}
%-----------------------------------------------------------------
\definition{Kategorie}:{Eine Kategorie $\mathcal{K}$ ist gegeben durch
 \begin{enumerate}
 \item[(i)] Eine Klasse $\Ob{K}$ von ''Objekten''\\
 ($A\in$ $\Ob{K}$ heisst Objekt von $\operatorname{Ob}_{\mathcal{K}}$)
 \item[(ii)] $\forall A,B\in $$\Ob{K}$ eine \underline{Menge}
 $\Mor(K;A;B)$ \\
 ($f\in \Mor(K;A;B)$ heisst Pfeil oder Morphismus in $\mathcal{K}$
 $f: A\longrightarrow B$ mit Ursprung A und Ziel B. Schreibe: $A= $ Ursprp(f), $B=$ Ziel(f).)
 \item[(iii)] (Komposition von Pfeilen)\\
 $\forall A,B,C \in \Ob{K}$ eine Verkn"upfung \\
 \begin{tabular}{cccc}
  $v_{A,B,C}:$ & $\Mor(K;B;C)\times \Mor(K;A;B)$ & $\longrightarrow$ & $ \Mor(K;A;C) $\\
               & $(g,f)$                         & $\mapsto$         & $v_{A,B,C}(g,f):=g \circ f$
 \end{tabular}
 \end{enumerate}
 derart, dass gelten: \\
\begin{enumerate}
\item[(KAT1)] (Disjunktheitsaxiom)\\
Jeder Pfeil hat eindeutig bestimmten Ursprung und eindeutig bestimmtes Ziel.\\
(d.h. $\emptyset \neq \Mor(K;A;B) \cap \Mor(K;A';B') \Rightarrow A=A' \land B=B'$)
\item[(KAT2)] (Assoziativit"at der Komposition)\\
Wo definiert, ist die Komposition assoziativ.\\
$A\stackrel{f}{\longrightarrow} B\stackrel{g}{\longrightarrow} C\stackrel{h}{\longrightarrow}D$ \\
$\forall A,B,C,D\in \Ob{K}$ $\forall f\in \Mor(K;A;B)$,$g\in \Mor(K;B;C)$,
$h \in \Mor(K;C;D)$: $h(gf)=(hg)f$
\item[(KAT3)](Existenz der neutralen Pfeile)\\
$\forall A\in \Ob{K} \exists e_A\in \Mor(K;A;A):\forall B\in \Ob{K} $
$\forall f\in \Mor(K;A;B):$ $fe_A=f$ \\
\insertfig{algebra1_2_1_KAT3_1}
$\forall B\in \Ob{K} \forall g\in \Mor(K;B;A)$
$e_Ag=g$\\
\insertfig{algebra1_2_1_KAT3_2}
\end{enumerate}  
   }
%--------------------------------------------------------------------
\remark{}:{In Kategorien geht es zu wie bei Mengen und Abbildungen (als Pfeilen), 
 nur brauchen die Objekte keine Mengen und die Pfeile keine Abbildungen zu sein.\\
 klar: In der Liste stehen Kategorien. Pfeile sind Abbildungen mit gewissen 
 Eigenschaften.$e_A=id_A$ hat die gew"unschten Eigenschaften, die Komposition 
 ist die Abbildungsschachtelung. (ben"otigt wird: Verschachtelt man zwei Abbildungen
 mit einer bestimmten Eigenschaft, so entsteht wieder eine Abbildung 
 mit derselben Eigenschaft. )\\
 Zum letzten Eintrag: Sei $\mathcal{K}$ Kategorie mit $\Ob{K}=M$,
 wobei $M$ eine Menge ist. $\forall a,b\in M$ sei $|\Mor(K;a;b)|\leq 1$. 
 Schreibe $a\geq b\iff \Mor(K;a;b)=\{\geq\}$.
 Komposition gilt, da ''$\geq$'' transitiv, d.h.
 $a\stackrel{f}{\geq} b\stackrel{g}{\geq} c \Rightarrow a\stackrel{gf}{\geq}$.\\
 $a\geq a$ gilt immer, da ''$\geq$'' reflexiv ist.
 
 Gesehen: Kategorien dieser Art sind genau die Mengen mit transitiver und
 reflexiver Relation.
  }   
%--------------------------------------------------------------------------
\lemma{}:{}=>{\label{lem_KAT_1}
$A\in \Ob{K}$ mit $e_A$ bestimmen sich gegenseitig eindeutig.}
%---------------------------------------------------------------------------
\proof \ref{lem_KAT_1}:{\begin{enumerate}
  \item[(a)] Disjunktheitsaxiom
  \item[(b)] Wie bei Monoiden zeigt man: $e_A,e_{A'}$ beide neutral
  $\Rightarrow e_A=e_Ae_{A'}=e_{A'}$.
  \insertfig{algebra1_2_1_lemma1}
  \end{enumerate}
  \qed  }	           
%--------------------------------------------------------------------------
\remark{}:{Man k"onnte also auf $\Ob{K}$ ganz verzichten und nur Axiome
 f"ur die Klasse aller Pfeile ($\operatorname{Mor}_{\mathcal{K}}$)
 formulieren.(Algebraische Struktur mit partieller Verkn"upfung (i.a.
 nicht "uberall definiert) mit Axiomen.)}
%---------------------------------------------------------------------------
\remark{Kommentare zu den Axiomen}:{\begin{enumerate}
\item Warum Klasse $\Ob{K}$ statt Menge? \\
Grund:z.B. Der Begriff Menge aller Mengen ist unsinnig. Man sagt ''Klasse
aller Mengen''. In der axiomatischen Mengenlehre nach G"odel/Bernaup sagt man
''Klassen'' statt ''Mengen'' in der naiven Mengenlehre. Nicht alle Konstruktionen,
die man f"ur Mengen hat, sind f"ur Klassen erlaubt, d.h. d"urfen Klassen, die nicht
Mengen sind, nicht Elemente anderer Klassen sein. Eine Kategorie, bei der $\Ob{K}$
eine Menge bildet, heisst klein.
\item In jeder Kategorie kann man ''Diagramme'' und Kommutativit"at von Diagrammen
wie in \underline{Set} erkl"aren.(Kommutieren wird durch Schraffur angedeutet.)\\
\insertfig{algebra1_2_1_kommutieren}
Regel: Setzt man mehrere kommutative Diagramme zusammen und ist dann das ganze
Diagramm schraffiert, so ist auch dieses kommutativ.
\end{enumerate}}                                    
%---------------------------------------------------------------------------
\example{$\underline{\operatorname{Set}}_*$, ''Kategorie der punktierten Mengen''}:{
Die Objekte sind die Paare $(X,x)$, wobei $ X\in \Ob{\operatorname{Set}}, x\in X$ (ausgezeichneter Punkt).\\
Pfeil: $f:(X,x)\longrightarrow (Y,y)$
ist Abbildung $f:X\longrightarrow Y$ mit $f(x)=y$. Die zugeh"orige Komposition 
ist die Schachtelung.}
%---------------------------------------------------------------------
\remark{}:{$\mathcal{K}$ \label{aequivalent_1}
sei eine Kategorie. Dann sind "aquivalent:
\begin{enumerate}
\item[(i)] $|\Ob{K}|=1$, etwa $\Ob{K}=\{A\}$
\item[(ii)] $\operatorname{Mor}_{\mathcal{K}}$=$\Mor(K;A;A)$
\item[(iii)]$\operatorname{Mor}_{\mathcal{K}}$ ist ein Monoid
\end{enumerate}}                          		
%---------------------------------------------------------------------
\proof \ref{aequivalent_1}:{
  \begin{description}
    \item{(i)$\longrightarrow$(ii)}  trivial
    \item{(ii)$\longrightarrow$(iii)}
	  klar, da in jeder Kategorie die $\Mor(K;A;A)$
      ein Monoid bilden (neutrales Element ist $e_A$). Insbesondere hat man
      man die Gruppe   $\operatorname{Aut}(A)=\Mor(K;A;A)$  
    \item{(iii)$\longrightarrow$(i)}  $\forall f,g\in \operatorname{Mor}_{\mathcal{K}}$:
    $\operatorname{Ziel}(f)=\operatorname{Ursp}(g)$
    $\Rightarrow \exists^{1}$  Ursp $\land \Rightarrow \exists^{1}$ Ziel
    $\Rightarrow$ genau ein Objekt. Ist z.B $M$ ein Monoid, so erfinde
    Objekt (d.h. z.B Buchstabe $A$). Setze $\Ob{K}=\{A\}$,
    $\Mor(K;A;A):=M$ dann ist $\mathcal{K}$ eine Kategorie.
    \end{description}
\qed	
}
%--------------------------------------------------------------------
\subsection{Isomorphismus, Funktoren}
%--------------------------------------------------------------------
\remark{}:{Ein weiterer Vorzug des Begriffs ist zum Beispiel die 
einheitliche Definition f"ur Isomorphismus (und Automorphismus)
bei allen Klassen.}
%--------------------------------------------------------------------
\definition{Isomorphismus}:{\begin{enumerate}
\item[(i)] $\mathcal{K}$ sei eine Kategorie. Ein \indexthis{Isomorphismus} $f:A\longrightarrow B$
ist ein invertierbarer Pfeil. Genauer (wie be Monoiden):\\
$f:A\longrightarrow B$ invertierbar $:\iff \exists g,h\in \Mor(K;B;A)$
mit $gf=e_A$, $fh=e_B$
\insertfig{algebra1_2_1_dreieck_iso}
Wie bei Monoiden zeigt man: $g=h$  und eindeutig bestimmt. Bezeichnung ist
$g=h=f^{-1}$.
$\operatorname{Aut}(A)$ besteht aus den Isomorhismen $f:A\longrightarrow A$
\item[(ii)] Zwei Objekte $A,B$ heissen isomorph $\iff \exists$
Isomorphismus $f:A\longrightarrow B$. Bezeichnung: $A\cong B$.
``$\cong$'' ist "Aquivalenzrelation auf $\Ob{K}$. (Transitivit"at gilt, da
wie bei Monoiden $(gf)^{-1}=f^{-1}g^{-1}$)
\end{enumerate}}
%----------------------------------------------------------------------
\definition{Unterkategorie}:{Sei $\mathcal{K}$ eine Kategorie.
$\mathcal{L}$ (Kategorie) heisst Unterkategorie von $\mathcal{K}$, wenn gelten:
\begin{enumerate}
\item[(i)] $\Ob{L} \subseteq \Ob{K}$
\item[(ii)] $\forall A,B\in \Ob{L}: \Mor(L;A;B)\subseteq \Mor(K;A;B)$
\item[(iii)] Die Komposition in $\mathcal{L}$ ist die aus $\mathcal{K}$.
\end{enumerate}
$\mathcal{L}$ heisst {\emph volle} Unterkategorie
$\iff \forall A,B \in \Ob{L}: \Mor(L;A;B)=\Mor(K;A;B)$.
Zu jeder Teilklasse $\mathcal{M}$ kann man die volle Unterkategorie
$\mathcal{L}$ um $\Ob{L}=\mathcal{M}$ bilden.
}
%------------------------------------------------------------------------
\example{}:{\begin{enumerate}
\item \underline{Ab} (Kategorie der abelschen Gruppen) bildet volle
Unterkategorie von \underline{Gp}.
\item Die endlichdimensionalen K-Vektorr"aume bilden volle Unterkategorie
der K-Vektorr"aume \underline{K-Mod}
\end{enumerate}}
%------------------------------------------------------------------------
\motivation{zur Begriffsbildung Funktor}:{Die ''Homomorphismen'' zwischen Kategorien 
(sie werden Funktoren genannt) sind wie im Spezialfall der Monoide
die Abbildungen\\
$\mathcal{F}: \operatorname{Mor}_{\mathcal{A}}\longrightarrow \operatorname{Mor}_{\mathcal{B}}$, \\
die die Komposition und die Neutralen respektieren. Zu $e_A$ soll
$\mathcal{F}(e_A)$ neutrales Element in $\mathcal{B}$ sein, d.h.
$\mathcal{F}(e_A)=e_{A'}$ mit $A' \in \Ob{B}$.}
%-----------------------------------------------------------------------
\definition{Funktor}:{Eine Abbildung $\mathcal{F}$ heisst kovarianter Funktor 
\index{Funktor, kovarianter}
von der Kategorie $\mathcal{A}$ in die Kategorie $\mathcal{B}$, wenn folgende
Forderungen erf"ullt sind:
\begin{enumerate}
\item[(Fun1)] Man hat die Abblidung $\mathcal{F}: \Ob{A}\longrightarrow \Ob{B}$,
$A\mapsto A'=: \mathcal{F}(A)$.
\item[(Fun2)] Man hat die Abbildung $\mathcal{F}: \operatorname{Mor}_{A}
\longrightarrow \operatorname{Mor}_{B}$, mit
  \begin{enumerate}
  \item[(a)] $\forall A,B \in \Ob{A}: \mathcal{F}(\Mor(A;A;B)) \subseteq$
  $\Mor(B;\mathcal{F}(A);\mathcal{F}(B))$
  \item[(b)] (Homomorphieregel)\\
  $\forall A,B,C \in \Ob{A}$, $f\in \Mor(A;A;B)$, $g\in \Mor(A;B;C)$:
  $\mathcal{F}(gf)=\mathcal{F}(g) \mathcal{F}(f)$
  \item[(c)] $\forall A\in \Ob{A}$: $\mathcal{F}(e_A)=e_{\mathcal{F}(A)}$.
  \end{enumerate} 
\end{enumerate}
Ein {\emph kontravarianter} Funktor \index{Funktor, kontravarianter}
$\mathcal{G}$ ist definiert wie ein kovarianter
Funktor, nur dass der Pfeil im Bild in die umgekehrte Richtung zeigt, d.h.\\
$\mathcal{G}(\Mor(A;A;B)) \subseteq \Mor(B;\mathcal{G}(B);\mathcal{G}(A))$.\\
Die Homomorphieregel lautet dann:\\
$\mathcal{G}(gf)=\mathcal{G}(f)\mathcal{G}(g)$.
}
%---------------------------------------------------------
\remark{zu Funktoren}:{
\begin{enumerate}
\item(Fun2) folgt aus dem Respektieren von Kompositionen, 
d.h. $f,g\in \operatorname{Mor}_\mathcal{A}$, so dass $gf$ definiert ist,
so soll $\mathcal{F}(g)\mathcal{F}(f)$ in $\mathcal{B}$ definiert sein und
$\mathcal{F}(gf)= \mathcal{F}(g) \mathcal{F}(f)$.
Weiter: $f\circ e_A=f \Leftrightarrow A$ Ursprung von $f$, $\mathcal{F}(f\circ e_A)$
$=\mathcal{F}(f)\mathcal{F}(e_A)$, d.h. $\mathcal{F}(e_A)=e_{\mathcal{F}(A)}$
$\Rightarrow e_{\mathcal{F}(A)}$ ist Ursprungsneutralelement von $\mathcal{F}(f)$.
Analog f"ur Ziel. Insgesamt ergibt dies (Fun2).
\item Ein Funktor bildet kommutative Diagramme in $\mathcal{A}$ auf
kommutative Diagramme in $\mathcal{B}$ ab.

in $\mathcal{A}$ 
\insertfig{algebra1_2_1_diagramm1} 
in $\mathcal{B}$ (kovarianter Funktor)
\insertfig{algebra1_2_1_diagramm2}
in $\mathcal{B}$ (kontavarianter Funktor) 
\insertfig{algebra1_2_1_diagramm3} 

\item gesehen: Ist $\Ob{A}=\{A\}$, $\Ob{B}=\{B\}$, dann:

$\mathcal{F}: \mathcal{A}\longrightarrow$
$\mathcal{B}$ ist Funktor 
$\iff$ $\mathcal{F}: \Mor(A;A;A)\longrightarrow \Mor(B;B;B)$ 
ist Monoidhomomorphismus.
\item Wenn nicht ausdr"ucklich erw"ahnt, so sei nachfolgend ein Funktor stets kovariant.
\end{enumerate}
}

%--------------------------------------------------------------
\subsection{Beispiele zu Funktoren}
%--------------------------------------------------------------
Praxiserfahrung: Konstruiert man zu $A\in \Ob{A}$ ein $\mathcal{F}\in \Ob{B}$
in vern"unftiger Weise, so dass die Konstruktion f"ur alle $A$ ''gleichartig''
gemacht wird, so ist $\mathcal{F}$ fast immer ein Funktor.

%--------------------------------------------------------------
\example{1}:{
$\mathcal{F}:$ \underline{Mon} $\longrightarrow$ \underline{Gp}, $\mathcal{F}(M)=M^{\times}$
(Einheitengruppe) ergibt einen Funktor.

Sei $\phi :M\longrightarrow N$ ein Pfeil in \underline{Mon}.
Was soll dann $\mathcal{F}(\phi)=: \phi _{\times}$ sein?\\
$\phi_{\times}=\phi _{| M^\times}:M^{\times}\longrightarrow N^{\times}$.
Es gilt n"amlich $u\in M^{\times} \Rightarrow \phi(u)\in N^{\times}$, und zwar wegen:
$\exists u^{-1}:uu^{-1}=e_{M^{\times}}=u^{-1}u$
$\Rightarrow \phi(uu^{-1})=\phi(u)\phi(u^{-1})=e_{N^{\times}}=\phi(u^{-1})\phi(u)$,
also $\phi(u^{-1})=\phi(u)^{-1}$. Ausserdem gilt: $\phi(id_M)=id_{\phi(M)}$, 
$id_M$ neutraler Pfeil in \underline{Mon}. 
}
%-----------------------------------------------------------------
\example{2}:{ $G$ Gruppe, $\mathcal{F}(G)=2^G$ das Teilmengenmomoid von $G$.
$\phi: G\longrightarrow H$ ein Gruppenhomomorphismus, $A\subseteq G$, d.h.
$A\in 2^G$, so $\phi _*(A)=\phi(A)$,
$\phi _*: 2^G \longrightarrow 2^H$, $\phi _*(AB)=\phi(AB)=\phi(A)\phi(B)=\phi _*(A)\phi _*(B)$,
also $\phi _*$ Pfeil in \underline{Mon}. $\mathcal{F}(\phi)=\phi_*$
}
%-----------------------------------------------------------------------
\example{3 Der Dualraum}:{$V$ (ein K-Vektorraum) $\longrightarrow \mathcal{G}(V):=V^*$,
wobei $V^*=\operatorname{Hom}(V,K)$ der Dualraum sei.

$V\stackrel{f}{\longrightarrow} W$, $W^* \stackrel{f^*}{\longrightarrow} V^*$,
$f^*$ die zu $f$ (K-linear) geh"orige adjungierte Abbildung (bzw. duale Abbildung).
\insertfig{algebra1_2_1_dualraum1}
Setze $\mathcal{G}(f)=f^*$. Aus der Linearen Algebra ist bekannt: $id_V^*=id_{V^*}$,
$(g\circ f)^*=f^*g^*$.
}
%-----------------------------------------------------------------------
\example{Verallgemeinerung der Beispiele}:{
Sei $\mathcal{K}$ eine Kategorie und $A\in \Ob{K}$. Dann ist:

\begin{tabular}{ccc}
$\mathcal{F}$: & $X\mapsto \Mor(K;A;X)$ & kovarianter Funktor \\
$\mathcal{G}$: & $X\mapsto \Mor(K;X;A)$ & kontravarianter Funktor \\
\end{tabular}
von $\mathcal{K}\longrightarrow$ \underline{Set}.\\
"Ubertragung der Pfeile: zu $f\in \Mor(K;X;Y)$, $\alpha \in \Mor(K;Y;A)$ sei $f^*(\alpha)=\alpha f$
\insertfig{algebra1_2_1_verall_beispiel1}
$f^*=\mathcal{G}(f)$ ist Abbildung (also Pfeil in \underline{Set})

$\Mor(K;Y;A) \longrightarrow \Mor(K;X;A)$, $\mathcal{G}(Y) \longrightarrow \mathcal{G}(X)$.

Nachpr"ufen der Homomorphieregel: $f:X \longrightarrow Y$, $g:Y \longrightarrow Z$,
$(gf)^*(\alpha)=\alpha(gf)=(\alpha g)f=g^*(\alpha)f=f^*(g^*(\alpha))$, also
$(gf)^*=f^*g^*$
\insertfig{algebra1_2_1_verall_beispiel2}
$\mathcal{F}(f)=f_*$:$\Mor(K;A;X) \longrightarrow \Mor(K;A;Y)$

trivial: kovarianter Funktor
}
%-----------------------------------------------------------------------
\subsection{Der Vergiss-Funktor}
%-----------------------------------------------------------------------
Idee: Vergiss-Funktor ''vergisst'' Struktur.

\example{}:{$\mathcal{V}$:$\underline{Gp} \longrightarrow$ \underline{Set},
$(G,v)\mapsto \mathcal{V}((g,v)):= G \in$ \underline{Set},
$\phi$:$G \longrightarrow H$, so $\mathcal{V}(\phi):=\phi$
}
\example{}:{\underline{Ring} $ \longrightarrow$\underline{Ab},
$(R,+,\cdot) \longrightarrow (R,+)$
}
\remark{}:{Eine Abbildung $A \longrightarrow G$, wobei $A$ eine Menge und $G$ eine Gruppe
sind, ist genau genommen ein Pfeil in \underline{Set} $A \longrightarrow \mathcal{V}(G)$
 }
%-----------------------------------------------------------------------
\subsection{Beliebte Anwendungen von Funktoren}
\lemma{Ein Funktor respektiert Isomorphismus}:{$\mathcal{F}:\mathcal{A}$
$\longrightarrow \mathcal{B}$ sei ein Funktor}=>{\label{lem_KAT_2}
$f$ ist ein Isomorphismus in $\mathcal{A}$
$\Rightarrow \mathcal{F}(f)$ ist ein Isomorphismus in $\mathcal{B}$.

Liefert {\emph Nichtisomorphiekriterium}: \\
$\mathcal{F}(A)\not\cong \mathcal{F}(B)$
in $\mathcal{B}\Rightarrow A\not\cong B$ in $\mathcal{A}$.}
%-----------------------------------------------------------------------
\proof \ref{lem_KAT_2}:{Sei $\mathcal{F}$ kovariant (kontravariant "ahnlich). 
$f$ Isomorphismus $f:A\longrightarrow B \Rightarrow \exists f^{-1}:B\longrightarrow A:$
$ff^{-1}=e_B$, $f^{-1}f=e_A \Rightarrow \mathcal{F}(f)\mathcal{F}(f^{-1})=e_{\mathcal{F}(B)}$,
$\mathcal{F}(f^{-1})\mathcal{F}(f)=e_{\mathcal{F}(A)}\Rightarrow \mathcal{F}(f^{-1})=\mathcal{F}(f)^{-1}$
\qed
}

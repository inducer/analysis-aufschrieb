% -----------------------------------------------------------------------------
\para{Erzeugende und Relationen}
% -----------------------------------------------------------------------------
\example Erzeugende der Diedergruppen:{
  $D_m\cong Z_m \rtimes Z_2=:G $ hat die Form $G=\product{g,u}$ mit den Relationen 
  \begin{enumerate}
    \item $g^m=u^2=1$
    \item $\phi_u(g):=ugu^{-1}=g^{-1}$
    \item 2 $\implies (ug)^2=1$
    \end{enumerate}
  Das Rechnen in $G$ ist durch die Erzeugenden und durch die Relationen 
  1 und 3 eindeuteig bestimmt. Diese Relationen sind ``definierend'' f"ur $G$.
  
  Ziel ist es jetzt, diese Begrifflichkeit zu pr"azisieren.
  }
% -----------------------------------------------------------------------------
\remark Erinnerung an freie Gruppen:{
  \index{Gruppe>freie}
  \index{Abbildungseigenschaft>universelle}
  \index{UAE}
  \index{Alphabet}
  Sei $A$ eine Menge (das ``Alphabet''). Sei dann $A^{-1}$ eine Menge
  von ``Inversen'' zu den Buchstaben in $A$.
  
  Dann gibt es die \emph{freie Gruppe} "uber $A$
  $Fr_G(A)$. Sie besteht aus den gek"urzten W"ortern $w=(a_1,\dots,a_n)$ mit
  $a_1,\dots,a_n\in A\uplus A^{-1}$ (wobei gek"urzt bedeutet, dass
  $a_i^{-1}=a_{i+1}$ nicht vorkommt). Einzige Operation in $Fr_G$ ist die
  Konkatenation $*$.
  
  Universelle Abbildungseigenschaft:
  \insertfig{algebra1_uae}
  Beachte $w=(a_1,\dots,a_n)=a_1* \cdots * a_n$, dann 
  $\tilde f(w)=f(a_1)\cdots f(a_n)$ (in $G$).
  
  Falls $A\subset G$ ($G$ Gruppe), so wird $f:A\to G$ zur Inklusion $\tilde j$.
  Es gilt $G=\product A\iff f$ surjektiv. $G\cong Fr_G(A)/\ker \tilde j$.
  
  Betrachte die linken Seiten der obigen Relationen mit $*$ statt $\cdot$ als
  Elemente $w\in Fr_G(A)$. Man sagt, die Relation $w\in Fr_G(A)$ gilt in $G$
  $:\iff \tilde j(w)=a_1\cdots a_n=1$.
  }
% -----------------------------------------------------------------------------
\definition Relation:{
  \index{Relation>definierende}
  Eine \emph{Relation} "uber $A$ in $G\supset A$ ist ein $w\in Fr_G(A)$ mit 
  $\tilde j(w)=1$ ($\iff w\in \ker \tilde j$).
  }
% -----------------------------------------------------------------------------
\convention{
  $\tilde j$ bezeichne stets eine bestimmte Injektion der freien Gruppe
  in eine Bildgruppe $G$.
  }
% -----------------------------------------------------------------------------
\remark Idee zum Hineinschneiden von Relationen:{  
  Gelte nun $\mathcal R\subset Fr_G(A)$, $G=\product A$. Gesucht: 
  Gr"o"stm"ogliche Faktorgruppe $\overline G=G/N$, in der die
  Relationen in $\cal R$ f"ur die Bilder $\kappa(\tilde j(w))$ ($w\in\mathcal R$)
  gelten. (das bedeutet $\tilde j(w)\in N$)
  
  Idee: gesucht ist ein kleinstm"oglicher Normalteiler $N$ mit 
  $\tilde j(\mathcal R)\subset N$.
  }
% -----------------------------------------------------------------------------
\definition Normalteilererzeugnis:{
  Sei $G$ Gruppe, $B\subset G$. Dann ist 
  \[\product B_{Nor}:=\bigcap_{B\subset N\unlhd G} N\unlhd  G
    \]
  $\product B_{Nor}$ hei"st das \emph{Normalteilererzeugnis} von $B$ in $G$.
  }
% -----------------------------------------------------------------------------
\remark Eigenschaft des Normalteilererzeugnisses:{
  \label{rem:nt-product}
  Es gilt $\product B_{Nor}=\product{\bigcup_{g\in G} gBg^{-1}}$. 
  
  ``$\supset$'' ist leicht.
  
  ``$\subset$'': Es gilt zun"achst f"ur Teilmengen $C\subset G$ und alle $g\in G$:
  \[\product{inn_g(C)}=inn_g(\product C)
    \]
  also insbesondere
  \begin{align*}
    inn_x(\product{\bigcup_{g\in G} gBg^{-1}})
    &=\product{\bigcup_{g\in G} inn_x( gBg^{-1})}
    =\product{\bigcup_{g\in G} inn_x( gBg^{-1})}\\
    &=\product{\bigcup_{g\in G}  xgB(xg)^{-1}}
    \subset \product{\bigcup_{g\in G}  gBg^{-1}}
    \end{align*}
  d.h. $\product{\bigcup_{g\in G} gBg^{-1}}$ ist Normalteiler, der $B$ enth"alt.
  }
% -----------------------------------------------------------------------------
\definition Hineinschneiden von Relationen:{
  Sei $G=\product A$, $\mathcal R\subset Fr_G(A)$. Dann hei"st
  \[\overline G=G/\product{\tilde j(\mathcal R)}_{Nor}
    \]
  die Gruppe, die aus $G$ durch Hineinschneiden der Relation $\mathcal R$ 
  entsteht.
  }
% -----------------------------------------------------------------------------
\example Hineinschneiden von Zyklizit"at:{
  Sei $G=Z_m=\product g$. Wir schneiden die Relation $g^n(\overset!=1)$ hinein.
  Es entsteht die Gruppe $Z_{\ggT(n,m)}$.
  }
% -----------------------------------------------------------------------------
\definition Kommutator:{
  Sei $G$ Gruppe. Zu $g,h\in G$ hei"st $[g,h]:=ghg^{-1}h^{-1}$ der
  \emph{Kommutator} von $g$ und $h$.
  \[G':=\product{[g,h]\mid g,h \in G}_{Nor}=\product{[g,h]\mid g,h \in G}
    \]
  hei"st \emph{Kommutatorgruppe}.
  $G/G'$ ist die gr"o"ste abelsche Faktorgruppe von $G$, sie entsteht
  durch Hineinschneiden der Relation $[g,h]=1\iff gh=hg$.
  $G/G'$ hei"st ``$G$ abelsch gemacht'' oder ``\emph{Kommutatorfaktorgruppe}''.
  }
% -----------------------------------------------------------------------------
\definition Charakteristische Untergruppe:{
  \index{Untergruppe>charakteristische}
  Sei $G$ eine Gruppe. Eine Untergruppe $U\leq G$ hei"st \emph{charakteristisch}
  $:\iff$ f"ur jedes $\phi\in Aut(G)$ gilt $\phi(U)=U$. 
  
  Jede charakteristische Untergruppe ist ein Normalteiler.\qed
  }
% -----------------------------------------------------------------------------
\remark:{
  Die Kommutatorgruppe ist charakteristisch.
  }
% -----------------------------------------------------------------------------
\theorem UAE des Hineinschneidens:
  $A$ Menge, $G=\product A$ Gruppe, $\mathcal R\subset F_{Gr}(A)$,
  $\overline G$ entstehe aus $G$ durch Hineinschneiden von $\mathcal R$.
  $H$ Gruppe, $\phi:G\to H$ Homomorphismus
  =>{
  \label{the:uae-relationen}
  Dann gilt: Genau dann faktorisiert $\phi$ durch $\overline G$ 
  (d.h. genau dann existiert ein Homomorphismus $\overline \phi$, der in
  untenstehendes kommutative Diagramm hineinpasst), wenn 
  $\mathcal R$ f"ur $\phi(A)$ gilt, d.h. f"ur jedes $w=b_1*\cdots *b_m\in\mathcal R$
  ($b_i\in A\cup A^{-1}$) gilt $\phi(b_1)\cdots \phi(b_m)=1$.
  \insertfig{algebra1_uae_hineinschneiden}
  }
% -----------------------------------------------------------------------------
\proof \ref{the:uae-relationen}:{
  $\overline \phi$ ist wohldefiniert $\iff $ 
  $\phi(\product{\tilde j(\mathcal R)}_{Nor})=\{1\}$.
  
  $\overline \phi$ ist dann eindeutig bestimmt.\qed
  }
% -----------------------------------------------------------------------------
\theorem UAE des Hineinschneidens von Relationen in die freie Gruppe:
  Sei $G$ Gruppe, $A\subset G$, $\mathcal R\subset F_{Gr}(A)=:F$=>{
  \label{the:uae-relationen-fgr}
  Genau dann gibt es einen Homomorphismus $\overline j$ mit untenstehendem
  kommutativen Diagramm, wenn alle Relationen $w\in\mathcal R$ in $G$ gelten.
  $\overline j$ ist dann eindeutig bestimmt.
  \insertfig{algebra1_1_5_uae_fgr}
  }
% -----------------------------------------------------------------------------
\proof \ref{the:uae-relationen-fgr}:{
  \insertfig{algebra1_1_5_uae_fgr_proof}
  Der Beweis geht in zwei Schritten vonstatten: Zun"achst konstruieren
  wir mittels der UAE der freien Gruppe den Homomorphismus $\tilde j:F\to G$,
  ausgehend vom gegebenen $j:A\to G$.
  
  Dann wenden wir die UAE des Hineinschneidens an und schneiden 
  $\product{\mathcal R}_{Nor}$ in $F$ hinein und erhalten so
  $\overline F:=F/\product{\mathcal R}_{Nor}$ sowie die dazugeh"origen
  Abbildungen $\overline j$ und $\overline i:=i\circ k$.\qed
  }
% -----------------------------------------------------------------------------
\definition Definierende Relation:{
  \index{Relation>definierend}
  \label{def:def-relation}
  Sei $A$ Menge, $G=\product{A}$ Gruppe, $F:=F_{Gr}(A)$, $\mathcal R\subset F$.
  $\mathcal R$ bestehe aus in $G$ geltenden Relationen. $\mathcal R$ hei"st
  \emph{definierend} $:\iff$ $\overline j:F/\product{\mathcal R}_{Nor} \to G$
  ist ein Isomorphismus.
  
  Man schreibt in diesem Fall $G=\product{A|\mathcal R}$.
  }
% -----------------------------------------------------------------------------
\annotation \ref{def:def-relation}:{
  Ich denke, man kann sich die Definition so vorstellen: 
  $F/\product{\mathcal R}_{Nor}$ ergibt die gr"o"ste Gruppe, die mit den
  gegebenen Grundelementen $A$ und den Rechenregeln in $\mathcal R$ 
  machbar ist, $G$ kann durchaus etwas kleiner sein. Erst wenn
  durch die Rechenregeln $F/\product{\mathcal R}_{Nor}$ auf die Gr"o"se
  von $G$ gezwungen wird (d.h. ein Isomorphismus existiert), 
  legen die Regeln genau $G$ fest.
  
  (Dass $G$ tats"achlich gr"o"ser ist als $F/\product{\mathcal R}_{Nor}$
  kommt nicht vor, denn $F/\product{\mathcal R}_{Nor}$ ist stets die
  \emph{gr"o"ste} solche Gruppe!)
  }
% -----------------------------------------------------------------------------
\definition Endliche Pr"asentation:{
  \index{Pr"asentation>endliche}
  Eine Gruppe $G=\product{A|\mathcal R}$ hei"st \emph{endlich pr"asentiert} $:\iff$
  $A$ und $\mathcal R$ sind endlich.
  }
% -----------------------------------------------------------------------------
\theorem Kriterium f"ur Endliche Pr"asentation:
  $G=\product A$ endlich, 
  $\mathcal R\subset F_{Gr}(A)$ eine Menge von in $G$ geltenden Relationen=>{
  \label{the:krit-endliche-praesentation}
  Dann gilt: $\mathcal R$ ist genau dann definierend, wenn 
  $|F/ \product{\mathcal R}_{Nor}|\leq |G|$ gilt.
  }
% -----------------------------------------------------------------------------
\proof \ref{the:krit-endliche-praesentation}:{
  In der Situation von \ref{the:uae-relationen-fgr} ist $\overline j$
  surjektiv, denn es ist ja $G=\product A$, und jedes $a\in\product A$ erreicht
  man ja mit Elementen der freien Gruppe. F"ur Injektivit"at gen"ugt dann
  $|\overline F\leq | G|$.
  }
% -----------------------------------------------------------------------------
\example:{
  \begin{enumerate}
    \item Betrachte
      \begin{align*}
        D_n&=Z_m\rtimes Z_2
	=\product{\underbrace{g,h}_A\mid\underbrace{g^n,h*h,h*g*h*g}_{\mathcal R}}\\
	&=\product{g,h\mid g^n=h^2=(hg)^2=1}
	\end{align*}
      Beh. $\mathcal R$ ist definierend.
      
      Es gen"ugt nach \ref{the:krit-endliche-praesentation} 
      $|\overline F:=F/ \product{\mathcal R}_{Nor}|\leq |D_m|=2m$.
      
      In $\overline F$ gilt $g^m=h^2=(hg)^2=1$.
      Aus $\mathcal R$ ergeben sich Folgerelationen: 
      $h^2=1\Rightarrow h=h^{-1}$, $1=hghg\Rightarrow g^{-1}=hgh^{-1}$.
      Daher $\overline F=\{g^ih^j\mid i\in\{0,\ldots,m\}, j=0,1\}$
      (Betrachte eine beliebiges Wort in $G$ und k"urze mit Hilfe obiger
      Regeln!)
      Also auch $|\overline F|\leq 2m$.
    \item Es gilt $S_n=\product{A:=\{\tau_1,\ldots,\tau_{n-1}\}}$ mit $\tau_i=(i,i+1)$ und
      $\mathcal R=\{\tau_i^2=1 (i=1,\ldots,n),(\tau_i\tau_{i+1})^3=1 (i=1,\ldots,n-1), (\tau_i\tau_j)^2=1 (1\leq j<i-1\leq n-1)\}$.
      
      "Ubung: $\mathcal R$ ist definierend.
    \end{enumerate}
  }
% -----------------------------------------------------------------------------
\remark Anwendung:{
  $G=\product{A,\mathcal R}$ sei endlich pr"asentiert, $H$ endliche Gruppe.
  $Hom(G,H)$ kann wie folgt bestimmt werden:
  
  Sei $A=(a_1,\ldots,a_n)\subset G$. Bestimme alle $A'=(a_1',\ldots,a_n')\subset H$, so dass
  $\mathcal R$ f"ur die $a_i'$ in $H$ gilt.
  
  $\phi\in Hom(G,H)$ sind dann genau die Fortsetzungen der Abbildungen $a_i\mapsto a_i'$.
  (laut UAE)
  
  Falls $G=H$ folgt: $\phi\in Aut(G)\iff G=\product{a_1',\ldots,a_n'}$ und 
  $\mathcal R$ gilt f"ur die $a_i'$.
  }
% -----------------------------------------------------------------------------
\remark Erfahrung der Gruppentheoretiker:{
  Meist ist es sehr schwer, aus $A,\mathcal R$ etwas "uber 
  $G=\product{A,\mathcal R}$ abzuleiten.
  
  Wortproblem: Gegeben sei eine endliche Pr"asentierung $G=\product{A,\mathcal R}$,
  au"serdem $w=a_1*\cdots*a_n\in F_{Gr}(A)$.
  Gesucht: Algorithmus, der entscheidet, ob $w$ in $G$ gilt oder nicht, d.h.
  $\tilde j(w)=1$. In Spezialf"allen, wie z.B. f"ur $D_m$ oder die freie
  Gruppe $F_{Gr}(A)$ (K"urzungsalgorithmus!) ist das Wortproblem zu l"osen,
  wie gesehen. Aber allgemein?
  
  Satz von Novikow (ca. 1956): Das Wortproblem ist algorithmisch nicht l"osbar.
  Genauer: Es gibt endlich pr"asentierte Gruppen $G$, f"ur die kein
  Algorithmus zu L"osung des Wortproblems existiert.
  
  Daher kann erst recht kein Algorithmus existieren, der das ``Isomorphieproblem''
  l"ost, d.h. zu zwei endlichen Pr"asentierungen entscheidet, ob die
  resultierenden Gruppen isomorph sind.
  
  Ist $G=\product{A,\mathcal R}$ endlich und endlich pr"asentiert, so gibt es 
  einen Algorithmus, der $|G|$ berechnet (``Coxeter-Todd-Algorithmus'').
  Allerdings ist es sehr schwierig, zu entscheiden ob $\product{A,\mathcal R}$
  endlich ist\ldots
  
  Kostprobe aus der Theorie der freien Gruppen:
  
  Satz von Schreier (1927):
  Jede Untergruppe $U$ einer freien Gruppe $F$ ist frei. Hat
  $F$ $n$ freie Erzeugende ($n\in N$) und ist $(F:U)<\infty$, so hat $U$
  genau $(F:U)(n-1)+1$ freie Erzeugende.
  F"ur $n=2$ gibt es nicht endlich erzeugbare Untergruppen. 
  (Gut lesbarer Beweis dazu z.B. in M. Hall jr.: The Theory of groups, pp. 95-106)
  }

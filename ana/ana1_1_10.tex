% -----------------------------------------------------------------------------
\para{Reelle Zahlen}
% -----------------------------------------------------------------------------
\definition K"orper:{
  Ein K"orper $({\mathbb K};+;\cdot)$ ist eine Menge $\mathbb K$ mit zwei
  Verkn"upfungen $+$ und $\cdot$, f"ur die
  die folgenden Axiome gelten:
  {
    \def\inK{ \in {\mathbb K} }
    \def\inKNull{ \in {\mathbb K \setminus \{0\}}}
    \begin{stmts}
      \item $(\forall a,b,c \inK)((a+b)+c=a+(b+c))$ 
        (\emph{Assoziativgesetz $+$})
      \item $(\exists 0 \inK)(\forall a \inK) (a+0=a)$ 
        (\emph{Neutralelement $+$})
      \item $(\forall a \inK)(\exists (-a) \inK) (a+(-a)=0)$ 
        (\emph{Inverses $+$})
      \item $(\forall a,b,c, \inK)(a+b=b+a)$ 
        (\emph{Kommutativgesetz $+$})
      \item $(\forall a,b,c \inK)((a\cdot b)\cdot c = a \cdot (b \cdot c))$ 
        (\emph{Assoziativgesetz $\cdot$})
      \item $(\exists 1 \inKNull)(\forall a \inKNull)(1\cdot a=a)$ 
        (\emph{Neutralelement $\cdot$})
      \item $(\forall a \inKNull)(\exists a^{-1} \inKNull)(a \cdot a^{-1}=1)$ 
        (\emph{Inverses $\cdot$})
      \item $(\forall a,b \inKNull)(a\cdot b=b\cdot a)$ 
        (\emph{Kommutativgesetz $\cdot$})
      \item $(\forall a,b,c \inK)(a\cdot(b+c)=a\cdot b + a \cdot c)$ 
        (\emph{Distributivgesetz})
      \end{stmts}
    }
  }
% -----------------------------------------------------------------------------
\remark:{
  Insbesondere ist $(\SetR,+,\cdot)$ ein K"orper. Alle bekannten
  Rechenregeln lassen sich aus den obigen Axiomen ableiten.
  }
% -----------------------------------------------------------------------------
\definition Kurzschreibweisen:{
  Der K"urze halber definiert man folgende Schreibweisen:
  \begin{align*}
    ab &:= a \cdot b\\
    a-b &:= a+(-b)\\
    \frac a b &:= a\cdot b^{-1}\qquad (b\ne 0)
    \end{align*}
  }
% -----------------------------------------------------------------------------
\definition Anordnung:{
  Eine Menge $M$ hei"st angeordnet, wenn eine Relation ``$\le$'' gegeben
  ist, die die folgenden Axiome erf"ullt:
  \begin{stmts}
    \item $(\forall a,b \in M)(a \le b \lor b \le a)$
    \item $(\forall a,b \in M)((a \le b \land b \le a) \implies a=b)$ 
    \item $(\forall a,b,c \in M)((a \le b \land b \le c) \implies a \le c)$ 
      (\indexthis{Transititivit"at})
    \item $(\forall a,b,c \in M)(a \le b \implies a+c \le b+c)$ 
    \item $(\forall a,b,c \in M)((a \le b \land 0 \le c) \implies ac \le bc)$ 
    \end{stmts}
  }
% -----------------------------------------------------------------------------
\definition Kurzschreibweisen:{
  F"ur die Anordnung definiert man weiterhin:
  \begin{align*}
    b \ge a &:\equiv a \le b \\
    a < b   &:\equiv \neg(a \ge b) \\
    b > a   &:\equiv a < b
    \end{align*}
  }
% -----------------------------------------------------------------------------
\remark:{
  Es ist einfach zu zeigen, da"s s"amtliche obigen Axiome auch mit den
  obigen Kurzschreibweisen gelten.
  }
% -----------------------------------------------------------------------------
\definition Betrag:{
  Sei $a \real$. Dann ist der Betrag von a:
  \[ |a|:=
    \begin{cases}
      a  & \text{falls $a \ge 0$} \\ 
      -a & \text{falls $a<0$}
    \end{cases}
  \]
}
% -----------------------------------------------------------------------------
\lessertheorem:$a,b \real$=>{
  Es gelten:
  \begin{align*}
    |a-b| &= |b-a| \\
    |a|=0 &\equiv a=0 \\
    |a|   &= |-a| \\
    |a||b| &= |ab| \\
    \pm a &\le |a| \\
    |a+b| &\le |a| + |b| \tag{Dreiecksungleichung} \\
    |a-b| &\ge \left| |a|-|b| \right|
    \end{align*}
  }
% -----------------------------------------------------------------------------
\definition Beschr"anktheit:{
  Sei $M \subseteq \SetR$, $M \ne \emptyset$. Dann hei"st
  M\ nach \tstack(oben;unten) beschr"ankt $:\equiv$
  \[(\exists \gamma\real)(\forall x \real)(x\stack(\le;\ge)\gamma)
    \]
  Dann hei"st $\gamma$ \tstack(obere;untere) Schranke.
  
  M\ hei"st beschr"ankt $:\equiv$ M\ ist nach oben und unten
  beschr"ankt.
  }
% -----------------------------------------------------------------------------
\definition Infimum/ Supremum und Minimum/ Maximum:{
  $\gamma$ hei"st \tstack(Supremum;Infimum) $:\equiv$
  \[(\forall \tilde\gamma \real)(\tilde\gamma
    \text{\tstack(OS;US)} \implies \tilde\gamma \stack(\ge;\le) \gamma)
    \]
  Kurzschreibweise: $\gamma=\stack(\sup;\inf)M$.
  
  Gilt $\stack(\sup;\inf)M \in M$, so nennt man $\stack(\sup;\inf)M$ auch
  gleichzeitig \tstack(Minimum;Maximum) von M.
  
  Kurzschreibweise: $\gamma=\stack(\min;\max)M$.
  }
% -----------------------------------------------------------------------------
\definition Intervall:{
  Sei $a,b \real,a<b$.\\
  $(a,b):=\left\{x\real\mid a<x<b\right\}$ hei"st offenes,\\
  $(a,b]:=\left\{x\real\mid a<x\le b\right\}$ und\\
  $[a,b):=\left\{x\real\mid a\le x< b\right\}$ halboffenes und\\
  $[a,b]:=\left\{x\real\mid a\le x\le b\right\}$ geschlossenes Intervall von
  $a$ bis $b$.
  }
% -----------------------------------------------------------------------------
\definition Vollst"andigkeitsaxiom:{
  Ein Axiom fehlt noch zur Bestimmung der reellen Zahlen,
  das Vollst"andigkeitsaxiom:
  \[(\forall M \subseteq \SetR,M \ne \emptyset)
      (M \text{ nach oben beschr"ankt} \implies \exists \sup M)
    \]
  }
% -----------------------------------------------------------------------------
\theorem:$M \subseteq \SetR,M \ne \emptyset,M
  \text{ nach unten beschr"ankt}$=>{
  Dann existiert $\inf M$.
  }
% -----------------------------------------------------------------------------
\lessertheorem:$M \subseteq \SetR,M \ne \emptyset$=>{
  $M$ ist beschr"ankt $:\equiv (\exists c>0)(\forall x \in M)(|x|<c)$.
  }
% -----------------------------------------------------------------------------
\theorem:$\emptyset\ne B\subseteq A\subseteq \SetR)$=>{
  Dann gilt das folgende:
  \begin{stmts}
    \item $A$ ist beschr"ankt $\equiv\inf A\le\sup A$
    \item $A$ ist nach \tstack(oben;unten) beschr"ankt $\implies$
      $B$ ist nach \tstack(oben;unten) beschr"ankt und
      $\stack(\sup B\le\sup A;\inf B \ge\inf A)$.
    \item $A$ ist nach \tstack(oben;unten) beschr"ankt und
      $\gamma$ eine \tstack(obere;untere) Schranke von $A$ $\implies$
      $\gamma=\stack(\sup;\inf)A \equiv (\forall \varepsilon>0)
      (\exists x\in A)(x\stack(>;<)\gamma-\epsilon)$
    \end{stmts}
  }
% -----------------------------------------------------------------------------
\para{Nat"urliche Zahlen}
% -----------------------------------------------------------------------------
\definition Induktionsmenge:{
  \index{Menge>induktive}
  $A\subseteq\SetR$ hei"st Induktionsmenge/IM/induktiv, wenn die
  folgenden Axiome gelten:
  \begin{stmts}
    \item $1 \in A$ 
    \item $x\in A \implies x+1\in A$ 
    \end{stmts}
  }
% -----------------------------------------------------------------------------
\definition Nat"urliche Zahlen:{
  Die nat"urlichen Zahlen sind wie folgt festgelegt:\par
  $\SetN:=\bigcap\limits_{A\text{IM}}A$.
}
% -----------------------------------------------------------------------------
\theorem:=>{
  Dann gelten
  \begin{stmts}
    \item $\SetN$ ist eine Induktionsmenge
    \item $\SetN$ ist nicht nach oben beschr"ankt.
    \item $x\real\implies(\exists n\natural)(n>x)$
    \item $A \subseteq\SetN \land A\text{ Induktionsmenge}\implies A=\SetN $
      (Prinzip der vollst. Induktion)
    \end{stmts}
  }
% -----------------------------------------------------------------------------
\remark Beweisverfahren der vollst"andigen Induktion:{
  \index{Induktion>vollst"andige}
  \index{vollst"andige Induktion}
  F"ur jedes $a\natural$ sei eine Aussage $A(n)$ definiert.
  Sei $A:=\left\{n\natural\mid A(n)\right\}$. Kann man zeigen,
  da"s $A(1)$ richtig ist und $A(n)\implies A(n+1)$, so ist $A$
  Induktionsmenge. Weil $A\subseteq\SetN$ und A Induktionsmenge ist,
  gilt $A=\SetN$. $A(n)$ gilt also f"ur jedes $n\natural$.
  }
% -----------------------------------------------------------------------------
\lessertheorem Wohlordnungsprinzip f"ur die nat. Zahlen:
  $\emptyset\ne M\subseteq\SetN$=>{
  Dann existiert $\min M$.
  }
% -----------------------------------------------------------------------------
\definition Ganze Zahlen,Br"uche:{
  Die folgenden Mengen bauen auf den nat"urlichen Zahlen auf:
  \begin{align*}
    \SetNN&:=\SetN \cup \{0\}\\
    \SetZ&:=\{-n\mid n\natural\}\cup\SetNN\\
    \SetQ&:=\{\frac p q\mid p\integer,q\natural\}
    \end{align*}
  }
% -----------------------------------------------------------------------------
\theorem:$x,y\real,x<y$=>{
  Dann existiert ein $r\rational$ mit $x<r<y$.
  }
% -----------------------------------------------------------------------------
\para{Folgen/Abz"ahlbarkeit}
% -----------------------------------------------------------------------------
\definition In-/Sur-/Bijektivit"at:{
  Seien $A,B$ beliebige Mengen, wobei $A\ne\emptyset\ne B$
  und $f:A\rightarrow B$ eine Funktion. Dann ist
  \[f(A):=\{f(x)\mid x\in A\} 
    \] 
  die Bildmenge von $f$. $f$ hei"st dann
  \begin{stmts}
    \item injektiv $:\equiv f(x_1)=f(x_2)\implies x_1=x_2$
    \item surjektiv $:\equiv f(A)=B$
    \item bijektiv $:\equiv$ $f$ injektiv und surjektiv
    \end{stmts}
  }
% -----------------------------------------------------------------------------
\definition Folge:{
  Sei $A\ne\emptyset$ eine beliebige Menge. Eine Funktion
  $a:\SetN\rightarrow A$ hei"st Folge in $A$.
  Schreibweise: 
  \begin{align*}
    a_n&:=a(n) \tag{``n-tes Folgenglied''}\\
    (a_n), {(a_n)}_{n=1}^{\infty}, (a_1,a_2,\ldots)&:=a\\
    \end{align*}
  }
% -----------------------------------------------------------------------------
\definition endlich,unendlich,abz"ahlbar:{
  Sei $X\ne\emptyset$ beliebige Menge.
  \begin{stmts}
    \item $X$ hei"st endlich $:\equiv (\exists n\natural)
      (\exists f:\{1,\ldots,n\}\to X)(f \text{ surjektiv})$
    \item $X$ hei"st unendlich $:\equiv$ $X$ ist nicht endlich.
    \item $X$ hei"st abz"ahlbar $:\equiv
      (\exists f:\SetN\to X)(f \text{ surjektiv})$
    \item $X$ hei"st abz"ahlbar unendlich $:\equiv$
      $X$ ist abz"ahlbar und unendlich.
    \item $X$ hei"st "uberabz"ahlbar $:\equiv$
      $X$ ist nicht abz"ahlbar, aber unendlich.
    \end{stmts}
  }
% -----------------------------------------------------------------------------
\remark:{
  $\SetN$, $\SetZ$ und $\SetQ$ sind abz"ahlbar unendlich. $\SetR$ ist
  "uberabz"ahlbar. Die Menge aller Folgen in $\{0,1\}$ ist
  "uberabz"ahlbar.
  }
% -----------------------------------------------------------------------------
\theorem:$\emptyset\ne B\supseteq A$ abz"ahlbar=>{
  Dann ist auch $B$ abz"ahlbar.
  }
% -----------------------------------------------------------------------------
\theorem:$X_1,X_2,X_3\ldots$ abz. viele Mengen=>{
  $\bigcup\limits_{n\natural} X_n$ abz"ahlbar
  }
% -----------------------------------------------------------------------------
\para{Einige Formeln}
% -----------------------------------------------------------------------------
\theorem Summenformel:$n\natural$=>{
  Dann gilt
  \[\sum_{k=1}^n k=\frac{n(n+1)}2
    \]
  }
% -----------------------------------------------------------------------------
\definition Nat"urliche Potenzen:{
  Sei $a\real,n\natural$.
  Dann ist $a^n:=\overbrace{a\cdot a\cdot a\cdots}^{n\text{-mal}}$ und $a^0:=1$.
}
% -----------------------------------------------------------------------------
\definition Fakult"at:{
  Sei $n\natural$. Dann ist $n!:=1\cdot 2\cdot 3\cdots n$ und $0!:=1$.
}
% -----------------------------------------------------------------------------
\definition Binomialkoeffizient:{
  Sei $n\natural;k\nnatural;k\le n$. Dann ist 
  \[\binom n k :=\frac{n!}{k!(n-k)!}
    \]
}
% -----------------------------------------------------------------------------
\theorem:$n\natural;k\nnatural;k\le n$=>{
  Dann gilt
  \[\binom n k +\binom n {k+1}=\binom{n+1} k
    \]
  }
% -----------------------------------------------------------------------------
\theorem Bernoulli'sche Ungleichung:$x\ge -1,n\natural$=>{
  \index{Ungleichung>Bernoulli'sche}
  Es gilt
  \[(1+x)^n\ge 1+nx
    \]
  }
% -----------------------------------------------------------------------------
\theorem Allgemeine 3. binomische Formel:$a,b\real;n\natural$=>{
  \index{Binomische Formel>allgemeine}
  In Verallgemeinerung von $a^2-b^2=(a-b)(a+b)$ gilt
  \[a^{n+1}-b^{n+1}=(a-b)\sum_{k=0}^{n}a^kb^{n-k}
    \]
  Insbesondere $a=1,t\ne 1$:
  \[\sum_{k=0}^{n}t^k=\frac{1-t^{n+1}}{1-t}
    \]
}
% -----------------------------------------------------------------------------
\theorem Binomialformel:$a,b\real;n\natural$=>{
  Es gilt
  \[(a+b)^n=\sum_{k=0}^n{n\choose k}a^kb^{n-k}
    \]
}
% -----------------------------------------------------------------------------
\lessertheorem Monotonie nat"urlicher Potenzen:$x,y\ge 0;n\natural$=>{
  Dann gilt 
  \[x\le y\implies x^n\le y^n
    \]
  }
% -----------------------------------------------------------------------------
\para{Wurzeln}
% -----------------------------------------------------------------------------
\definition Wurzel:{
  Sei $a\ge 0,n\natural$. Dann ex. ein $x\real$ mit $x^n=a$.
  Schreibweise: $x=\root n \of a$.
  }
% -----------------------------------------------------------------------------
\definition Rationale Potenzen:{
  \index{Potenzen>rationale}
  Sei $a\ge 0;r=\frac m n;m,n \natural$. Dann ist
  \[a^r=a^\frac m n:=\left(\sqrt[n]a\right)^m
    \]
  Diese Darstellung ist unabh"angig davon, ob $\frac m n $ gek"urzt ist
  oder nicht.
  }
% -----------------------------------------------------------------------------
\definition Negative Potenzen:{
  Sei $a>0;r\rational;r<0$. Dann ist
  \[a^r:=\frac 1{a^{-r}}
    \]
  }
% -----------------------------------------------------------------------------
\para{Konvergente Folgen}
% -----------------------------------------------------------------------------
\definition Beschr"anktheit einer Folge:{
  \index{Folge>Beschr"anktheit}
  $(a_n)$ hei"st beschr"ankt (nach oben/unten)$:\equiv$ die Bildmenge
  beschr"ankt ist (nach oben/unten).
  
  Analog "ubertragen sich die Definitionen von $\min,\max,\inf,\sup$.
  }
% -----------------------------------------------------------------------------
\definition Epsilon-Umgebung:{
  Sei $x_0\real,\epsilon>0$. Dann hei"st $U_\epsilon(x_0)$ die
  $\epsilon$-Umgebung von $x_0$.
  \[U_\epsilon(x_0):=(x_0-\epsilon;x_0+\epsilon)
    \]
}
% -----------------------------------------------------------------------------
\definition Konvergenz einer Folge:{
  \index{Folge>Konvergenz einer}
  Sei $(a_n)$ eine reelle Folge. $(a_n)$ hei"st konvergent $:\equiv$
  \[(\exists a\real)(\forall \epsilon>0)(\exists n_0(\epsilon)\natural)
    (\forall n\ge n_0\natural)(a_n\in U_\epsilon(a))
    \]
  Ist $(a_n)$ konvergent, so hei"st $a$ Grenzwert/GW von $(a_n)$.
  Schreibweise: $a_n \to a$ f"ur $n\to\infty$ oder $\limn a_n=a$.
  Eine nicht konvergente Folge hei"st divergent.
}
% -----------------------------------------------------------------------------
\theorem:$(a_n)$ konvergente reelle Folge=>{
  \begin{stmts}
    \item $\limn a_n$ ist eindeutig bestimmt.
    \item $(a_n)$ ist beschr"ankt.
    \end{stmts}
  }
% -----------------------------------------------------------------------------
\remark:{
  In Konvergenzfragen kommt es auf endlich viele Folgenglieder nicht an.
  }
% -----------------------------------------------------------------------------
\example:{
  \begin{stmts}
    \item $c\real\implies\limn c=c$ 
    \item $\limn \frac 1 n=0$ 
    \item $(n)$ ist divergent.
    \item $((-1)^n)$ ist divergent.
    \item $\limn \frac 1 {\sqrt{n}}=0$
    \end{stmts}
  }
% -----------------------------------------------------------------------------
\definition Folge (II):{
  Sei $k\integer$ fest. Eine Fkt.
  $a:\{n\integer\mid n\ge k\}\to X\ne\emptyset$ hei"st
  ebenfalls Folge.
  
  Schreibweise: $(a_n)_{n\ge k}$ oder $(a_n)_{n\ge k}^{\infty}$
}
% -----------------------------------------------------------------------------
\definition f"ur fast alle:{
  Sei $k,n\integer$, $A(n)$ eine Aussage f"ur alle $n\ge k$.
  Dann sagt man:\par

  $A(n)$ gilt f"ur fast alle $n\ge k$ $:\equiv$
  $(\exists n_0\ge k)(\forall n\ge n_0)(A(n) \text{gilt})$.
}
% -----------------------------------------------------------------------------
\theorem: $(a_n),(b_n),(c_n)$ reelle Folgen=>{
  Dann gilt
  \begin{stmts}
    \item $\limn a_n=a \implies \limn |a_n-a|=0$ 
    \item $\limn b_n=0 \land |a_n-a|\le b_n \ffa n\natural
           \implies \limn a_n=a$
    \item $\limn a_n=a\implies \limn |a_n|=|a|$ $(\not\equiv)$
    \end{stmts}
  Sei nun $\limn a_n=a,\limn b_n=b$. Dann gilt weiter:
  \begin{stmts}
    \item $\limn (a_n+b_n)=a+b$
    \item $\alpha\real$ $\implies$ $\limn \alpha a_n=\alpha a$ 
    \item $\limn a_n b_n=ab$ 
    \item $b\ne 0\implies(\exists n_0\natural)(\forall n\ge n_0)(b_n\ne 0,
      \limn \frac{a_n}{b_n}=\frac a b$) 
    \item $a_n\le b_n \ffa n\natural \implies a\le b$ 
    \item $a=b \land a_n\le c_n \le b_n \ffa n\natural\implies
      \limn c_n=a$
    \end{stmts}
  }
% -----------------------------------------------------------------------------
\example:{
  Sei $p\natural$. Dann ist $\limn \frac 1 {n^p}=0$.
  }
% -----------------------------------------------------------------------------
\definition Monotonie:{
  $a_n$ sei reelle Folge.
  \begin{stmts}
    \item $(a_n)$ hei"st monoton wachsend $:\equiv (\forall n\natural)(a_{n+1}\ge a_n)$
    \item $(a_n)$ hei"st monoton fallend $:\equiv (\forall n\natural)(a_{n+1}\le a_n)$
    \item $(a_n)$ hei"st monoton $:\equiv$ $(a_n)$ monoton wachsend $\lor$ fallend
    \item $(a_n)$ hei"st streng monoton wachsend $:\equiv (\forall n\natural)(a_{n+1}>a_n)$
    \item $(a_n)$ hei"st streng monoton fallend $:\equiv (\forall n\natural)(a_{n+1}<a_n)$
    \end{stmts}
  }
% -----------------------------------------------------------------------------
\theorem Monotoniekriterium:
  $(a_n)$ sei monoton \tstack(wachsend;fallend) und nach \tstack(oben;unten) 
  beschr"ankt=>{
  Dann gilt $\limn a_n=\stack(\sup a_n;\inf a_n)$.
  }
% -----------------------------------------------------------------------------
\para{Wichtige Beispiele}
% -----------------------------------------------------------------------------
\theorem:$p\natural,(a_n)$ reelle Folge,
  $(\forall n\natural)(a_n\ge0)$,$\limn a_n=a$=>{
  Es ist $\limn \root p\of{a_n}=\sqrt[p]a$.
  }
% -----------------------------------------------------------------------------
\theorem:$x\real,n\natural,a_n:=x^n$=>{
  \begin{stmts}
    \item $x=0$: $\limn a_n=0$ 
    \item $x=1$: $\limn a_n=1$ 
    \item $x=-1$: $(a_n)$ divergent 
    \item $|x|>1$: $(a_n)$ divergent 
    \item $|x|<1$: $\limn a_n=0$ 
    \end{stmts}
  }
% -----------------------------------------------------------------------------
\theorem:=>{
  Es ist $\limn \sqrt[n]n=1$.
  }
% -----------------------------------------------------------------------------
\theorem:$c>0$=>{
  Es ist $\limn \sqrt[n]c=1$.
  }
% -----------------------------------------------------------------------------
\theorem:$a_n=(1+\frac 1 n)^n,b_n=\sum_{k=0}^n \frac 1{k!}$=>{
  Dann sind $\limn a_n=\limn b_n=e$.
  }
% -----------------------------------------------------------------------------
\definition Eulersche Zahl e:{
  Man legt fest $e:=\limn(1+\frac 1 n)^n$.
  }
% -----------------------------------------------------------------------------
\para{Teilfolgen und H"aufungswerte}
% -----------------------------------------------------------------------------
\definition Teilfolge:{
  Sei $(a_n)$ Folge und $\phi: \SetN\to\SetN$ streng monoton wachsend.
  $b_n:=a_{\phi(n)}$ hei"st dann Teilfolge von $a_n$.
}
% -----------------------------------------------------------------------------
\definition H"aufungswert:{
  $\alpha\real$ ist H"aufungswert/HW von $(a_n)$ $:\equiv$
  $\epsn(a_n\in U_\epsilon(\alpha))$ gilt f"ur unendlich viele Folgenglieder.
}
% -----------------------------------------------------------------------------
\remark:{Jede reelle Zahl ist H"aufungswert der rationalen Zahlen.}
% -----------------------------------------------------------------------------
\theorem:$(a_n)$ reelle Folge=>{
  \begin{stmts}
    \item $\alpha$ ist HW von $(a_n)$ $\equiv$
      $\exists$ Teilfolge von $(a_n)$ mit $\limes k->\infty a_{n_k}=\alpha$\\
    \item $\limn a_n=a$ $\implies$
      $\forall$ Teilfolgen $(a_{n_k})$ von $(a_n)$:
      $\limes k->\infty a_{n_k}=a$\\
    \item $(a_n)$ konvergent $\implies$ $(a_n)$ hat genau einen HW, n"amlich
      den Grenzwert.
    \end{stmts}  
  }
% -----------------------------------------------------------------------------
\theorem:$(a_n)$ reelle Folge=>{
  Es gibt eine monotone Teilfolge von $(a_n)$.
}
% -----------------------------------------------------------------------------
\theorem Satz von Bolzano-Weierstra"s:
  $(a_n)$ beschr"ankte reelle Folge=>{
  \label{the:bolzano-weierstrass}
  \index{Bolzano-Weierstra"s>Satz von}
  $(a_n)$ hat mindestens einen H"aufungswert.
  }
% -----------------------------------------------------------------------------
\para{Oberer und unterer Limes}
% -----------------------------------------------------------------------------
\definition:{
  Sei $(a_n)$ eine reelle Folge.
  $H(a_n):=\left\{\alpha\real\mid\alpha\text{ ist HW von $(a_n)$} \right\}$
}
% -----------------------------------------------------------------------------
\theorem:
  $(a_n)$ beschr"ankte reelle Folge=>{
  \begin{stmts}
    \item $H(a_n)$ ist beschr"ankt.
    \item $\sup H(a_n),\inf H(a_n)\in H(a_n)\equiv\exists\min H(a_n),\max H(a_n)$.
      (Wegen \ref{the:bolzano-weierstrass}: $H(a_n)\ne\emptyset$)
    \end{stmts}
  }
% -----------------------------------------------------------------------------
\definition limsup / liminf:{
  Sei $(a_n)$ beschr"ankte reelle Folge. Dann hei"st
  \begin{stmts}
    \item $\limsupn a_n := \max H(a_n)$ (\indexthis{limes superior}/oberer Limes)
    \item $\liminfn a_n := \min H(a_n)$ (\indexthis{limes inferior}/unterer Limes)
    \end{stmts}
  }
% -----------------------------------------------------------------------------
\theorem:
  $(a_n)$ beschr"ankte reelle Folge, $\alpha\real$=>{
  \begin{align*}
    \alpha=\liminf a_n \equiv
    & \epsn(\alpha-\epsilon<a_n) \ffa n\natural\land \\
    & \epsn(a_n<\alpha+\epsilon) \text{ f"ur unendlich viele } n\natural \\
    \alpha=\limsup a_n \equiv
    & \epsn(\alpha-\epsilon<a_n) \text{ f"ur unendlich viele } n\natural\land \\
    & \epsn(a_n<\alpha+\epsilon) \ffa n\natural 
    \end{align*}
  }
% -----------------------------------------------------------------------------
\theorem:
  $(a_n)$ beschr"ankte reelle Folge, $\alpha\real$=>{
  Die folgenden drei Aussagen sind "aquivalent:
  \begin{stmts}
    \item $\liminfn a_n = \limsupn a_n$
    \item $(a_n)$ hat genau einen HW.
    \item $(a_n)$ ist konvergent.
    \end{stmts}
  }
% -----------------------------------------------------------------------------
\lessertheorem:$(a_n),(b_n)$ beschr"ankte reelle Folgen=>{
  \begin{stmts}
    \item $a_n\le b_n \ffa n\natural\implies
      \stack(\limsupn;\liminfn) a_n \le
      \stack(\limsupn;\liminfn) b_n$ 
    \item $\limsupn (a_n+b_n) \le
      \limsupn a_n+\limsupn b_n$ 
      
      $\liminfn (a_n+b_n) \le
      \liminfn a_n+\liminfn b_n$ 
    \item $\alpha\ge 0$ $\implies$ $\limsupn \alpha a_n = \alpha\limsupn a_n$
    \item $\limsupn (-a_n)=-\liminfn a_n$ 
    \item $\liminfn (-a_n)=-\limsupn a_n$ 
    \end{stmts}
  }
% -----------------------------------------------------------------------------
\para{Das Cauchy-Kriterium}
% -----------------------------------------------------------------------------
\definition Cauchy-Folge:{
  Eine reelle Folge $(a_n)$ hei"st Cauchy-Folge $:\equiv$
  \[\epsn(\exists n_0=n_0(\epsilon)\natural)(\forall n,m\ge n_0)(|a_n-a_m|<\epsilon)
    \]
  }
% -----------------------------------------------------------------------------
\theorem Cauchy-Kriterium:=>{
  $(a_n)$ ist konvergent $\equiv$ $(a_n)$ ist Cauchy-Folge.
  }
% -----------------------------------------------------------------------------
\remark:{
  $(a_n)$ ist Cauchy-Folge $\equiv$
  \[\epsn(\exists n_0(\epsilon)\natural)(\forall n\ge n_0)(\forall k\natural)
    (|a_n-a_{n+k}|<\epsilon)
    \]
  }

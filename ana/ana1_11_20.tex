% -----------------------------------------------------------------------------
\para{Unendliche Reihen}
% -----------------------------------------------------------------------------
\definition Reihe:{
  \index{Konvergenz>Reihen}
  Sei $(a_n)$ reelle Folge. F"ur $n\natural$ sei
  $s_n:=\sum_{i=1}^n a_i$. Die Folge $(s_n)$ hei"st unendliche Reihe
  (kurz:Reihe) und wird bezeichnet $\sumn 1 a_n$.
  
  $(s_n)$ hei"st ($n$-te) Teilsumme der Reihe.
  
  $\sumn 1 a_n$ hei"st \tstack(konvergent;divergent) $:\equiv$
  $(s_n)$ ist \tstack(konvergent;divergent).
  
  Ist $\sumn 1 a_n$ konvergent, so hei"st $\limn s_n$ der
  Reihenwert/die Reihensumme.
  
  Schreibweise: $\sumn 1 a_n = \limn s_n$
  }
% -----------------------------------------------------------------------------
\remark:{
  Ist $p\integer$ und $(a_n)_{n=p}^\infty$ eine reelle Folge, so def.
  man entsprechend: $s_n:=\sum_{i=p}^n a_i$ und $\sumn p$. Die folgenden
  S"atze und Definitionen "ubertragen sich entsprechend.
  }
% -----------------------------------------------------------------------------
\example Geometrische Reihe:{
  \index{Reihe>geometrische}
  F"ur $|x|<1$ ergibt sich
  \[\sumn 0 x^n=\frac 1 {1-x}
    \]
  }
% -----------------------------------------------------------------------------
\example Harmonische Reihe:{
  \index{Reihe>harmonische}
  Die harmonische Reihe
  \[\sumn 1 \frac 1 n
    \]
  divergiert.
  }
% -----------------------------------------------------------------------------
\remark:{
  $\SetQ$ kann von abz"ahlbar vielen Intervallen mit bel. kleiner L"angensumme
  "uberdeckt werden.
  }
% -----------------------------------------------------------------------------
\theorem:$(a_n)$ reelle Folge=>{
  \begin{stmts}
    \item $a_n\ge 0$ und $(s_n)$ beschr"ankt $\implies$ $\sumn 1 a_n$ konv.
      (\emph{\indexthis{Monotoniekriterium}})
    \item $\sumn 1 a_n$ konv. $\equiv$
      \[\epsn(\exists n_0=n_0(\epsilon))(\forall m,n)
        (m > n\implies\left|\sum_{k=n+1}^m a_k\right|<\epsilon)
	\]
      (\emph{\indexthis{Cauchykriterium}})
    \item $\sumn 1 a_n$ konv. $\implies$ $(\forall \nu\natural)(\sumn \nu a_n \text{ konv.})$. 
    
      Weiterhin gilt f"ur $\nu\to\infty$ auch $\sumn\nu a_n\to 0$.
    \item $\sumn 1 a_n$ konvergiert $\implies$ $\limn a_n=0$ 
    \end{stmts}
  }
% -----------------------------------------------------------------------------
\theorem Linearit"at von Reihen:
  $\sumn 1 a_n,\sumn 2 b_n$ konv. Reihen, $\alpha,\beta \real$=>{
  \[\sumn 1 (\alpha a_n+\beta b_n) = \alpha \sumn 1 a_n + \beta \sumn 1 b_n
    \]
  }
% -----------------------------------------------------------------------------
\remark:{
  $\sumn 1 (a_n+b_n)$ kann konvergieren, obwohl $\sumn 1 a_n$ und
  $\sumn 1 b_n$ divergieren.
  }
% -----------------------------------------------------------------------------
\definition Absolute Konvergenz:{
  \index{Konvergenz>absolute}
  Eine Reihe hei"st absolut konvergent $:\equiv$
  $\sumn 1 |a_n|$ konvergiert.
  }
% -----------------------------------------------------------------------------
\theorem Dreiecksungleichung f"ur Reihen:$\sumn 1 |a_n|$ absolut konvergent=>{
  \index{Reihen>Dreiecksungleichung f"ur}
  $\sumn 1 a_n$ konvergiert und es gilt
  \[
    \left|\sumn 1 a_n\right| \le \sumn 1 |a_n|
  \]
}
% -----------------------------------------------------------------------------
\para{Konvergenzkriterien f"ur Reihen}
% -----------------------------------------------------------------------------
\theorem Leibniz-Kriterium:
  $(a_n)$ monoton fallende Nullfolge, $b_n=(-1)^{n+1}a_n$=>{
  Dann konvergiert $\sumn 1 b_n$.
  }
% -----------------------------------------------------------------------------
\theorem Majorantenkriterium:
  $|a_n|\le b_n \ffa n\natural$ und $\sumn 1 b_n$ konvergent=>{
  Dann konv. $\sumn 1 a_n$ absolut.
  }
% -----------------------------------------------------------------------------
\lessertheorem Minorantenkriterium:
  $0\le b_n\le a_n \ffa n\natural$ und $\sumn 1 b_n$ divergent=>{
  Dann ist $\sumn 1 a_n$ divergent.
  }
% -----------------------------------------------------------------------------
\example:{
  $\sumn 1 \frac 1 {n^2}=\frac {\pi^2} 6$
  }
% -----------------------------------------------------------------------------
\remark:{
  $\sumn 1 \frac 1 {n^\alpha}$ konvergiert $\equiv$ $\alpha>1$.
  }
% -----------------------------------------------------------------------------
\theorem Wurzelkriterium:
  $(a_n)$ reelle Folge=>{
  \begin{stmts}
    \item Ist $\sqrt[n]{|a_n|}$ unbeschr"ankt, so div. $\sumn 1 a_n$.
    \item Ist $\sqrt[n]{|a_n|}$ beschr"ankt, so ex.
          $\alpha:=\limsupn \sqrt[n]{|a_n|}$.
    \item $\alpha<1 \implies \sumn 1 a_n$ konv. absolut
    \item $\alpha=1:$ keine Aussage m"oglich
    \item $\alpha>1 \implies \sumn 1 a_n$ divergiert
    \end{stmts}
  }
% -----------------------------------------------------------------------------
\remark:{
  $\sqrt[n]{|a_n|}<1$ f"ur alle $n\natural$ gen"ugt im allgemeinen nicht
  f"ur Konvergenz.
  }
% -----------------------------------------------------------------------------
\theorem Quotientenkriterium:
  $(a_n)$ reelle Folge mit $a_n\ne 0 \ffa n\natural$=>{
  \begin{stmts}
    \item $\left|\frac {a_{n+1}}{a_n}\right|$ beschr"ankt,
          $\limsupn \left|\frac {a_{n+1}}{a_n}\right|<1$ $\implies$
          $\sumn 1 a_n$ absolut konv.
    \item $\left|\frac {a_{n+1}}{a_n}\right|\ge 1\ffa n\natural \implies \sumn 1 a_n$
          divergent
    \item $\left|\frac{a_{n+1}}{a_n}\right|$ beschr"ankt,
          $\liminfn \left|\frac{a_{n+1}}{a_n}\right|>1$ $\implies$
          $\sumn 1 a_n$ divergent
    \end{stmts}
  }
% -----------------------------------------------------------------------------
\remark:{
  Ist $(a_n)$ Folge in $\SetR\setminus\{0\}$ und $\left|\frac{a_n+1}{a_n}\right|$
  beschr"ankt, so gilt
  \[\liminfn \left|\frac {a_n+1}{a_n}\right| \, \le \,
    \liminfn \root n \of{|a_n|} \, \le \,
    \limsupn \root n \of{|a_n|} \, \le \,
    \limsupn \left|\frac {a_n+1}{a_n}\right|
    \]
  Liefert das Quotientenkriterium keine Entscheidung, so braucht man
  es mit dem Wurzelkriterium gar nicht erst zu versuchen, d.h. das
  Quotientenkriterium ist das ``empfindlichere Werkzeug''.
}
% -----------------------------------------------------------------------------
\example Exponentialfunktion:{
  $E(x):=\sumn 0 \frac {x^n}{n!}$ konvergiert f"ur alle $x\real$.
  (Quotientenkriterium)
  
  Es gilt $E(0)=1, E(1)=e$.
  }
% -----------------------------------------------------------------------------
\remark Klammern in Reihen:{
  In konverg. Reihen darf man im allgemeinen ``Klammern'' nicht weglasssen,
  ohne die Konvergenz zu beeinflussen. Hinzuf"ugen von Klammern ist jedoch
  kein Problem, wie der folgende Satz zeigt:
  }
% -----------------------------------------------------------------------------
\theorem:
  $\sumn 1 a_n$ konvergent, $(n_k)$ streng wachsende Folge in $\SetN$=>{
  Setze $b_1:=a_1+\cdots+a_{n_1}$, $b_2=a_{n_1+1}+\cdots+a_{n_2}$ $\ldots$.
  
  Dann konvergiert $\sumn 1 b_n$ und es ist $\sumn 1 b_n = \sumn 1 a_n$.
  }
% -----------------------------------------------------------------------------
\para{Umordnung von Reihen}
% -----------------------------------------------------------------------------
\definition Umordnung:{
  Sei $(a_n)$ reelle Folge und $\phi:\SetN\to\SetN$ bijektiv.
  Setze $b_n:=a_{\phi(n)}$.
  
  Dann hei"st $(b_n)$ Umordnung von $(a_n)$ und 
  $\sumn 1 b_n$ Umordnung von $\sumn 1 a_n$.
  }
% -----------------------------------------------------------------------------
\remark:{
  Die Relation ``ist Umordnung von'' ist symmetrisch.
  }
% -----------------------------------------------------------------------------
\lessertheorem:$\phi:\SetN\to\SetN$ bijektiv, $n_0\natural$=>{
  Dann existiert ein $n_0\natural$ so, dass 
  $(\forall n\ge n_0)(\phi(n)\ge n_0)$.
  }
% -----------------------------------------------------------------------------
\theorem Umordnungssatz:
  $(b_n)$ Umordnung von $(a_n)$=>{
  \begin{stmts}
    \item $(a_n)$ konvergiert $\implies$ $(b_n)$ konvergiert und
      $\limn a_n=\limn b_n$.
    \item $\sumn 1 a_n$ konv. absolut $\implies$
      $\sumn 1 b_n$ konv. absolut und $\sumn 1 a_n=\sumn 1 b_n$
    \end{stmts}
  }
% -----------------------------------------------------------------------------
\lessertheorem Riemann'scher Umordnungssatz:
 $\sumn 1 a_n$ konvergent, aber nicht absolut konvergent=>{
 \begin{stmts}
   \item Es ex. eine divergente Umordnung $\sumn 1 b_n$ von $\sumn 1 a_n$.
   \item Sei $s\real$. Dann ex. eine Umordnung $\sumn 1 c_n$ von $\sumn 1 a_n$ mit 
     $\sumn 1 c_n$ konv. und $\sumn 1 c_n=s$.
   \end{stmts}
 }
% -----------------------------------------------------------------------------
\remark:{
  $\sumn 1 a_n$ konvergiert absolut $\equiv$ Jede Umordnung $\sumn 1 b_n$ von
  $\sumn 1 a_n$ konvergiert (gegen $\sumn 1 a_n$).
}
% -----------------------------------------------------------------------------
\definition Produktreihe:{
  Seien $\sumn 0 a_n,\sumn 0 b_n$ absolut konv. Reihen. Sei
  $\Phi:\SetN\times\SetN\to\SetN$ bijektiv und $a_j b_k =p_{\Phi(j,k)}$.
  $\sumn 0 p_n$ hei"st dann Produktreihe von $\sumn 0 a_n,\sumn 0 b_n$.
}
% -----------------------------------------------------------------------------
\remark:{
  Sind $\sumn 0 p_n$ und $\sumn 0 \tilde p_n$ zwei solche Produktreihen,
  dann ist jede Umordnung der anderen.
}
% -----------------------------------------------------------------------------
\theorem:
  $\sumn 0 a_n,\sumn 0 b_n$ abs. konv.,$\sumn 0 p_n$ sei eine ihrer
  Produktreihen.=>{
  $\sumn 0 p_n$ konv absolut und 
  $\sumn 0 p_n=\left(\sumn 0 a_n\right)\left(\sumn 0  b_n\right)$
  }
% -----------------------------------------------------------------------------
\definition Cauchyprodukt:{
  Seien $\sumn 0 a_n,\sumn 0 b_n$ absolut konvergent. Sei weiterhin
  $c_n:=\sum_{k=0}^n a_k b_{n-k}$. Dann hei"st $\sumn 0 c_n$ das
  Cauchyprodukt von $\sumn 0 a_n$ und $\sumn 0 b_n$.
}
% -----------------------------------------------------------------------------
\theorem:
  $\sumn 0 a_n,\sumn 0 b_n$ abs. konv.,$\sumn 0 c_n$ sei ihr Cauchyprodukt=>{
  $\sumn 0 c_n$ konv. absolut und 
  $\sumn 0 c_n=\left(\sumn 0 a_n\right)\left(\sumn 0  b_n\right)$.
  }
% -----------------------------------------------------------------------------
\remark Funktionalgleichung der Exponentialfunktion:{
  Aus (13.3) ergibt sich die Funktionalgleichung f"ur die Exponentialfunktion:
  \[E(x)\cdot E(y)=E(x+y)
    \]
  }
% -----------------------------------------------------------------------------
\para{Potenzreihen}
% -----------------------------------------------------------------------------
\definition Potenzreihe:{
  \label{def:potenzreihe}
  Sei $(a_n)_{n=0}^\infty$ eine reelle Folge. Eine Reihe der Form
  $\sumn 0 a_n x^n$ hei"st Potenzreihe/PR.
  }
% -----------------------------------------------------------------------------
\theorem:$\sumn 0 a_n x^n$ Potenzreihe=>{
  \begin{stmts}
    \item Ist $\sqrt[n]{|a_n|}$ unbeschr"ankt,
      so konv. die PR nur f"ur $x=0$.
    \item Ist $\sqrt[n]{|a_n|}$ beschr"ankt,
      so ex. $\rho:=\limsupn \sqrt[n]{|a_n|}$:
    \item $\rho=0 \implies$ PR konv. abs. f"ur alle $x\real$
    \item $\rho>0 \implies$ PR konv. abs. f"ur alle $|x|<\frac 1 \rho$
      und div. f"ur alle $|x|>\frac 1 \rho$.
      F"ur $|x|=\frac 1 \rho$ ist keine allgemeine Aussage m"oglich.
    \end{stmts}
  }
% -----------------------------------------------------------------------------
\definition Konvergenzradius:{
  Sei $\sumn 0 a_n x^n$ eine Potenzreihe und $\rho$ wie oben. 
  Dann hei"st
  \[r:=\begin{cases}
      0                 & \sqrt[n]{|a_n|} \text{unbeschr"ankt} \\
      \infty            & \sqrt[n]{|a_n|} \text{beschr"ankt und $\rho=0$} \\
      \frac 1 \rho      & \sqrt[n]{|a_n|} \text{beschr"ankt und $\rho>0$}
      \end{cases}
    \]
  der Konvergenzradius/KR der PR.
  }
% -----------------------------------------------------------------------------
\definition Konvergenzbereich:{
  Sei $\sumn 0 a_n x^n$ eine Potenzreihe und $r$ ihr KR. Dann hei"st die Menge 
  \[K:=\left\{x\in\SetR\mid\sumn 0 a_n x^n \text{ konv.} \right\}
    \]
  der Konvergenzbereich/KB der PR.
  }
% -----------------------------------------------------------------------------
\remark:{
  In Abh"angigkeit von $r$ hat der KB folgende Gestalt:
  \[KB=\begin{cases}
      \{0\}                     	& \text{falls $r=0$} \\
      \SetR                     	& \text{falls $r=\infty$} \\
      \stack({(};[)-r,r\stack({)};])    & \text{falls $r\in(0;\infty)$}
      \end{cases}
    \]
  }
% -----------------------------------------------------------------------------
\definition Sinus/Cosinus:{
  Sinus und \indexthis{Cosinus} werden hier "uber ihre 
  Potenzreihenentwicklung festgelegt:
  \begin{align*}
    \cos:\SetR\to\SetR &\quad \cos x:=\sumn 0 (-1)^n \frac{x^{2n}}{(2n)!}\\
    \sin:\SetR\to\SetR &\quad \sin x:=\sumn 0 (-1)^n \frac{x^{2n+1}}{(2n+1)!}
    \end{align*}
  Die beiden Potenzreihen konvergieren absolut f"ur alle $x\real$.
  }
% -----------------------------------------------------------------------------
\theorem:
  $\sumn 0 a_n x^n$,$\sumn 0 b_n x^n$ PR mit KRen $r_1,r_2>0$.=>{
  Sei $R:=\min\{r_1,r_2\}$, $c_n:=\sum_{k=0}^n a_k b_{n-k}$.
  
  Dann ist der Konvergenzradius von $\sumn 0 c_n x^n$ mindestens $R$ und
  es gilt
  \[\sumn 0 c_n x^n=\left(\sumn 0 a_n x^n\right)\left(\sumn 0 b_n x^n\right)
    \]
  }
% -----------------------------------------------------------------------------
\theorem:$E(x)=\sumn 0 \frac{x^n}{n!}$=>{
  \begin{stmts}
    \item $(\forall x,y\real)(E(x)E(y)=E(x+y))$ 
    \item $E(0)=1 \quad E(1)=e$ 
    \item $(\forall r\in\SetQ)(E(r)=e^r)$ 
    \item $(\forall x\in\SetQ)(E(-x)=\frac{1}{E(x)} \land E(x)>0)$ 
    \item $E:\SetR\to\SetR$ streng monoton wachsend 
    \end{stmts}
  }
% -----------------------------------------------------------------------------
\para{$g$-adische Entwicklung}
% -----------------------------------------------------------------------------
\definition Gau"s-Klammer:{
  Sei $a\real$. Dann ex. genau ein $k\integer$ mit $k\le a\le k+1$.\par
  Dann ist $[a]:=k$ die gr"o"ste ganze Zahl kleiner oder gleich a.
  }
% -----------------------------------------------------------------------------
\convention{
  $a\ge 0\quad g\natural\quad g>1$
  }
% -----------------------------------------------------------------------------
\definition $g$-adische Entwicklung:{
  Ist die Zahl $a\real$ gegeben, so erh"alt man die g-adische Entwicklung von
  $a$ durch diese Folge:
  \begin{align*}
    z_0&:=[a]\\
    z_{n+1}&:=\left[\left(a-z_0-\frac{z_1}{g}-\cdots\frac{z_n}{g^n}\right)g^{n+1}\right]
  \end{align*}
  }
% -----------------------------------------------------------------------------
\theorem:$a\real, (z_n)_{n=0}^\infty$ die g-adische Entwicklung von a=>{
  \begin{stmts}
    \item $z_0+\frac{z_1}{g}+\cdots+\frac{z_n}{ g^n}\le a <z_0+
          \frac{z_1}{ g}+\cdots+\frac{z_n}{ g^n}+\frac{1}{ g^n}$ 
    \item $(\forall n\nnatural)(z_n\nnatural)$
    \item $(\forall n\nnatural)(z_n\le g-1)$
    \item Die Folge $(z_n)$ ist bei festem $a$ eindeutig bestimmt.
    \item $\sumn 0 z_n=a$
    \end{stmts}
  }
% -----------------------------------------------------------------------------
\definition g-adische Schreibweise:{
  Seien $a\real, (z_n)_{n=0}^\infty$ die g-adische Entwicklung von a. Dann schreibt man
  \[a = z_0,z_1 z_2 z_3\ldots 
    \]
  }
% -----------------------------------------------------------------------------
\theorem:$a\real, (z_n)_{n=0}^\infty$ die g-adische Entwicklung von a=>{
  $(z_n)=g-1 \; \ffa n\natural$ ist nicht m"oglich.
  }
% -----------------------------------------------------------------------------
\theorem:=>{
  $\SetR$ ist "uberabz"ahlbar.
  }
% -----------------------------------------------------------------------------
\para{Grenzwerte bei Funktionen}
% -----------------------------------------------------------------------------
\definition H"aufungspunkt:{
  Sei $D\subseteq\SetR$ und $x_0\real$. Dann hei"st $x_0$
  H"aufungspunkt/HP von D $:\equiv$ \par
  $\epsn(D\cap U_\epsilon(x_0)\setminus \{x_0\} \ne\emptyset)$
  }
% -----------------------------------------------------------------------------
\remark:{
  Endliche Mengen haben keinen H"aufungspunkt.
  }
% -----------------------------------------------------------------------------
\lessertheorem:$D\subseteq\SetR,x_0\real$=>{
  $x_0$ ist HP von D\ $\equiv$ Es ex. eine Folge $(x_n)$ in
  $D\setminus \{x_0\}$ mit $\limn x_n=x_0$.
  }
% -----------------------------------------------------------------------------
\convention{
  $D\subseteq\SetR,x_0$ H"aufungspunkt von D.
  }
% -----------------------------------------------------------------------------
\definition Grenzwert einer Funktion:{
  Sei $f:D\to\SetR$ eine Funktion und $a\real$. Dann hei"st a der Grenzwert von
  $f$ $:\equiv$
  \[(\forall (x_n) \text{ in } D\setminus\{x_0\})
    (\limn x_n = x_0\implies \limn f(x_n)=a)
    \]
  Schreibweise: $\limx {x_0}=a$ oder $f(x) \tocond{x\to x_0} a$.
  }
% -----------------------------------------------------------------------------
\remark:{
  Falls $x_0\defined$, so ist der Funktionswert an dieser Stelle nicht
  relevant. F"ur Existenz und Gr"o"se von $\limx \xnull f(x)$ ist nur das
  Verhalten von $f$ in der ``N"ahe'' von $x_0$ relevant.
  }
% -----------------------------------------------------------------------------
\definition Einseitige Grenzwerte:{
  Man legt fest
  \begin{align*}
    \limx {x_0+} f(x) &:= \lim_{\substack{ x\to x_0 \\ x>x_0 }} f(x)\\
    \limx {x_0-} f(x) &:= \lim_{\substack{ x\to x_0 \\ x<x_0 }} f(x)
    \end{align*}
  }
% -----------------------------------------------------------------------------
\theorem Epsilon-Delta-Charakterisierung des Grenzwert:$f:D\to\SetR,a\real$=>{
  \index{Grenzwert>Epsilon-Delta-Charakterisierung}
  Es ist $\limx \xnull f(x)=a$ $\equiv$
  \[\epsn(\exists \delta=\delta(\epsilon))(\forall x\defined\setminus\{x_0\})
    (|x-x_0|<\delta \implies |f(x)-a|<\epsilon)
    \]
  }
% -----------------------------------------------------------------------------
\theorem Folgen-Charakterisierung des Grenzwerts:$f:D\to\SetR$=>{
  \index{Grenzwert>Folgen-Charakterisierung}
  Der Grenzwert $\limx \xnull f(x)$ existiert $\equiv$
  \[(\forall (x_n) \text{ in }D\setminus\{x_0\})
    (\limn x_n=x_0\implies (f(x_n))\text{ konvergiert})
    \]
  }
% -----------------------------------------------------------------------------
\theorem:$f,g,h:D\to\SetR,
  \exists a:=\limx \xnull f(x),
  \exists b:=\limx \xnull g(x)$=>{F"ur $x\to x_0$: 
  \begin{stmts}
    \item $f(x)+g(x)\to a+b$,
      $f(x)\cdot g(x)\to a\cdot b$,
      $|f(x)| \to |a|$
    \item $(\exists \delta>0)(\forall x\defined\cap U_\delta(x_0)\setminus\{x_0\})
      (f(x)\le g(x))\implies a\le b$
    \item $(\exists \delta>0)(\forall x\defined\cap U_\delta(x_0)\setminus\{x_0\})
      (f(x)\le h(x)\le g(x)), a=b \implies h(x)\to a$
    \item $b\ne 0\implies
      (\exists \delta>0)(\forall x\defined\cap U_\delta(x_0)\setminus\{x_0\})
      (|g(x)|>\frac{|b|}{ 2})$
      $\implies \frac{f(x)}{ g(x)}\to \frac ab$
    \end{stmts}
  }
% -----------------------------------------------------------------------------
\definition:{
  Sei $(x_n)$ reelle Folge. Dann sagt man
  \begin{align*}
    x_n\to\infty &:\equiv (\forall c\real)(\exists n_0\natural)
                          (\forall n\ge n_0)(x_n>c)\\
    x_n\to -\infty &:\equiv (\forall c\real)(\exists n_0\natural)
                          (\forall n\ge n_0)(x_n<c)
  \end{align*}
  }
% -----------------------------------------------------------------------------
\definition:{
  Sei $D\subseteq\SetR:f:D\to\SetR$. Dann ist
  $\lim f(x)=\pm\infty$ $:\equiv$
  \[(\forall (x_n) \text{ in } D\setminus\{x_0\})
    (\limn x_n=x_0 \implies \limn f(x_n)=\pm\infty)
    \]
  }
% -----------------------------------------------------------------------------
\definition:{
  $D\subseteq\SetR$ sei nicht nach \tstack(oben;unten) beschr"ankt,
  $f:D\to\SetR$. Dann ist $\limx {\pm\infty}$ f(x)=a $:\equiv$
  \[(\forall (x_n) \text{ in } D)
    (\limn x_n=\pm\infty \implies \limn f(x_n)=a)
    \]
  ($a=\pm\infty$ zugelassen)
  }
% -----------------------------------------------------------------------------
\example:{
  Es gilt $\limx \infty E(x)=\infty$ und $\limx {-\infty} E(x)=0$.
  }
% -----------------------------------------------------------------------------
\para{Stetigkeit}
% -----------------------------------------------------------------------------
\definition Stetigkeit:{
  Sei $D\subseteq\SetR,f:D\to\SetR$. $f$ hei"st stetig in $x_0\defined$ $:\equiv$
  \[(\forall (x_n)\text{ in }D)(\limn x_n=x_0\implies \limn f(x_n)=f(x_0))
    \]
  Schreibweise: $f\cont$, wobei
  $\SetCont(D):=\{f:D\to\SetR\mid\text{ $f$ ist auf D\ stetig}\}$
  }
% -----------------------------------------------------------------------------
\theorem Epsilon-Delta-Charakterisierung der Stetigkeit:
  $D\subseteq\SetR,f:D\to\SetR,x_0\defined$=>{
  \index{Stetigkeit>Epsilon-Delta-Charakterisierung}
  \begin{stmts}
    \item $f$ stetig in $x_0$ $\equiv$
      $\epsn(\exists \delta=\delta(\epsilon)>0)(\forall x\defined)$ 
      $(|x-x_0|<\delta \implies |f(x)-f(x_0)|<\epsilon)$
    \item $x_0$ HP von D $\implies$ $f$ ist stetig in $x_0$ $\equiv$
      $\limx \xnull f(x)=f(x_0)$
    \end{stmts}
  }
% -----------------------------------------------------------------------------
\theorem:$f,g:D\to\SetR$, $f,g$ seien stetig in $x_0\defined$.=>{
  Dann sind auch $f+g,f\cdot g,|f|$ stetig.
  Ist $\tilde D=\{x\defined\mid g(x)\ne 0\}, x_0\in\tilde D$, so ist
  $\frac f g$ auch stetig in $x_0$.\par
  Somit ist $\SetCont(D)$ ein Vektorraum.
  }
% -----------------------------------------------------------------------------
\theorem:$f:D\to\SetR$ sei stetig in $x_0$, $g: E\to\SetR$, wobei
  $f(D)\subseteq E$, $g$ stetig in $f(x_0)$=>{
  Dann ist $f\after g$ stetig in $x_0$.
  }
% -----------------------------------------------------------------------------
\theorem:=>{
  Potenzreihen mit Konvergenzradius $r>0$ sind stetig.
  }
% -----------------------------------------------------------------------------
\remark:{
  $E(x)$, $\sin x$, $\cos x$ sind stetig auf $\SetR$.
  }
% -----------------------------------------------------------------------------
\para{Eigenschaften stetiger Funktionen}
% -----------------------------------------------------------------------------
\theorem Zwischenwertsatz:
  $f\in \SetCont([a;b])$ und $f(a)\le y_0\le f(b)$ oder $f(a)\ge y_0\ge f(b)$=>{
  Dann existiert ein $x_0\in [a;b]$ mit $f(x_0)=y_0)$.
  }
% -----------------------------------------------------------------------------
\theorem Nullstellensatz von Bolzano:
  $f\in \SetCont([a;b])$ und $f(a)<0<f(b)$=>{
  Dann existiert ein $x_0\in (a;b)$ mit $f(x_0)=0$.
  }
% -----------------------------------------------------------------------------
\remark:{
  $E(\SetR)=(0;\infty)$
  }
% -----------------------------------------------------------------------------
\definition abgeschlossen/offen:{
  $D\subseteq\SetR$ hei"st abgeschlossen $:\equiv$
  $(\forall (x_n)\text{ in }D)(\limn x_n\defined)$
  
  $D\subseteq\SetR$ hei"st offen $:\equiv$
  $\SetR\setminus D$ ist abgeschlossen.
  }
% -----------------------------------------------------------------------------
\remark:{
  $\SetR$ ist offen und abgeschlossen,
  $[a,b)$ ist weder offen noch abgeschlossen. 
  Das bedeutet, offen und abgeschlossen sind bei Mengen keine Gegens"atze,
  w"ahrend dies im allgemeinen z.B. bei T"uren der Fall ist. :-)
  }
% -----------------------------------------------------------------------------
\lessertheorem:=>{
  $D$ ist abgeschlossen $\equiv$ Jeder HP von $D\in D$
  
  $D$ ist offen $\equiv$
  $(\forall x\defined)(\exists \delta>0)(U_\delta(x)\subseteq D)$
}
% -----------------------------------------------------------------------------
\lessertheorem:$D\subseteq\SetR$ abgeschlossen und beschr"ankt,
  $(x_n)$ sei Folge in $D$=>{
  $(x_n)$ enth"alt konvergente TF in $D$. (``$D$ ist folgenkompakt.'')
  }
% -----------------------------------------------------------------------------
\definition Beschr"anktheit einer Funktion:{
  $f:D\to\SetR$ hei"st beschr"ankt, falls $f(D)$ beschr"ankt ist.
  }
% -----------------------------------------------------------------------------
\theorem:$\emptyset\ne D\subseteq\SetR$, $D$ abgeschlossen, beschr"ankt.
  $f\cont$=>{
  Dann existieren $x_1,x_2\in D$ so, dass 
  $(\forall x\in D)(f(x_1)\le f(x)\le f(x_2))$,
  d.h.: $f$ ist beschr"ankt und $f(D)$ besitzt Minimum und Maximum.
  }
% -----------------------------------------------------------------------------
\remark:{
  Sei $ I\subseteq\SetR$ ein Intervall und $f: I\to\SetR$ sei
  streng mon. \tstack(wachsend;fallend) $\implies$ $f$ injektiv und
  $f^{-1}:f( I)\to I$.

  $f$ streng mon. \tstack(wachsend;fallend) $\equiv$
  $f^{-1}$ streng mon. \tstack(wachsend;fallend)
}
% -----------------------------------------------------------------------------
\theorem:Sei $ I$ ein Intervall, $f\in\SetCont( I)$=>{
  $f( I)$ ist Intervall
  }
% -----------------------------------------------------------------------------
\lessertheorem:
  $f\in\SetCont([a;b])$, $A:=\min f([a;b])$, $B:=\max f([a;b])$=>{
  Dann ist $f([a;b])=[A;B]$.
  }
% -----------------------------------------------------------------------------
\theorem: $ I$ Intervall,$f\in\SetCont( I)$ streng monoton=>{
  $f^{-1}\in\SetCont(f(I))$
  }
% -----------------------------------------------------------------------------
\definition Logarithmus:{
  Auf $\log:(0;\infty)\to\SetR$ wird der Logarithmus $\log x$ folgenderma"sen
  definiert: $\log x:=E^{-1}(x)$.
  }
% -----------------------------------------------------------------------------
\remark Eigenschaften von $\log$:{
  \begin{stmts}
    \item $\log$ ist auf $(0;\infty)$ streng monoton wachsend und stetig.
    \item $\log 1=0$, $\log e=1$, $\limx {0+} \log x=-\infty$, 
      $\limx \infty \log x=\infty$.
    \item $\log (x\cdot y)=\log x+\log y$
    \item $\log (\frac x y)=\log x-\log y$
    \end{stmts}
  }
% -----------------------------------------------------------------------------
\definition Allgemeine Potenz:{
  Sei $a>0,x\real$. Speziell f"ur $a=e$:
  \[e^x:=E(x \log e)=E(x)
    \]
  Dann ist
  \[a^x:=E(x \log a)=e^{x \log a}
    \]
}
% -----------------------------------------------------------------------------
\remark Eigenschaften der allgemeinen Potenz:{
  \begin{stmts}
    \item $x\mapsto a^x$ stetig
    \item $a^x>0$ 
    \item $a^{x+y}=a^x\cdot a^y$
    \item $a^{-x}=\frac{1}{a^x}$
    \item $\left(a^x\right)^y=a^{x\cdot y}$
    \end{stmts}
  }
% -----------------------------------------------------------------------------
\para{Funktionenfolgen und -reihen}
 % -----------------------------------------------------------------------------
\convention{
  Sei $D\subseteq\SetR$ und $(f_n)$ eine Folge von Funktionen.
  
  $f_n:D\to\SetR,n\natural$, $s_n:=\sum_{k=1}^{n} f_k$.
  
  $\sumn 1 f_n$ bezeichnet die Funktionenfolge $(s_n)$.
  }
% -----------------------------------------------------------------------------
\definition Punktweise Konvergenz:{
  Die Funktionenfolge $(f_n),\sumn 1 f_n$ hei"st punktweise/pw konvergent
  $:\equiv$ 
  \[\limn f_n(x)/\sumn 1 f_n(x) \text{ex. f"ur alle $x\in D$}
    \]
  Dann hei"st $f(x):=\limn f_n(x)$ bzw. $s(x):=\sumn 1 f_n(x)$ die
  Grenz- bzw. Summenfunktion von $(f_n)$/$\sumn 1 f_n$.
  }
% -----------------------------------------------------------------------------
\remark:{
  Punktweise Konvergenz in Quantorenschreibweise:
  \[(\forall x\defined)\epsn(\exists n_0=n_0(x,\epsilon))(\forall n\ge n_0)
      \stack({(|f_n(x)-f(x)|<\epsilon)};{(|s_n(x)-s(x)|<\epsilon)})
    \]
}
% -----------------------------------------------------------------------------
\definition Gleichm"a"sige Konvergenz:{
  $(f_n)$ bzw. $\sumn 1 f_n$ hei"st gleichm"a"sig/glm konvergent $:\equiv$
  Es ex. ein $f:D\to\SetR$ bzw. $s:D\to\SetR$ mit:
  \[
    \epsn(\exists n_0=n_0(\epsilon))(\forall n\ge n_0)(\forall x\defined)
      \stack({(|f_n(x)-f(x)|<\epsilon)};{(|s_n(x)-s(x)|<\epsilon)})
  \]
}
% -----------------------------------------------------------------------------
\remark:{
  $(f_n)$ konv. glm gegen $f$ $\implies$ $(f_n)$ konv. pw gegen $f$.
  (analog f"ur $(s_n)$)
}
% -----------------------------------------------------------------------------
\theorem:
  $(f_n)$ Funktionenfolge und $f:D\to\SetR$ eine Fkt.=>
  {
    Genau dann, wenn eine Nullfolge $(\alpha_n)$ und ein $n\natural$ 
    mit
    \[(\forall n\ge n_0)(\forall x\defined)(|f_n(x)-f(x)|<\alpha_n)\]
    existieren, konvergiert $(f_n)$ glm gegen $f$.
  }
% -----------------------------------------------------------------------------
\theorem Majorantenkriterium von Weierstra"s:
  $(f_n)$ Funktionenfolge und $f:D\to\SetR$ eine Fkt.=>
  {
    Existiert eine Folge $(c_n)$ und ein $n_0\natural$ mit
    \[(\forall n\ge n_0)(\forall x\defined)(|f_n(x)|\le c_n)\]
    und ist $\sumn 1 c_n$ konvergent, so konv. $\sumn 1 f_n$ glm auf D.
  }
% -----------------------------------------------------------------------------
\remark:{
  Potenzreihen konvergieren im allgemeinen nicht glm auf ihrem
  Konvergenzbereich, jedoch konvergieren sie gleichm"a"sig auf einem 
  abgeschlossenen Intervall, das Teilmenge ihres Konvergenzbereiches ist:
  }
% -----------------------------------------------------------------------------
\theorem:$\sumn 0 a_n x^n$ Potenzreihe, $D$ ihr Konvergenzbereich,
  $[a;b]\subseteq D$=>{
  $\sumn 0 a_n x^n$ konvergiert glm auf $[a;b]$.
  }
% -----------------------------------------------------------------------------
\theorem:
  $(f_n)$/$\sumn 0 f_n$ glm gegen $f:D\to\SetR$ konvergente Funktionenfolge=>{
  Dann gilt:
  \begin{stmts}
    \item $(\forall n\natural)(f_n \text{ ist stetig in $x_0$})$ $\implies$
            $f$ ist stetig in $x_0$
    \item $(\forall n\natural)(f_n\cont) \implies f\cont$
    \end{stmts}
  }
% -----------------------------------------------------------------------------
\remark:{
  Ist $(f_n)$ pw konvergent auf $D$ gegen $f:D\to\SetR$, so gilt:\par
  $f_n\cont$, aber $f\not\in\SetCont(D)$ $\implies$ $(f_n)$ konvergiert nicht
  glm.
  }
% -----------------------------------------------------------------------------
\remark:{
  Ist $(f_n)$/$\sumn 0 f_n$ eine glm gegen $f:D\to\SetR$ konvergente Funktionenfolge,
  sind alle $(f_n)$ in $x_0\in D$ stetig und ist $x_0$ HP von $D$,
  so gilt:
  \[\limx \xnull\left(\limn f_n(x)\right) \; = \;
    \limx \xnull f(x)=f(x_0)=\limn f_n(x_0) \; = \;
    \limn \left( \limx \xnull f_n(x) \right)
    \]
  }
% -----------------------------------------------------------------------------
\theorem Identit"atssatz:
  $f(x)=\sumn 0 a_n x^n, g(x)=\sumn 0 b_n x^n$ PRen mit KRen $r_1,r_2>0$=>{
    Sei $R:=\min\{r_1,r_2\}$. Sei $(x_k)$ eine Folge in $(-R,R)\setminus\{0\}$
    mit $\limes k->\infty x_k=0$.\par
    Dann gilt $(\forall k\natural)(f(x_k)=g(x_k))$ $\implies$
      $(\forall n\nnatural)(a_n=b_n)$
  }
% -----------------------------------------------------------------------------
\para{Gleichm"a"sige Stetigkeit}
% -----------------------------------------------------------------------------
\convention{
  $D\subseteq\SetR \qquad f:D\to\SetR$
  }
% -----------------------------------------------------------------------------
\remark:{
  Es ist $f\cont$ $\equiv$
  \[(\forall x'\defined)\epsn(\exists \delta(\epsilon,x'))(\forall x\defined)
    (|x-x'|<\delta \implies |f(x)-f(x')|<\epsilon)
    \]
  }
% -----------------------------------------------------------------------------
\definition Gleichm"a"sige Stetigkeit:{
  $f:D\to\SetR$ hei"st gleichm"a"sig stetig/glm stetig $:\equiv$
  \[\epsn(\exists \delta(\epsilon))(\forall x,x'\defined)
    (|x-x'|<\delta \implies |f(x)-f(x')|<\epsilon)
    \]
  }
% -----------------------------------------------------------------------------
\remark:{
  $f$ ist glm stetig auf $D$ $\implies$ $f\cont$
  }
% -----------------------------------------------------------------------------
\theorem:$D$ beschr"ankt und abgeschlossen, $f\cont$=>{
  $f$ ist glm stetig
  }
% -----------------------------------------------------------------------------
\definition Lipschitz-Stetigkeit:{
  $f:D\to\SetR$ hei"st Lipschitz-stetig $:\equiv$
  \[(\exists L>0)(\forall x,x'\defined)(|f(x)-f(x')|<L(x-y))
    \]
  }
% -----------------------------------------------------------------------------
\remark:{
  $f$ ist Lipschitz-stetig auf $D$ $\implies$ $f$ ist glm stetig auf $D$
  }

% -----------------------------------------------------------------------------
\para{Das Beste aus "Ubungen und Bl"attern}
% -----------------------------------------------------------------------------
\definition Cantor'sche Mengen:{
  Setze $M_0:=[0,1]$, $M_1:=M_0\setminus (\frac 1 3,\frac 2 3)$, 
  $M_2:=M_1\setminus ((\frac 1 9,\frac 2 9 ) \cup (\frac 7 9, \frac 8 9))$ usw.
  Setze dann
  \[M:=\bigcap_n M_n
    \]
  Dann kann man zeigen, dass $M$ "uberabz"ahlbar ist, aber dennoch 
  $\lambda(M)=0$ gilt.
  }
% -----------------------------------------------------------------------------
\definition Vitali'sche Mengen:{
  Definiere "Aquivalenzrelation ``$\sim$'': $x\sim y:\equiv x-y\rational$. Dann
  ergeben sich "Aquivalenzklassen
  \[K_x:=\{y\mid x\sim y\}=x+\SetQ
    \]
  Von jeder Klasse wird nun ein Vertreter $x\in(-1,1)$ gew"ahlt.
  Dann gilt f"ur die Vitali'sche Menge
  \[V:=\{x\mid x\in(-1,1)\land \text{ $x$ ist Vertreter einer Klasse}\}
    \]
  mit $V_r:=V+r$ (wobei $r\rational$)
  \begin{stmts}
    \item $V_r\cap V_q=\emptyset$ ($r\ne q,r,q\rational$)
    \item $(-1,1)\subseteq \bigcap_{|r|<2} V_r$
    \item $\bigcap_{|r|<2} V_r\subseteq (-3,3)$
    \item $\lambda(V_r)=\lambda(V)$
    \item Es ergibt sich, dass die $\sigma$-Additivit"at f"ur $V$ nicht gilt.
    \end{stmts}
  }
% -----------------------------------------------------------------------------
\theorem:
  $M\subseteq \SetR^n$ messbar, $f\in\SetFLMeas(M)$, $g:\SetR\to\SetR$=>{
  Dann gilt
  \begin{stmts} 
    \item $g\in\SetCont(\SetR,\SetR)$ $\implies$ $g\circ f\in\SetFLMeas(M)$
    \item $g$ monoton $\implies$ $g\circ f\in\SetFLMeas(M)$
    \end{stmts}
  }

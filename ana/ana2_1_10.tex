% -----------------------------------------------------------------------------
\section{Normierte lineare R"aume und Konvergenz}
% -----------------------------------------------------------------------------
\convention{
  Mit $\VR U,\VR V,\VR W,\VR X,\VR Y,\VR Z$ bezeichnen wir in Zukunft Vektorr"aume. 
  Falls diese endlich sind, so erscheint ihre Dimension im Index, wie z.B. 
  in $\VR V_3$.
  
  Die Menge der Homomorphismen von $V$ nach $W$ bezeichnen wir dann mit 
  $\VRL(\VR V, \VR W):=\text{Hom}(\VR V,\VR W)$.
  }
% -----------------------------------------------------------------------------
\definition Norm:{
  Eine Abbildung $\norm \cdot:\VR V\to\SetR$ mit den Eigenschaften
  \begin{stmts}
    \item 
      $(\forall x\in \VR V)(\norm x \geq 0;\norm x = 0 \equiv x = \nullvec)$
    \item
      $(\forall x\in \VR V)(\norm{\lambda x} = |\lambda| \norm x$
      (Homogenit"at)
    \item
      $(\forall x,y\in \VR V)(\norm{x+y} \leq \norm x+\norm y)$
      (Dreiecksungleichung)
    \end{stmts}
  hei"st eine Norm auf $\VR V$. Ist $\norm \cdot$ eine Norm auf $\VR V$, so hei"st
  $(\VR V;\norm \cdot)$ ein normierter linearer Raum (NLR). F"ur $x,y\in\VR V$
  hei"st $\norm{x-y}$ der Abstand von $x$ und $y$ (bez"uglich $\norm \cdot$).
  }
% -----------------------------------------------------------------------------
\example Beispiele f"ur Normen:{
  Im Vektorraum $\VR V_1=\SetR$ existiert die Betragsfunktion als Norm mit 
  $\norm x:=|x|$.
  
  In $\SetR^n$ existieren folgende Normen (sei $x=(\ntuple \xi)$):
  \begin{stmts}
    \item $\norm x:= \max \{ \ntuple \xi \}$ (Maximumsnorm)
    \item $\norm x:= \sum_{k=1}^n |\xi_k|$ (Betragssummennorm)
    \item $\norm x:= \sqrt{\sum_{k=1}^n \xi_k^2}$ (Euklid-Norm)
    \end{stmts}
  
  In $\VR V=\SetCont([a;b];\SetR)$ existieren folgende Normen:
  \begin{stmts}
    \item $\norm x:= \max \{ \norm{f(t)}\mid t\in [a;b] \}$ 
    \item $\norm x:= \int_a^b |f(t)| dt$ 
    \end{stmts}
  }
% -----------------------------------------------------------------------------
\lessertheorem:=>{
  \begin{stmts}
    \item $| \norm x - \norm y | \leq \norm{x-y}$ 
    \item $\norm{\sum_{k=1}^n x_k} \leq \sum_{k=1}^n \norm{x_k}$
    \end{stmts}
  }
% -----------------------------------------------------------------------------
\definition Innenprodukt:{
  \index{Skalarprodukt}
  Seien $x=(\ntuple \xi),y=(\ntuple \eta)$. 
  Dann hei"st $x\cdot y:=\sum_{k=1}^n \xi_k\eta_k$ Innenprodukt/Skalarprodukt 
  von $x$ und $y$.
  }
% -----------------------------------------------------------------------------
\remark:{
  Beachte
  \[\sqrt{x\cdot x} = \text{Euklidnorm}
    \]
  }
% -----------------------------------------------------------------------------
\theorem Cauchy-Schwarz-Ungleichung:=>{
  Die Euklidnorm ist eine Norm auf endlich dimensionalen VRen, und es gilt
  \[|x\cdot y|\leq\norm x \norm y
    \]
  }
% -----------------------------------------------------------------------------
\remark:{
  Ist $\norm \cdot_0$ eine Norm auf $\SetR^n$ und $\VR V_n$ ein $n$-dimensionaler
  VR, so kann man $\VR V_n$ folgenderma"sen normieren: Man w"ahlt eine Basis
  $\{\ntuple b\}$. $x=\sum_{k=1}^n \xi_k b_k$. Dann ist 
  $\norm x:=\norm{(\ntuple \xi)}$ Norm auf $\VR V_n$.
  
  Ist umgekehrt $\norm \cdot$ eine Norm auf $V_n$ mit $\{\ntuple b\}$ als Basis,
  so ist $\norm{(\ntuple \xi)}_0:=\norm{\sum_{k=1}^n \xi_k b_k}$ eine Norm
  auf $\SetR^n$.
  }
% -----------------------------------------------------------------------------
\definition Beschr"anktheit:{
  Eine Menge $\LSet A\subseteq \VR V$ hei"st beschr"ankt $:\equiv$
  $(\exists c>0)(\forall x\in\LSet A)(\norm x < c)$.
  }
% -----------------------------------------------------------------------------
\lesserdefinition Konvergenz:{
  Eine Folge $(x_n)$ in $\VR V$ hei"st konvergent, falls gilt
  \[(\exists x\in\VR V)\epsn(\exists n_0\natural)(\forall n\geq n_0)(\norm{x_n-x}<\epsilon)
    \]
  In diesem Fall hei"st $x$ Grenzwert/GW von $(x_n)$. S"amtliche Schreibweisen
  von reellen Grenzwerten "ubertragen sich entsprechend.  
  }
% -----------------------------------------------------------------------------
\lesserdefinition Cauchy-Folge:{
  Eine Folge $(x_n)$ in $\VR V$ hei"st Cauchy-Folge, falls gilt 
  \[\epsn(\exists n_0\natural)(\forall n,m\geq n_0)(\norm{x_n-x_m}<\epsilon)
    \]
  }
% -----------------------------------------------------------------------------
\remark Analog wie in $\SetR$ definiert man:{
  \begin{itemize}
    \item Divergenz
    \item Beschr"anktheit
    \item Teilfolge
    \end{itemize}
  }
% -----------------------------------------------------------------------------
\remark Analog wie in $\SetR$ beweist man:{
  \begin{itemize}
    \item Konvergente Folgen sind beschr"ankt
    \item Eindeutigkeit des Grenzwertes
    \item $x_n\to x \equiv \norm{x_n-x}\to 0$
    \item $x_n\to x,y_n\to y \implies x_n+y_n\to x+y$
    \item $\lambda\real: \lambda x_n\to\lambda x$
    \item $\norm {x_n}\to \norm x$
    \item Jede Teilfolge einer konv. Folge konvergiert gegen den selben Grenzwert.
    \end{itemize}
  }
% -----------------------------------------------------------------------------
\remark Konvergenz von Cauchyfolgen:{
  Wie in $\SetR$ gilt auch in NLRen, dass jede konvergente Folge eine
  Cauchyfolge ist, aber die Umkehrung gilt nur noch in speziellen R"aumen:
  }
% -----------------------------------------------------------------------------
\definition Banach-Raum:{
  Ein NLR $(\VR V;\norm \cdot)$ hei"st vollst"andig oder Banach-Raum, wenn in 
  $\VR V$ jede Cauchy-Folge konvergiert.
  }
% -----------------------------------------------------------------------------
\remark:{
  Die Definitionen 3 und 4 h"angen von der gew"ahlten Norm ab.
  }
% -----------------------------------------------------------------------------
\definition "Aquivalenz von Normen:{
  Zwei Normen $\norm \cdot_1$, $\norm \cdot_2$ auf $\VR V$ hei"sen "aquivalent, wenn
  gilt
  \[(\exists \alpha,\beta)(\forall x\in\VR V)(\alpha\norm x_1\leq\norm x_2\leq\beta\norm x_1)
    \]
  Schreibweise: $\norm \cdot_1\sim\norm \cdot_2$
  }
% -----------------------------------------------------------------------------
\theorem:=>{
  ``$\sim$'' ist eine "Aquivalenzrelation auf der Menge aller Normen auf $\VR V$.
  }
% -----------------------------------------------------------------------------
\theorem:$\norm \cdot_1$, $\norm \cdot_2$ Normen auf VR $\VR V_n$, $\dim \VR V_n=n$=>{
  Dann ist $\norm \cdot_1\sim\norm \cdot_2$
  }
% -----------------------------------------------------------------------------
\theorem:$(\VR V_n;\norm \cdot)$ NLR, $\dim V_n=n\natural$=>{
  \begin{stmts}
    \item
      Ist $\{\ntuple b\}$ eine Basis von $\VR V_n$ und 
      $(x_k)=\sum_{j=1}^n \xi_j^{(k)} b_j$ eine Folge in $\VR V_n$, so gilt:
      $(x_n)$ konvergent $\equiv$ die Folgen $(\xi_j^{(k)})$ sind konvergent 
      $(j=1\ldots n)$. 
      
      In diesem Fall gilt 
      $\limes k->\infty x_k=\sum_{j=1}^n \limes k->\infty \xi_j^{(k)} b_j$.
    \item
      $\VR V_n$ ist vollst"andig.
    \end{stmts}
  }
% -----------------------------------------------------------------------------
\theorem Satz von Bolzano-Weierstra"s:
  $(\VR V_n,\norm \cdot)$ endlich dimensionaler NLR, 
  $(x_k)$ beschr"ankte Folge in $\VR V_n$=>{
    $(x_k)$ besitzt eine konvergente Teilfolge.
  }
% -----------------------------------------------------------------------------
\theorem:$(\VR V_n,\norm \cdot_1)$ und $(\VR W,\norm \cdot_2)$ normierte NLRe, 
  $\Phi\in\VRL(\VR V_n,\VR W)$=>{
  Die Menge 
  \[M:=\left\{\frac{\norm{\Phi(x)}_2}{\norm x_1} \mid x\in\VR V_n\setminus \{0\} \right\}
    \]
  ist beschr"ankt und $\fnorm \cdot: \VRL(\VR V_n,\VR W)\to\SetR$ mit
  $\fnorm \Phi:=\sup M$ ist eine Norm auf $\VRL(\VR V_n,\VR W)$.
  F"ur diese Norm gilt:
  \[(\forall x\in\VR V)(\norm{\Phi(x)}_2\leq\fnorm \Phi\norm x_1)
    \]
}
% -----------------------------------------------------------------------------
\remark:{
  Ist $\norm \cdot_1$ Norm auf $\SetR^n$, $\norm._2$ Norm auf $\SetR^m$, so
  ist 
  \[\fnorm A:=\sup_{x\neq\nullvec}\frac{\norm{Ax}_2}{\norm x_1}
    \] 
  eine Norm auf $\SetR^{m\times n}$. Es gilt dann: 
  \[\norm{Ax}_2 \leq \fnorm A \cdot \norm x_1
    \]
  
  Ist $\norm \cdot=\norm \cdot_1=\norm \cdot_2$, so gilt f"ur $A,B\in\SetR^{n\times n}$
  \[\fnorm{A\cdot B}\leq \fnorm A \cdot \fnorm B
    \]
  Diese Eigenschaft nennt sich \indexthis{Submultiplikativit"at}.
  }
% -----------------------------------------------------------------------------
\section{Topologische Grundbegriffe}
% -----------------------------------------------------------------------------
\convention{
  Im folgenden sei stets $(\VR V,\norm \cdot)$ ein NLR. 
  }
% -----------------------------------------------------------------------------
\lesserdefinition Offene Kugel, Delta-Umgebung:{ 
  \index{Delta-Umgebung}
  F"ur $x_0\in\VR V,\delta>0$ hei"st 
  \[U_\delta:=\{x\in\VR V\mid\norm{x-x_0}<\delta\}
    \] 
  die Umgebung oder offene Kugel um $x_0$ mit Radius $\delta$.
  }
% -----------------------------------------------------------------------------
\remark:{
  Es gilt:
  \begin{align*}
    A\subseteq\VR V \text{ beschr"ankt} & \equiv (\exists c>0)(A\subseteq U_c(\nullvec)) \\
      & \equiv (\exists c>0)(\exists x_0\in\VR V)(A\subseteq U_c(x_0))
    \end{align*}
  }
% -----------------------------------------------------------------------------
\definition Innerer Punkt:{
  Sei $A\subseteq\VR V$.
  \begin{stmts}
    \item $x_0\in\VR V$ hei"st innerer Punkt von $A$ $:\equiv$ 
      $(\exists \delta>0)(U_\delta(x_0\subseteq A))$
    \item $\inner A:=$ Menge der inneren Punkte von $A$.
    \item $A$ hei"st offen $:\equiv$ $\inner A=A$.
    \end{stmts}
  }
% -----------------------------------------------------------------------------
\theorem:=>{
  \begin{stmts}
    \item Die Vereinigung beliebig vieler offener Mengen ist wieder offen.
    \item Der Durchschnitt endlich vieler offener Mengen ist wieder offen.
    \end{stmts}
  }
% -----------------------------------------------------------------------------
\remark:{
  Der Schnitt unendlich vieler offener Mengen ist im allgemeinen nicht mehr 
  offen.
  }
% -----------------------------------------------------------------------------
\definition H"aufungspunkt:{
  Sei $A\subseteq\VR V$.
  \begin{stmts}
    \item $x_0\in\VR V$ hei"st H"aufungspunkt/HP von $A$ $:\equiv$
      $(\forall \delta>0)(U_\delta(x_0)\setminus\{x_0\}\cap A \neq \emptyset)$.
    \item $H(A):=$ Menge aller H"aufungspunkte von $A$.
    \item $\closure A:=A\cup H(A)$ hei"st \indexthis{Abschluss} von $A$.
    \item $A$ hei"st abgeschlossen $:\equiv$ $\closure A=A$
      \index{Abgeschlossenheit}
    \end{stmts}
  }
% -----------------------------------------------------------------------------
\theorem:$A\subseteq\VR V$=>{
  \begin{stmts}
    \item $A$ ist abgeschlossen $\equiv$ $\VR V\setminus A$ ist offen. \\
    \item Die Vereinigung endl. vieler abgeschlossener Mengen ist abgeschlossen. \\
    \item Der Durchschnitt bel. vieler abgeschlossener Mengen ist abgeschlossen. \\
    \item A ist abgeschlossen $\equiv$ $H(A)\subseteq A$
    \end{stmts}
  }
% -----------------------------------------------------------------------------
\definition Randpunkt:{
  Sei $A\subseteq\VR V$.
  \begin{stmts}
    \item 
      $x_0\in\VR V$ hei"st Randpunkt von $A$, falls gilt
      \[(\forall \delta>0)(U_\delta(x_0)\cap A\neq\emptyset \land 
        U_\delta(x_0)\cap V\setminus A\neq\emptyset)\]
    \item
      $\rim A$ $:=$ Menge aller Randpunkte/Rand von $A$
    \end{stmts}
  }
% -----------------------------------------------------------------------------
\lessertheorem:=>{
  Es gilt $\rim A=\closure A\setminus\inner A$.
  }
% -----------------------------------------------------------------------------
\definition Umgebung:{
  Sei $x_0\in\VR V$. Dann hei"st $U\subseteq\VR V$ eine Umgebung von $x_0$, falls gilt
  \[(\exists \delta>0)(U_\delta(x_0)\subseteq U
    \]
  }
% -----------------------------------------------------------------------------
\theorem:=>{
  Sei $A\subseteq\VR V$.
  \begin{stmts}
    \item 
      $x_0\in H(A)$$\equiv$ Es existiert eine Folge $(x_n)$ in 
      $A\setminus\{x_0\}$ mit $\limn x_n = x_0$.
    \item
      $A$ ist abgeschlossen $\equiv$ Der Grenzwert jeder konvergenten Folge in $A$
      liegt in $A$.
    \end{stmts}
  }
% -----------------------------------------------------------------------------
\section{Reihen, Grenzwerte, Stetigkeit}
% -----------------------------------------------------------------------------
\lesserdefinition Konvergenz von Reihen:{
  \index{Reihen>Konvergenz von}
  Sei $(\VR V,\norm \cdot)$ ein NLR. Ist $(x_n)$ Folge in $\VR V$, so ist die 
  Konvergenz der Reihe $\sumn 1 x_n$ durch die Konvergenz der Folge
  $\left(\sum_{k=1}^n x_n\right)$ definiert.

  \index{Konvergenz>absolute}
  Absolute Konvergenz wird durch die Konvergenz der Folge 
  $\left(\sum_{k=1}^n \norm{x_n}\right)$ definiert.
  
  F"ur Banachr"aume gelten sinngem"a"s wie in $\SetR$:
  \begin{itemize}
    \item \indexthis{Majorantenkriterium}
    \item \indexthis{Wurzelkriterium}
    \item \indexthis{Quotientenkriterium}
    \item \indexthis{Cauchykriterium}
    \end{itemize}
  }
% -----------------------------------------------------------------------------
\example:{
  Betrachte $\SetR^{n\times n}$ mit einer submultiplikativen Norm $\fnorm .$
  Sei $A\in\SetR^{n\times n}$ Dann konvergiert die Reihe
  \[e^A:=\sum_{k=0}^\infty \frac{A^k}{k!}
    \]
  absolut.
  }
% -----------------------------------------------------------------------------
\definition Grenzwert einer Funktion:{
  \index{Funktion>Grenzwert einer}
  Seien $(\VR V,\norm \cdot_1)$, $(\VR W,\norm \cdot_2)$ NLRe, $D\subseteq\VR V$,
  $x_0\in D$ sei H"aufungspunkt von $D$ und $f:D\to\VR W$ eine Funktion,
  weiterhin sei $y_0\in\VR W$. Dann definiert man
  \[\limes x->{x_0} f(x)=y_0 :\equiv
    (\forall (x_n)\subseteq D\setminus \{x_0\})
    (x_n\to x_0\implies f(x_n)\to y_0)
    \]
  }
% -----------------------------------------------------------------------------
\theorem: 
  $(\VR V,\norm \cdot_1)$, $(\VR W,\norm \cdot_2)$ NLRe, $D\subseteq\VR V$,
  $x_0\in D$ HP von $D$, $f:D\to\VR W$ Funktion,
  $y_0\in\VR W$=>{
  Dann ist $\limes x->{x_0} f(x)=y_0$ $\equiv$
  \[\epsn(\exists \delta>0)(\forall x\in D\setminus\{x_0\})
    (|x-x_0|<\delta\implies |f(x)-y_0|<\epsilon)
    \]
  }
% -----------------------------------------------------------------------------
\definition Stetigkeit:{
  \index{Funktion>Stetigkeit einer}
  Seien $(\VR V,\norm \cdot_1)$, $(\VR W,\norm \cdot_2)$ NLRe, $D\subseteq\VR V$,
  $x_0\in D$ und $f:D\to\VR W$ eine Funktion.
  
  Dann hei"st $f$ stetig in $x_0$ $:\equiv$
  \[(\forall (x_n) \text{ in} D\setminus \{x_0\})
    (x_n\to x_0\implies f(x_n)\to f(x_0))
    \]
}
% -----------------------------------------------------------------------------
\theorem:
  $(\VR V,\norm \cdot_1)$, $(\VR W,\norm \cdot_2)$ NLRe, $D\subseteq\VR V$,
  $x_0\in D$, $f:D\to\VR W$=>{
  Dann ist $f$ genau dann stetig in $x_0$, wenn gilt
  \[\epsn(\exists \delta>0)(\forall x\in D\setminus\{x_0\})
    (|x-x_0|<\delta\implies |f(x)-f(x_0)|<\epsilon)
    \]  
  }
% -----------------------------------------------------------------------------
\lessertheorem:
  $(\VR V,\norm \cdot_1)$, $(\VR W,\norm \cdot_2)$ NLRe, $D\subseteq\VR V$,
  $f,g:D\to\VR W$, $w:D\to\SetR$, $\alpha,\beta\real$=>
{
  \begin{stmts}
    \item 
      Sei $x_0$ H"aufungspunkt von $D$ und sei 
      $\limx \xnull f(x)=y_0$, 
      $\limx \xnull g(x)=z_0$ und  
      $\limx \xnull h(x)=\lambda$.
      Dann gilt
      \begin{itemize}
        \item $\limx \xnull h(x)\cdot(\alpha f(x)+\beta g(x))=
          \lambda(\alpha y_0+\beta z_0)$
        \item $\lambda\neq 0$ $\implies$ 
	  \[(\exists \delta>0)(\forall x\in U_\delta(x_0)\cap D\setminus\{x_0\})
            (h(x)\neq 0, \limx \xnull \frac 1 {h(x)} = \frac 1 \lambda)
            \]
        \end{itemize}
    \item
      Sei $x_0\in D$ und $f,g,h$ stetig in $x_0$. Dann ist
      $h\cdot(\alpha f+\beta g)$ stetig in $x_0$. 
      Falls $h(x_0)\neq 0$, gilt au"serdem 
      \[(\exists\delta>0)(\forall x\in U_\delta(x_0)\cap D\setminus\{x_0\})(h(x)\neq 0)
        \] 
      und $\frac 1 h:U_\delta(x_0)\cap D\to\SetR$ ist stetig in $x_0$.
    \end{stmts}
  }
% -----------------------------------------------------------------------------
\lessertheorem:
  $(\VR V,\norm \cdot_1)$, $(\VR W,\norm \cdot_2)$, $(\VR Z,\norm \cdot_3)$  NLRe, 
  $D\subseteq\VR V$, $\LSet E\subseteq\VR W$, 
  $f:D\to\VR W$, $g:\LSet E\to\VR Z$, $f(D)\subseteq\LSet E$, $x_0\in D$, 
  $f$ stetig in $x_0$, $g$ stetig in $f(x_0)$=>{
  Dann ist $g\circ f:D\to\VR W$ stetig in $x_0$.
  }
% -----------------------------------------------------------------------------
\definition Stetigkeit auf Mengen:{
  \index{Stetigkeit>gleichm"a"sige}
  \index{Stetigkeit>Lipschitz-}
  Seien $(\VR V,\norm \cdot_1)$, $(\VR W,\norm \cdot_2)$, $D\subseteq\VR V$, 
  $f:D\to\VR W$.
  
  Dann hei"st $f$
  \begin{itemize}
    \item stetig auf $D$ genau dann, wenn $f$ stetig ist 
      in jedem Punkt $x\in D$.
    \item gleichm"a"sig stetig auf $D$ $:\equiv$
      \[\epsn(\exists \delta>0)(\forall x,y\in D)
        (\norm{x-y}_1<\delta \implies \norm{f(x)-f(y)}_2)
        \]
    \item Lipschitz-stetig auf $D$ $:\equiv$
      \[\epsn(\exists L\geq0)(\forall x,y\in D)
        (\norm{f(x)-f(y)}_2\leq L\norm{x-y}_1)
        \]
    \end{itemize}
  }
% -----------------------------------------------------------------------------
\remark:{
  Lineare Abbildungen sind Lipschitz-stetig.
  }
% -----------------------------------------------------------------------------
\lesserdefinition Konvergenz bei Funktionenfolgen:{
  \index{Funktionenfolgen>Konvergenz bei}
  \index{Konvergenz>punktweise}
  \index{Konvergenz>gleichm"a"sige}
  Seien $(\VR V,\norm \cdot_1)$, $(\VR W,\norm \cdot_2)$ NLRe, $D\subseteq\VR V$,
  $(f_n):D\to\VR W$, $f:D\to\VR W$
  
  Dann hei"st $(f_n)$
  \begin{itemize}
    \item punktweise/pw konvergent gegen $f$, wenn gilt 
      \[(\forall x\in D)(\limn f_n(x)=f(x))
        \]
    \item gleichm"a"sig/glm konvergent gegen $f$, wenn gilt
      \[\epsn(\forall n\geq n_0)(\forall x\in D)(\norm{f_n(x)-f(x)}_2<\epsilon) 
        \]
    \end{itemize}
  }
% -----------------------------------------------------------------------------
\theorem: $D\subseteq\VR V$, $(f_n):D\to\VR W$, $x_0\in D$,
  $(\forall n\natural)(f_n \text{ stetig in $x_0$})$=>{
  Konvergiert $f_n$ gleichm"a"sig gegen $f$, so ist auch $f$ stetig in $x_0$.
  }  
% -----------------------------------------------------------------------------
\section{Kompakte Mengen}
% -----------------------------------------------------------------------------
\convention{
  Sei $(\VR V,\norm \cdot)$ ein normierter linearer Raum.
  }
% -----------------------------------------------------------------------------
\definition Kompaktheit:{
  $A\subseteq\VR V$ hei"st kompakt genau dann, wenn gilt:
  
  Sind $O_j$ ($j\in\LSet J$) beliebige offene Teilmengen von $\VR V$ mit 
  $A\subseteq\bigcup_{j\in\LSet J} O_j$, so gibt es f"ur jede solche
  \indexthis{"Uberdeckung} endlich viele Teilmengen
  $O_{j_1},\ldots O_{j_n}$ mit $A\subseteq\bigcup_{i\in \{1,\ldots,n\}} O_{j_n}$.
  }
% -----------------------------------------------------------------------------
\theorem:$A\subseteq\VR V$ kompakt=>{
  $A$ ist abgeschlossen und beschr"ankt.
  }
% -----------------------------------------------------------------------------
\remark:{
  In unendlich dimensionalen R"aumen sind beschr"ankte und abgeschlossene
  Mengen im allgemeinen nicht kompakt.
  }
% -----------------------------------------------------------------------------
\theorem Satz von Heine-Borel:
$(\VR V_n;\norm \cdot)$ $n$-dimensionaler NLR, $\emptyset\neq\ A\subseteq\VR V_n$
  abgeschlossen und beschr"ankt=>{
  \index{Heine-Borel>Satz von}
  $A$ ist kompakt. ($\emptyset$ ist auch kompakt)
  }
% -----------------------------------------------------------------------------
\theorem:
$(\VR V_n;\norm \cdot)$ $n$-dimensionaler NLR, $A\subseteq\VR V_n$=>{
  Dann sind folgende Aussagen "aquivalent:
  \begin{stmts}
    \item $A$ ist kompakt
    \item $A$ ist beschr"ankt und abgeschlossen
    \item Jede Folge $(x_n)$ in $A$ besitzt eine konvergente Teilfolge 
      $(x_{n_l})$ mit $\limes l->\infty x_{k_l}\in A$.
    \end{stmts}
  }
% -----------------------------------------------------------------------------
\remark:{
  Die "Aquivalenz von (1) und (3) gilt auch in unendlich dimensionalen NLRen.
  }
% -----------------------------------------------------------------------------
\theorem:
  $(\VR V,\norm \cdot_1)$, $(\VR W,\norm._2)$ endlich dimensionale NLRe, 
  $\emptyset\neq D\subseteq\VR V_n$ kompakt,
  $f:D\to\VR W_n$ stetig=>{
  \begin{stmts}
    \item $f(D)$ kompakt
    \item $\VR W_m=\SetR \implies (\exists x_1,x_2\in D)(f(x_1)\leq f(x)\leq f(x_2))$
    \end{stmts}
  }
% -----------------------------------------------------------------------------
\theorem:
  $(\VR V,\norm \cdot_1)$, $(\VR W,\norm._2)$ endlich dimensionale NLRe, 
  $\emptyset\neq D\subseteq\VR V_n$ kompakt,
  $f:D\to\VR W_n$ stetig=>{ 
  Dann ist $f$ gleichm"a"sig stetig auf $D$. 
  }
% -----------------------------------------------------------------------------
\section{Differenzierbarkeit}
% -----------------------------------------------------------------------------
\convention{
  Ab jetzt betrachten wir nur noch endlich dimensionale NLRe, o.B.d.A. 
  repr"asentiert durch $(\SetR^n,\norm.)$. $\norm.$ bezeichnet fast immer
  die Euklidnorm. $D\subseteq\SetR^n$ sei stets offen, $\fnorm A$ sei wie
  in Satz 6 definiert. $\norm x$ bezieht sich auf den Raum, aus dem $x$ 
  stammt.
  }
% -----------------------------------------------------------------------------
\definition Differenzierbarkeit:{
  Sei $x_0\in D$, $f:D\to\SetR^n$ eine Funktion. Dann hei"st $f$ in
  $x_0$ differenzierbar/db/total-differenzierbar/Fr{\'e}chet-differenzierbar 
  genau dann, wenn eine Matrix $A\in\SetR^{m\times n}$ existiert mit
  \[\frac{f(x_0+h)-f(x_0)-Ah}{\norm h}\to 0\quad (h\to 0)
    \]
  In diesem Fall hei"st $A$ \indexthis{Ableitung} von $f$ an der Stelle $x_0$.
  
  Schreibweise: $f'(x_0):=A$.
  }
% -----------------------------------------------------------------------------
\remark:{
  \begin{stmts}
    \item Diese Def. ist in $\SetR$ "aquivalent zur alten Definition.
    \item Falls sie existiert, ist die Ableitung eindeutig bestimmt.
    \item Ist $f:D\to\SetR^n$ in jedem Punkt $x\in D$ differenzierbar,
      so hei"st $f$ differenzierbar auf $D$. In diesem Fall ist durch 
      $x\mapsto f'(x)$ eine Abbildung $f':D\to\SetR^{m\times n}$ gegeben. 
      Ist $f'$ auch noch stetig auf $D$, so hei"st $f$ 
      \indexthis{stetig differenzierbar} auf $D$.
      \index{Differenzierbarkeit>stetige}
      
      Schreibweise: $f\in\SetCont^1(D,\SetR^n)$
    \end{stmts}
  }
% -----------------------------------------------------------------------------
\theorem:
  $f:D\to\SetR^m$ in $x_0\in D$ db=>{ 
  Dann ist $f$ stetig in $x_0$.
  }
% -----------------------------------------------------------------------------
\example:{
  Ist $f=Ax+b$, so ist $f'(x)=A$.
  }
% -----------------------------------------------------------------------------
\theorem Kettenregel:
  $D\subseteq \SetR^n$, $\LSet E\subseteq \SetR^m$, $g:D\to\SetR^m$, 
  $f:\LSet E\to\SetR^k$, $g(D)\subseteq\LSet E$, $g$ diffbar in $x_0$,
  $f$ diffbar in $g(x_0)$=>{
  Sei $F:=f\circ g$, $F:D\to\SetR^k$. Dann ist $F$ diffbar in $x_0$, 
  und es gilt:
  \[F'(x_0)=f'(g(x_0))\cdot g'(x_0)
    \]
  }
% -----------------------------------------------------------------------------
\remark:{
  Die Menge der differenzierbaren Funktionen ist ein Vektorraum, 
  das Differenzieren ist eine lineare Abbildung.
  }
% -----------------------------------------------------------------------------
\definition Partielle Ableitung:{
  \index{Ableitung>partielle}
  Seien $(\ntuple e)$ und $(u_1,\ldots,u_m)$ Standardbasen von $\SetR^n$ bzw.
  $\SetR^m$. $D\subseteq\SetR^n$ sei offen und $f:D\to\SetR^m$ mit
  den Komponentenfunktionen $f_k:D\to\SetR (k=1\ldots m)$, so dass
  \[f(x)=\sum_{k=1}^n f_k(x)\cdot u_k
    \]
  Dann definiert man f"ur $x_0=(\ntuple \xi)\in D$, $1\leq i\leq m$ und 
  $1\leq j\leq n$ definiert man
  \begin{align*}
    \frac{\partial f_i}{\partial \xi_j}(x_0):=&
      (f_i)_{\xi_j}(x_0):=
      (D_jf_i)(x_0):= \\
    & \limes t->0 \frac{f(x_0+t\cdot e_j)-f(x_0)}t
    \end{align*}
  Falls der Grenzwert existiert, hei"st er partielle Ableitung von $f_i$
  nach der $j$-ten Koordinate an der Stelle $x_0$.
  
  Existieren alle partiellen Ableitungen, so hei"st $f$ an der Stelle $x_0$
  partiell differenzierbar. Sind dann auch noch alle partiellen Ableitungen
  auf $D$ stetig, so hei"st $f$ auf $D$ stetig partiell 
  differenzierbar.
  }
% -----------------------------------------------------------------------------
\theorem:
  $f:D\to\SetR^m$, $f$ in $x_0\in D$ db=>{
  Dann ist $f$ in $x_0$ partiell db und es gilt f"ur $j=1,\ldots,n$
  \[f'(x_0)e_j=\sum_{i=1}^n D_jf_iu_i
    \]
  }
% -----------------------------------------------------------------------------
\definition Konvexit"at:{
  Sei $\VR V$ ein VR. Eine Menge $\LSet A\subseteq\VR V$ hei"st konvex genau
  dann, wenn f"ur alle $x,y\in\LSet A$ gilt:
  \[s(x,y):=\{tx+(1-t)y\mid t\in[0;1]\}\subseteq\LSet A
    \]
  }
% -----------------------------------------------------------------------------
\theorem Mittelwertsatz:
  $f:D\to\SetR$ db auf $D$, $a,b\in D,a\neq b,s(a,b)\subseteq D$=>{
  Dann existiert ein $\xi\in s(a,b)\setminus\{a,b\}$ mit 
  \[f(b)-f(a)=f'(\xi)\cdot(b-a)
    \]
  }
% -----------------------------------------------------------------------------
\remark:{
  Sind die Voraussetzungen des Mittelwertsatzes f"ur eine Funktion $f$ 
  erf"ullt, ist $D$ konvex und die Ableitung beschr"ankt, so ist $f$
  Lipschitz-stetig.
  }
% -----------------------------------------------------------------------------
\theorem:
  $f:D\to\SetR^m$=>{
  Dann gilt:
  \[f\in \SetCont^1(D,\SetR)
    \equiv
    \text{$f$ ist auf $D$ stetig partiell db.}
    \]
  }
% -----------------------------------------------------------------------------
\lesserdefinition Gradient:{
  Ist $f:D\to\SetR$ an der Stelle $x_0$ partiell db, so schreibt man auch
  \[\grad f(x_0):=((D_1f)(x_0),\ldots,(D_nf)(x_0))
    \]
  $\grad f(x_0)$ hei"st dann der Gradient von $f$ an der Stelle $x_0$. Ist
  $f$ in $x_0$ diffbar, so ist $f'(x_0)=grad f(x_0)$. Ist $f:D\to\SetR^m$
  in $x_0$ db, so ist
  \[f'(x_0)=\begin{pmatrix} \grad f_1(x_0) \\ \vdots \\ \grad f_n(x_0) \end{pmatrix}
    \]
  }
% -----------------------------------------------------------------------------
\definition Richtungsableitung:{
  Sei $f:D\to\SetR$ eine Funktion, $x_0\in D$ und $v\in\SetR^n$ mit
  $\norm v=1$. Dann ist
  \[\frac{\partial f}{\partial v}(x_0) := \limes t->0 \frac{f(x_0+tv)-f(x_0)}t
    \]
  falls dieser Grenzwert existiert. In diesem Fall hei"st der Grenzwert die
  Richtungsableitung von $f$ an der Stelle $x_0$ in Richtung $v$.
  }
% -----------------------------------------------------------------------------
\remark:{
  Ist $f:D\to\SetR$ in $x_0$ partiell db, so ist 
  \[(D_jf)(x_0)=\frac{\partial f}{\partial e_j}(x_0)
    \]
  also die Richtungsableitung von $f$ in Richtung $e_j$. $(j=1,\ldots,n)$
  }
% -----------------------------------------------------------------------------
\theorem:
  $f:D\to\SetR$ in $x_0$ db, $v\in\SetR^n$, $\norm v=1$=>{
  \begin{stmts}
    \item 
      \[\frac{\partial f}{\partial v}(x_0) = f'(x_0)\cdot v = \grad f(x_0)\cdot v 
        \]
    \item 
      \[\max \{ \frac{\partial f}{\partial w}(x_0) \mid 
                w\in\SetR^n, \norm w=1
             \} = \norm{\grad f(x_0)}
        \]
    \end{stmts}
  }
% -----------------------------------------------------------------------------
\section{Der Fixpunktsatz von Banach}
% -----------------------------------------------------------------------------
\lesserdefinition Kontraktion:{
  Sei $(\VR V,\norm\cdot)$ NLR, $\emptyset\neq A\subseteq\VR V$ eine Teilmenge.
  Dann hei"st eine Abbildung $g:A\to A$ mit $\norm{g(x)-g(y)}\leq q \norm{x-y}$ 
  f"ur ein $q\in[0;1)$ eine Kontraktion.
  
  (Hoffe ich mal. So steht es im Beweis zum Satz "uber die Umkehrfkt.)
  }
% -----------------------------------------------------------------------------
\theorem Fixpunktsatz von Banach:
  $(\VR V,\norm\cdot)$ vollst"andiger NLR, $\emptyset\neq A\subseteq\VR V$ 
  abgeschlossen, $g:A\to A$ Kontraktion=>{ 
  Dann existiert genau ein Fixpunkt $x\in A$ mit $g(x)=x$. 
  }
% -----------------------------------------------------------------------------
\section{Vorbereitungen zum Satz "uber die Umkehrfunktion}
% -----------------------------------------------------------------------------
\theorem:
  $\Omega:=\{A\mid A\in\SetR^{n\times n} \text{ regul"ar}\}$=>{
  \begin{stmts}
    \item Ist $A\in\Omega, B\in \SetR^{n\times n}$ und gilt 
      \[\fnorm{A-B}<\frac 1 {\fnorm{A^{-1}}}
        \] 
      so ist $B$ regul"ar.
    \item $\Omega$ ist offene Teilmenge von $\SetR^{n\times n}$ und 
      $A\mapsto A^{-1}$ ist eine stetige Abbildung in $\Omega$.
    \end{stmts}
  }
% -----------------------------------------------------------------------------
\theorem:
  $D\subseteq\SetR^n$ offen und konvex, $f\in\SetCont^1(D,\SetR^n)$,
  $\fnorm{f'(x)}<L$ auf $D$=>{
  Dann gilt f"ur $x,y\in D$
  \[\norm{f(x)-f(y)}\leq L\norm{x-y}
    \]
  }
% -----------------------------------------------------------------------------
\section{Satz "uber die Umkehrfunktion}
% -----------------------------------------------------------------------------
\theorem Satz "uber die Umkehrfunktion:
  $D\subseteq\SetR^n$ offen, $f\in\SetCont^1(D,\SetR^n)$, $x_0\in D$
  und $f'(x_0)$ regul"ar=>{
  \begin{stmts}
    \item Es existiert eine offene Umgebung $U\subseteq D$ von $x_0$ und 
      eine offene Umgebung $\tilde U$ von $y_0:=f(x_0)$ derart, dass
      $f$ auf $U$ injektiv ist und $f(U)=\tilde U$ gilt.
    \item Ist $f^{-1}:\tilde U\to U$ die Inverse von $f:U\to\tilde U$ 
      (wie in (1)), so ist $f^{-1}\in\SetCont^1(\tilde U;\SetR^n)$ und es gilt
      \[(f^{-1})'(x)=(f'\circ f^{-1}(x))^{-1}\quad (x\in\tilde U)
        \]
    \end{stmts}
  }
% -----------------------------------------------------------------------------
\theorem:
  $D\subseteq\SetR^n$, $f\in\SetCont^1(D,\SetR^n)$, $f'(x)$ regul"ar 
  f"ur alle $x\in D$=>{
  Dann ist $f(E)$ offen f"ur jede offene Teilmenge $E\subseteq D$.
  }
% -----------------------------------------------------------------------------
\remark:{
  Unter den obigen Voraussetzungen ist $f$ auf einer Umgebung jedes
  beliebigen Punktes $x\in D$ injektiv. $f$ hei"st dann lokal-injektiv.
  }
% -----------------------------------------------------------------------------
\section{Der Satz "uber implizit definierte Funktionen}
% -----------------------------------------------------------------------------
\convention{
  Ist $x=(\xi_1,\ldots,\xi_n)\in\SetR^n, y=(\eta_1,\ldots,\eta_m)\in\SetR^m$,
  so schreiben wir einfach $(x,y)$ f"ur $(\xi_1,\ldots,\xi_n,\eta_1,\ldots,\eta_m)$.
  
  Jedes $A\in\SetR^{n\times(n+m)}$ l"asst sich in zwei lineare Abbildungen
  $A_x$ und $A_y$ zerlegen, so dass
  \begin{align*}
    A_x x &:= A\begin{pmatrix} x \\ 0 \end{pmatrix} \\
    A_y y &:= A\begin{pmatrix} 0 \\ y \end{pmatrix}
    \end{align*}
  Dann gilt $A\begin{pmatrix} x \\ y \end{pmatrix} = A_xx+A_yy$.
  }
% -----------------------------------------------------------------------------
\theorem:
  $A\in\SetR^{n\times(n+m)}$, $A_x$ invertierbar=>{
  Zu jedem $y\in\SetR^m$ existiert genau ein $x\in \SetR^n$ mit 
  $A\begin{pmatrix} x \\ y \end{pmatrix} = 0$, n"amlich
  \[x=-(A_x)^{-1} A_y y
    \]
  }
% -----------------------------------------------------------------------------
\theorem Satz "uber implizit definierte Funktionen:
  $D\subseteq\SetR^{n+m}$ offen, $f\in\SetCont^1(D,\SetR^n)$, 
  $(x_0,y_0)\in D$, $f(x_0,y_0)=0$ und 
  f"ur $A=f'(x_0,y_0)$ sei $A_x$ regul"ar=>{ 
  \index{implizit definierte Funktionen>Satz "uber}
  \index{Funktionen>Satz "uber implizit definierte}
  Dann gibt es offene Mengen $U\subseteq D$ und $\tilde U\subseteq\SetR^m$
  mit $(x_0;y_0)\in U$ und $y\in\tilde U$ mit folgenden Eigenschaften:
  \begin{stmts}
    \item Zu jedem $y\in \tilde U$ ex. genau ein $x\in\SetR^n$ mit 
      $(x,y)\in U$ und $f(x,y)=0$.
    \item Definiert man $g:\tilde U\to\SetR^n$ durch $g(y):=$das eindeutig
      bestimmte $x$ aus (1), so gilt $g(y_0)=x_0$, $f(g(y_0),y_0)=0 (y\in\tilde U)$
      und $g\in\SetCont^1(\tilde U,\SetR^n)$.
      
      Ferner gilt: $g'(y_0) = -(A_x)^{-1} A_y$ bzw. auch allgemeiner
      \[g'(y)=-f'(g(y),y)_x^{-1} f'(g(y),y)_y \quad(y\in \tilde U)
        \]
    \end{stmts}
  }
% -----------------------------------------------------------------------------
\trick Merkregel zum Satz:{
  Sei $h:\SetR^m\to\SetR^n$ mit $h(y):=f(g(y),y)$. Sei weiter $y\in\tilde U$, also
  \begin{align*}
    \frac{dh}{dy}&=0
      \underset{h\in\SetCont^1}=\frac{\partial h}{\partial y} \\
      &=\frac{\partial f}{\partial g{,}y} \frac{\partial g{,}y}{\partial y} \\
      &=\begin{pmatrix}
          \frac{\partial f}{\partial g} &
          \frac{\partial f}{\partial y}
        \end{pmatrix}
        \cdot
        \begin{pmatrix}
          \frac{\partial g}{\partial y}\\
          I_m \\
        \end{pmatrix} \\
      &=\frac{\partial f}{\partial g}\cdot\frac{\partial g}{\partial y}+
        \frac{\partial f}{\partial y}=0
    \intertext{Deswegen gilt:}
    \frac{\partial g}{\partial y}
      &=-\left(\frac{\partial f}{\partial g}\right)^{-1}
        \frac{\partial f}{\partial y}
    \end{align*}
  }
% -----------------------------------------------------------------------------
\section{Ableitungen h"oherer Ordnung}
% -----------------------------------------------------------------------------
\convention{
  Sei $D\subseteq\SetR^n$ offen, $f:D\to\SetR$ Funktion.
  }
% -----------------------------------------------------------------------------
\definition Mehrfache partielle Differentiation:{
  \index{Differentiation>mehrfache}
  Sei $f$ partiell db auf $D$ mit den partiellen Ableitungen
  $D_1f,D_2f,\ldots,D_nf$. Sind $D_1f,D_2f,\ldots,D_nf:D\to\SetR$
  partiell db, so sind die partiellen Ableitungen zweiter Ordnung von $f$
  definiert durch
  \[D_{ij}f:=f_{\xi_j\xi_i}:=\frac{\partial^2 f}{\partial \xi_i \xi_j}:=
    D_i(D_jf)
    \]
  Faustregel: Immer vom $f$ aus lesen ergibt die Reihenfolge von innen nach au"sen.
  }
% -----------------------------------------------------------------------------
\definition Mehrfache stetige Differenzierbarkeit:{
  \index{Differenzierbarkeit>stetige>mehrfache}
  $f$ hei"st auf $D$ $k$-mal stetig db $:\equiv$
  Jede $k$-fache partielle Ableitungen von $f$ nach einer beliebigen
  Kombination von Koordinaten existiert und ist stetig.
  
  Schreibweise: $f\in\SetCont^k(D,\SetR)$
  
  Analog definiert man:
  \begin{align*}
    \SetCont^k(D,\SetR^m)&:=\{f=(f_1,\ldots,f_m) \mid 
      (\forall 1\leq j\leq m)(f_j\in\SetCont^k(D,\SetR^m))\} \\
    \SetCont^0(D,\SetR^m)&:=\SetCont(D,\SetR^m) \\
    \SetCont^\infty(D,\SetR^m)&:=\bigcap_{k\geq 0} \SetCont^k(D,\SetR^m)
    \end{align*}
  }
% -----------------------------------------------------------------------------
\theorem Satz von Schwarz:
  $D\subseteq\SetR^n$ offen, $f:D\in\SetCont^2(D,\SetR)$=>{
  Dann gilt f"ur alle $i,j\in\{1,\ldots,n\}$, dass $D_{ij}f=D_{ji}f$.
  }
% -----------------------------------------------------------------------------
\remark:{
  Ist $f\in\SetCont^k(D,\SetR)$, so folgt aus dem Satz von Schwarz, dass
  bei $\leq k$-facher Differentiation die Reihenfolge keine Rolle spielt.
  }
% -----------------------------------------------------------------------------

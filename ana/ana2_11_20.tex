% -*- LaTeX -*-
% -----------------------------------------------------------------------------
\section{Der Satz von Taylor}
% -----------------------------------------------------------------------------
\lesserdefinition Multi-Index:{
  Sei $p=(p_1,\ldots,p_n)\in\SetN_0^n$. Dann nennt man $p$ auch einen
  Multiindex und definiert
  \begin{align*}
    |p|&:=p_1+\ldots+p_n \\
    p!&:=p_1!\cdot\ldots\cdot p_n! \\
  \intertext{und f"ur $x=(\ntuple\xi)$}
    x^p&:=\xi_1^{p_1}\cdot\xi_2^{p_2}\cdots\xi_n^{p_n}
    \end{align*}
  
  Sei $D\subseteq\SetR^n$, $f\in\SetCont^{|p|}(D,\SetR)$. Dann definiert man
  \[D^pf:=D^{p_1}_1 \ldots D^{p_n}_n f:=
      \frac{\partial^{|p|} f}{\partial x_1^{p_1} \ldots x_n^{p_n}}
    \]
}
% -----------------------------------------------------------------------------
\theorem:
  $D\subseteq\SetR^n$ offen, $k\in\SetN$, $f\in\SetCont^k(D,\SetR)$,
  $x\in D$, $h\in\SetR^n$ mit $s(x,x+h)\subseteq D$,
  $\Phi:[0,1]\to\SetR$, $\Phi(t)=f(x+th)$=>{
  Dann ist $\Phi\in\SetCont^k([0,1],\SetR)$ und es gilt
  \[\Phi^{(\nu)}(t)= \sum_{|p|=\nu} \frac{\nu!}{p!} (D^pf)(x+th)h^p
    \]
  }
% -----------------------------------------------------------------------------
\theorem Satz von Taylor:
  $D\subseteq\SetR^n$ offen, $k\in\SetN$, $f\in\SetCont^{k+1}(D,\SetR)$,
  $x\in D$, $h\in\SetR^n$ mit $s(x,x+h)\subseteq D$=>{
  \index{Taylor>Satz von}
  Dann existiert ein $\Theta\in[0,1]$ mit
  \[f(x+h)=
      \underbrace{
        \sum_{|p|\leq k} \frac{(D^pf)(x)}{p!}h^p
      }_{
        k\text{-tes Taylorpolynom}
      }
      +
      \underbrace{
        \sum_{|p|=k+1} \frac{(D^pf)(x+\Theta h)}{p!}h^p
      }_{
        \text{Restglied}
      }
    \]
  }
% -----------------------------------------------------------------------------
\definition Hesse-Matrix:{
  Sei $D\subseteq\SetR^n$ offen, $f\in\SetCont^2(D,\SetR)$ und 
  $x\in D$. Dann hei"st
  \[H_f(x):=\begin{pmatrix}
      \grad f_{\xi_1}(x) \\
      \vdots \\
      \grad f_{\xi_n}(x) \\
      \end{pmatrix}
    \]
  die Hesse-Matrix von $f$ an der Stelle $x$.
  
  F"ur $k=1$, $f\in\SetCont^2(D,\SetR)$ lautet der Satz von Taylor also:
  Zu $x\in D$, $h\in\SetR^n$ mit $s(x,x+h)\subseteq D$ existiert
  ein $\Theta\in[0,1]$ mit
  \[f(x+h)=f(x)+(\grad f(x))\cdot h + \frac 1 2 h (H_f(x+\Theta h) h)
    \]
  
  Wegen $f\in\SetCont^2(D,\SetR)$ ist nach dem Satz von Schwarz
  $H_f(x)=H_f^T(x)$.
  }
% -----------------------------------------------------------------------------
\section{Extremwerte}
% -----------------------------------------------------------------------------
\definition Extremum:{
  Sei $M\subseteq\SetR^n$. Dann hat $f:M\to\SetR$ in $x_0\in M$ ein lokales
  \tstack(Maximum;Minimum) $:\equiv$
  \[(\exists\delta>0)(\forall x\in U_\delta(x_0))(f(x)\stack(\leq;\geq)f(x_0))
    \]
  $x_0$ hei"st lokales Extremum genau dann, wenn $x_0$ lokales Minimum oder
  Maximum ist.
  \index{Extremum>lokales}
  \index{Minimum>lokales}
  \index{Maximum>lokales}
}
% -----------------------------------------------------------------------------
\theorem:
  $D\subseteq\SetR^n$ offen, $f:D\to\SetR$, $x_0$ lok. Extremum von
  $f$, $f$ in $x_0$ partiell db=>{
  Dann gilt
  \[\grad f(x_0)=0
    \]
  }
% -----------------------------------------------------------------------------
\remark:{
  Obiger Satz gibt nur ein \textit{notwendiges}, kein \textit{hinreichendes}
  Kriterium f"ur ein Extremum an.
  }
% -----------------------------------------------------------------------------
\definition Quadratische Form:{
  Sei $A=((a_{ij}))\in\SetR^{n\times n}$ symmetrisch.
  Dann hei"st die Abbildung $Q_A:\SetR^n\to\SetR$ definiert durch
  \[Q_A(x):=x^T(Ax)
    \]
  die zu $A$ geh"orende quadratische Form. \index{Form>quadratische}
  Sie hei"st
  
  \begin{tabular}{l@{$:\equiv$}l}
    positiv definit (pd) & $(\forall x\in\SetR^n\setminus\{0\})(Q_A(x)>0)$ \\
    negativ definit (nd) & $(\forall x\in\SetR^n\setminus\{0\})(Q_A(x)<0)$ \\
    positiv semidefinit (psd) & $(\forall x\in\SetR^n\setminus\{0\})(Q_A(x)\geq 0)$ \\
    negativ semidefinit (nsd) & $(\forall x\in\SetR^n\setminus\{0\})(Q_A(x)\leq 0)$ \\
    indefinit (id) & $(\exists u,v\in\SetR^n\setminus\{0\})(Q_A(u)\cdot Q_A(v)<0)$ \\
    \end{tabular}
  \index{Definitheit>positive}
  \index{Definitheit>negative}
  \index{Definitheit>positive Semi-}
  \index{Definitheit>negative Semi-}
  \index{Semidefinitheit>positive}
  \index{Semidefinitheit>negative}
  \index{Indefinitheit}
  }
% -----------------------------------------------------------------------------
\remark:{
  \begin{stmts}
    \item Sei $x=(\ntuple \xi)\in\SetR^n$. Dann gilt:
      \[Q_A(x)=\sum_{i,j=1}^n \xi_j a_{ij} \xi_i
        \]
    \item Ist $n=2$, so gilt:
      \begin{align*}
        A \text{ pd} & \equiv a_{11}>0 \land \det A>0 \equiv \text{ Alle Eigenwerte von $A$ $>0$ } \\
        A \text{ nd} & \equiv a_{11}<0 \land \det A>0 \equiv \text{ Alle Eigenwerte von $A$ $<0$ } \\
        A \text{ id} & \equiv \det A<0 \equiv \text{ Es ex. EW $>0$ und $<0$ }
        \end{align*}
    \item $A$ ist pd $\equiv$ $-A$ ist nd.
    \item $x\in\SetR^n,\alpha\in\SetR$ $\implies$ $Q_A(\alpha x)=\alpha^2Q_A(x)$
    \end{stmts}
  }
% -----------------------------------------------------------------------------
\theorem:
  $A\in\SetR^{n\times n}$ symmetrisch=>{
  \begin{stmts}
    \item $A$ ist \tstack(positiv;negativ) definit $\equiv$
      \[(\exists c>0)(\forall x\in\SetR^n)(Q_A(x)\stack(\geq;\leq -) c\cdot \norm x^2)
        \]
    \item Sei $A$ pd (nd,id). Dann existiert ein $\epsilon>0$, so dass
      \[(\forall B\in\SetR^{n\times n})(\fnorm{A-B}\leq\epsilon \implies 
        B \text{ pd (nd,id)})
        \]
      
      Zus"atzlich: Ist $A$ id, so existieren $u,v\in\SetR^n$ mit 
      $Q_B(u)>0$, $Q_B(v)<0$ f"ur jede symmetrische Matrix $B$ mit
      $\fnorm{A-B}\leq\epsilon$.
    \end{stmts}
  }
% -----------------------------------------------------------------------------
\theorem:
  $D\subseteq\SetR^n$ offen, $f\in\SetCont^2(D,\SetR)$, $x_0\in D$ und
  $\grad f(x_0)=0$=>{
  \begin{stmts}
    \item $H_f(x_0)$ ist pd $\implies$ $f$ hat in $x_0$ ein lokales Minimum
    \item $H_f(x_0)$ ist nd $\implies$ $f$ hat in $x_0$ ein lokales Maximum
    \item $H_f(x_0)$ ist id $\implies$ $f$ hat in $x_0$ kein lokales Extremum
    \end{stmts}
  }
% -----------------------------------------------------------------------------
\remark:{
  \begin{itemize}
    \item In $\SetR$ tritt Fall (3) nicht ein.
    \item Ist $H_f(x_0)$ positiv/negativ semidefinit, so folgt \textit{nichts}
      "uber die Existenz von Extremstellen.
    \end{itemize}
  }
% -----------------------------------------------------------------------------
\section{Extremstellen unter Nebenbedingungen}
% -----------------------------------------------------------------------------
\convention{
  Es sei stets $D\subseteq\SetR^n$ offen, $f:D\to\SetR$ eine Funktion,
  $m,n\in\SetN$ mit $m<n$, $\phi:D\to\SetR^m$ eine Funktion und
  $T:=\{ x\in D\mid \phi(x)=0 \}$
  }
% -----------------------------------------------------------------------------
\definition Extremum unter Nebenbedingungen:{
  \index{Nebenbedingungen>Extremum unter}
  $f$ hat in $x_0\in D$ ein lokales \tstack(Maximum;Minimum) unter der
  Nebenbedingung (Nb) $\phi(x_0)=0$ genau dann, wenn gilt
  \[x_0\in T \land (\exists \delta>0)(U_\delta(x_0)\subseteq D) \land
    (\forall x \in U_\delta(x_0)\cap T)(f(x)\stack(\leq;\geq)f(x_0))
    \]
  }
% -----------------------------------------------------------------------------
\convention{  
  Man definiert sich zus"atzlich eine Hilfsfunktion 
  $H:D\times\SetR^m\to\SetR$
  \begin{align*}
    H(x,\lambda):= & f(x)+\lambda \phi(x) \\
    = & f(x) + \sum_{k=1}^m \lambda_k \phi_k(x)
    \end{align*}
  }
% -----------------------------------------------------------------------------
\remark:{
  $D\times\SetR^m\subseteq\SetR^{n+m}$ ist offen.
  Falls existent, gilt:
  \[H'(x,\lambda)=\grad H(x,\lambda)=
    \big(f'(x)+\lambda\phi'(x)\quad \phi(x)\big)\in\SetR^{1\times n+m}
    \]
  D.h. f"ur $x_0\in D$ mit $x_0=(\ntuple \xi)$ und 
  $\lambda_0\in\SetR^m$ mit $\lambda_0=(\lambda^{(0)}_1,\ldots,\lambda^{(0)}_m)$ gilt:
  \[H'(x_0,\lambda_0)=0 \equiv f'(x_0)+\lambda\phi'(x_0)=0 \land x_0\in T
    \]
  genau dann, wenn 
  \begin{align*}
    H_{\xi_1}(x_0,\lambda_0)&=
      f_{\xi_1}(x_0)+
      \lambda^{(0)}_1\frac{\partial}{\partial \xi_1} \phi_1(x_0)+\ldots+
      \lambda^{(0)}_m\frac{\partial}{\partial \xi_1} \phi_m(x_0)
      =0 \\
    & \vdots \\
    H_{\xi_n}(x_0,\lambda_0)&=
      f_{\xi_n}(x_0)+
      \lambda^{(0)}_1\frac{\partial}{\partial \xi_m} \phi_1(x_0)+\ldots+
      \lambda^{(0)}_m\frac{\partial}{\partial \xi_m} \phi_m(x_0)
      =0 \\
    H_{\lambda_1}(x_0,\lambda_0)&=
      \phi_1(x_0)=0 \\
    & \vdots \\
    H_{\lambda_m}(x_0,\lambda_0)&=
      \phi_m(x_0)=0
    \end{align*}
  }
% -----------------------------------------------------------------------------
\theorem Multiplikatorenregel von Lagrange:
  $f\in\SetCont^1(D,\SetR)$, $\phi\in\SetCont^1(D,\SetR^m)$, 
  $x_0\in D$ lokales Extremum von $f$ mit Nb $\phi(x_0)=0$ und
  $\rank \phi'(x_0)=m$=>{
  \index{Lagrange>Multiplikatorenregel von}
  Dann existiert ein $\lambda_0\in\SetR^m$ mit $H'(x_0,\lambda_0)=0$ 
  (s. auch obiges Gleichungssystem)
  }
% -----------------------------------------------------------------------------
\annotation Fragen, die man sich zum Verfahren stellen kann:{
  Warum bringen einen die lokalen Extrema von $f$ meist nicht weiter?

  Dazu stelle man sich den Graphen der Funktion $f(x,y)=x^2+y^2$ als
  Berglandschaft vor (das ist ein Tal bei $(0,0)$, mit nach jeder
  Seite parabolischem Anstieg). $f$ hat genau ein (lokales und globales)
  Extremum im Nullpunkt. Man suche aber Extrema z.B. unter der Bedinung
  $|x|=1 \lor |y|=1$, d.h. z.B. $\phi(x,y)=(|x|-1)\cdot(|y|-1)$. Dann hat
  die Funktion zwar Extrema unter Nebenbedingung, diese fallen aber
  nicht mit dem Extremum von $f$ zusammen!

  Wie funktioniert das Verfahren?

  $H$ bestimmen, Ableitung nullsetzen f"uhrt auf eine Menge von
  Kandidatenstellen. Man sichert dann die Existenz des gesuchten
  Extremums (Z.B. mit Hilfe des weiter unten stehenden
  Tricks) und sucht unter den (hoffentlich wenigen) verbleibenden
  Kandidatenstellen das globale Minimum unter Nebenbedingung.

  Was passiert mit $H$, wenn die Nebenbedingung nicht erf"ullt ist?

  Dazu nehme man an, man habe ein (lokales) Extremum von $f$ in $x_1$
  aufgesp"urt, das aber die Nebenbedingung nicht erf"ullt.  Dann
  existiert mindestens ein $j$, so dass man mittels $\lambda_j$ die
  Hilfsfunktion $H$ in einer beliebig kleinen Umegubng in beliebige
  Richtung ver"andern kann. Dies bedeutet: $H$ hat keine lokalen
  Minima dort, wo die Nebenbedingung nicht erf"ullt ist.  (Dies sieht
  man auch leicht an der Ableitung $H'$: Notwendige Bedingung f"ur ein
  lokales Extremum ist $\phi(x)=0$.)

  Was passiert mit $H$, wenn die Nebenbedingung erf"ullt ist?

  In diesem Moment stimmt $H$ vollst"andig mit $f$ "uberein, daran
  "andert $\lambda$ nichts. Wenn es einem also gelingt, $H$ lokal zu minimieren,
  dann erh"alt man eine Stelle, die (s.o.!) die Nebenbedingung erf"ullt
  und auch $f$ (lokal) minimiert. Bleibt die entscheidende Frage:

  Erwischt man so auch garantiert alle solchen Stellen, d.h. gilt f"ur jedes
  lokale Extremum von $f$  unter Nebenbedingung auch $H'=0$?

  Ja, in gewissen Grenzen. Das ist genau die Aussage des obigen
  Satzes.  "Aquivalent formuliert: Wird die Ableitung von $H$ auch
  einmal Null, wenn $f$ ein Extremum unter Nebenbedingung hat? 
  D.h. findet man dann ein passendes $\lambda$, so dass auch $H'=0$ wird?
  $\rank \phi'=m$ stellt sicher, dass man an dieser Stelle $\lambda_0$
  geeignet w"ahlen kann, damit $-\grad f=(\phi'\lambda_0)^T\iff H'=0$ gilt (Theorie
  der LGSe!)
  }
% -----------------------------------------------------------------------------
\trick Beweis der Existenz eines Extremums:{
  Sei $T:=\{x\in D\mid \phi(x)=0\}$. Dann definiert man sich eine 
  kompakte Menge $K:=T\cap\{x\mid x_1\leq r,\ldots,x_n\leq r\}$ und zeigt,
  dass auf $T\setminus K$ alle Funktionswerte gr"o"ser/kleiner als die am
  betrachteten kritischen Punkt sind. ($r$ durch Ausprobieren w"ahlen)
  Dann existiert ein Minimum/Maximum von $f$ auf $K$, und es gilt
  \[\min f(T)=\min f(K)
    \]
  so dass die Existenz eines Extremums gesichert ist. (es sollte halt nur kein 
  Randextremum von $K$ sein)
  }
% -----------------------------------------------------------------------------
\section{Wege}
% -----------------------------------------------------------------------------
\definition Weg:{
  Eine stetige Abbildung $\gamma:[a,b]\to\SetR^n$ ($[a,b]\subseteq\SetR$)
  hei"st ein Weg.
  
  $\gamma(a)$ hei"st Anfangspunkt von $\gamma$.
  
  $\gamma(b)$ hei"st Endpunkt von $\gamma$.
  
  $[a,b]$ hei"st Parameterintervall von $\gamma$.
  
  $\gamma$ ist ``orientiert'', d.h. $\gamma(t_1)$ wird ``vor'' $\gamma(t_2)$
  durchlaufen $\equiv$ $t_1<t_2$.
  
  $\gamma^-(t):=\gamma(a+b-t)$ mit $t\in[a,b]$ hei"st zu $\gamma$ inverser Weg.
  \index{Weg>inverser}
  }
% -----------------------------------------------------------------------------
\lesserdefinition Bogen:{
  Sei $\gamma:[a,b]\to\SetR^n$ ein Weg. Dann hei"st 
  \[\Gamma:=\Gamma_\gamma:=\gamma([a,b])
    \]
  der zum Weg $\gamma$ geh"orende Bogen.
  }
% -----------------------------------------------------------------------------
\remark:{
  $\Gamma$ ist kompakt, da $[a,b]$ kompakt und $\gamma$ stetig.
  }
% -----------------------------------------------------------------------------
\definition L"ange:{
  \index{Weg>rektifizierbarer}
  Sei $\gamma:[a,b]\to\SetR^n$ ein Weg und $z=\{t_0,\ldots,t_m\}$ eine 
  Zerlegung von $[a,b]$.
  \[L(\gamma,z):=\sum_{k=1}^m \norm{\gamma(t_k)-\gamma(t_{k-1})}
    \]

  $\gamma$ hei"st \indexthis{rektifizierbar}/rb $:\equiv$
  \[(\exists M\geq 0)(\forall z \text{ Zerlegung von $[a,b]$})(L(\gamma,z)\leq M)
    \]
  
  In diesem Fall hei"st 
  \[L(\gamma):=\sup\{L(\gamma,z)\mid z\text{ Zerlegung von $[a,b]$}\}
    \]
  die L"ange des Weges $\gamma$.
  }
% -----------------------------------------------------------------------------
\theorem:
  $\gamma:[a,b]\to\SetR^n$ Weg, $\gamma=(\ntuple \gamma)$=>{
  Dann gilt $\gamma$ rektifizierbar $\equiv$
  \[(\forall j\in\{1,\ldots,n\})(\gamma_j\in\SetBV([a,b],\SetR))
    \]
  }
% -----------------------------------------------------------------------------
\remark:{
  Ist der Weg $\gamma:[a,b]\to\SetR^n$ rb, so ist $\gamma|_{[c,d]}$ ebenfalls rb.
  ($[c,d]\subseteq [a,b]$)
  }
% -----------------------------------------------------------------------------
\remark:{
  Ist der Weg $\gamma:[a,b]\to\SetR^n$ rb, so ist $\gamma^-$ auch rb und es
  gilt
  \[L(\gamma)=L(\gamma^-)
    \]
  }
% -----------------------------------------------------------------------------
\definition Wegl"angenfunktion:{
  Sei $\gamma:[a,b]\to\SetR^n$ ein rektifizierbarer Weg. Dann hei"st
  \[s(t):=\begin{cases}
      L(\gamma|_{[a,t]}) & t\in (a,b] \\
      0 & t=a
      \end{cases}
    \]
  Wegl"angenfunktion von $\gamma$.
  }
% -----------------------------------------------------------------------------
\remark:{
  Es gilt $s(b)=L(\gamma)$.
  
  Analog wie bei Funktionen von beschr"ankter Variation zeigt man, dass $s$
  auf $[a,b]$ monoton w"achst und es gilt
  \[s(t_2)-s(t_1)=L(\gamma|_{[t_1,t_2]})
    \]
  }
% -----------------------------------------------------------------------------
\definition Integral "uber einen Weg:{
  Sei $\gamma:[a,b]\to\SetR^n$ ein Weg, $\gamma=(\ntuple \gamma)$. 
  Dann definiert man
  \[\int_a^b \gamma(t) dt:=
      \begin{pmatrix}
        \int_a^b \gamma_1(t) dt & \ldots & \int_a^b \gamma_n(t) dt
        \end{pmatrix}
    \]
  }
% -----------------------------------------------------------------------------
\theorem:
  $\gamma:[a,b]\to\SetR^n$ Weg=>{
  Dann gilt
  \[\norm{\int_a^b \gamma(t) dt} \leq \int_a^b \norm{\gamma(t)} dt
    \]
  }
% -----------------------------------------------------------------------------
\theorem:
  $\gamma\in\SetCont^1([a,b],\SetR^n)$, $\gamma=(\ntuple \gamma)$=>{
  \begin{stmts}
    \item $\gamma$ ist rb.
    \item $s\in\SetCont^1([a,b],\SetR)$ und $s'(t)=\norm{\gamma'(t)}$.
    \item $L(\gamma)=\int_a^b \norm{\gamma'(t)} dt $, also
      $s(t)=\int_a^t \norm{\gamma'(\tau)} d\tau$
    \end{stmts}
  }
% -----------------------------------------------------------------------------
\definition St"uckweise stetige Differenzierbarkeit:{
  \index{Differenzierbarkeit>stetige>st"uckweise}
  Ein Weg $\gamma:[a,b]\to\SetR^n$ hei"st st"uckweise stetig db $:\equiv$
  \begin{multline*}
    (\exists z=\{t_0,\ldots,t_m\} \text{ Zerl. von $[a,b]$})
    (\forall k\in\{1,\ldots,m\}) \\
    (\gamma|_{[t_{k-1},t_k]}\in\SetCont^1([t_{k-1},t_k],\SetR))
  \end{multline*}
  Man definiert au"serdem dann  zu gegebener Zerlegung $z=\{t_0,\ldots,t_m\}$
  \[\gamma_k:=\gamma|_{[t_{k-1},t_k]}
    \] 
  }
% -----------------------------------------------------------------------------
\remark:{
  St"uckweise stetig differenzierbare Wege sind rb und es gilt
  \[L(\gamma)=\sum_{k=1}^m L(\gamma_k)=
    \sum_{k=1}^m \int_{t_{k-1}}^{t_k} \norm{\gamma'(t)} dt
    \]
  }
% -----------------------------------------------------------------------------
\definition Parametertransformation:{
  Seien $\gamma_1:[a,b]\to\SetR^n$, $\gamma_2:[\alpha,\beta]\to\SetR^n$ Wege.
  Dann ist 
  \[\gamma_1 \sim \gamma_2 :\equiv
    (\exists h:[a,b]\to[\alpha,\beta] \text{ stetig, bijektiv, wachsend})
    (\gamma_1=\gamma_2\circ h)
    \]
  }
% -----------------------------------------------------------------------------
\remark:{
  \begin{stmts}
    \item Die Eigenschaften von $h$ implizieren, dass $h$ streng monoton ist
      und $h(a)=\alpha$, $h(b)=\beta$.
    \item Ist $\gamma_1\sim\gamma_2$, so ist $\Gamma_{\gamma_1}=\Gamma_{\gamma_2}$.
    \item ``$\sim$'' ist "Aquivalenzrelation.
    \item $\gamma_1$, $\gamma_2$ haben gleiche Orientierung, da $h$ wachsend.
    \end{stmts}
  }
% -----------------------------------------------------------------------------
\definition Glattheit:{
  Ein Weg $\gamma:[a,b]\to\SetR^n$ hei"st glatt $:\equiv$ 
  $\gamma\in\SetCont^1([a,b],\SetR^n)$ und $\gamma'(t)\neq 0$. ($t\in[a,b]$)
  }
% -----------------------------------------------------------------------------
\theorem:
  $\gamma_1:[a,b]\to\SetR^n$, $\gamma_2:[\alpha,\beta]\to\SetR^n$ Wege,
  $\gamma_1\sim\gamma_2$ und $h$ Parametertransformation so, dass
  $\gamma_1=\gamma_2\circ h$=>{
  Dann gilt:
  \begin{stmts}
    \item $\gamma_1$ rb $\equiv$ $\gamma_2$ rb. 
      In diesem Fall $L(\gamma_1)=L(\gamma_2)$.
    \item $\gamma_1,\gamma_2$ glatt $\implies$ $h\in\SetCont^1(\SetR,\SetR)$ 
      und $h'(t)>0$ ($t\in[a,b]$)
    \end{stmts}
  }
% -----------------------------------------------------------------------------
\theorem:
  $\gamma:[a:b]\to\SetR$ rb Weg, $s:[a,b]\to\SetR$ Wegl"angenfunktion=>{
  $s$ ist wachsend und stetig.
  }
% -----------------------------------------------------------------------------
\remark Wegl"ange als Parameter:{
  Sei $\gamma:[a,b]\to\SetR^n$ rektifizierbarer Weg und $s:[a,b]\to\SetR$
  die zugeh"orige Wegl"angenfunktion. Dann gilt: 
  \begin{stmts}
    \item $s\nearrow$, $s\in\SetCont([a,b]\to\SetR)$
    \item $s(a)=0$, $s(b)=L(\gamma)$, $s([a,b])=[0,L(\gamma)]$
    \end{stmts}
  Sei nun $s$ streng wachsend auf $[a,b]$. (z.B. wenn $\gamma$ auf $[a,b]$ 
  injektiv oder glatt ist) Dann ist $s$ bijektiv und man definiert
  $\sigma:=s^{-1}$, $\sigma:[0,L(\gamma)]\to[a,b]$. Dann gilt:
  \begin{stmts}
    \item $\sigma\nearrow$, $\sigma\in\SetCont([0,L(\gamma)],[a,b])$
    \item $\sigma(0)=a$, $\sigma(L(\gamma))=b$, $\sigma([0,L(\gamma)])=[a,b]$
    \end{stmts}
  Setze nun $\tilde\gamma:=\gamma\circ\sigma$ und es gilt 
  $\gamma\sim\tilde\gamma$. $\tilde\gamma$ hei"st Parameterdarstellung von
  $\Gamma:=\Gamma_\gamma$. Sei $\phi:[0,L(\gamma)]\to\SetR$ die
  Wegl"angenfunktion von $\tilde\gamma$. Sei $\tau\in[0,L(\gamma)]$.
  Dann existiert genau ein $t\in[a,b]$ mit $s(t)=\tau$ und es gilt
  $\sigma([0,\tau])=[a,t]$. Also auch
  \[\phi(\tau)=L(\tilde\gamma|_{[0,\tau]})=L(\gamma|_{[a,t]})=s(t)=\tau
    \]
  f"ur alle $t\in[0,L(\gamma)$.
  }
% -----------------------------------------------------------------------------
\theorem:
  $\gamma:[a:b]\to\SetR$ glatter Weg, 
  $\tilde\gamma$ Parameterdarstellung von $\Gamma_\gamma$=>{
  $\tilde\gamma$ ist glatt und $\norm{\tilde\gamma'(t)}=1$. ($t\in[0,L(\gamma)$)
  }
% -----------------------------------------------------------------------------
\remark:{
  Es gilt: $\gamma'(t)=h'(t)\cdot\tilde\gamma'(h(t))>0$.
  
  Im Falle $n=2$ hei"st $u:=(-\gamma_2(t),\gamma_1(t))$ positiver Normalenvektor
  und es gilt $u\perp \gamma'(t)$.
  }
% -----------------------------------------------------------------------------
\section{Wegintegrale}
% -----------------------------------------------------------------------------
\definition Wegintegral:{
  Sei $\gamma:[a,b]\to\SetR^n$ ein rber Weg und 
  $f\in\SetCont(\Gamma_\gamma,\SetR^n)$. Sei weiterhin
  $\gamma=(\ntuple \gamma)$, $f=(\ntuple f)$. Dann hei"st
  \[\int_\gamma f(x)dx:=\int_\gamma f(x)\cdot dx:=
    \int_\gamma \sum_{j=1}^n f_j(x)dx_j:=
    \sum_{j=1}^n \int_a^b f_j(\gamma(t)) d\gamma_j(t)
    \]
  Wegintegral von $f$ l"angs $\gamma$. Au"serdem definiert man
  \[\int_\gamma f_j(x)dx_j:=\int_a^b f_j(\gamma(t)) d\gamma_j(t)
    \]
  f"ur alle $j=1\ldots n$.
  }
% -----------------------------------------------------------------------------
\theorem:
  $\gamma:[a,b]\to\SetR^n$ rb Weg, $f,g\in\SetCont(\Gamma_\gamma,\SetR^n)$=>{
  \begin{stmts}
    \item \[ \int_\gamma \alpha f(x)+\beta g(x) dx = 
      \alpha \int_\gamma f(x) dx+\beta \int_\gamma f(x) dx \]
    \item F"ur jedes $c\in[a,b]$: \[ \int_\gamma f(x) dx=
      \int_{\gamma|_{[a,c]}} f(x) dx+\int_{\gamma|_{[c,b]}} f(x) dx \]
    \item \[ \int_{\gamma^-} f(x) dx=-\int_\gamma f(x) dx \]
    \item \[ \left|\int_\gamma f(x) dx\right| \leq
      L(\gamma) \max_{x\in\Gamma_\gamma} \{\norm{f(x)} \}\]
    \end{stmts}
  }
% -----------------------------------------------------------------------------
\theorem:
  $\gamma\in\SetCont^1([a,b],\SetR^n)$, $f\in\SetCont(\Gamma_\gamma,\SetR^n)$=>{
  Dann gilt
  \[\int_\gamma f(x)dx=
    \int_a^b \underbrace{f(\gamma(t))\cdot\gamma'(t)}_{\text{Skalarprodukt!}} dt
    \]
  }
% -----------------------------------------------------------------------------
\remark:{
  Ist $\gamma:[a,b]\to\SetR^n$ ein st"uckweise stetig db Weg und
  $z=\{t_0,\ldots,t_m\}$ eine Zerlegung von $[a,b]$ mit 
  $\gamma_k:=\gamma|_{[t_{k-1},t_k]}\in\SetCont^1([t_{k-1},t_k],\SetR^n)$, 
  $k=1,\ldots,m$, so gilt
  \[\int_\gamma f(x)dx=\sum_{k=1}^m \int_{\gamma_k} f(x)dx
    \]
  }
% -----------------------------------------------------------------------------
\theorem:
  $\gamma:[a,b]\to\SetR^n$, $\tilde\gamma:[\alpha,\beta]\to\SetR^n$
  rb Wege mit $\gamma\sim\tilde\gamma$, 
  $\Gamma:=\Gamma_\gamma=\Gamma_{\tilde\gamma}$,
  $f\in\SetCont(\Gamma,\SetR^n)$=>{
  Dann gilt
  \[\int_\gamma f(x)dx=\int_{\tilde\gamma} f(x)dx
    \]
  }
% -----------------------------------------------------------------------------
\definition Wegintegral bzgl. Wegl"ange:{
  Sei $\gamma:[a,b]\to\SetR^n$ ein rb Weg. Da $s:[a,b]\to\SetR$ mon.
  wachsend ist, gilt $s\in\SetBV([a,b],\SetR)$. 
  Sei weiter $f\in\SetCont(\Gamma_\gamma,\SetR)$. 
  Dann ist $f\in\SetRInt_s([a,b],\SetR)$
  und es hei"st
  \[\int_\gamma f(x) ds:=\int_a^b f(\gamma(t))ds(t)
    \]
  Wegintegral von $f$ l"angs $\gamma$ bez"uglich der Wegl"ange.
  }
% -----------------------------------------------------------------------------
\remark:{
  \begin{stmts}
    \item Ist $\gamma\in\SetCont^1([a,b],\SetR^n)$, so ist 
      $s\in\SetCont^1([a,b],\SetR)$ mit $s'(t)=\norm{\gamma'(t)}$ ($t\in[a,b]$)
      (nach Satz 40). Somit ist
      \[\int_\gamma f(x)ds=\int_a^b f(\gamma(t))\norm{\gamma'(t)} dt
        \]
    \item Ist $\gamma$ glatt, $\sigma:=s^{-1}$, so gilt:
      \[\int_\gamma f(x) ds = \int_0^{L(\gamma)} f(\gamma(\sigma(\tau)))d\tau
        \]
    \end{stmts}
  }
% -----------------------------------------------------------------------------
\section{Stammfunktionen}
% -----------------------------------------------------------------------------
\definition Gebiet:{
  Eine Menge $\emptyset\neq G\subseteq\SetR^n$ hei"st Gebiet $:\equiv$
  $G$ ist offen und f"ur alle Punkte $x_0,y_0\in G$ existiert ein Weg
  $\gamma:[a,b]\to G$ mit $\gamma(a)=x_0$, $\gamma(b)=y_0$.
  }
% -----------------------------------------------------------------------------
\remark:{
  Jede offene nicht-leere konvexe Menge ist ein Gebiet.
  }
% -----------------------------------------------------------------------------
\definition Stammfunktion:{
  Sei $G\subseteq\SetR^n$ ein Gebiet und $f:G\to\SetR^n$ eine Funktion.
  $F:G\to\SetR$ hei"st Stammfunktion/SF von $f$ auf $G$ $:\equiv$
  $F$ db auf $G$ und $F'(x)=\grad F(x)=f(x)$ ($x\in G$)
  
  Schreibweise:
  \[F(x)=\int f(x) dx
    \]
  }
% -----------------------------------------------------------------------------
\remark:{
  Im Fall $n=1$ gilt: $G$ Gebiet $\equiv$ $G$ offenes Intervall.
  
  In diesem Fall hat jede auf $G$ stetige Funktion eine Stammfunktion.
  }
% -----------------------------------------------------------------------------
\remark:{
  Ist $f\in\SetCont(G,\SetR^n)$ und $F:G\to\SetR$ eine Stammfunktion von $f$
  auf $G$, so ist $F\in\SetCont^1(G,\SetR)$.
  }
% -----------------------------------------------------------------------------
\theorem:
  $G\subseteq\SetR^n$ Gebiet, $x_0,y_0\in G$, $\gamma:[a,b]\to G$ ein Weg
  mit $\gamma(a)=x_0$, $\gamma(b)=y_0$=>{
  Dann ex. eine Zerlegung $z=\{t_0,\ldots,t_m\}$ von $[a,b]$ so, dass
  \[s(\gamma(t_{k-1}),\gamma(t_k))\subseteq G
    \]
  f"ur alle $k=1,\ldots,n$.
  }
% -----------------------------------------------------------------------------
\remark:{
  Insbesondere erh"alt man damit einen st"uckweise stetig db Weg in $G$,
  der $x_0$ mit $y_0$ verbindet.
  }
% -----------------------------------------------------------------------------
\theorem:
  $G\subseteq\SetR^n$ Gebiet, $g:G\to\SetR$ auf $G$ db mit $g'(x)=0$=>{
  Dann existiert ein $c\in\SetR$, so dass $g(G)=c$. (also $g$ konstant auf $G$)
  }
% -----------------------------------------------------------------------------
\remark:{
  Dies bedeutet, dass Stammfunktionen, falls vorhanden, bis auf eine additive
  Konstante eindeutig bestimmt sind.
  }
% -----------------------------------------------------------------------------
\theorem Integrabilit"atsbedingung:
  $G\subseteq\SetR^n$ Gebiet, $f\in\SetCont^1(G,\SetR^n)$=>{
  Besitzt $f$ eine Stammfunktion $F$ auf $G$, so gilt ($f=(f_1,\ldots,f_n)$)
  \[\frac{\partial f_j}{\partial \xi_k}=\frac{\partial f_k}{\partial \xi_j}
    \]
  f"ur alle $k,j\in\{1,\ldots,n\}$.
  }
% -----------------------------------------------------------------------------
\remark:{
  Die Umkehrung dieses Satzes gilt im Allgemeinen nicht, f"ur 
  $\SetCont^1$-Funktionen gibt er jedoch ein notwendiges Kriterium f"ur
  Integrabilit"at an.
  }
% -----------------------------------------------------------------------------
\theorem:
  $G\subseteq\SetR^n$ Gebiet, $f\in\SetCont(G,\SetR^n)$, 
  $F:G\to\SetR$ SF von $f$, $\gamma:[a,b]\to G$ st"uckweise stetig db Weg=>{
  \label{the:geb.sfausw}
  Dann gilt
  \[\int_\gamma f(x) dx = F(\gamma(b))-F(\gamma(a))
    \]
  }
% -----------------------------------------------------------------------------
\definition Wegunabh"angigkeit:{
  Sei $G\subseteq\SetR^n$ Gebiet, $f\in\SetCont^1(G,\SetR^n)$.
  Dann hei"st $\int f(x)dx$ in $G$ wegunabh"angig $:\equiv$
  Es existiert ein $\alpha\real$, so dass f"ur beliebige zwei Punkte 
  $x,y\in G$ und jeden st"uckweise stetig db Weg $\gamma:[a,b]\to G$ mit 
  $\Gamma_\gamma\subseteq G$, $\gamma(a)=x$, $\gamma(b)=y$ 
  \[\int_\gamma f(x)dx=\alpha
    \]
  In diesem Fall schreibt man
  \[\int_x^y f(x) dx\text{ statt }\int_\gamma f(x)dx
    \]
  }
% -----------------------------------------------------------------------------
\remark:{
  Die Aussage von \ref{the:geb.sfausw} bedeutet, dass die Stammfunktion einer
  durch die Voraussetzungen des Satzes bestimmten Funktion wegunabh"angig ist.
  }
% -----------------------------------------------------------------------------
\theorem:
  $G\subseteq\SetR^n$ Gebiet, $f\in\SetCont(G,\SetR^n)$, 
  $\int f(x)dx$ wegunabh. in $G$=>
  {
  Dann besitzt $f$ eine Stammfunktion auf $G$. Eine solche ist z.B.
  \[F(z):=\int_{x_0}^z f(x) dx\quad (x_0,z\in G, x_0 \text{ fest})
    \]
  }
% -----------------------------------------------------------------------------
\theorem:
  $G\subseteq\SetR^n$ konvexes Gebiet, $f\in\SetCont^1(G,\SetR^n)$=>{
  Dann sind die folgenden Aussagen "aquivalent:
  \begin{stmts}
    \item $f$ besitzt auf $G$ eine SF
    \item $f$ erf"ullt die Integrabilit"atsbedingung
    \item $\int f(x)dx$ ist wegunabh"angig in $G$
    \end{stmts}
  }
% -----------------------------------------------------------------------------
\section{Quader}
% -----------------------------------------------------------------------------
\definition Koordinatenvergleich:{
  F"ur $a=(\alpha_1,\ldots,\alpha_n)\in\SetR^n$, 
  $b=(\beta_1,\ldots,\beta_n)\in\SetR^n$
  definiert man
  \begin{align*}
    a \leq b &:\equiv \alpha_j\leq \beta_j \quad (j=1,\ldots,n) \\
    a \ll b &:\equiv \alpha_j<\beta_j \quad (j=1,\ldots,n)
    \end{align*}
  F"ur $\alpha\real$ seien
  \[[\alpha,\alpha]=(\alpha,\alpha)=[\alpha,\alpha)=(\alpha,\alpha]=\emptyset
    \]
  Dann hei"sen
  \begin{align*}
    Q[a,b]&:=\{x\in\SetR^n\mid a\leq x\leq b\}
       = [\alpha_1,\beta_1]\times\cdots\times[\alpha_n,\beta_n] \\
    Q(a,b)&:=\{x\in\SetR^n\mid a\ll x\ll b\}
       = (\alpha_1,\beta_1)\times\cdots\times(\alpha_n,\beta_n)
    \end{align*}
  Quader.
  }
% -----------------------------------------------------------------------------
\lessertheorem:=>{
  Es gilt $Q[a,b]$ ist abgeschlossen, $Q(a,b)$ ist offen. Quader haben 
  folgende Eigenschaften
  \begin{stmts}
    \item $Q[a,b]=\emptyset \equiv (\exists j\in\{1,\ldots,n\})(\alpha_j=\beta_j)$ 
    \item $Q(a,b)=\inner{Q[a,b]}$ und $\rim{Q[a,b]}=Q[a,b]\setminus Q(a,b)$ ($a\leq b$)
    \item $\closure{Q(a,b)}=Q[a,b]$ ($a\ll b$)
    \end{stmts}
  }
% -----------------------------------------------------------------------------
\definition Inhalt:{
  Sei $Q[a,b]$ ein Quader und $a=(\alpha_1,\ldots,\alpha_n)\in\SetR^n$, 
  $b=(\beta_1,\ldots,\beta_n)\in\SetR^n$. Dann hei"st
  \[|Q[a,b]|:=\prod_{j=1}^n (\beta_j-\alpha_j)
    \]
  Inhalt von $Q$.
  }
% -----------------------------------------------------------------------------
\definition Zerlegung:{
 $z$ hei"st Zerlegung von $Q[a,b]$ $:\equiv$
 \[z=z_1\times\cdots\times z_n
   \]
 mit $z_j=\{t_1^{(j)},\ldots,t_n^{(j)}\}$ Zerlegung von $[\alpha_j,\beta_j]$.
}
% -----------------------------------------------------------------------------
\remark:{
  Eine Zerlegung $z$ erzeugt Teilquader $Q_1,\ldots,Q_m$ der Form
  \[Q_k=[t_{l_1}^{(1)},t_{l_1+1}^{(1)}]\times \ldots \times
        [t_{l_n}^{(n)},t_{l_n+1}^{(n)}]
    \]
  }
% -----------------------------------------------------------------------------
\theorem:
  $Q[a,b]$ Quader, $z$ Zerlegung von $Q[a,b]$, $Q_1,\ldots, Q_m$ die von
  $z$ erzeugten Teilquader=>{
  Dann gilt
  \[|Q[a,b]| = \sum_{k=1}^n |Q_k|\text{ und } Q[a,b]=\bigcup_{k=1}^m Q_k
    \]
  }
% -----------------------------------------------------------------------------
\definition Quadersumme:{
  $S$ hei"st Quadersumme genau dann, wenn es abgeschlossene Quader 
  $Q_1,\ldots,Q_m$ mit
  \begin{stmts}
    \item $S=\bigcup_{k=1}^m Q_k$
    \item $\inner{Q_i}\cap \inner{Q_k}=\emptyset$ ($i,k=1,\ldots,n,i\neq k$)
    \end{stmts}
  gibt.
  
  Ist $S$ Quadersumme, so hei"st
  \[|S|:=\sum_{k=1}^m |Q_k|
    \]
  Inhalt von $S$.
  }
% -----------------------------------------------------------------------------
\theorem:
  $Q_1,\ldots,Q_m$ abgeschlossene Quader=>{
  Dann gilt:
  \begin{stmts}
    \item Dann ist 
      \[S:=\bigcup_{k=1}^m Q_k
        \]
      eine Quadersumme, und es gilt weiter
      \[|S|\leq\sum_{k=1}^m |Q_k|
        \]
    \item Sind $T,S\subseteq\SetR^n$ Quadersummen mit $S\subseteq T$, so gilt
      $|S|\leq|T|$
    \end{stmts}
  }
% -----------------------------------------------------------------------------
\convention{
  Sei
  \[\cal M:=\{(\alpha,\beta),(\alpha,\beta],[\alpha,\beta),[\alpha,\beta]
      \mid \alpha,\beta\real,\alpha\leq\beta\}
    \]
  Mit $\SetQuad n$ bezeichnet man die Menge aller beschr. Quader in $\SetR^n$:
  \[\SetQuad n:=\{I_1\times\ldots\times I_n\mid I_j\in \cal M\}\quad (j=1,\ldots,n)
    \]
}
% -----------------------------------------------------------------------------
\theorem:
  $Q,\tilde Q\in \SetQuad n$=>{
  Dann gilt
  \begin{stmts}
    \item $Q\cap\tilde Q\in \SetQuad n$
    \item Es existieren $Q_1,\ldots,Q_n\in \SetQuad m$ mit 
      $Q\setminus\tilde Q=\bigcup_{k=1}^m Q_k$
    \end{stmts}
  }
% -----------------------------------------------------------------------------
\theorem:
  $Q_1,Q_2,\ldots\in \SetQuad n$ h"ochstens abz"ahlbar viele Quader mit 
  $\cal M=\bigcup_k Q_k$=>{
  Dann gibt es h"ochstens abz"ahlbar viele $\tilde Q_1,\tilde Q_2,\ldots\in \SetQuad n$
  mit $\cal M=\bigcup_k \tilde Q_k$, wobei die $\tilde Q_k$ paarweise disjunkt
  sind.
  }
% -----------------------------------------------------------------------------
\theorem:
  $D\subseteq\SetR^n$ offen=>{
  Dann existiert eine Menge von h"ochstens abz"ahlbar vielen paarweise disjunkten 
  $Q_1,Q_2,\ldots\in \SetQuad n$ mit $D=\bigcup_k Q_k$ und $\closure{Q_k}\subseteq D$
  f"ur alle $k$.
  }
% -----------------------------------------------------------------------------
\section{Das "au"sere Lebesgue-Ma"s}
% -----------------------------------------------------------------------------
\definition "Au"seres Lebesgue-Ma"s:{
  Sei $A\subseteq\SetR^n$. Sei 
  \[J_A:=\{
      \{Q_1,Q_2,\ldots\}\subseteq\SetQuad n \mid 
      A\subseteq\bigcup_k Q_k \land \text{ h"ochstens abz. unendl. }
      \}
    \]
  Dann ist $J_A\neq\emptyset$ und es hei"st
  \[\lambda(A):=\inf\{\sum_k |Q_k| \mid \{Q_1,Q_2,\ldots\}\in J_A\}
    \]
  "au"seres Lebesgue-Ma"s (L-Ma"s) von $A$. 
  ($\infty$ ist als Wert von $\lambda(A)$ zugelassen)
  }
% -----------------------------------------------------------------------------
\example:{
  \begin{stmts}
    \item $\lambda(\{x\})=0$
    \item $\lambda(\SetQ)=0$
    \item Sei $H$ eine Hyperebene der Form $\xi_j=c$. Dann ist $\lambda(H)=0$
    \item $\lambda(\emptyset)=0$, $\lambda(\SetR^n)=\infty$
    \item Sei $Q\in\SetQuad n$ abgeschlossen. Dann gilt $\lambda(Q)=|Q|$.
    \end{stmts}
  }
% -----------------------------------------------------------------------------
\theorem:=>{
  Es gilt
  \begin{stmts} 
    \item $(\forall A\subseteq\SetR^n)(0\leq\lambda(A)\leq\infty)$
    \item $(\forall A,B\subseteq\SetR^n)
      (A\subset B\implies \lambda(A)\leq\lambda(B)$
    \item $\sigma$-Subadditivit"at: Sei $(A_n)\subset\cal P(\SetR^n)$ 
      endlich. Dann ist
      \[\lambda(\bigcup_k A_k)\leq\sum_k\lambda(A_k)
        \]
    \end{stmts}
  \label{the:lebesgue-prop}
  }
% -----------------------------------------------------------------------------
\example:{
  Sei $Q\in\SetQuad n$ beliebig. Dann gilt $\lambda(Q)=|Q|$.
  }
% -----------------------------------------------------------------------------
\remark:{
  Ist $A\subset\SetR^n$ beschr"ankt, so ist $\lambda(A)<\infty$.
  }
% -----------------------------------------------------------------------------
\definition Nullmenge:{
  $N\subset\SetR^n$ hei"st Nullmenge/NM $:\equiv$ $\lambda(N)=0$.
  }
% -----------------------------------------------------------------------------
\remark:{
  Es ergibt sich
  \begin{stmts}
    \item Die Vereinigung endlich vieler Nullmengen hat das Lebesgue-Ma"s $0$.
    \item $N$ endlich/ abz. unendlich $\implies$ $N$ ist Nullmenge.
    \item Jede Teilmenge einer NM ist NM.
    \item Sei $H$ eine Hyperebene der Form $\xi_j=c$. Dann ist $H$ Nullmenge.
    \end{stmts}
  }
% -----------------------------------------------------------------------------
\definition fast "uberall:{
  Es sei $A\subseteq\SetR^n$ und $(E)$ eine Eigenschaft von Punkten aus $A$.
  Dann sagt man: $(E)$ gilt fast "uberall/f"u in A (bzw. f"ur fast alle 
  $x\in A$) $:\equiv$ 
  $\lambda(\{x\in A\mid \text{$(E)$ gilt nicht f"ur $x$} \})=0$
  }
% -----------------------------------------------------------------------------
\section{Das Lebesgue'sche Ma"s}
% -----------------------------------------------------------------------------
\definition Lebesgue-Messbarkeit:{
  $A\in\SetR^n$ hei"st Lebesgue-messbar genau dann, wenn
  \[(\forall E\subseteq\SetR^n)(\lambda(E)=\lambda(E\cap A)+
    \lambda(E\cap(\SetR^n\setminus A)))
    \]
  }
% -----------------------------------------------------------------------------
\remark:{
  Wegen \ref{the:lebesgue-prop} reicht in der obigen Definition die
  Eigenschaft
  \[(\forall E\subseteq\SetR^n)(\lambda(E)\geq\lambda(E\cap A)+
    \lambda(E\cap(\SetR^n\setminus A)))
    \]
  Man definiert $\SetLMeas:=\{A\subseteq\SetR^n\mid A \text{ Lebesgue-messbar\}}$.
  }
% -----------------------------------------------------------------------------
\example:{
  \begin{stmts}
    \item Ist $N$ eine Nullmenge, so gilt $N\in\SetLMeas$.
    \item $\SetR^n\in\SetLMeas$
    \end{stmts}
  }
% -----------------------------------------------------------------------------
\definition Volumen:{
  F"ur $A\in\SetLMeas$ hei"st $\lambda(A)$ das Volumen von $A$.
  }
% -----------------------------------------------------------------------------
\theorem:=>{
  $\SetLMeas$ hat folgende Eigenschaften:
  \begin{stmts}
    \item $A\in\SetLMeas$ $\implies$ $\SetR^n\setminus A\in\SetLMeas$
    \item $\SetR^n,\emptyset$ und jede Nullmenge sind $\in\SetLMeas$
    \item $A,B\in\SetLMeas$ $\implies$ $A\setminus B\in\SetLMeas$
    \item $A_1,A_2,\ldots\in\SetLMeas$ $\implies$
      $\bigcap_k A_k\in\SetLMeas$, $\bigcup_k A_k\in\SetLMeas$
    \end{stmts}
  $\lambda:\SetLMeas\to[0,\infty)\cup\{\infty\}$ hat folgende Eigenschaften:
  \begin{stmts}
    \item $\sigma$-Additivit"at: $A_1,A_2,\ldots\in\SetLMeas$ paarweise 
      disjunkt $\implies$
      \[\lambda(\bigcup_k A_k)=\sum_k \lambda(A_k)
        \]
    \item $A_1,A_2,\ldots\in\SetLMeas$, $A_1\subseteq A_2 \subseteq \ldots$ $\implies$
      \[\lambda(\bigcup_k A_k)=\limes k->\infty \lambda(A_k)
        \]
    \item $A_1,A_2,\ldots\in\SetLMeas$, $A_1\supseteq A_2 \supseteq \ldots$ 
      und $\lambda(A_1)<\infty$ $\implies$
      \[\lambda(\bigcap_k A_k)=\limes k->\infty \lambda(A_k)
        \]
    \end{stmts}
  } 
% -----------------------------------------------------------------------------
\theorem:=>{
  Es gilt $\SetQuad n\subseteq\SetLMeas$.
  }
% -----------------------------------------------------------------------------
\remark:{
  Aus dem vorigen Satz geht direkt hervor:
  \begin{stmts}
    \item $A\subseteq\SetR^n \text{ offen} \implies A\in\SetLMeas$
    \item $A\subseteq\SetR^n \text{ abgeschlossen} \implies A\in\SetLMeas$
    \item $A_1,A_2,A_3,\ldots \subseteq\SetR^n \text{ offen} \implies 
      \bigcup_k A_k\in\SetLMeas$ (``$G_\delta$-Menge'')
    \item $A\subseteq\SetR^{n-1}$, $A\in\SetLMeas$, $[\alpha,\beta]\subset\SetR$.
      Dann gilt
      \[A\times[\alpha,\beta]\in\SetLMeas \text{ und } 
        \lambda(A\times[\alpha,\beta])=\lambda(A)\cdot(\beta-\alpha)
        \]
    \end{stmts}
  }
% -----------------------------------------------------------------------------
\section{Messbare Funktionen}
% -----------------------------------------------------------------------------
\convention{
  Sei $M\in\SetLMeas$. F"ur Funktionen $f,g:M\to\SetR$ sei vereinbart
  \[\{f<g\}_M:=\{x\in M\mid f(x)<g(x)\}
    \]
  Analog f"ur $\leq$, $>$, $\geq$.
  }
% -----------------------------------------------------------------------------
\theorem:
  $f:M\to\SetR$, $t\real$=>{
  Dann sind folgende Aussagen "aquivalent: 
  \[\{f<t\}_M\in\SetLMeas \equiv 
    \{f\leq t\}_M\in\SetLMeas \equiv 
    \{f>t\}_M\in\SetLMeas \equiv 
    \{f\geq t\}_M\in\SetLMeas 
    \]
  }
% -----------------------------------------------------------------------------
\definition Lebesgue-Messbarkeit einer Funktion:{
  \index{Funktion>Lebesgue-Messbarkeit einer}
  Eine Funktion $f:M\to\SetR$ hei"st (Lebesgue-)messbar auf $M$ $:\equiv$
  \[(\forall t\in\SetR)(\{f\leq t\}_M\in\SetLMeas)
    \]
  Schreibweise: $f\in\SetFLMeas(M)$
}
% -----------------------------------------------------------------------------
\theorem:
  $f,g:M\to\SetR$ Funktionen $s,t\real$=>{
  Dann gilt:
  \begin{stmts}
    \item $f,g\in\SetFLMeas(M) \implies 
      \{f < g\}_M,\{f > g\}_M,\{f \leq g\}_M,\{f \geq g\}_M \in\SetLMeas$ bzw.
      $\{s\stack(<;\leq)f\stack(<;\leq) t\}\in\SetLMeas$
    \item $B\in\SetLMeas,B\subseteq M,f\in\SetFLMeas(M)\implies f\in\SetFLMeas(B)$
    \item $B_1,B_2,\ldots\in\SetLMeas$ endlich/abz"ahlbar viele Mengen, 
      $M=\bigcup_k B_k$, $f\in\SetFLMeas(B_k)$ f"ur $k\natural$ $\implies$ 
      $f\in\SetFLMeas(M)$
    \item $N\subseteq M$ Nullmenge (also $N,M\setminus N\in\SetLMeas$) und 
      $h\in\SetFLMeas(M\setminus N)$, $f|_{M\setminus N}=h\implies f\in\SetFLMeas(M)$
    \end{stmts}
  }
% -----------------------------------------------------------------------------
\theorem:
  $f,g:M\to\SetR$, $f=g$ f."u. auf $M$=>{
  Dann gilt $f\in\SetFLMeas(M)\equiv g\in\SetFLMeas(M)$.
  }
% -----------------------------------------------------------------------------
\definition Maximum/Minimum-Verkn"upfung bei Funktionen:{
  Seien $f,g:M\to\SetR$ Funktionen. Dann definiert man f"ur $x\in M$
  \begin{align*}
    (f\lor g)(x)&:=\max\{f(x),g(x)\} \\
    (f\land g)(x)&:=\min\{f(x),g(x)\} \\
    f^+&:=f\lor 0 \\
    f^-&:=-(f\land 0)
    \end{align*}
  Es ergibt sich $f^+,f^-\geq 0,f=f^+-f^-,|f|=f^++f^-$.
  }
% -----------------------------------------------------------------------------
\theorem:=>{
  Es gilt:
  \begin{stmts}
    \item $f_1,f_2,\ldots:M\to\SetR$, $f_i\in\SetFLMeas(M)$ $(i\natural)$ 
      endlich/abz. viele Funktionen. F"ur jedes $x\in M$ existiere 
      $\underline H(x):=\inf\{f_k(x)\mid k\natural\}$ und
      $\overline H(x):=\sup\{f_k(x)\mid k\natural\}$. Dann gilt
      $\underline H,\overline H\in\SetFLMeas(M)$.
    \item $(f_k)\subset\SetFLMeas(M)$ Folge und f"ur jedes $x\in M$ existiere 
      $\underline h(x):=\limesinf k->\infty f_k(x)$ und
      $\overline h(x):=\limessup k->\infty f_k(x)$. Dann gilt
      $\underline h,\overline h\in\SetFLMeas(M)$.
    \item $f,g\in\SetFLMeas(M)$. Dann sind $f\land g,f\lor g,f^+,f^-\in\SetFLMeas(M)$.
    \item $(f_k)\subset\SetFLMeas(M)$ Folge, $(f_k)$ konvergiere f."u. auf $M$
      gegen eine Funktion $f:M\to\SetR$ $\implies$ $f\in\SetFLMeas(M)$
    \end{stmts}
  }
% -----------------------------------------------------------------------------
\theorem:=>{
  Es gilt:
  \begin{stmts}
    \item $f\in\SetLMeas(M)$ $\equiv$ F"ur alle offenen/abgeschlossenen
      Teilmengen $U\subseteq\SetR$ gilt $f^{-1}(U)\in\SetLMeas$.
    \item Seien $g\in\SetCont(\SetR,\SetR)$, $f\in\SetLMeas(M)$. Dann ist
      $g\circ f\in\SetLMeas(M)$
    \end{stmts}
  }  
% -----------------------------------------------------------------------------
\theorem:=>{
  Es gilt:
  \begin{stmts}
    \item $\SetFLMeas(M)$ ist reeller Vektorraum
    \item $f,g\in\SetFLMeas(M)$ $\implies$ $f\cdot g\in\SetFLMeas(M)$
    \item $f,g\in\SetFLMeas(M)$, $g(x)\neq 0$ $\implies$ $f/g\in\SetFLMeas(M)$
    \item $f\in\SetFLMeas(M)$, $p>0$ $\implies$ $|f|^p\in\SetFLMeas(M)$
    \end{stmts}
  }
% -----------------------------------------------------------------------------
\definition Treppenfunktion:{
  Man definiert folgendes:
  \begin{stmts}
    \item \index{Funktion>charakteristische}
      Sei $A\subseteq\SetR^n$. Dann nennt man
      \[1_A(x):=\begin{cases} 1 & x\in A \\ 0 & x\not\in A \end{cases}
        \]
      \indexthis{charakteristische Funktion} von $A$. 
      (Bemerkung: $1_A\in\SetFLMeas(\SetR^n)\equiv A\in\SetLMeas$)
    \item Eine Funktion $f:\SetR^n\to\SetR$ hei"st Treppenfunktion/Tfkt $:\equiv$
      $f(\SetR^n)$ ist endlich.
      
      Ist $f$ eine Treppenfunktion und $f(\SetR^n)=\{\alpha_1,\ldots,\alpha_m\}$,
      so legt man fest $A_0:=\emptyset$ und 
      \[A_j:=\{x\in\SetR^n\setminus \bigcup_{i=0}^{j-1} A_i \mid f(x)=\alpha_j\}
        \]
      f"ur $j=1,\ldots,m$. Dann hat $f$ folgende Darstellung
      \[f=\sum_{j=1}^m \alpha_j 1_{A_j}
        \]
      und es gilt $\SetR^n=A_1\cup\ldots\cup A_m$, daher 
      $f\in\SetFLMeas(M)\equiv A_1,\ldots,A_m\in\SetLMeas$
      
      Schreibweise: $T:=\{f:\SetR^n\to\SetR\mid f\in\SetFLMeas(M) 
        \text{ Treppenfunktion}\}$, $T_+:=\{f\in T\mid\ f \geq 0\}$
    \end{stmts}
  }
% -----------------------------------------------------------------------------
\remark:{
  Treppenfunktionen haben folgende Eigenschaften:
  \begin{stmts}
    \item $A\in\SetLMeas\implies 1_A\in\SetFLMeas(\SetR^n)$
    \item $B_1,B_2\in\SetLMeas$, $B_1\cap B_2=\emptyset$, 
      $B:=B_1\cup B_2\in\SetLMeas$ $\implies$ $1_B=1_{B_1}+1_{B_2}$
      
      Also ist die obige Darstellung einer Treppenfunktion nicht eindeutig.
    \end{stmts}
  }
% -----------------------------------------------------------------------------
\theorem:
  $f\in\SetFLMeas(M)$, $f(x)\geq 0$ ($x\in M$)=>{
  Dann existiert eine Folge $(f_m)\subset T_+$ mit
  \begin{stmts}
    \item $f_1(x)\leq f_2(x) \leq \ldots$
    \item $\limes m->\infty f_m(x)=f(x)$
    \end{stmts}
  f"ur alle $x\in M$.
  }

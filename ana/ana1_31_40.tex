% -------------------------------------------------------------------------
\para{Der zweite Hauptsatz der Integralrechnung}
% -------------------------------------------------------------------------
\definition:{
  Sei $f\in\SetRInt([a,b])$. Dann ist $\int_b^a f(x)dx:=-\int_a^b f(x)dx$
  
  Sei $c\in[a,b]$. Dann ist $\int_c^c f(x)dx:=0$
  }
% -------------------------------------------------------------------------
\remark:{
  $f\in\SetRInt([a,b]),x\in[a,b]$. Dann ex. $\int_a^x f(x)dx$.
  }
% -------------------------------------------------------------------------
\theorem 2. Hauptsatz der Differential- und Integralrechnung:
  $f\in\SetRInt([a,b])$,
  $F:[a,b]\to\SetR$, $x\mapsto F(x):=\int\nolimits_a^x f(t)dt$=>{
  \begin{stmts}
    \item $F$ ist Lipschitz-stetig auf $[a,b]$, insbes. $F\in\SetCont([a,b])$
    \item Ist $f$ in $x_0\in[a,b]$ stetig, so ist $F$ in $x_0$ db und 
          $F'(x_0)=f(x_0)$
    \item $f\in\SetCont([a,b])\implies F\in\SetCont^1([a,b])$ und $F'=f$ auf 
          $[a,b]$
    \end{stmts}
  }
% -------------------------------------------------------------------------
\remark:{
  Jede Funktion $f\in\SetCont([a,b])$ besitzt auf $[a,b]$ eine Stammfunktion.
  }
% -------------------------------------------------------------------------
\para{Integrationsregeln}
% -------------------------------------------------------------------------
\theorem:$I\subseteq\SetR$ bel. Intervall, $f\in\SetCont(I)$, 
  $\xi\in I$ fest, 
  $F:I\to\SetR$, $x\mapsto F(x):=\int\nolimits_a^x f(t)dt$=>{
  $F'=f$ auf $I$
  }
% -------------------------------------------------------------------------
\definition Unbestimmtes Integral:{
  Sei $I\subseteq\SetR$ bel. Intervall, $f:I\to\SetR$ eine Funktion.
  Besitzt $f$ auf $I$ eine Stammfunktion $F$, so schreibt man f"ur eine 
  solche auch 
  \[\int f(x)dx:=F
    \]
  }
% -------------------------------------------------------------------------
\theorem Partielle Integration:=>{
  \begin{stmts}
    \item $f,g\in\SetRInt([a,b])$, $F,G$ Stammfunktionen von $f,g$.\\
      Dann: $\int_a^b F(x)g(x)dx =[F(x)G(x)]_a^b-\int_a^b f(x)G(x)dx$
    \item $f,g\in\SetCont^1([a,b])$\\
      Dann: $\int_a^b f'(x)g(x)dx= [f(x)g(x)]_a^b-\int_a^b f(x)g'(x)dx$
    \item $I\subseteq\SetR$ beliebiges Intervall, $f,g\in\SetCont^1(I)$\\
      Dann: $\int f'(x)g(x)dx = f(x)g(x)-\int f(x)g'(x) dx$
    \end{stmts}
  }
% -------------------------------------------------------------------------
\definition Hilfsschreibweise zur Substitutionsregel:{
  F"ur den folgenden Satz legen wir fest
  \[<\alpha;\beta>:=[\min\{\alpha,\beta\};\max\{\alpha,\beta\}]
    \]
  }
% -------------------------------------------------------------------------
\theorem Substitutionsregel:
  $f\in\SetRInt(<\alpha;\beta>)$ besitze auf $<\alpha;\beta>$ eine Stammfunktion.=>{
  \begin{stmts}
    \item $g:<\alpha;\beta>\to\SetR$ db, $(f\circ g)\cdot g' \in\SetRInt((\alpha;\beta))$ \\
      $g(<\alpha;\beta>)\subseteq <\alpha;\beta>,a:=g(\alpha),b:=g(\beta)$ \\
      Dann: $\displaystyle\int_a^b f(x) dx=\int_\alpha^\beta f(g(t))\cdot g'(t) dt$ 
    \item $f\in\SetCont(<a;b>),g\in\SetCont^1(<\alpha;\beta>)$ \\
      $g(<\alpha;\beta>)\subseteq <\alpha;\beta>,a:=g(\alpha),b:=g(\beta)$ \\
      Dann: $\displaystyle\int_a^b f(x) dx=\int_\alpha^\beta f(g(t))\cdot g'(t) dt$ 
    \item $I, J$ seien bel. Intervalle, 
      $f\in\SetCont(I),g\in\SetCont^1( J)\subseteq I$
    \item $\int f(g(t))\cdot g'(t) dt=\int f(x)dx|_{x=g(t)}$ auf  $J$.
    \item $g$ streng monoton $\implies$ $(\exists g^{-1}:g( J)\to  J)$\\
      $\displaystyle\int f(x)dx=\int f(g(t))g'(t) dt\big|_{t=g^{-1}(x)}$
    \end{stmts}
  }
% -------------------------------------------------------------------------
\remark Merkregel f"ur die Substitution:{
  Sei $g=g(x)$ eine db Funktion. Dann schreibt man $\frac{dg}{dx}$ f"ur $g'$.
  Subst. $x=g(t)$. Dann ist $\frac{dx}{dt}=g'(t) \equiv dx=g'(t)dt$. 
  Keine Mathematik, aber schick zum Merken.
  }
% -------------------------------------------------------------------------
\para{Verschiedenes}
% -------------------------------------------------------------------------
\definition Feinheitsma"s:{
  Sei $z$ eine Zerlegung von $[a,b]$ und $|I_j|,j=1\ldots n$ die L"ange des
  $j$-ten Intervalls.
  
  Dann ist $|z|:=\max\{|I_j||j=1\ldots n\}$ das Feinheitsma"s von $z$.
  }
% -------------------------------------------------------------------------
\theorem:$f:[a,b]\to\SetR, (\forall x\in[a,b])(|f(x)|\le M)$, 
  $z,\tilde z$ Zerlegungen von $[a,b]$, $z\subseteq\tilde z$, 
  $\tilde z$ enth"alt $p$ Teilpunkte mehr als $z$.=>{
  Dann gelten
  \begin{align*}
    \lsum(\tilde z) &\le \lsum(z) + 2pM|z|\\
    \usum(\tilde z) &\ge \usum(z) - 2pM|z|\\
    \end{align*}
  }
% -------------------------------------------------------------------------
\theorem:$f:[a,b]\to\SetR, (\forall x\in[a,b])(|f(x)|\le M)$, 
  $(z_n)$ Folge von Zerlegungen von $[a,b]$, $\limn|z_n|=0$,
  zu jedem $z_n$ existiert ein passender Zwischenvektor $\xi(n)$.=>{
  \begin{stmts}
    \item $\limn \lsum(z_n)=\lint_a^b f(x)dx$ \\
      $\limn \usum(z_n)=\uint_a^b f(x)dx$ 
    \item $f\in\SetRInt([a,b])\implies
      \limn \sigma_f(z_n;\xi(n))=\int_a^b f(x)dx$
    \end{stmts}
  }
% -------------------------------------------------------------------------
\theorem: $(f_n)$ Folge in $C^1([a,b])$. $f_n(a)$ konvergiert, 
  $f_n'$ konvergiert glm auf $[a,b]$ gegen $g:[a,b]\to\SetR$=>{
  $f_n$ konvergiert glm auf $[a,b]$ und f"ur die Grenzfunktion gilt
  \begin{gather*}
    f(x)=\limn f_n(x)\qquad f\in\SetCont^1[a,b]\\
    (\forall x\in[a,b])(f'(x)=g(x))     
    \end{gather*}
  }
% -------------------------------------------------------------------------
\remark:{
  Vorstehender Satz l"a"st sich ausdr"ucken (mit den obigen Voraussetzungen):
  \[(\limn f_n)'=f'=g=\limn (f_n)' 
    \]
  Obiger Satz gilt entsprechend f"ur Funktionenreihen.
  }
% -------------------------------------------------------------------------
\theorem Satz von Taylor mit Integralrestglied:
  $I\subseteq\SetR$ Intervall, $f\in\SetCont^{n+1}(I)$, $a,b\in I,a<b$=>
  {
  Es gilt
  \[f(b)=\sum_{k=0}^n \frac {f^{(k)}(a)}{k!} (b-a)^k+
         \int_a^b \frac{(b-x)^n}{n!}f^{(n+1)}(x) dx
    \]
  }
% -------------------------------------------------------------------------
\para{Uneigentliche Integrale}
% -------------------------------------------------------------------------
\convention{$a,b\real$, 
  $\alpha\real\cup\{-\infty\},\beta\in\SetR\cup\{\infty\}$
  }
% -------------------------------------------------------------------------
\definition Uneigentliches Integral:{
  $f:\stack({[\alpha;\beta)};{(\alpha;\beta]})\to\SetR$ sei ib "uber jedem
  Intervall $\stack({[a;t]};{[t;b]})$ mit $t\in\stack({(a;\beta)};{(\alpha;b)})$.
  Existiert der Grenzwert 
  $\stack({\limes t->\beta \int_a^t f(x)dx};{\limes t->\alpha \int_t^b f(x) dx})$,
  so hei"st $f$ uneigentlich integrierbar "uber $\stack({[a;\beta)};{(\alpha;b]})$.
  und das Integral
  $
    \stack({\int_a^\beta f(x)dx};{\int_\alpha^b f(x) dx})
    =
    \stack({\limes t->\beta \int_a^t f(x)dx};{\limes t->\alpha \int_t^b f(x) dx})
  $
  hei"st konvergent. Andernfalls hei"st
  $\stack({\int_a^\beta f(x)dx};{\int_\alpha^b f(x) dx})$ divergent.
  }
% -------------------------------------------------------------------------
\lessertheorem: $f:[\alpha;\beta)\to\SetR$ ib f"ur alle 
  $[a;t)$ mit $t\in(\alpha;\beta)$=>
  {
  Es ist
  \[\int_a^\beta f(x)dx \text{ konvergent} \equiv
    (\exists c\in(\alpha;\beta))(\int_c^\beta f(x)dx \text{ konvergent})
    \]
  In diesem Fall:
  \[\int_a^\beta f(x)dx =\int_a^c f(x)dx+\int_c^\beta f(x)dx
    \]
  F"ur untere Grenzen gilt dieser Satz analog.
  }
% -------------------------------------------------------------------------
\definition Beidseitiges uneigentliches Integral:{
  $f:(\alpha,\beta)\to\SetR$ hei"st uneigentlich ib "uber $(\alpha;\beta)$ 
  $:\equiv$
  \[(\exists c\in (\alpha;\beta))
      (\int_\alpha^c f(x)dx \text{ konvergent} \land 
       \int_c^\beta f(x)dx \text{ konvergent})
    \]
  In diesem Fall:
  \[\int_\alpha^\beta f(x)dx:=\int_\alpha^c f(x)dx+\int_c^\beta f(x)dx 
    \]
  }
% -------------------------------------------------------------------------
\remark:{
  Gilt obige Beziehung f"ur ein $c\in(\alpha;\beta)$, so gilt sie f"ur alle
  $c\in(\alpha;\beta)$.
  }
% -------------------------------------------------------------------------
\remark:{
  Aus der Existenz von $\limes t->\infty \int_{-t}^t f(x)dx$ folgt nicht
  die Konvergenz des entsprechenden Integrals.
  }
% -------------------------------------------------------------------------
\convention{
  Die folgenden S"atze und Definitionen sind f"ur Integrale der Form
  $\int_a^\beta f(x) dx$ formuliert, gelten aber sinngem"a"s auch f"ur 
  den anderen Typ.
  
  Sei ab hier $f\in\SetRInt([a,b])$, $t\in(\alpha,\beta)$.
}
% -------------------------------------------------------------------------
\theorem Cauchy-Kriterium f"ur uneigentliche Integrale:=>{
  Es gilt
  \begin{multline*}
    \int_a^\beta f(x)dx \text{ konvergent} \equiv \\
    \epsn(\exists c=c(\epsilon)\in(a,\beta))(\forall u,v \in [c,\beta))
      (\left|\int_u^v f(x)dx\right| <\epsilon)
    \end{multline*}
  }
% -------------------------------------------------------------------------
\definition Absolute Konvergenz bei uneigentlichen Integralen:{
  $\int_a^\beta f(x)dx$ hei"st absolut konvergent $:\equiv$ 
  $\int_a^\beta |f(x)| dx$ konvergiert.
  }
% -------------------------------------------------------------------------
\theorem Dreiecksungleichung f"ur uneigentliche Integrale:
  $\int_a^\beta f(x)dx$ absolut konvergent=>{
  $\int_a^\beta f(x)dx$ konvergent und es gilt
  \[\left|\int_a^\beta f(x)dx\right|\le\int_a^\beta |f(x)|dx
    \]
  }
% -------------------------------------------------------------------------
\theorem Majoranten-/Minoranten\-kriterium f"ur uneigentliche Integrale:=>{
  \begin{stmts}
    \item $(\forall x \in[a,\beta))(|f(x)|\le g(x))$, 
      $\int_a^\beta g(x)dx$ konv.
      $\implies$ $\int_a^\beta f(x)dx$ konv. absolut
      (\emph{\indexthis{Majorantenkriterium}})
    \item $(\forall x \in[a,\beta))(f(x)\ge g(x)\ge 0)$, 
      $\int_a^\beta g(x)dx$ divergent 
      $\implies$ $\int_a^\beta f(x)dx$ divergent
      (\emph{\indexthis{Minorantenkriterium}})
    \end{stmts}
  }
% -------------------------------------------------------------------------
\remark:{
  Sei $ W$ die Menge aller "uber $[a,\beta)$ uneigentlich ib'en 
  Funktionen. Dann ist $ W$ ein reeller Vektorraum und die 
  Abbildung $f\mapsto\int_a^\beta f(x)dx\,(f\in W)$ ist linear.
  }
% -------------------------------------------------------------------------
\para{Funktionen von beschr"ankter Variation}
% -------------------------------------------------------------------------
\definition Variation:{
  Sei $[a,b]\subseteq\SetR,a<b$ und $f:[a,b]\to\SetR$ eine Funtkion.
  $z=\{x_0,...x_n\}$ sei Zerlegung von $[a,b]$.
  
  Dann hei"st $V_f(z):=\sum_{j=1}^n |f(x_j)-f(x_{j-1})|$ die Variation
  von $f$ bezgl. $z$.
  }
% -------------------------------------------------------------------------
\lessertheorem:
  $z_1,z_2$ Zerlegungen von $[a,b]$ mit $z_1\subseteq z_2$=>{
  $V_f(z_1)\le V_f(z_2)$
  }
% -------------------------------------------------------------------------
\definition Beschr"ankte Variation:{
  $f:[a,b]\to\SetR$ hei"st von beschr"ankter Variation (bV) $:\equiv$
  \[(\exists M\ge 0)(\forall z \text{ Zerlegung von $[a,b]$})(V_f(z)\le M)
    \]
  In diesem Fall hei"st 
  $V_f([a,b]):=\sup \{V_f(z)\mid z \text{ Zerlegung von $[a,b]$}\}$
  die \indexthis{Totalvariation} von $f$.
  $\SetBV([a,b])$ bezeichnet die Menge aller Funktionen von beschr"ankter
  Variation auf $[a,b]$
}
% -------------------------------------------------------------------------
\lessertheorem:=>{
  Es ist $\SetCont^1([a,b])\subset \SetBV([a,b])$, aber
  $\SetCont([a,b])\not\subset \SetBV([a,b])$.
  }
% -------------------------------------------------------------------------
\theorem:$f:[a,b]\to\SetR$, $c\in[a,b]$=>{
  \begin{stmts}
    \item $f\in\SetBV([a,b])\equiv f\in\SetBV([a;c])\cap\SetBV([c;b])$
      $(\forall f\in\SetBV([a,b]))(V_f([a,b])=V_f([a;c])+V_f([c;b])$ 
    \item $f\in\SetBV([a,b])$ $\implies$ $f$ ist beschr"ankt 
    \item $f$ monton auf $[a,b]$ $\implies$ $f\in\SetBV([a,b])$ 
    \item $f$ hat beschr"ankte Ableitung $\implies$ $f\in\SetBV([a,b])$ 
      Dieser Fall tritt z.B. ein, wenn $f\in\SetCont^1([a,b])$ oder $f$ 
      Lipschitz-stetig.
    \item $\SetBV([a,b])$ ist ein reeller Vektorraum.
    \end{stmts}
  }
% -------------------------------------------------------------------------
\theorem:=>{
  Es ist $f\in\SetBV([a,b])$ $\equiv$
  \[(\exists f_1,f_2:[a,b]\to\SetR \text{ monton $\nearrow$})(f=f_1-f_2)
    \]
  }
% -------------------------------------------------------------------------
\theorem:$f\in\SetCont^1([a,b])$=>{
  Dann gilt 
  \[V_f([a,b])=\int_a^b |f'(x)| dx
    \]
  }
% -------------------------------------------------------------------------
\para{Das Riemann-Stieltjes-Integral}
% -------------------------------------------------------------------------
\convention{
  $f,g:[a,b]\to\SetR$ beschr"ankt
  }
% -------------------------------------------------------------------------
\definition Riemann-Stieltjes-Integral:{
  Sei $z=\{x_0,...x_n\}$ Zerlegung von $[a,b]$ und $\xi$ ein zu $z$ 
  passender ZV.
  
  $\sigma_f(z;\xi;g):=\sum_{j=1}^n f(\xi_j)(g(x_j)-g(x_{j-1}))$
  hei"st dann \indexthis{Riemann-Stieltjes-Summe}.
  
  $f$ hei"st (Riemann-Stieltjes-)ib bzgl. $g$ $:\equiv$
  \begin{multline*}
    (\exists s\in\SetR)\epsn(\exists \delta>0)
    (\forall z \text{ Zerlegung von $[a,b]$ mit $|z|<\delta$})\\
    (\forall \xi \text{ ZV passend zu $z$})(|\sigma_f(z;\xi;g)-s|<\epsilon)
    \end{multline*}
  $\SetRInt_g([a,b])$ bezeichnet die Menge aller "uber $[a,b]$ bzgl. $g$
  Riemann-Stieltjes-ib'en Funktionen.
  
  Man kann zeigen, da"s $s$ in diesem Fall eindeutig bestimmt ist. $s$
  hei"st dann das Riemann-Stieltjes-Integral von $f$ bzgl. $g$
  "uber $[a,b]$
  
  Schreibweise:
  \[\int_a^b f(x)dg(x):=s
    \]
}
% -------------------------------------------------------------------------
\remark:{
  Es gilt
  \begin{stmts}
    \item Sei $g:[a,b]\to\SetR, g(x):=x$. Dann ist 
      $\sigma_f(z,\xi,g)=\sigma_f(z,\xi)$ und es gilt 
      $\SetRInt_g([a,b])=\SetRInt([a,b])$.
    \item Sei $g:[a,b]\to\SetR, g(x):=c, c\real$. Dann gilt:
      $\SetRInt_g([a,b])=\{f:[a,b]\to\SetR\mid f \text{ beschr"ankt}\}$
    \end{stmts}
  }
% -------------------------------------------------------------------------
\theorem:=>{
  Es gilt
  \[\begin{split}
    f\in\SetRInt_g([a,b]) \equiv 
    & (\forall (z_n) \text{ $z_n$ Zerl. von $[a,b]$ mit $|z_n|\to 0 (n\to\infty)$})\\
    & (\forall (\xi_n) \text{ $\xi_n$ zu $z_n$ passender ZV})
      (\sigma_f(z_n,\xi_n,g)\to\int_a^b f(x)dg(x))
    \end{split}\]
  }
% -------------------------------------------------------------------------
\theorem:
  $g_1,g_2:[a,b]\to\SetR$ beschr"ankt=>{
  \begin{stmts}
    \item $\SetRInt_g([a,b])$ ist reeller Vektorraum.
      Die Abbildung $f\mapsto\int_a^bf(x)dg(x)\,(f\in\SetRInt_g([a,b]))$ ist linear.
    \item $f\in\SetRInt_{g_1}([a,b])\cap\SetRInt_{g_2}([a,b])$, 
      $\alpha,\beta\in\SetR$.
      Dann: $f\in\SetRInt_{\alpha g_1+\beta g_2}$ und 
      \[\int_a^b f(x) d\alpha g_1(x)+\int_a^b f(x) d\beta g_2(x)=
	\int_a^b f(x) d\alpha g_1(x)+\beta g_2(x)
        \]
    \item $a<c<b; f\in\SetRInt_g([a;c])\cap \SetRInt_g([c;b])\cap\SetRInt_g([a,b])$ 
      Dann: $\int_a^b f(x)dg(x)=\int_a^c f(x)dg(x)+\int_c^b f(x)dg(x)$
    \end{stmts}
  }
% -------------------------------------------------------------------------
\remark:{
  $f\in\SetRInt_g([a;c])\cap\SetRInt_g([c;b]) \not\implies f\in\SetRInt_g([a,b])$
  }  
% -------------------------------------------------------------------------
\theorem Partielle Integration (Riemann-Stieltjes):
  $f\in\SetRInt_g([a,b])$=>{
    $g\in\SetRInt_f([a,b])$ und es gilt
    \[ \int_a^b f(x)dg(x)= [f(x)g(x)]_a^b-\int_a^b g(x)df(x) \]
  }
% -------------------------------------------------------------------------
\theorem:
  $f\in\SetCont([a,b]),g\in\SetBV([a,b])$=>{
  Dann ist $f\in\SetRInt_g([a,b])$.
  }  
% -------------------------------------------------------------------------
\theorem:
  $f\in\SetCont([a,b]),g\in\SetCont^1([a,b])$=>{
  $f\in\SetRInt_g([a,b])$ und es gilt
  \[\int_a^b f(x)dg(x)=\int_a^b f(x)\cdot g'(x)dx 
    \]
  }  
% -------------------------------------------------------------------------
\theorem:
  $g:[a,b]\to\SetR \nearrow$, $f_1,f_2\in\SetRInt_g([a,b]), f_1\le f_2$=>
  {
  Dann gilt 
  \[\int_a^b f_1(x)dg(x)\le\int_a^b f_2(x)dg(x)
    \]
  }
% -------------------------------------------------------------------------
\remark:{
  $f\in\SetRInt_g([a,b])$, $g\nearrow$, $m,M\real, m\le f\le M$ auf $[a,b]$.
  Dann gilt
  \[m[g(x)]_a^b\le\int_a^b f(x)dg(x)\le M[g(x)]_a^b
    \]
  }
% -------------------------------------------------------------------------
\theorem:
  $f\in\SetCont([a,b])$, $g\in\SetBV([a,b])$, $F(x):=\int_a^x f(t)dg(t)$=>{
  \begin{stmts}
    \item $F\in\SetBV([a,b])$
    \item $g$ in $x_0$ stetig $\implies$ $F$ in $x_0$ stetig
    \end{stmts}
  }
% -------------------------------------------------------------------------
\remark:{
  Sind $f\in\SetCont([a,b])$, $g\in\SetCont^1([a,b])$, so gilt
  \begin{align*}
    F(x)&=\int_a^x f(t)dg(t)=\int_a^x f(t)g'(t) dt\implies F\in\SetCont^1([a,b]) \\
    F'(x)&=f(x)g'(x)\text{ auf $[a,b]$}
    \end{align*}
  }
% -------------------------------------------------------------------------
\framedmsg{Das war's --- weiter geht's in Ana II :-)}

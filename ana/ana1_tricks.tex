% -*- LaTeX -*-
% -----------------------------------------------------------------------------
\para{Tricks for kicks}
% -----------------------------------------------------------------------------
\trick Grenzwert einer rekursiv def. Folge:{
  Monotonie/Beschr"anktheit beweisen, Grenzwert durch Einsetzen in Definition.
  }
% -----------------------------------------------------------------------------
\trick Geht nicht?:{
  Sch"atz' es ab. Geht immer noch nicht? Versuch's mit Bernoulli.
  Immer noch nicht? Schreib' den Anfang der Reihe hin. Immer noch nicht?
  Versuch' doch mal, den Mittelwertsatz als Absch"atzung zu verwurschten.
  Auch nicht? Dann versuch' mal eine Formel aus Kapitel 4! Geht immer 
  noch nicht? Substituier doch mal wieder!
  }
% -----------------------------------------------------------------------------
\trick Polynom im Z"ahler, Polynom im Nenner?:{
  Klammer' doch $x^z,z\integer$ aus.
  }
% -----------------------------------------------------------------------------
\trick Steht da was von $|a+c|$?:{
  Dann hilft Dir vielleicht $|a+c|=|a-b+b+c|\le|a-b|+|b+c|$.
  }
% -----------------------------------------------------------------------------
\trick Reihenwert berechnen:{
  Versuch mal, zu integrieren, dann zu berechnen, und dann wieder zu 
  differenzieren! (Vor. glm konv.)
  }
% -----------------------------------------------------------------------------
\trick Integriert sich nicht von selbst?:{
  Zwing es: Partielle Integration mit $1$ als $f'$? Partielle Integration,
  bei der immer das gleiche rauskommt: nach Integral aufl"osen!
  So "ahnlich wie $\frac 1{\sqrt{1+x^2}}$? Da steckt die Trigonometrie dahinter!
  Wo wir bei Trigonometrie sind: $\sin^2=1-\cos^2$ und umgekehrt.
  
  Rationale Funktion in Abh"angigkeit von $\sin x$ und $\cos x$? 
  
  Substituiere $t=\tan \frac x 2$. Dann wird aus:
  
  \begin{tabular}{ccc}
    $\sin x$ &$\to$& $\frac{2t}{1+t^2}$ \\
    $\cos x$ &$\to$& $\frac{1-t^2}{1+t^2}$ \\
    $\sin^2 \frac x 2$ &$\to$& $\frac{t^2}{1+t^2}$ \\
    $\cos^2 \frac x 2$ &$\to$& $\frac{1}{1+t^2}$ \\
    \end{tabular}
  
  Rationale Funktion in Abh"angigkeit von $\sqrt{ax^2+bx+c}$ und $x$?
  
  Substituiere $t=\frac{2(ax+b)}{\sqrt{\pm D}}$ mit $D=4ac-b^2$. 
  Dann bleibt irgendetwas von der Form $\sqrt{1+t^2}$,$\sqrt{1-t^2}$ stehen.
  Versuch es dann so:
  
  \begin{tabular}{ccc}
    $\sqrt{1+t^2}$ &$\to$& $t=\sinh s$ \\
    $\sqrt{t^2-1}$ &$\to$& $t=\cosh s$ \\
    $\sqrt{1-t^2}$ &$\to$& $t=\sin/\cos s$
    \end{tabular}
  
  Rationale Funktion in Abh"angigkeit von $\sqrt{ax+b}$, $\sqrt{cx+d}$ und $x$?

  Substituiere $x=\frac{t^2-b}a$.  
  }
% -----------------------------------------------------------------------------
\trick Hilft alles nichts?:{
  Schlaf dr"uber.
  }

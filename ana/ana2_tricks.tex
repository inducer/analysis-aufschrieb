% -----------------------------------------------------------------------------
\para{Tricks for Kicks}
% -----------------------------------------------------------------------------
\trick Operatornormen bestimmen:{
  Benutze folgende Beziehung:
  \[\fnorm A = \sup_{x\ne 0} \frac{\norm{Ax}}{\norm{x}}
             = \sup_{x\ne 0} \norm{ A \frac x {\norm{x}} }
	     = \sup_{\norm{x} = 1} \norm{Ax}
    \]
  Zeige dann eine obere Absch"atzung f"ur den so gewonnenen Ausdruck und 
  konstruiere einen Vektor, so dass der Ausdruck tats"achlich erreicht wird.
  Dann muss der gefundene Term Maximum, und damit Supremum, und damit
  die gesuchte Operatornorm sein.
  }
% -----------------------------------------------------------------------------
\trick Absch"atzen geht nicht?:{
  Hilfe naht! 
  \begin{itemize}
    \item Nutze die "Aquivalenz von Normen auf endlich dimensionalen R"aumen 
      aus.
    \item Sch"atze auf jeden Fall den Betrag ab, nichts anderes!
    \end{itemize}
  }
% -----------------------------------------------------------------------------
\trick Dinge, die man nicht vergessen sollte:{
  Zum Beispiel:
  \begin{itemize}
    \item Banach'scher Fixpunktsatz, falls gezeigt werden soll, dass eine 
      Gleichung (genau) eine L"osung hat.
    \end{itemize}
  }
% -----------------------------------------------------------------------------
\theorem Definitheit einer Matrix:=>{
  Eine Matrix $A$ ist 
  \begin{itemize}
    \item positiv definit $\equiv$ alle Hauptunterdeterminanten positiv
    \item negativ definit $\equiv$ $k$-te Hauptunterdeterminante hat Vz $(-1)^k$
    \item positiv semidefinit $\equiv$ alle Hauptunterdeterminanten $\ge 0$
    \item negativ semidefinit $\equiv$ $k$-te Hauptunterdeterminante hat Vz $(-1)^k$
      oder ist $0$
    \item indefinit $\Leftarrow$ $(\exists k)$($2k$-te Hauptunterdeterminante 
      hat negatives Vz)
    \end{itemize}
  (Quelle: D. Unruh, dominique@unruh.de)
  }
% -----------------------------------------------------------------------------
\trick Als Ungleichung gegebene Mengen:{
  Hier ist das gleiche Verfahren (quadratische Erg"anzung) angebracht, das
  auch bei Quadriken in LA funktioniert.
  }
% -----------------------------------------------------------------------------
\definition $\sigma$-Algebra:{
  Ein Tupel $(X,B)$ mit $B\subseteq\cal P(x)$ hei"st $\sigma$-Algebra $:\equiv$
  \begin{stmts}
    \item $X\in B$
    \item $A\in B$ $\implies$ $X\setminus A\in B$
    \item $A_1,A_2,\ldots\in B$ $\implies$
      \[\bigcap_{i=1}^\infty A_i\in B
        \]
    \end{stmts}
  }

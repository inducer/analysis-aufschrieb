% -*- LaTeX -*-
% -----------------------------------------------------------------------------
\section{Lineare Systeme mit konstanten Koeffizienten}
\label{sec:linear-const-system}
% -----------------------------------------------------------------------------
\definition Exponentialfunktion auf Matrizen:{
  \index{Matrizen>Exponentialfunktion auf}
  Sei $A\in\SetR^{p\times p}$. Dann wird definiert
  \[e^A:=\sum_{k=0}^\infty \frac{A^k}{k!}
    \]
  Diese Reihe ist absolut konvergent.
  }
% -----------------------------------------------------------------------------
\remark Eigenschaften von $e^A$:{
  Sei $A\in\SetR^{p\times p}$.
  \begin{stmts}
    \item $AB=BA$ $\implies$ $e^{A+B}=e^Ae^B=e^Be^A$ (im Allgemeinen sind
      alle drei verschieden)
    \item $e^A$ ist invertierbar und es gilt $(e^A)^{-1}=e^{-A}$
    \item Sei $Y:\SetR\to\SetR^{p\times p}$ definiert durch $Y(x):=e^{xA}$ mit
      $A\in\SetR^{p\times p}$ fest.  Dann ist 
      $Y\in\SetCont^\infty(\SetR,\SetR^{p\times p})$ und es gilt
      $Y^{(n)}(x)=A^ne^{xA}$. Insbesondere ist also $e^{xA}$ ein 
      Fundamentalsystem von $Y'(x)=AY(x)$, und zwar genau dasjenige,
      das bei $x=0$ die Einheitsmatrix $I$ liefert.
    \item Mit jeder submultiplikativen Matrixnorm $\fnorm\cdot$ gilt
      $\fnorm {e^{xA}}\leq e^{|x|\,\fnorm A}$.
    \end{stmts}
  }
% -----------------------------------------------------------------------------
\remark Eigenvektoren:{
  Ist nun $\lambda$ reeller Eigenwert zu $A\in\SetR^{p\times p}$, 
  $v$ zugeh"origer reeller EV, so gilt:
  \[e^{xA}v=\sum_{k=0}^\infty \frac {(xA)^kv}{k!}
    =\sum_{k=0}^\infty \frac {x^k\lambda^k}{k!} v
    =e^{\lambda x}v
    \]
  Dies bedeutet ohne Umwege, dass $y(x):=e^{\lambda x}v$ eine L"osung
  des Dgl-Systems ist.
  }
% -----------------------------------------------------------------------------
\deduction Berechnung eines Fundamentalsystems:{
  Die explizite Berechnung von $e^{xA}$ ist in vielen F"allen nicht 
  durchf"uhrbar. Hier soll deswegen ein anderes Verfahren vorgestellt werden, 
  das mit Hilfe der Jordan'schen Normalform ebenfalls auf ein FS f"uhrt.
  
  Da i.A. EWe und EVen nicht reell sind, ist es naheliegend,
  $y'(x)=Ay(x)$ im $\SetC^p$ statt im $\SetR^p$ zu betrachten.
  Sei $y:=\xi+i\eta$.
  F"ur die (weiterhin reelle) Matrix $A$ ist dann 
  \[\Re \{Ay\}=A\xi\qquad \Im \{Ay\}=A\eta
    \]
  Fast alle Definitionen und S"atze aus Ana II "ubertragen sich w"ortlich,
  wenn man den $\SetC^p$ mit der Euklidnorm
  \[\norm y_C=\sqrt{\norm\xi^2+\norm\eta^2}
    \]
  versieht.
  Falls ein EW $\lambda_1\not\real$ zum EV $v\not\real^p$ existiert,
  so sind doch $\Re\{e^{\lambda_1x}v_1\}$ und $\Im\{e^{\lambda_1x}v_1\}$
  reelle L"osungen. Allgemein gilt: Ist $y:I\to\SetC^p$ mit
  $y(x)=\xi(x)+i\eta(x)$ eine L"osung von $y'(x)=A(x)y(x)$, so
  gilt
  \[\xi'(x)+i\eta'(x)=A\xi(x)+iA\eta(x)
    \]
  Zu $A$ existiert nun ein $T\in\SetC^{p\times p}$ 
  so, dass gilt $J:=T^{-1}AT$ und $J$ Jordan-Normalform besitzt.
  Dann ergibt sich $J^k=T^{-1}A^kT$ f"ur $k\natural$. und damit
  \[e^{xJ}=T^{-1}e^{xA}T
    \]
  wobei man, um $e^{xJ}$ zu berechnen, lediglich $e^{xJ_i}$ f"ur jedes
  Jordank"astchen $J_i$ (mit $j=1,\ldots,q$) einzeln berechnen muss,
  dies geht relativ einfach folgenderma"sen: Habe $J_i$ die Gr"o"se
  $r\times r$.
  
  Ist $r=1$, so haben wir kein Problem. Andernfalls betrachte die Matrix
  \[\tilde J_i:=\begin{pmatrix}
      0 & 0 & \cdots & 0 & 0 \\
      1 & 0 & \cdots & 0 & 0 \\
      0 & 1 & \cdots & 0 & 0 \\
      \vdots & \vdots & \ddots & \vdots & \vdots \\
      0 & 0 & \cdots & 1 & 0 \\
      \end{pmatrix}
    \]
  Damit ist $J_i=\lambda I+\tilde J_i$. F"ur $\tilde J_i^k$ ist zu beachten,
  dass die $1$er-Diagonale von $J_i^0=I$ aus gez"ahlt, um $k$ Schritte
  nach links unten ``gewandert'' ist. Damit ergibt sich sofort
  \[e^{x\tilde J_i}=\begin{pmatrix}
      1 & 0 & 0 & \cdots & 0 \\
      x & 1 & 0 & \cdots & 0 \\
      \frac {x^2} 2 & x & 1 & \cdots & 0 \\
      \vdots & \vdots & \vdots & \ddots & \vdots \\
      \frac {x^{r-1}}{(r-1)!} & \cdot & \cdot & \cdots & 1 
      \end{pmatrix}
    \]
  und weiterhin $e^{xJ_i}=e^{\lambda x}e^{x\tilde J_i}$.
  
  Nun gilt: $T\cdot e^{xJ}=e^{xA}T$, und da nun $e^{xA}T$ ebenfalls
  ein Fundamentalsystem (bestehend aus LK der Spalten von $e^{xA}$),
  muss $Te^{xJ}$ ebenfalls eines sein. Wie oben gesehen, sind die
  Spalten von $Te^{xJ}$ Polynome mit Koeffizienten $q_i\in\SetC^p$
  vom Grad $\leq r-1$, wobei $r$ in diesem Fall die Dimension des
  gr"o"sten zu einem EW $\lambda$ geh"origen Jordanblocks darstellen soll.
  
  Insgesamt ergibt sich also f"ur den Fall mehrfacher Eigenwerte das
  folgende
  }
% -----------------------------------------------------------------------------
\remark Rechenverfahren bei mehrfachen Eigenwerten:{
  Treten mehrfache Eigenwerte auf, so f"uhrt das schlichte Ausrechnen
  von Eigenwerten und -vektoren nicht mehr zwingend auf ein Fundamentalsystem.
  Sei $\lambda$ nun mehrfacher Eigenwert, w"ahle EV $v_1$ zu $\lambda$.
  Damit ist $y_1(x):=e^{\lambda x}v_1$ eine L"osung.
  
  N"achster Ansatz: $y_2(x):=e^{\lambda x}(xv_1+v_2)$, dies f"uhrt auf
  ein LGS f"ur $v_2$, n"amlich $(A-\lambda I)v_2=v_1$. Induktiv 
  fortgesetzt ergibt sich der Ansatz
  \[y_i(x):=e^{\lambda x}\left(\sum_{j=0}^{i-1} \frac 1 {j!} v_{i-j} x^j\right)
    =e^{\lambda x}\left(\frac{v_i}{0!}+\frac{v_{i-1}x}{1!}+\cdots+\frac{v_1x^{i-1}}{(i-1)!}\right)
    \]
  der wiederum auf das LGS $(A-\lambda I)v_i=v_{i-1}$ f"uhrt.
  }
% -----------------------------------------------------------------------------
\section{Explizite Dgl'en h"oherer Ordnung}
% -----------------------------------------------------------------------------
\deduction "Ubertragung von Ergebnissen aus Ordnung 1:{
  Sei $D\subseteq\SetR\times\SetR^{kp}$ und $f:D\to\SetR^p$. Wir 
  betrachten die Dgl
  \begin{equation*}
    \tag{*} y^{(k)}=f(x,y(x),y^{(2)}(x),\ldots,y^{(k-1)}(x))
    \end{equation*}
  Viele Ergebnisse "uber Dgl'en 1. Ordnung lassen sich folgenderma"sen auf
  $(*)$ "ubertragen: Betrachte $F:D\to\SetR^{kp}$ definiert durch
  \[F(x,z):=(z_2,z_3,\ldots,z_k,f(x,z))
    \]
  Beachte $z=(z_1,\ldots,z_k)\in\SetR^{kp}$ mit $z_j\in\SetR^p$. 
  Betrachte nun die zugeh"orige Dgl. 1. Ordnung
  \begin{equation*}
    \tag{**} z'(x)=F(x,z(x))
    \end{equation*}
  gleichbedeutend mit
  \[\begin{pmatrix}
      z_1'(x)\\ \vdots \\ z_{k-1}'(x) \\ z_k'(x)
      \end{pmatrix}=
    \begin{pmatrix}
      z_2(x)\\ \vdots \\ z_k(x) \\ f(x,z_1(x),\ldots,z_k(x))
      \end{pmatrix}
    \]
  Damit ist klar, wie sich eine L"osung einer Gleichung der Form $(*)$ 
  in eine L"osung einer Gleichung der Form $(**)$ und umgekehrt verwandeln
  l"asst. In diesem Sinne sind $(*)$ und $(**)$ "aquivalent.
  }
% -----------------------------------------------------------------------------
\definition Lipschitz-Stetigkeit:{
   \index{Stetigkeit>Lipschitz-}
   Sei $D=[a,b]\times\SetR^{kp}$, $f:D\to\SetR^p$. 
   Weiter existiere ein $L>0$ so, dass
   \[(\forall (x,z),(x,\tilde z)\in D)(\norm{f(x,z)-f(x,\tilde z)}_1\leq L\norm{z-\tilde z}_2)
     \]
   Dann nennt man $f$ Lipschitz-stetig auf $D$ bzgl. der letzten 
   $kp$ Perameter. 
   (Achtung: $\norm\cdot_1$ ist Norm auf $\SetR^p$, $\norm\cdot_2$ dagegen
   Norm auf $\SetR^{kp}$)
   }
% -----------------------------------------------------------------------------
\theorem Satz von Picard-Lindel"of:
  $f\in\SetCont([a,b]\times\SetR^{kp},\SetR^{p})$ Lipschitz-stetig bez"uglich
  der letzten $kp$ Parameter=>{
  \index{Picard-Lindel"of>Satz von}
  Dann ist das AWP
  \[y^{(k)}=f(x,y(x),y^{(2)}(x),\ldots,y^{(k-1)}(x))\qquad 
    y^{(i)}(x_0)=y_i\;(0\leq i\leq i-1)
    \]
  (mit $x_0\in[a,b],y_i\in\SetR^p$ fest) eindeutig l"osbar auf $[a,b]$.
  }
% -----------------------------------------------------------------------------
\theorem Satz von Peano:
  $D\subseteq\SetR^p$ offen, $f\in\SetCont(D,\SetR^p)$,
  $(x_0,y_0,\ldots,y_{k-1})\in D$=>{
  \index{Peano>Satz von}
  Dann besitzt das AWP 
  \[y^{(k)}=f(x,y(x),y^{(2)}(x),\ldots,y^{(k-1)}(x))\qquad 
    y^{(i)}(x_0)=y_i\;(0\leq i\leq i-1)
    \]
  (mit $x_0\in[a,b],y_i\in\SetR^p$ fest) eine L"osung.
  }
% -----------------------------------------------------------------------------
\subsection{Lineare Differentialgleichungen $k$-ter Ordnung}
% -----------------------------------------------------------------------------
\definition Lineare Differentialgleichung (III):{
\index{Differentialgleichung>lineare}%
\index{Differentialgleichung>homogene}%
\index{Differentialgleichung>inhomogene}%
\index{Lineare Differentialgleichung}%
\index{Homogene Differentialgleichung}%
\index{Inhomogene Differentialgleichung}%
  Eine Gleichung der Form
  \begin{equation*}
    \tag{*} y^{(k)}(x)+\sum_{i=0}^{k-1}a_iy^{(i)}(x)=r(x)
    \end{equation*}
  mit $a_i,r\in\SetCont([a,b],\SetR)$.
  hei"st lineare Differentialgleichung $k$-ter Ordnung.
  Bezeichnung: $(*)$ hei"st homogen $\equiv$ $r=0$, ansonsten
  inhomogen.
  }
% -----------------------------------------------------------------------------
\remark:{
  Beachte:
  \begin{itemize}
    \item Hier ist $f:[a,b]\to\SetR$ die Funktion
      \[f(x,z_1,\ldots,z_k)=r(x)-\sum_{i=0}^{k-1}a_i(x)z_{i+1}
        \]
    \item $f$ erf"ullt gerade die Voraussetzungen des Satzes von
      Picard-Lindel"of. Damit ist $(*)$ eindeutig l"osbar auf $[a,b]$.
    \end{itemize}
  }
% -----------------------------------------------------------------------------
\remark Der homogene Fall $(r=0)$:{
  Das zugeh"orige System 1. Ordnung ist $z'(x)=A(x)z(x)$ mit 
  $A(x)\in\SetCont([a,b],\SetR^{p\times p}$ und
  \[A(x):=\begin{pmatrix}
      0 & 1 & 0 & \dots & 0 \\
      0 & 0 & 1 & \dots & 0 \\
      \vdots & \vdots & \vdots & \ddots & \vdots \\
      -a_0(x) & -a_1(x) & -a_2(x) & \dots & -a_{k-1}(x) 
      \end{pmatrix}
    \]
  }
% -----------------------------------------------------------------------------
\theorem:
  $r=0$, $V=\{y\in\SetCont^k([a,b],\SetR)\mid y \text{ l"ost } (*)\}$=>{
  Dann ist $V$ VR und es gilt $\dim V=k$.
  }
% -----------------------------------------------------------------------------
\remark:{
  Auch hier hei"st jede Basis von $V$ \indexthis{Fundamentalsystem}.
  }
% -----------------------------------------------------------------------------
\subsection{Lineare Dgl'en $k$-ter Ordnung mit konstanten Koeffizienten}
% -----------------------------------------------------------------------------
\remark:{
  Falls $a_0,\ldots,a_{k-1}\in\SetR$ nicht von $x$ abh"angen, ist
  \[A:=\begin{pmatrix}
      0 & 1 & 0 & \dots & 0 \\
      0 & 0 & 1 & \dots & 0 \\
      \vdots & \vdots & \vdots & \ddots & \vdots \\
      0 & 0 & 0 & \dots & 1 \\
      -a_0 & -a_1 & -a_2 & \dots & -a_{k-1}
      \end{pmatrix}
    \]
  eine $k\times k$-Matrix mit dem charakteristischen Polynom
  \[p(x):=\det(A-\lambda I)=(-1)^k(\lambda^k+a_{k-1}\lambda^{k-1}+\cdots+\lambda a_1+a_0)
    \]
  }
% -----------------------------------------------------------------------------
\theorem:
  $\lambda$ $r$-fache Nullstelle von $p(x)$=>{
  Dann entsprechen $\lambda$ die $r$ L"osungen
  \[e^{\lambda x},xe^{\lambda x},\dots,x^{r-1}e^{\lambda x}\qquad (x\real)
    \]
  Ist dabei $\lambda=\xi+i\eta\not\real$, so erh"alt man $2r$ L"osungen
  \[x^qe^{\xi x}\cos(\eta x),x^qe^{\xi x}\sin(\eta x)
    \qquad (x\real,q=0,\ldots,r-1)
    \]
  Dabei streicht man diejenigen $r$ L"osungen, die zu $\conjug \lambda$
  geh"oren. Auf diese Weise erh"alt man ein reelles FS f"ur die
  homogene Dgl
  \[y^{(k)}(x)+\sum_{i=0}^{k-1}a_iy^{(i)}(x)=0
    \]
  }
% -----------------------------------------------------------------------------
\remark Im Falle der Inhomogenit"at:{
  Bei speziellen $r(x)$ (z.B. bei Kombinationen aus Polynomen,
  Exponential- und trigonometrischen Funktionen)
  kommt man evtl. mit besonderen Ans"atzen durch,
  die in der Regel recht "ahnlich zu $r(x)$ aussehen, evtl. noch mit
  einem Polynom/Faktor multipliziert.
  }
% -----------------------------------------------------------------------------
\section{Potenzreihenentwicklung von L"osungen}
% -----------------------------------------------------------------------------
\definition Mehrdimensionale Potenzreihe:{
  \index{Potenzreihe>mehrdimensionale}
  Seien $a_{jk}\real$ ($j,k\nnatural$) und $(x_0,y_0)\in\SetR^2$. Dann hei"st
  \[f(x,y):=\sum_{j,k=0}^\infty a_{jk}(x-x_0)^j(y-y_0)^k
    \]
  eine zweidimensionale Potenzreihe und weiterhin $f(x,y)$ in eine Potenzreihe
  entwickelbar, falls die Reihe in einer $\epsilon$-Umgebung von $(x_0,y_0)$
  absolut konvergiert.
  
  Definition f"ur mehr Dimensionen analog.
  }
% -----------------------------------------------------------------------------
\convention{
  Seien $D\subseteq\SetR\times\SetR$ offen, $(x_0,y_0)\in D$ und
  $f\in\SetCont^\infty(D,\SetR)$ eine Funktion, die in einer
  Umgebung von $(x_0,y_0)$ in eine Potenzreihe entwickelbar ist,
  sei also
  \[f(x,y)=\sum_{j,k=0}^\infty a_{jk}(x-x_0)^j(y-y_0)^k
    \]
  }
% -----------------------------------------------------------------------------
\remark:{
  Ohne weitere Umst"ande erhalten wir, dass das AWP 
  $y'(x)=f(x,y(x)),y(x_0)=y_0$ eindeutig l"osbar ist, denn
  aus der $\SetCont^\infty$-Eigenschaft von $f$ folgen unmittelbar 
  Stetigkeit und lokale Lipschitz-Stetigkeit.
  }
% -----------------------------------------------------------------------------
\deduction Majorantenmethode:{
  \index{Majorantenmethode}
  Sei also nun $I:=(\omega_-,\omega_+)$ und $y:I\to\SetR$ die L"osung.
  
  Ist $y$ in der N"ahe von $x_0$ in eine PR entwickelbar? 
  Ja, dies kann mit Hilfe des folgenden Verfahrens gezeigt werden.
  
  Zun"achst einmal kann das AWP $y'(x)=f(x,y(x)),y(x_0)=y_0$ durch
  $z(x):=y(x+x_0)-y_0$ so transformiert werden, dass gilt:
  $z'(x)=f(x,z(x)),z(0)=0$. Daher betrachten wir im Folgenden
  o.B.d.A. den Fall, dass $(x_0,y_0)=(0,0)$ ist.
  \begin{align*}
    y''(x)=&f_x(x,y(x))+f_y(x,y(x))f(x,y(x))\\
    y'''(x)=&f_{xx}(x,y(x))+f_{xy}(x,y(x))f(x,y(x))+f_{yx}(x,y(x))f(x,y(x))+\\
      &f_{yy}(x,y(x))(f(x,y(x)))^2+\\
      &f_y(x,y(x))(f_x(x,y(x))+f_y(x,y(x)))f(x,y(x))
    \end{align*}
  Beobachtung: $y^{(k)}(x)$ kann f"ur jedes $k\natural$ dargestellt werden
  als endliche Summe von nichtnegativen Vielfachen von Produkten der Art 
  $f(x,y(x))\cdot g(x,y(x))$, wobei $g$ eine partielle Ableitung von $f$
  ist.
  
  Beachte, dass unter den gegebenen Voraussetzungen gilt: ($a_{jk}$ waren
  die Koeffizienten der Potenzreihe von $f$)
  \[a_{jk}=\frac 1 {j!k!} \left( \frac{\partial^j}{\partial x^j}
    \frac{\partial^k}{\partial y^k} f\right)(0,0)
    \]
  Somit erh"alt man speziell f"ur $x=x_0=0$ einige Abh"angigkeiten zwischen 
  den $a_{jk}$ und den folgenderma"sen definierten
  \[a_n:=\frac{y^{(n)}(0)}{n!}
    \]
  und zwar z.B. (Beachte: Die Fakult"aten dienen lediglich zur Korrektur
  der entsprechenden Vorfaktoren der $a_{jk}$ und $a_n$)
  \begin{align*}
    a_0&=0,a_1=f(0,0)=a_{00}\\
    a_2&=\frac 1 {2!} (a_{10}+a_{01}a_{00})\\
    a_3&=\frac 1 {3!} (2!a_{20}+2!a_{00}a_{11}+2!a_{02}a_{00}^2+a_{10}a_{01}+a_{01}^2a_{00})
    \end{align*}
  Also sind die $a_n$ darstellbar als Summe von nichtnegativen
  Vielfachen von Produkten der $a_jk$. 
  
  Wegen $\limes {j,k}->\infty |a_{jk}|\epsilon^j\epsilon^k=0$ existiert
  ein $M>0$ so, dass $|a_{jk}|\epsilon^j\epsilon^k\leq M$ f"ur alle
  $j,k\nnatural$. Man setzt nun
  \[A_{jk}:=\frac M {\epsilon^j\epsilon^k}
    \]
  in einer Reihe 
  \[F(x,y):=\sum_{j,k=0}^\infty A_{jk} x^jy^k
    =\sum_{j,k=0}^\infty \frac M {\epsilon^j\epsilon^k} x^jy^k
    \] 
  Es gilt $|a_{jk}|\leq A_{jk}$. Sei au"serdem $\epsilon>0$ so, 
  dass $F$ absolut konvergent f"ur alle $(x,y)\in(-\epsilon,\epsilon)^2$. 
  F"ur $F$ l"asst sich ein expliziter Ausdruck angeben. Weil $F$ absolut
  konvergiert, gilt folgende Umordnung:
  \[F(x,y)=M \sum_{j=0}^\infty \left(\frac x \epsilon\right)^j\sum_{k=0}^\infty \left(\frac y \epsilon\right)^k
    =\frac M {\left(1-\frac x \epsilon\right)\left(1-\frac y \epsilon\right)}
    \]
  Mit Hilfe des Verfahrens f"ur Differentialgleichungen mit getrennten 
  Ver"anderlichen erhalten wir eine L"osung des AWP
  $w'(x)=F(x,y(x)),w(0)=0$, n"amlich
  \[w(x)=\epsilon-\epsilon\sqrt{1+2M\log\left(1-\frac x \epsilon\right)}
    \]
  Mit Hilfe der beiden Potenzreihenentwicklungen
  \begin{align*}
    \log(1-t)&=\sumn 1 -\frac {t^n}n\\
    \sqrt{1+t}&=\sumn 0 \combover{\frac 1 2}n t^n
    \end{align*}
  "uberlegt man sich leicht, dass $w(x)$ um $0$ eine Potenzreihendarstellung 
  mit Konvergenzradius $R<\epsilon$ besitzt. F"ur 
  \[A_n:=\frac{w^{(n)}(0)}{n!}
    \]
  folgt aus obigen Formeln $|a_n|\leq A_n$ auf die folgende Art und Weise 
  (hier als Beispiel f"ur $A_2$)
  \begin{multline*}
    |a_2|=\frac 1 2|a_{10}+a_{01}+a_{00}|
    \leq \frac 1 2 (|a_10|+|a_{01}|+a_{00}|)\\
    \leq \frac 1 2 (A_{10}+A_{01}+A_{00})
    = A_2
    \end{multline*}
  (wesentlich: die Nichtnegativit"at der Vorfaktoren in den obigen Summen!)
  Deswegen hat auch die Reihe
  \[\tilde y(x)=\sumn 0 a_n x^n
    \]
  mindestens den Konvergenzradius $R$. 
  (Bem: Es gilt $|w(x)|<\epsilon$ auf $U_R(0)$ wegen der 
  L"osbarkeitseigenschaften von $w'(x)=F(x,w(x))$)
  Es bleibt nun zu zeigen, dass gilt $y=\tilde y$ auf $U_R(0)$ gilt.
  Da $\tilde y'(x)$ und $f(x,\tilde y)$ Potenzreihen mit Konvergenzradien
  $\geq R$ sind und da die $a_n$ so bestimmt sind, dass
  \[(\tilde y')^{(n)}(0)=\tilde y^{(n+1)}=\frac {d^n}{dx^n} f(x,\tilde y(x))|_{x=0}
    \qquad (n\nnatural)
    \]
  folgt $\tilde y'(x)=f(x,\tilde y(x))$ ($x\in U_R(0)$), Weiter ist 
  $\tilde y(0)=a_0=0$. Wegen der Eindeutigkeit der L"osung des AWPs folgt
  nun $y(x)=\tilde y(x)$ auf $U_R(0)$.
  
  Beachte: diese Methode funktioniert in analoger Weise auch f"ur
  Systeme und Dgl'en h"oherer Ordnung.
  }
% -----------------------------------------------------------------------------
\remark:{
  Generell ist der Potenzreihen-Ansatz (unabh"angig von obiger Theorie) ein
  L"osungsverfahren f"ur AWPe, wenn man nachweisen kann, dass eine
  formal berechnete PRe eine L"osung darstellt. 
  
  Solche formalen Potenzreihen erh"alt man wegen des Identit"atssatzes 
  f"ur Potenzreihen induktiv durch Koeffizientenvergleich in der
  durch das AWP $y'(x)=f(x,y(x))$ (implizit) gegebenen Rekursionsformel.
  }
% -----------------------------------------------------------------------------
\section{Stetige Abh"angigkeit von L"osungen}
% -----------------------------------------------------------------------------
\theorem:
  $f,f_k\in\SetCont([a,b]\times\SetR^p,\SetR^p)$ mit
  $\norm{f_k(x,y)}\leq M$ ($k\natural,(x,y)\in[a,b]\times\SetR^p$),
  $\limes k->\infty f_k(x,y)=f(x,y)$ glm auf $[a,b]\times \closure{U_{M(b-a+1)}(0)}$.
  $(x_{0k})_{k=1}^\infty\subset [a,b]$, 
  $(y_{0k})_{k=1}^\infty\subset\closure{U_M(0)}$,
  $(y_k)_{k=1}^\infty\subset\SetCont([a,b],\SetR^p)$ Funktionenfolge mit
  $y_k'(x)=f_k(x,y_k(x)),y_k(x_{0k})=y_{0k}$ f"ur $k\natural$=>{
  Dann besitzt $(y_k)$ eine auf $[a,b]$ gleichm"a"sig konvergente Teilfolge,
  deren Grenzfunktion $y:[a,b]\to\SetR^p$ L"osung der
  Dgl $y'(x)=f(x,y(x))$ auf $[a,b]$ ist.
  
  Gilt zus"atzlich $x_{0k}\to x_0$, $y_{0k}\to y_0$ f"ur $k\to\infty$ und
  ist das AWP $y'(x)=f(x,y(x)),y(x_0)=y_0$ eindeutig l"osbar, so konv.
  $(y_k)$ glm gegen die L"osung des AWP.
  }
% -----------------------------------------------------------------------------
\theorem:
  $R>0$, $y_0\in\SetR^p$, $f\in\SetCont([a,b]\times U_R(y_0),\SetR^p)$.
  $y'(x)=f(x,y(x)),y(x_0)=y_0$ eindeutig l"osbar auf $[a,b]$, 
  $y:[a,b]\to\SetR^p$ die L"osung=>{
  Dann existiert zu jedem $\epsilon>0$ ein $\delta>0$ mit folgenden 
  Eigenschaften:
  
  Sind $\tilde f\in\SetCont([a,b]\times U_R(y_0),\SetR^p)$, 
  $\tilde y_0\in U_R(y_0),\tilde x_0\in[a,b]$ und gilt
  \[\norm{f(x,y)-\tilde f(x,y)}\leq\delta \qquad (x,y)\in[a,b]\times U_R(y_0)
    \]
  weiterhin $\norm{y_0-\tilde y_0}<\delta$, $|x_0-\tilde x_0|<\delta$, so
  gilt f"ur jede nicht fortsetzbare L"osung $\tilde y:J\to\SetR^p$
  des AWPs $\tilde y'(x)=\tilde f(x,\tilde y(x)),\tilde y(\tilde x_0)=\tilde y_0$:
  
  $J=[a,b]$ und $\norm{y(x)-\tilde y(x)}<\epsilon$ f"ur $x\in[a,b]$.
  }
% -----------------------------------------------------------------------------
\lessertheorem:
  $D=[a,b]\times\SetR^p$, $f\in\SetCont(D,\SetR^p)$ und $f$ Lipschitz-stetig
  bzgl. $y$ mit Konstante $L$=>{
  Beachte: Obige Voraussetzungen entsprechen gerade denen des Satzes
  von Picard-Lindel"of.
  Betrachte die AWP'e
  \begin{align*}
    y_1'(x)&=f(x,y_1(x)),y_1(x_0)=y_{01}\\
    y_2'(x)&=f(x,y_2(x)),y_2(x_0)=y_{02}
    \end{align*}
  Dann ergibt sich, dass 
  \[\norm{y_1(x)-y_2(x)}\leq e^{L(b-a)}\norm{y_{01}-y_{02}}\qquad (x\in [a,b])
    \]
  (etwa ``Lipschitz-stetige Abh"angigkeit vom Anfangswert'')
  Daher ergibt sich die stetige Abh"angigkeit der L"osung des AWP vom
  Anfangswert in folgender Form:
  \[\epsn(\exists \delta>0)(\norm{y_{01}-y_{02}}<\delta\implies
    \norm{y_1-y_2}_\infty<\epsilon)
    \]
  (W"ahle $\delta:=\frac\epsilon{e^{L(b-a)}}$)
  }
% -----------------------------------------------------------------------------
\section{Differentialungleichungen}
% -----------------------------------------------------------------------------
\convention{
  Gelte stets $p=1$.
  }
% -----------------------------------------------------------------------------
\theorem:
  $D\subseteq\SetR^2$ beliebig, $f:D\to\SetR$,
  $u,v\in\SetCont^1([a,b],\SetR)$ mit $(x,u(x)),(x,v(x))\in D$ f"ur
  $x\in[a,b]$,$u(a)\leq v(a)$, $u'(x)-f(x,u(x))<v'(x)-f(x,v(x))$ auf $[a,b]$=>{
  \label{the:diff-ineq}
  Dann gilt $u(x)<v(x)$ f"ur $x\in(a,b]$.
  }
% -----------------------------------------------------------------------------
\remark:{
  \ref{the:diff-ineq} gilt nicht f"ur ``$\leq$'' anstelle von $<$ in der
  Voraussetzung. (Gegenbeispiel: $f(x,y)=2\sqrt{|y|}$ mit $u(x)=x^2,v(x)=0$
  an der Stelle $0$)
  }
% -----------------------------------------------------------------------------
\remark Gewinnung von Absch"atzungen der L"osungen:{
  Insbesondere gilt unter der zus"atzlichen Voraussetzung, dass 
  $D$ offen, $f\in\SetCont(D,\SetR)$,
  $u,v\in\SetCont^1([x_0,b],\SetR)$ mit $u(x_0)\leq y_0\leq v(x_0)$ und
  \[u'(x)-f(x,u(x))<0<v'(x)-f(x,v(x))
    \]
  dass f"ur eine L"osung $y:[x_0,b]\to\SetR$ des AWP's 
  $y'(x)=f(x,y(x)),y(x_0)=y_0$ gilt $u(x)<y(x)<v(x)$ f"ur $x\in(x_0,b]$.
  
  Funktionen mit obigen Eigenschaften hei"sen \indexthis{Oberfunktionen}
  bzw. \indexthis{Unterfunktionen}.
  }
% -----------------------------------------------------------------------------
\theorem:
  $D=[a,b)\times\SetR$ ($b=\infty$ zugelassen),
  $f\in\SetCont(D,\SetR)$, $(x_0,y_0)\in D$=>{
  \label{the:ode-extremal}
  Dann besitzt das AWP $y'(x)=f(x,y(x)),y(x_0)=y_0$ 
  eine gr"o"ste nach rechts nicht fortsetzbare L"osung 
  $\overline y:[x_0,\overline\omega_+)\to\SetR$ 
  und eine kleinste nach rechts nicht fortsetzbare L"osung 
  $\underline y:[x_0,\underline\omega_+)\to\SetR$, so dass
  f"ur jede nicht fortsetzbare L"osung
  $y:[x_0,\omega_+)\to\SetR$ mit
  $\eta^*:=\min\{\overline\omega_+,\underline\omega_+\}$ gilt
  $\omega_+\geq\eta^*$ und $\underline y\leq y\leq\overline y$ auf
  $[x_0,\eta^*)$.
  
  Die L"osungen $\overline y,\underline y$ sind dabei eindeutig bestimmt
  und monoton wachsend vom Anfangswert abh"angig.
  }
% -----------------------------------------------------------------------------
\remark:{
  Dazu gibt es zu sagen:
  \begin{stmts}
    \item Ist das AWP $y'(x)=f(x,y(x)),y(x_0)=y_0$ f"ur jedes $y_0\in\SetR$
      nach rechts eindeutig l"osbar, so h"angt die L"osung monoton wachsend
      vom AW ab.
    \item Eine entsprechende "Uberlegung ist nat"urlich auch ``nach links''
      m"oglich.
    \item Der anschlie"sende \ref{the:ode-extremal-weak} ist eine 
      Abschw"achung von \ref{the:ode-extremal}.
    \end{stmts}
  }
% -----------------------------------------------------------------------------
\theorem:
  $f\in\SetCont([a,b]\times\SetR,\SetR)$ und beschr"ankt, $y_0\in\SetR$=>{
  \label{the:ode-extremal-weak}
  Dann besitzt das AWP $y'(x)=f(x,y(x)),y(a)=y_0$ eine gr"o"ste und
  eine kleinste L"osung $\overline y,\underline y:[a,b]\to\SetR$. Diese
  sind eindeutig bestimmt und monoton wachsend vom Anfangswert abh"angig.
  }
% -----------------------------------------------------------------------------
\lessertheorem:
  $f\in\SetCont([a,b],\SetR)$ lokal Lipschitz-stetig bzgl. $y$,
  $u,v\in\SetCont^1([a,b],\SetR)$ mit $u'(x)-f(x,u(x))<0<v'(x)-f(x,v(x))$,
  $u(a)\leq y_0\leq v(a)$=>{
  Dann ist das AWP $y'(x)=f(x,y(x))$ eindeutig l"osbar und die L"osung 
  $y:[a,b]\to\SetR$ dieses AWP's erf"ullt
  \[u(x)\leq y(x)\leq v(x) \qquad (x\in[a,b])
    \]
  In diesem Fall hei"sen $u,v$ wieder Ober- bzw \indexthis{Unterfunktion}
  \index{Oberfunktion} des AWP.
  }
% -----------------------------------------------------------------------------
\section{Randwertprobleme}
% -----------------------------------------------------------------------------
\definition Randwertproblem 2. Ordnung:{
  Sei $D\subseteq\SetR^{2p}$ und $f:[a,b]\to\SetR^p$ eine Funktion.
  Das Randwertproblem/\indexthis{RWP} besteht darin, Funktionen 
  $y:[a,b]\to\SetR^p$ zu finden mit
  \[\left\{\begin{array}{l}
      y''(x)=f(x,y(x))y'(x))\\
      R_a(y)=0\qquad R_b(y)=0
      \end{array}\right.
    \]
  Dabei sind $R_a(y)$ und $R_b(y)$ Bedingungen f"ur $x=a$ und $x=b$.
  Typische Beispiele sind mit $\gamma\real$ etwa:
  \begin{center}\begin{tabular}{lll}
    $R_a(y)=y(a)-\gamma$ & Randbedingung 1. Art & Dirichlet'sche RB \\
    $R_a(y)=y'(a)-\gamma$ & Randbedingung 2. Art & Neumann'sche RB \\
    $R_a(y)=\alpha_1y(a)+\alpha_2y'(a)-\gamma$ & Randbedingung 3. Art 
      & Gemischte RB
    \end{tabular}\end{center}
  Im folgenden betrachten wir meist den Fall $p=1$.
  }
% -----------------------------------------------------------------------------
\example:{
  Es k"onnen jede Menge F"alle auftreten:
  \begin{itemize}
    \item $y''(x)-\pi^2y(x)$, $y(0)=0$, $y(1)=0$ 
      $\implies$ $y(x)=c\sin(\pi x)$, $c\real$
    \item $y''(x)-\pi^2y(x)+1$, $y(0)=0$, $y(1)=0$ 
      $\implies$ keine L"osung
    \item $y''(x)-\pi^2y(x)$, $y(0)=0$, $y'(1)=0$ 
      $\implies$ $y(x)=0$ eindeutig
    \end{itemize}
  wobei in jedem Fall von den m"oglichen L"osungen der Dgl ausgegangen
  wird.  
  }
% -----------------------------------------------------------------------------
\subsection{Der lineare Fall}
% -----------------------------------------------------------------------------
\deduction L"osbarkeit von linearen RWP:{
  Seien $r,a_0,a_1\in\SetCont([a,b],\SetR)$. Wir betrachten das RWP
  \[(*)\left\{\begin{array}{l}
      y''(x)+a_1(x)y'(x)+a_0(x)y(x)=r(x)\\
      \alpha_1y(a)+\alpha_2y'(a)=\gamma_a\\
      \beta_1y(b)+\beta_2y'(b)=\gamma_b\\      
    \end{array}\right.
    \]
  Seien dabei $\alpha_i,\beta_i\real,\alpha_1^2+\alpha_2^2>0,
  \beta_1^2+\beta_2^2>0$. Nun sei $y_1,y_2:[a,b]\to\SetR$ ein 
  Fundamentalsystem der homogenen Gleichung
  \[y''(x)+a_1(x)y'(x)+a_0(x)y(x)=0
    \]
  und $y_s:[a,b]\to\SetR$ eine L"osung der Dgl in $(*)$. Die L"osung dieser
  Dgl sind also Funktionen der Form $y=y_s+c_1y_1+c_2y_2$ mit $c_1,c_2\real$.
  Die Randbedingungen sind genau dann erf"ullt, wenn 
  $\begin{pmatrix}c_1\\c_2\end{pmatrix}$ L"osung des folgenden LGS ist:
  \[\left(\begin{array}{ll|l}
      \alpha_1y_1(a)+\alpha_2y_1'(a) & \alpha_1 y_2(a)+\alpha_2y_2'(a) &
      \gamma_a-\alpha_1y_s(a)-\alpha_2y_s'(a)\\
      \beta_1y_1(b)+\beta_2y_1'(b) & \beta_1 y_2(b)+\beta_2y_2'(b) &
      \gamma_a-\beta_1y_s(b)-\beta_2y_s'(b)\\
      \end{array}\right)
    \]
  Bezeichnet man die linke $2\times 2$-Matrix mit $R$, so folgt, dass $(*)$
  genau dann eindeutig l"osbar ist, wenn $R$ regul"ar ist. Daher kann
  die Regularit"at von $R$ nicht vom gew"ahlten Fundamentalsystem abh"angen.
  $(*)$ ist insbesondere genau dann eindeutig l"osbar, wenn es f"ur das
  folgende zugeh"orige, sogenannte ``homogene'' RWP nur die triviale 
  L"osung gibt:
  \index{homogenes Randwertproblem}
  \index{Randwertproblem>homogenes}
  \[(**)\left\{\begin{array}{l}
      y''(x)+a_1(x)y'(x)+a_0(x)y(x)=0\\
      \alpha_1y(a)+\alpha_2y'(a)=0\\
      \beta_1y(b)+\beta_2y'(b)=0\\      
    \end{array}\right.
    \]
  Hat man also ein Fundamentalsystem f"ur die Dgl in einem Randwertproblem
  der Art $(*)$ gefunden, so ist die eindeutige L"osbarkeit gleichwertig mit 
  der Frage nach der Regularit"at von $R$. 
  
  Falls $(**)$ mehrdeutig l"osbar ist, ist wegen $\alpha_1^2+\alpha_2^2>0,
  \beta_1^2+\beta_2^2>0$ die L"osungsmenge ein Vektorraum der Dimension $1$.
  In diesem Fall ist $(*)$ mehrdeutig oder nicht l"osbar.
  }
% -----------------------------------------------------------------------------
\subsection{Nichtlineare Randwertprobleme}
% -----------------------------------------------------------------------------
\deduction Das Dirichlet-Randwertproblem:{
  Sei $f\in\SetCont([0,1]\times\SetR,\SetR)$ und
  \[(*)\left\{\begin{array}{l}
      y''(x)=f(x,y(x))\\
      y(0)=0,y(1)=0
      \end{array}\right.
    \]
  Man kann $(*)$ wie folgt in eine Integralgleichung umschreiben:
  Wir betrachten die Funktion $G:[0,1]^2\to\SetR$ mit
  \[G(x,\xi)=\begin{cases}
      \xi(x-1)& 0\leq\xi\leq x\leq 1\\
      x(\xi-1)& 0\leq x\leq\xi\leq 1\\
      \end{cases}
    \]
  $G$ ist stetig. Weiterhin betrachten wir die Abbildung
  $T:\SetCont([0,1],\SetR)\to\SetCont([0,1],\SetR)$ mit
  \[T(u)(x):=\int_0^1G(x,\xi)f(\xi,u(\xi))d\xi
    \]
  Dann gilt: F"ur jedes $u\in\SetCont([0,1],\SetR)$ ist 
  $T(u)\in\SetCont^2([0,1],\SetR)$, $T(u)(0)=T(u)(1)=0$ (wegen
  $G(0,\xi)=G(1,\xi)=0$ f"ur $\xi\in[0,1]$) und
  $T''(u)(x)=f(x,u(x))$ (nachrechnen). Damit gilt: 
  \begin{displaytext}
    $y\in\SetCont^2([0,1],\SetR)$ ist 
    eine L"osung von $(*)$ $\equiv$ $y\in\SetCont([0,1],\SetR),T(y)=y$
    \end{displaytext}
  ``$\implies$'' wurde nur f"ur eindeutig l"osbare RWPe gerechtfertigt,
  dort gilt f"ur die L"osung obige Gleichheit notwendigerweise.
  
  Allgemein hei"st ein solches $G$ eine \indexthis{Green'sche Funktion}.
  \index{Funktion>Green'sche} Man kann auch andere RWP'e mit Hilfe
  Green'scher Funktionen in Integralgleichungen umschreiben, z.B. obiges
  Problem f"ur ein beliebiges Intervall $[a,b]$ mit:
  \[G(x,\xi)=\begin{cases}
      \frac 1a\xi(x-a)& 0\leq\xi\leq x\leq a\\
      \frac 1ax(\xi-a)& 0\leq x\leq\xi\leq a\\
      \end{cases}
    \]
  (Mehr zu Green'schen Funktion in Walter, Gew"ohnliche 
  Differentialgleichungen, Kapitel 26)
  }
% -----------------------------------------------------------------------------
\theorem:
  $f\in\SetCont([a,b]\times\SetR,\SetR)$, 
  $f$ Lipschitz-stetig bzgl. $y$ mit Lipschitz-Konstante $L<\frac{\pi^2}{(b-a)^2}$=>{
  Dann ist das RWP
  \[\left\{\begin{array}{l}
      y''(x)=f(x,y(x))\\
      y(a)=0,y(b)=0
      \end{array}\right.
    \]
  eindeutig l"osbar.
  }
% -----------------------------------------------------------------------------
\remark:{
  Am Beispiel am Anfang des Kapitels sieht man, dass die gegebene Schranke
  wirklich optimal ist.
  }
% -----------------------------------------------------------------------------
\theorem Satz von Scorza-Dragoni:
  $f\in\SetCont([a,b]\times\SetR,\SetR)$ mit $\norm f_\infty\leq M>0$=>{
  \label{the:scorza}
  \index{Scorza-Dragoni>Satz von}
  Dann besitzt das RWP
  \[\left\{\begin{array}{l}
      y''(x)=f(x,y(x))\\
      y(a)=0,y(b)=0
      \end{array}\right.
    \]
  eine L"osung.
  }
% -----------------------------------------------------------------------------
\remark:{
  Erg"anzend ist folgendes zu sagen:
  \begin{itemize}
    \item Die L"osung ist nicht notwendigerweise eindeutig. Beispiel:
      Betrachte
      \[f(y)=\begin{cases}
        1 & y\leq -1\\
        -y & -1<y<1\\
        -1 & y\geq 1
        \end{cases}
        \]
      Dann sind u.a. alle Funktionen $y:[0,\pi]\to\SetR$ mit 
      $y(x)=\alpha\sin x$ mit $|\alpha|\leq 1$ L"osung des RWP.
    \item Der Satz gilt auch dann, wenn man noch $y'(x)$ in den
      Argumenten von $f$ zul"asst.
    \item Eine Beweism"oglichkeit f"ur \ref{the:scorza} benutzt
      den Fixpunktsatz von Brouwer, der im folgenden erw"ahnt werden soll.
    \end{itemize}
  }
% -----------------------------------------------------------------------------
\lessertheorem Fixpunktsatz von Brouwer:
  $\emptyset\neq K\subseteq\SetR^{n\times n}$ kompakt und konvex,
  $f\in\SetCont(K,K)$=>{
  \index{Brouwer>Fixpunktsatz von}
  Dann hat $f$ einen Fixpunkt.
  }
% -----------------------------------------------------------------------------
\lessertheorem Fixpunktsatz von Schauder:
  $(E,\norm\cdot)$ Banachraum,
  $\emptyset\neq K\subseteq E$ kompakt und konvex,
  $\closure{f(K)}$ kompakt=>{
  \index{Brouwer>Fixpunktsatz von}
  Dann hat $f$ einen Fixpunkt.
  }
% -----------------------------------------------------------------------------
\remark Leichte Verallgemeinerung von \ref{the:scorza}:{
  Aus \ref{the:scorza} erh"alt man auch die L"osbarkeit von RWPen der Form
  \begin{equation}
    \tag{*}y''(x)=f(x,y(x)),y(a)=\gamma_a,y(b)=\gamma_b
    \end{equation}
  mit stetigem und beschr"anktem $f$. Betrachte dazu 
  \[\tilde f(x,z)=f(x,z+\gamma_b\frac{x-a}{b-a}+\gamma_a\frac{b-x}{b-a})
    \]
  Nach \ref{the:scorza} ist also 
  \[z''(x)=f(x,z(x)),z(a)=0,z(b)=0
    \]
  l"osbar, sei $z$ diese L"osung. Betrachte dann $y:[a,b]\to\SetR$ mit
  \[y(x):=z(x)+\gamma_b\frac{x-a}{b-a}+\gamma_a\frac{b-x}{b-a}
    \]
  Dann ist offenbar $y(x)$ L"osung von $(*)$.
  }
% -----------------------------------------------------------------------------
\section{Autonome Differentialgleichungen und Stabilit"at}
% -----------------------------------------------------------------------------
\remark Gegenstand der Betrachtung:{
  Sei $D\subseteq\SetR^p$, $f\in\SetCont(D,\SetR^p)$. Wir betrachten
  \begin{equation*}
    \tag{*} y'(x)=f(y(x)),y(x_0)=y_0
    \end{equation*}
  mit $x_0\real,y_0\in D$. Man beobachtet zun"achst folgendes:
  \begin{itemize}
    \item $f(y_0)=0$ $\implies$ $y\ident y_0$ ist L"osung.
    \item Ist $y:[x_0,\infty)\to\SetR^p$ L"osung von $(*)$, so ist f"ur
      jedes $\xi\real$ auch $z:[x_0-\xi,\infty)\to\SetR^p$ mit $z(x):=y(x+\xi)$
      eine L"osung der Dgl. in $(*)$ (jedoch i.a. nicht des AWP)
    \item Ist $y:[x_0,\infty)\to\SetR^p$ eine L"osung von $(*)$ und 
      existiert $\limes x->\infty y(x)\in D$, so gilt 
      $f(\limes x->\infty y(x))=0=\limes x->\infty y'(x)$.
    \end{itemize}
  }
% -----------------------------------------------------------------------------
\definition Gleichgewicht:{
  Sei $D\subseteq\SetR^p$, $f\in\SetCont(D,\SetR^p)$ und
  \begin{equation*}
    \tag{*} y'(x)=f(y(x))
    \end{equation*}
  Eine Stelle $y_0\in D$ mit $f(y_0)=0$ hei"st
  \indexthis{Gleichgewichtspunkt}, die dazugeh"orige L"osung
  $y\ident y_0$ \indexthis{Gleichgewichtsl"osung} der Differentialgleichung.
  }
% -----------------------------------------------------------------------------
\definition Stabilit"at:{
  Sei $D\subseteq\SetR^p$ offen, $f\in\SetCont(D,\SetR^p)$ und $y_0\in D$
  eine Nullstelle von $f$. Dann hei"st
  \begin{stmts}
    \item $y_0$ \indexthis{stabiler Gleichgewichtspunkt}
      \index{Gleichgewichtspunkt>stabiler}
      $:\equiv$ F"ur jedes $\epsilon>0$ existiert
      ein $\delta>0$ so, dass falls f"ur $y_1\in D$ mit $\norm{y_1-y_0}<\delta$
      und $y:[x_0,\omega_+)\to\SetR^p$ eine nach rechts nicht fortsetzbare
      L"osung des AWPs $y'(x)=f(y(x)),y(x_0)=y_1$ folgt $\omega_+=\infty$
      und $\norm{y(x)-y_0}<\epsilon$ f"ur $x\geq x_0$.
    \item $y_0$ \indexthis{instabiler Gleichgewichtspunkt}
      \index{Gleichgewichtspunkt>instabiler}
      $:\equiv$ $y_0$ nicht stabil.
    \item $y_0$ \indexthis{asymptotisch stabiler Gleichgewichtspunkt}
      \index{Gleichgewichtspunkt>asymptotisch stabiler}
      $:\equiv$ wie (1), lediglich folgt 
      zus"atzlich, dass $\limes x->\infty y(x)=y_0$.
    \end{stmts}
  }
% -----------------------------------------------------------------------------
\remark:{
  Aha! Es ist folgendes zu beobachten:
  \begin{itemize}
    \item Die Definition ist unabh"angig von der 
      gew"ahlten Norm und der Wahl der Anfangsstelle $x_0$.
    \item Nach Definition gilt ``asymptotisch stabil $\implies$ stabil''.
      Die Eigenschaft $\limes x->\infty y(x)=y_0$ gen"ugt jedoch
      im Gegenzug nicht f"ur Stabilit"at.
    \item $y_0$ stabil $\implies$ AWP $y'(x)=f(y(x)),y(x_0)=y_0$ eindeutig
      l"osbar, weil jede tats"achlich verschiedene L"osung irgendwann
      einmal die in der Def. erw"ahnte $\epsilon$-Umgebung verl"asst.
    \end{itemize}
  }
% -----------------------------------------------------------------------------
\subsection{Ein Stabilit"atssatz}
% -----------------------------------------------------------------------------
\theorem:
  $D\subseteq\SetR^p$ offen, $f\in\SetCont^1(D,\SetR^p)$, $y_0\in D$,
  $f(y_0)=0$, f"ur jeden EW $\lambda\complex$ von $f'(y_0)\in\SetR^{p\times p}$
  gelte $\Re\{\lambda\}<0$=>{
  \label{the:stability}
  Dann ist $y_0$ asymptotisch stabil.
  }
% -----------------------------------------------------------------------------
\remark:{
  Ist $A\in\SetR^{p\times p}$ mit $\Re\{\lambda\}<0$ f"ur alle EWe 
  $\lambda$ von $A$, so ist nach \ref{the:stability} $y_0=0$ 
  asymptotisch stabiler Gleichgewichtspunkt von $y'(x)=Ay(x)$. Dies
  folgt auch nach Paragraph \ref{sec:linear-const-system}.
  
  Beachte: Falls die EW-Bedingung gilt, ist $A$ regul"ar, also $0$ einziger
  Gleichgewichtspunkt.
  
  Zum Beweis von \ref{the:stability} verwenden wir den folgenden Hilfssatz:
  }
% -----------------------------------------------------------------------------
\theorem:
  $A\in\SetR^{p\times p}$ $c\real$ so, dass f"ur alle EWe $\lambda\complex$ 
  von $A$ gilt: $\Re\{\lambda\}<c$=>{
  Dann existiert ein Skalarprodukt mit
  \[\iprod y {Ay} \leq c\norm{y}_0^2
    \]
  dabei sei $\norm{\cdot}_0$ die von $\iprod \cdot \cdot $ erzeugte Norm.
  }
% -----------------------------------------------------------------------------
\subsection{Ein Instabilit"atssatz}
% -----------------------------------------------------------------------------
\theorem:
  $D\subseteq\SetR^p$ offen, $f\in\SetCont^1(D,\SetR^p)$, $y_0\in D$
  mit $f(y_0)=0$. Es existiere ein EW $\lambda$ von $f'(y_0)$ mit
  $\Re\{\lambda\}>0$=>{
  Dann ist $y_0$ ein instabiler Gleichgewichtspunkt.
  }
% -----------------------------------------------------------------------------
\remark Charakterisierung der Stabilit"at bei linearen Systemen:{
  Im Falle $\Re\lambda\leq 0$ f"ur alle EWe der Matrix $A\in\SetR^{p\times p}$
  k"onnen "uber die Stabilit"at der 0 als Gleichgewichtspunkt von $y'(x)=Ay(x)$
  folgende Aussagen gemacht werden:
  \begin{itemize}
    \item Existiert zu einem EW $\lambda=0$ ein nichttrivialer
      Jordanblock (also einer, der Einsen au"serhalb der Diagonalen hat),
      so ist die $0$ instabil.
    \item Sind alle Jordanbl"ocke zu EWen $\lambda$ mit $\Re\lambda=0$
      trivial, so ist die $0$ stabil.
    \end{itemize}
  Diese Ergebnisse k"onnen leicht auch auf andere Gleichgewichtspunkte $x\neq 0$
  "ubertragen werden, w"ahrend die "Ubertragung auf nichtlineare Systeme
  in dieser Form nicht m"oglich ist. (Bsp. $y'(x)=\pm y^3(x)$ wird
  im Fall ``$-$'' bei $0$ asympt. stabil, im Fall ``$+$'' allerdings instabil)
  }
% -----------------------------------------------------------------------------
\subsection{Liapunov-Funktionen}
% -----------------------------------------------------------------------------
\convention{
  Sei $D\subseteq\SetR^p$ offen, $f:D\to\SetR^p$ sei lokal Lipschitz-stetig.
  Wir nehmen o.B.d.A. $0\in D$ und $f(0)=0$ an. (sonst Transformation)
  }
% -----------------------------------------------------------------------------
\definition Liapunov-Funktion:{
  \index{Funktion>Liapunov-}
  Eine stetige Funktion $V:U_r(0)\to\SetR$ mit $U_r(0)\subseteq D$ hei"st 
  Liapunov-Funktion (in $0$) $:\equiv$
  \begin{stmts}
    \item $V\in\SetCont^1(U_r(0),\SetR)$
    \item $V(0)=0$, $V(y)>0$ f"ur alle $y\in U_r(0)\setminus\{0\}$
    \item $\grad V(y)\cdot f(y)\leq 0$ f"ur alle $y\in U_r(0)$.
    \end{stmts}
  }
% -----------------------------------------------------------------------------
\theorem: $D$ offen, $0\in D$,
  $y'(x)=f(x,y(x))$ mit $f:D\to\SetR^p$ lokal Lipschitz-stetig,
  besitze eine Liapunov-Funktion $V:U_r(0)\to\SetR$=>{
  Dann ist $0$ stabiler Gleichgewichtspunkt. Gilt zus"atzlich
  $\grad V(y)\cdot f(y)<0$ f"ur alle $y\in U_r(0)\setminus\{y_0\}$, so
  ist $0$ sogar asymptotisch stabil.
  }
% -----------------------------------------------------------------------------
\trick Ansatz f"ur Liapunov-Funktionen:{
  Bei einfach aufgebauten Funktionen (z.B. Polynomen) f"uhren oft
  Polynomans"atze mit geraden Exponenten zum Erfolg.
  }
% -----------------------------------------------------------------------------
\subsection{Der Satz von Chetaev}
% -----------------------------------------------------------------------------
\theorem Satz von Chetaev:
  $D\subseteq\SetR^p$ offen, $f:D\to\SetR^p$ lok. Lipschitzstetig und 
  $f(0)=0$. Existiere ein $\epsilon_0>0$ und $G\subseteq D$ offen mit
  $U_{\epsilon_0}(0)\subseteq D$, $0\in\rim G$ und 
  $V\in\SetCont(\closure G,\SetR)\cap\SetCont^1(G,\SetR)$.
  =>{
  \label{the:chetaev}
  \index{Chetaev>Satz von}
  Gilt f"ur $V$, dass
  \begin{itemize}
    \item $V(y)>0$ f"ur alle $y\in G$, $V(y)\leq M$ f"ur alle $y\in\closure G$
      und ein $M>0$
    \item $\grad V(y)\cdot f(y)\geq h(V(y))$ f"ur eine Funktion 
      $h:[0,M]\to\SetR$, die stetig, streng wachsend sein und $h(0)=0$ erf"ullen
      muss.
    \item $V(y)=0$ f"ur alle $y\in\rim G\cap U_{\epsilon_0(0)}$
    \end{itemize}
  so ist $0$ instabiler Gleichgewichtspunkt.
  }
% -----------------------------------------------------------------------------
\remark Gleichgewichtspunkte au"ser $0$:{
  \ref{the:chetaev} gilt analog, wenn $0$ durch eine beliebige andere 
  Nullstelle $y_0$ von $f$ ersetzt wird.
  }
% -----------------------------------------------------------------------------
\remark Wie findet man Differentialgleichungen?:{
  Gegeben sei $\gamma\in\SetCont^1([0,\infty),\SetR^p)$. Betrachte dann
  \[f:\left\{\begin{array}{l}
      [0,\infty)\times\SetR^p\to\SetR^p\\
      (x,y)\mapsto \norm{\gamma(x)-y}y+\gamma'(x)
      \end{array}\right.
    \]
  Dann ist $y=\gamma$ genau die L"osung des AWPs $y'(x)=f(x,y(x)),y(0)=\gamma(0)$.
  }
% -----------------------------------------------------------------------------
\section{Der Satz von Kneser}
% -----------------------------------------------------------------------------
\subsection{Zusammenh"angende Mengen}
% -----------------------------------------------------------------------------
\convention{
  Sei $(V,\norm\cdot)$ ein NLR.
  }
% -----------------------------------------------------------------------------
\definition Zusammenhang:{
  Sei $M\subseteq V$.
  \begin{itemize}
    \item $M$ hei"st \indexthis{wegzusammenh"angend} $:\equiv$ 
      Sind $y,\tilde y\in M$,
      so existiert eine stetige Funktion $\gamma:[a,b]\to M$ mit
      $\gamma(a)=y,\gamma(b)=\tilde y$.
    \item $M$ hei"st \indexthis{unzusammenh"angend} $:\equiv$ 
      Es gibt offene Mengen $O_1,O_2$ mit
      \begin{itemize}
        \item $M\subseteq O_1\cup O_2$
        \item $O_1\cap O_2=\emptyset$
        \item $M\cap O_j\neq\emptyset$ f"ur $j=1,2$
        \end{itemize}
    \item $M$ hei"st \indexthis{zusammenh"angend} $:\equiv$ 
      $M$ ist nicht unzusammenh"angend.
    \end{itemize}
  }
% -----------------------------------------------------------------------------
\remark Verschiedenster Schnickschnack "uber zusammenh"angende Mengen:{
  Man beobachtet:
  \begin{itemize}
    \item Jede endliche Teilmenge von $V$, die aus mehr als einem 
      Punkt besteht, ist unzsuammenh"angend.
    \item Jede konvexe Menge ist wegzusammenh"angend, daher auch 
      zusammenh"angend.
    \item Sind $M_1,M_2\subseteq V$ zusammenh"angend und nicht disjunkt,
      so ist $M_1\cup M_2$ zusammenh"angend, aber $M_1\cap M_2$ im 
      allgemeinen nicht. (z.B. ``zwei W"urstchen'')
    \item Sei $V=\SetR$. Dann gilt $M\subseteq\SetR$ zusammenh"angend
      $\equiv$ $M$ wegzusammenh"angend $\equiv$ $M$ Intervall.
    \item $M\subseteq V$ wegzusammenh"angend $\implies$ $M$
      zusammenh"angend.
    \item $M\subseteq V$ zusammenh"angend $\implies$ $\closure M$ 
      zusammenh"angend.
    \item Ist $\dim V\geq 2$ (inklusive $\infty$), so ex. stets Mengen,
      die zusammenh"angend, aber nicht wegzusammenh"angend sind.
      
      Beispiel: $B:=\{(x,\sin x)\mid x\in (0,1]\}$ ist zwar noch
      beides, aber $\closure B$ nicht mehr. (Betr. z.B. Grenzwerte
      der Folgen der Maxima und Minima des Sinus-Graphen)
    \item Eine offene Teilmenge von $V$ ist zusammenh"angend $\equiv$
      sie ist wegzusammenh"angend.
    \item $(V,\norm\cdot_1),(W,\norm\cdot_2)$ NLRe, $D\subseteq V$ offen, 
      $M\subseteq D$ zusammenh"angend und $f\in\SetCont(D,W)$. Dann
      ist auch $f(M)$ zusammenh"angend.
    \end{itemize}
  }
% -----------------------------------------------------------------------------
\subsection{Der Satz von Kneser}
% -----------------------------------------------------------------------------
\convention{
  Sei $f\in\SetCont([0,1]\times\SetR^p,\SetR^p)$ beschr"ankt 
  (z.B. $\norm f_\infty\leq c$ mit $c>0$)
  Wir betrachten das AWP
  \begin{equation*}
    \tag{*} y'(x)=f(x,y(x)),y(0)=0
    \end{equation*}
  (alle folgenden "Uberlegungen gelten nat"urlich auch f"ur allgemeine AWPe)
  Nach \ref{the:peano} (Peano) existiert jede nicht fortsetzbare L"osung von
  $(*)$ auf $[0,1]$.
  
  Sei
  \[M:=\{y:[0,1]\to\SetR\mid y \text{ l"ost $(*)$}\}
    \]
  $M$ hei"st auch L"osungs- oder \indexthis{Knesertrichter}
  \index{L"osungstrichter} zum AWP $(*)$. Wir betrachten den Banachraum
  $\SetCont([0,1],\SetR^p)$ mit der Maximumnorm auf $[0,1]$. Offenbar ist
  $M\subseteq\SetCont([0,1],\SetR^p)$.
  }
% -----------------------------------------------------------------------------
\theorem Satz von Kneser:=>{
  \label{the:kneser}
  \index{Kneser>Satz von}
  Unter obigen Bedingungen ist $M$ eine zusammenh"angende Teilmenge von 
  $\SetCont([0,1],\SetR^p)$.
  }
% -----------------------------------------------------------------------------
\remark:{
  Man stellt fest:
  \begin{itemize}
    \item Sei $x_0\in[0,1]$ fest. Die Abbildung
      $P_{x_0}:\SetCont([0,1],\SetR^p)\to\SetR^p$ mit 
      $P_{x_0}(u)\mapsto u(x_0)$ ist stetig. Mit $M$ ist daher
      auch $P_{x_0}(M)$ zusammenh"angend.
      
      Im Fall $p=1$ ist $P_{x_0}(M)$ ein beschr"anktes und 
      abgeschlossenes Intervall, denn nach 
      \ref{the:ode-extremal-weak} existiert zu dem betrachteten AWP
      eine gr"o"ste und eine kleinste L"osung 
      $\overline y,\underline y:[0,1]\to\SetR$. Im Fall $p\geq 2$ ist $M$
      i.a. kompliziert.
    \item Wir beweisen den Satz f"ur $p=1$. ($p>1$ funktioniert
      bis auf die Lipschitz-stetige Approximation von $f$ v"ollig
      analog)
    \item \ref{the:kneser} wird auch oft formuliert als
      ``$M$ ist kompakt und zusammenh"angend.''
    \item In Ana II wurde f"ur endlichdimensionale NLRe bewiesen:
      (gilt auch f"ur $\infty$)
      $(V,\norm \cdot)$ NLR, $M\subseteq V$ kompakt $\equiv$
      Jede Folge $(y_n)\subseteq M$ besitzt eine konvergente TF, 
      deren Grenzwert in $M$ liegt.
      (``Folgenkompaktheit'')
    \item Die L"osungsmenge des AWPs ist folgenkompakt. Aus 
      \ref{the:ascoli} folgt allgemein:
      $M\subseteq\SetCont([0,1],\SetR^p)$ folgenkompakt, wenn
      $M$ gleichgradig stetig, abgeschlossen und beschr"ankt ist.
    \item Im Beweis von \ref{the:kneser} wird u.a. benutzt:
      $M\subseteq V$ folgenkompakt, $S\subseteq M$, $S$ abgeschlossen
      $\implies$ $S$ folgenkompakt.
    \end{itemize}
  }
% -----------------------------------------------------------------------------
\definition Grenzmenge:{
  Sei $y\in\SetCont([x_0,\infty),\SetR^p)$. Die Menge
  \[\omega_+(y):=\{z\in\SetR^p\mid \exists (x_n)\subseteq[x_0,\infty)\text{ mit }
      x_n\to\infty,y(x_n)\to z (n\to\infty)\}
    \]
  hei"st Grenzmenge der Funktion $y$. Analog definiert man $\omega_-(y)$.
  }
% -----------------------------------------------------------------------------
\remark Eigenschaften der Grenzmenge:{
  Ist $y$ beschr"ankt, so ist $\omega_+(y)$ beschr"ankt, abgeschlossen
  (daher kompakt), nichtleer und zusammenh"angend.
  }
% -----------------------------------------------------------------------------
\remark Grenzmengen und AWPe:{
  Ist $f:\SetR^p\to\SetR^p$ lokal Lipschitz-stetig und das AWP
  $y'(x)=f(y(x)),y(x_0)=y_0$ l"osbar, so kann man die Grenzmenge der
  L"osung betrachten und schreibt dann kurz $\omega_+(y_0)$, da sie
  aufgrund der Autonomie des AWPs nicht von $x_0$ abh"angt.
  
  Desweiteren gilt hier: Ist $y$ beschr"ankt, so ist $\omega_+(y_0)$ 
  kompakt, zusammenh"angend und invariant, d.h. f"ur $z_0\in\omega_+(y_0)$
  und $z:[x_0,\omega_+)\to\SetR^p$ die nach rechts nicht fortsetzbare 
  L"osung von $z'(x)=f(z(x)),z(x_0)=z_0$, so gilt $\omega_+=\infty$ und
  $z(x)\in\omega_+(y_0)$ f"ur alle $x\geq x_0$.
  }
% -----------------------------------------------------------------------------
\framedmsg{ Das war's f"ur Analysis I-III. Tsch"us! :-) }

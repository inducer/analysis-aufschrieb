% -*- LaTeX -*-
% -----------------------------------------------------------------------------
\section{Integrals"atze}
% -----------------------------------------------------------------------------
\example:{
  Sei z.B. $f\in\SetCont^1([a,b],\SetR)$. Dann gilt
  \[f(b)-f(a)=\int_a^b f'(x)dx
    \]
  Ziel ist es, dieses eindimensionale Ergebnis ins mehrdimensionale 
  zu "ubertragen.
  }
% -----------------------------------------------------------------------------
\convention{
  Sei $D\subseteq\SetR^p$ offen, $f\in\SetCont^1(D,\SetR^p)$, 
  $f=(f_1,\ldots,f_p)$.
  }
% -----------------------------------------------------------------------------
\definition Divergenz:{
  F"ur eine Funktion $f\in\SetCont^1(\SetR^k,\SetR^p)$ wird die Divergenz 
  $\div f$ wie folgt festgelegt:
  \[\div f:=D_1f_1+\ldots+D_pf_p
    \]
  }
% -----------------------------------------------------------------------------
\definition Rotation:{
  F"ur eine Funktion $f\in\SetCont^1(\SetR^k,\SetR^3)$ wird die Rotation
  $\rot f$ wie folgt festgelegt:
  \[\rot f:=\begin{pmatrix}
      D_2f_3-D_3f_2 \\
      D_3f_1-D_1f_3 \\
      D_1f_2-D_2f_1
      \end{pmatrix}
    \]
  }
% -----------------------------------------------------------------------------
\subsection{Der zweidimensionale Integralsatz von Gau"s}
% -----------------------------------------------------------------------------
\definition Zul"assigkeit:{
  Sei $(x_0,y_0)\in\SetR^2$ fest und $R:[0,2\pi]\to(0,\infty)$ st"uckweise
  stetig db mit $R(0)=R(2\pi)$. Der Weg $\gamma:[0,2\pi]\to\SetR^2$ sei
  gegeben durch
  \[\gamma(t)=\begin{pmatrix}
      \gamma_1(t)\\
      \gamma_2(t)
    \end{pmatrix}:=\begin{pmatrix}
      x_0+R(t)\cos t \\
      y_0+R(t)\sin t 
      \end{pmatrix}
    \]
  Dann ist $\gamma$ st"uckweise stetig db mit $\gamma(0)=\gamma(2\pi)$.
  Weiter sei
  \[B:=\{(x_0+r\cos t,y_0+r\sin t)\mid t\in[0,2\pi],0\leq r\leq R(t)\}
    \]
  Es gilt: B ist kompakt und $\rim B=\Gamma$.
  
  Eine Menge $B$, die auf diese Art und Weise erzeugt werden kann, hei"st
  zul"assig. Zul"assig sind z.B. Kreise, Ellipsen, konvexe Vielecke.
  Sei $(x,y)=\gamma(t)\in\rim B$ mit geeignetem $t\in[0,2\pi]$. Dann l"asst
  sich die "au"sere Normale von $B$ folgendermassen darstellen:
  \[n(x,y):=\frac {(\gamma_2'(t),-\gamma_1'(t))} 
            {\norm{(\gamma_2'(t),-\gamma_1'(t))}}
    \] 
  (es gilt stets $\norm{(\gamma_2'(t),-\gamma_1'(t))}\neq 0$).
  }
% -----------------------------------------------------------------------------
\theorem Integralsatz von Gau"s in der Ebene:
  $B\subseteq\SetR^2$ zul"assig, $U\supseteq B$ offen, 
  $f\in\SetCont^1(U,\SetR^2)$=>{
  \label{the:gauss2dim}
  \index{Integralsatz}
  \index{Gau"s >Integralsatz von}
  Dann gilt
  \[\int_B \div f(x,y) d(x,y)=
      \underbrace{ \int_{\rim B} f(x,y)\cdot n(x,y) ds }
      _{\text{Wegintegral}}
    \]
  }
% -----------------------------------------------------------------------------
\remark:{
  \ref{the:gauss2dim} gilt allgemeiner f"ur Bereiche $B$, deren Rand 
  durch einen rektifizierbaren Weg im math. pos Sinn einmal umlaufen werden
  kann.
  }
% -----------------------------------------------------------------------------
\remark Berechnung von Fl"acheninhalten:{
  Sei $\gamma\in\SetCont^1$ eine Parametrisierung von $\rim B$ mit 
  $\norm{\gamma'}\ident 1$. Dann erh"alt man:
  \begin{align*}
    \lambda(B) &= \int_B 1\,d(x,y) 
               = \frac 1 2 \int_B \id_x(x,y)+\id_y(x,y)d(x,y) \\
               &= \frac 1 2 \int_B \div \id(x,y) d(x,y) 
               = \frac 1 2 \int_{\partial B} \mat{x\\y}\cdot n(x,y) d(x,y) \\
               &= \frac 1 2 \int_\gamma \mat{\gamma_1(t)\\\gamma_2(t)}\cdot
	          \mat{-\gamma_2'(t)\\ \gamma_1'(t)} dt 
               = \frac 1 2 \int_\gamma \mat{-\gamma_2(t)\\\gamma_1(t)}\cdot
	          \gamma'(t) dt \\
               &= \frac 1 2 \int_{\rim B} x\,dy - y\,dx
    \end{align*}
  Analog ergibt sich:
  \[\lambda(B)=\int_{\rim B} x\,dy=-\int_{\rim B} y\,dx
    \]
  Den Fl"acheninhalt der Ellipse mit Randweg
  \[\gamma(t)=\begin{pmatrix}
      a \cos t \\ b \sin t
      \end{pmatrix}
    \]
  erh"alt man so einfach zu $ab\pi$.
  }
% -----------------------------------------------------------------------------
\subsection{Fl"achen im $\SetR^3$}
% -----------------------------------------------------------------------------
\definition Parameterdarstellung:{
  \index{Tangentialebene}
  \index{Normalenvektor}
  Sei $D\subseteq\SetR^2$ offen, $K\subseteq D$ kompakt und $\Phi=(\Phi_1,\Phi_2,
  \Phi_3)\in\SetCont^1(D,\SetR^3)$ und $\rank\Phi'=2$ auf $D$. Dann 
  hei"st $\Phi$ Parameterdarstellung der glatten Fl"ache 
  $\cal F:=\Phi(D)$ und $\Phi|_K$ Parameterdarstellung des glatten 
  Fl"achenst"ucks $\cal F_K:=\Phi(K)$.

  Es gilt:
  \[\Phi'=\begin{pmatrix}
             D_1\Phi_1 & D_2\Phi_1 \\
             D_1\Phi_2 & D_2\Phi_2 \\
             D_1\Phi_3 & D_2\Phi_3 \\
	     \end{pmatrix}
    \]
  Die Spaltenvektoren von $\Phi'$ sind linear unabh"angig und bestimmen die 
  Tangentialebene
  \[
    (u,v,w)=\Phi(x_0,y_0)+[\Phi_x(x_0,y_0),\Phi_y(x_0,y_0)]
    \]
  im Punkt $\Phi(x_0,y_0)$. Senkrecht auf dieser Ebene steht der Normalenvektor
  \[
    n(x_0,y_0):=\frac{\Phi_x(x_0,y_0)\times\Phi_y(x_0,y_0)}{\norm{\Phi_x(x_0,y_0)
    \times\Phi_y(x_0,y_0)}}
    \]
  }
% -----------------------------------------------------------------------------
\remark Wichtiger Spezialfall:{
  $\phi\in\SetCont(D,\SetR),((x,y)\in D),\Phi(x,y)=(x,y,\phi(x,y))$.
  Hier ist
  \[\Phi_x(x,y)=\begin{pmatrix}1 \\ 0 \\ \phi_x(x,y)\end{pmatrix} \qquad
    \Phi_y(x,y)=\begin{pmatrix}0 \\ 1 \\ \phi_y(x,y)\end{pmatrix}
    \]
  und damit $n=\frac{(-\phi_x,-\phi_y,1)}{\sqrt{1+(\phi_x)^2+(\phi_y)^2}}$ 
  ($n$ zeigt nach oben).
  }
% -----------------------------------------------------------------------------
\definition Fl"acheninhalt:{
  Ist $\cal F_k$ ein glattes Fl"achenst"uck, so ist dessen Fl"acheninhalt 
  definiert als
  \[I(\cal F_K):=\int_K\norm{\Phi_x(x,y)\times\Phi_y(x,y)}d(x,y)
    \]
  }
% -----------------------------------------------------------------------------
\subsection{Der Integralsatz von Stokes}
% -----------------------------------------------------------------------------
\definition:{
  Sei $\Phi_K$ Parameterdarstellung eines glatten Fl"achenst"ucks, 
  $f\in\SetCont(\cal F_K,\SetR)$ so definiert man
  \[\int_{\Phi|_K} f\, do:=
    \int_K f(\Phi(x,y))\norm{\Phi_x(x,y)\times\Phi_y(x,y)}d(x,y)
    \]
  Dieses Integral bleibt bei geeigneter Umparametrisierung erhalten, daher
  schreibt man auch
  \[\int_{\cal F_K}f\,do:=\int_{\Phi|_K} f\, do
    \]
  F"ur Fl"achen, die aus glatten Fl"achenst"ucken zusammengesetzt sind, 
  definiert man die entsprechenden Integrale als Summe der Integrale der
  einzelnen Fl"achenst"ucke.
  }
% -----------------------------------------------------------------------------
\theorem Integralsatz von Stokes:
  $B\subseteq\SetR^2$ kompakt und zul"assig, 
  $U\supseteq B$ offen, $\Phi\in\SetCont(U,\SetR^3)$ 
  $\gamma:[a,b]\to\SetR^2$ Parameterisierung von $\rim B$,
  $A\supseteq\Phi(B)$ offen, $f\in\SetCont^1(A,\SetR^3)$=>{
  \index{Stokes>Integralsatz von}
  Sei $\cal F_B=\Phi(B)$.
  Dann ist $\Phi\circ\gamma:[a,b]\to\SetR^3$ Parametrisierung von 
  $\rim{\cal F_B}$ und es gilt:
  \[\int_{\Phi|_B} \rot f\cdot n\,do=\int_{\Phi\circ\gamma}f(x,y,z)d(x,y,z)
    \]
  }
% -----------------------------------------------------------------------------
\subsection{Der Integralsatz von Gau"s im Raum}
% -----------------------------------------------------------------------------
\definition Normalbereich:{
  Sei $B\subseteq\SetR^2$ kompakt und zul"assig, $U\supseteq B$ offen,
  $\gamma:[a,b]\to\SetR^2$ glatt und Parametrisierung von $\rim B$,
  $\phi_1,\phi_2\in\SetCont^1(U,\SetR)$. Es sei $\phi_1<\phi_2$ auf $B$.
  Weiter sei
  \[\Phi_i(x,y)=\begin{pmatrix} x \\ y \\ \phi_i(x,y) \end{pmatrix}
    \qquad (i=1,2)
    \]
  Dann hei"st die Menge
  \[V:=\{(x,y,z)\mid (x,y)\in B,z\in[\phi_1(x),\phi_2(x)]\}
    \]
  ein Normalbereich bez"uglich der $z$-Achse. (Definition bez"uglich 
  $x$,$y$-Achsen analog) Die \indexthis{Mantelfl"ache} $\cal M$ von $V$
  hat die Parametrisierung
  \[\Phi_3(t,z)=(\gamma(t),z)\qquad 
    (a\leq t\leq b,\phi_1(\gamma(t))\leq z\leq \phi_2(\gamma(t)))
    \]
  Wir nennen den ins "Au"sere von $V$ zeigenden Normaleneinheitsvektor
  $n=(n_1,n_2,n_3)$ auf $\rim V$ die \indexthis{"au"sere Normale}
  \index{Normale>"au"sere} und erhalten
  \begin{align*}
    n&=\frac{(\phi_{1x},\phi_{1y},-1)}{\norm{(\phi_{1x},\phi_{1y},-1)}} 
        \qquad \text{ auf $\cal F_1$} \\
    n&=\frac{(-\phi_{2x},-\phi_{2y},1)}{\norm{(-\phi_{2x},-\phi_{2y},1)}} 
        \qquad \text{ auf $\cal F_2$} \\
    n&=\frac{(\gamma_2',-\gamma_1',0)}{\norm{(\gamma_2',-\gamma_1',0)}} 
        \qquad \text{ auf $\cal M$}
    \end{align*}
  }
% -----------------------------------------------------------------------------
\theorem Integralsatz von Gau"s im Raum:
  $V$ Normalbereich bzgl. aller drei Achsen, $U\supseteq V$ offen,
  $f\in\SetCont^1(U,\SetR^3)$=>{
  \label{the:int-gauss-3d}
  \index{Gau"s>Integralsatz im Raum}
  Dann gilt
  \[\int_V \div f(x,y,z)d(x,y,z)=\int_{\rim V} f\cdot n\,do
    \]
  }
% -----------------------------------------------------------------------------
\remark:{
  Durch Approximationsargumente kann \ref{the:int-gauss-3d} auf Kugeln,
  Ellipsoide u.v.a. Teilmengen des $\SetR^3$ erweitert werden.
  }
% -----------------------------------------------------------------------------
\section{Gew"ohnliche Differentialgleichungen}
% -----------------------------------------------------------------------------
\convention{
  $I$ sei ein Intervall positiver L"ange.
  }
% -----------------------------------------------------------------------------
\definition Gew"ohnliche Differentialgleichung:{
  \index{Differentialgleichung>gew"ohnliche}
  Sei $k\natural$, $y\in\SetCont^k(I,\SetR^p)$, $D\subseteq\SetR^{(k+1)p+1}$, 
  $F:D\to\SetR^p$.
  Eine Gleichung der Form
  \begin{equation*}
    F(x,y(x),y'(x),\ldots,y^{(k)}(x))=0 \tag{*}
    \end{equation*}
  hei"st gew"ohnliche Differentialgleichung/Dgl. Sei $I$ ein Intervall, 
  $y:I\to\SetR^p$ eine Funktion. $y$ hei"st L"osung von $(*)$ auf $I$,
  falls gilt:
  \begin{stmts}
    \item $y$ ist auf $I$ $k$-mal db
    \item $(x,y(x),\ldots,y^{(k)})\in D)$ f"ur alle $x\in I$.
    \item $(*)$ ist erf"ullt.
    \end{stmts}
  L"osungen "uber verschiedenen Intervallen werden als verschieden angesehen,
  selbst wenn sie in $y$ "ubereinstimmen.
  }
% -----------------------------------------------------------------------------
\example:{
  Hier einige Differentialgleichungen mit L"osung:
  \begin{itemize}
    \item $y^2(x)+(y'(x))^2=1$ $\to$ $y(x)=\cos x$
    \item $x+y(x)/y'(x)=0$ $\to$ $y:(-\infty,0)\to\SetR$, $y(x)=c/x$
    \item $f\in\SetCont([a,b],\SetR)$, $y'(x)-f(x)=0$ $\equiv$ $y'(x)=f(x)$
    \end{itemize}
  }
% -----------------------------------------------------------------------------
\definition Explizite Differentialgleichung:{
  \index{Differentialgleichung>explizite}
  Hat $(*)$ die Form
  \[y^{(k)}=F(x,y(x),y'(x),\ldots,y^{(k-1)}(x))
    \]
  so spricht man von einer expliziten Dgl $k$-ter \indexthis{Ordnung}.
  }
% -----------------------------------------------------------------------------
\remark:{
  Ist $y:I\to\SetR^p$ eine L"osung einer expliziten Dgl mit stetigem $F$, so 
  ist $y\in\SetCont^k(I,\SetR^p)$. 
  In dieser Vorlesung werden vorwiegend solche Dgl'en betrachtet. 
  }
% -----------------------------------------------------------------------------
\definition Anfangswertproblem:{
  Sei nun weiterhin $(x_0,y_0,\ldots,y_{k-1})\in D$, so hei"st das 
  Gleichungssystem
  \begin{align*}
    y^{(k)}(x)&=F(x,y(x),y'(x),\ldots,y^{(k-1)}(x)) \tag{***}\\
    y(x_0)&=y_0\\
    y'(x_0)&=y_1 \tag{**} \\
    &\vdots\\
    y^{(k-1)}(x_0)&=y_{k-1} 
    \end{align*}
  ein Anfangswertproblem/AWP. Eine L"osung des AWP ist eine L"osung der
  Dgl $(***)$ und der Gleichungen $(**)$.
  }
% -----------------------------------------------------------------------------
\example:{
  Hier einige Anfangswertprobleme mit L"osung:
  \begin{stmts}
    \item $y'(x)=2y(x)$ mit $y(0)=4$ $\to$ $y(x)=4e^{2x}$
    \item $y'(x)=2\sqrt{2y(x)}$, $y(0)=0$ $\to$ 
      $y(x)\in\{\max\{0,x-a\}\mid a>0\}$
    \end{stmts}
  }
% -----------------------------------------------------------------------------
\definition Eindeutigkeit einer L"osung:{
  Ein AWP $(**)$ hei"st eindeutig l"osbar, falls gilt
  \begin{stmts}
    \item Es existiert eine L"osung $y:I\to\SetR^p$
    \item Sind $y_1:I_1\to\SetR^p$, $y_2:I_2\to\SetR^p$ L"osungen, so
      gilt $y_1=y_2$ auf $I_1\cap I_2$
    \end{stmts}
  }
% -----------------------------------------------------------------------------
\remark Bezeichnungsweisen in der Literatur:{
  Die Variablenbezeichnungen von Dgl'en sind in der Literatur nicht
  einheitlich.
  }
% -----------------------------------------------------------------------------
\section{Explizit l"osbare Differentialgleichungen}
% -----------------------------------------------------------------------------
\convention{
  Wir betrachten stets $\SetR$, also $p=1$.
  }
% -----------------------------------------------------------------------------
\definition Lineare Differentialgleichung (I):{
\index{Differentialgleichung>lineare}%
\index{Differentialgleichung>homogene}%
\index{Differentialgleichung>inhomogene}%
\index{Lineare Differentialgleichung}%
\index{Homogene Differentialgleichung}%
\index{Inhomogene Differentialgleichung}%
  Sei $J\subseteq\SetR$ ein Intervall, $a,g\in\SetCont(J,\SetR)$.
  Eine Differentialgleichung der Form
  \[y'(x)=a(x)y(x)+g(x)
    \]
  hei"st linear. Sie hei"st au"serdem homogen, falls $g=0$, ansonsten 
  inhomogen.
  }
% -----------------------------------------------------------------------------
\subsection{Homogene lineare Dgl'en}
% -----------------------------------------------------------------------------
\lessertheorem:$y'(x)=a(x)y(x)$ Dgl, $a\in\SetCont(J,\SetR)$, 
  $A:J\to\SetR$, $A'=a$, $y:I\to\SetR$ ($I\subseteq J$) L"osung=>{
  F"ur $x\in I$ gilt
  \[e^{-A(x)}(y'(x)-a(x)y(x))=0\implies \frac d {dx}(e^{-A}y)(x)=0
    \]
  d.h. der letzte Term ist konstant, also gilt f"ur alle $c\in\SetR$, dass
  $y(x)=ce^{A(x)}$.
  }
% -----------------------------------------------------------------------------
\remark:{
  Ist also in diesem Fall $y(x)=0$ f"ur ein $x\in J$, so gilt $y=0$.
  }
% -----------------------------------------------------------------------------
\subsection{Inhomogene lineare Dgl'en}
% -----------------------------------------------------------------------------
\lessertheorem:$y'(x)=a(x)y(x)+g(x)$ Dgl, $a,g\in\SetCont(J,\SetR)$,
  $A:J\to\SetR$, $A'=a$=>{
  Mit dem Ansatz $y(x)=e^{A(x)}c(x)$ soll nun $c(x)$ so bestimmt werden, dass
  $y$ L"osung der Dgl wird.
  \begin{align*}
    y'(x)&=c(x)a(x)e^{A(x)}+c'(x)e^{A(x)} \\
         &\overset{!}{=}c(x)a(x)e^{A(x)}+g(x) \\
	 &\equiv c'(x)=e^{-A(x)}g(x)
    \end{align*}
  W"ahle $x_0\in J$ fest und setze
  \[c(x):=\int_{x_0}^x g(\xi)e^{-A(\xi)}d\xi
    \]
  Damit gilt
  \begin{align*}
    y(x)&=e^{A(x)}\int_{x_0}^x g(\xi) e^{-A(\xi)} d\xi \\
        &=\int_{x_0}^x e^{\int_\xi^x a(\eta)d\eta} g(\xi) d\xi
    \end{align*}
  Sind nun $y_1,y_2:I\to\SetR$ ($I\subseteq J$) L"osungen der inhomogenen
  Gleichung, so ist $y:=y_1-y_2$ L"osung der zugeh. homogenen Gleichung.
  Umgekehrt kann man auf diese Art und Weise auch aus einer inh. und einer 
  hom. L"osung eine weitere inh. L"osung erzeugen.
  }
% -----------------------------------------------------------------------------
\subsection{Lineare AWPe}
% -----------------------------------------------------------------------------
\theorem: $J\subseteq$ Intervall, $x_0\in J$, $y_0\real$, 
  $y'(x)=a(x)y(x)+g(x)$, $y(x_0)=y_0$ AWP mit $a,g\in\SetCont(J,\SetR)$ =>{
  Dann sind genau die Funktionen
  \[y(x)=\int_{x_0}^x e^{\int_\xi^x a(\eta)d\eta} g(\xi) d\xi
    +y_0e^{\int_{x_0}^x a(\xi)d\xi}
    \]
  L"osung des AWP auf $J$.
  }
% -----------------------------------------------------------------------------
\subsection{Dgl'en mit getrennten Ver"anderlichen}
% -----------------------------------------------------------------------------
\definition Differentialgleichung mit getrennten Ver"anderlichen:{
  \index{Ver"anderliche>Differentialgleichung mit getrennten}
  \index{Getrennte Ver"anderliche>Differentialgleichung mit}
  Seien $J_1,J_2$ Intervalle, $g\in\SetCont(J_1,\SetR)$, 
  $f\in\SetCont(J_2,\SetR)$. Eine Differentialgleichung der Form
  \[y'(x)=g(x)f(y(x))
    \]
  hei"st Differentialgleichung mit getrennten Ver"anderlichen.
  Im Spezialfall $g\ident 1$ hei"st die Dgl. autonom.
  \index{Autonome Differentialgleichung}
  \index{Differentialgleichung>autonome}
  }
% -----------------------------------------------------------------------------
\remark Eine spezielle L"osung:{
  Seien $(x_0,y_0)\in J_1\times J_2$. Ist $f(y_0)=0$, so ist
  $y:J_1\to\SetR$ mit $y(x)=y_0$ eine L"osung der obigen Dgl, oft
  sind aber noch weitere L"osungen vorhanden.
  }
% -----------------------------------------------------------------------------
\theorem:
  $J_1,J_2$ Intervalle, $g\in\SetCont(J_1,\SetR)$, 
  $f\in\SetCont(J_2,\SetR)$. $(x_0,y_0)\in \inner{J_1}\times\inner{J_2}$,
  $f(y_0)\neq 0$=>{
  Ist $y:I\to\SetR$ L"osung der Dgl $y'(x)=g(x)f(y(x))$ mit 
  $y(x_0)=y_0$ und $f(y(x))\neq 0$ ($x\in I$), so gilt:
  
  F"ur jedes $x\in I$ besitzt $f$ keine Nullstellen zwischen 
  $y_0=y(x_0)$ und $y(x)$, und damit
  \[\int_{x_0}^x g(\xi)d\xi=\int_{y_0}^{y(x)} \frac 1 {f(\eta)} d\eta
    \qquad (x\in I)
    \]
  denn es ist (abgelitten)
  \begin{equation}
    g(x)=\frac{y'(x)}{f(y(x))}
    \tag{$*$}
    \end{equation}
  Somit ist $z\mapsto\int_{y_0}^{y(z)} 1/f(\eta) d\eta$ streng monoton, also
  umkehrbar, $(*)$ kann also nach $y$ aufgel"ost werden. Insgesamt gilt:
  
  Es existiert ein $\epsilon>0$ so, dass f"ur $I:=(x_0-\epsilon,x_0+\epsilon)$ 
  gilt:
  \begin{stmts}
    \item Das AWP $y'(x)=g(x)f(y(x)),y(x_0)=y_0$ hat eine L"osung
      $y:I\to\SetR$ mit $f(y(x))\neq 0$ ($x\in I$)
    \item Sind $y_1:I_1\to\SetR$, $y_2:I_2\to\SetR$ L"osungen, so stimmen diese
      punktweise auf $I_1\cap I_2$ "uberein.
    \end{stmts}
  }
% -----------------------------------------------------------------------------
\remark Zum Autonomen Fall:{
  Sei $f\in\SetCont(J,\SetR)$. Betrachte $y'(x)=f(y(x))$ Ist $y:\SetR\to\SetR$
  L"osung, so ist $y_a:\SetR\to\SetR$ mit $y_a(x):=y(x+a)$ und $a\in\SetR$
  ebenfalls eine L"osung auf $\SetR$.
  }
% -----------------------------------------------------------------------------
\theorem:
  $J\subseteq\SetR$ Intervall, $f\in\SetCont(J,\SetR)$, $y:I\to\SetR$
  L"osung von $y'(x)=f(y(x))$=>{
  Dann ist $y$ monoton auf $I$.
  }
% -----------------------------------------------------------------------------
\section{Einige Substitutionen}
% -----------------------------------------------------------------------------
\lessertheorem N"utzliche Substitutionen:=>{
  F"ur die folgenden Dgl'en von spezieller Gestalt existieren Tricks:
  \begin{stmts}
    \item Die \indexthis{Homogene Differentialgleichung} 
      (Vorsicht: Homogene Dgl $\neq$ homogene lineare Dgl)
      \index{Differentialgleichung>homogene}
      \[y'(x)=f\left(\frac {y(x)} x\right)
        \]
      Substituiere $u(x)=y(x)/x$. Dann ist $y'(x)=u(x)+xu'(x)$ (Bew: Definition
      von $u$ ableiten, aufl"osen)
      Dann ergibt sich (einsetzen in Dgl)
      \[u'(x)=\frac{f(u(x))-u(x)} x
        \]
      also eine Dgl mit getrennten Ver"anderlichen.
    \item Betrachte
      \[y'(x)=f(ax+by(x))
        \]
      Setze $u(x):=ax+by(x)$. Dann ist $u'(x)=a+bf(u(x))$ (linear)
    \item Die \indexthis{Bernoulli'sche Dgl}
      \index{Differentialgleichung>Bernoulli'sche}
      \[y'(x)=g(x)y(x)+h(x)y^\alpha(x)\qquad \alpha\neq 1
        \]
      Setze $u(x):=y^{1-\alpha}(x)$. Es ergibt sich:
      \begin{align*}
        u'(x) &= (1-\alpha)y^{-\alpha}(x)y'(x) \\
              &= (1-\alpha)\left[g(x)y^{1-\alpha}(x)+h(x)\right] \\
	      &= (1-\alpha)g(x)u(x)+(1-\alpha)h(x)
        \end{align*}
      also eine lineare Dgl in $u$.
    \item Die \indexthis{Riccati'sche Differentialgleichung}
      \index{Differentialgleichung>Riccati'sche}
      \[y'(x)=k(x)+g(x)y(x)+h(x)y^2(x)
        \]
      Sind $y,\tilde y$ L"osungen dieser Dgl, so gen"ugt $u=y-\tilde y$
      einer Bernoulli'schen Dgl mit $\alpha=2$, n"amlich
      \[u'(x)=\left[g(x)+2\tilde y(x)h(x)\right]u(x)+h(x)u^2(x)
        \]
      die mit $v:=1/u$ in eine lineare Dgl "uberf"uhrt werden kann.
      Kennt man also eine L"osung $\tilde y$ der Riccati'schen
      Dgl, so k"onnen weitere L"osungen durch L"osen einer linearen
      Dgl berechnet werden.
    \end{stmts}
  }
% -----------------------------------------------------------------------------
\remark:{
  Bis jetzt ist kein ``Rechenverfahren'' bekannt, das eine ausreichend gro"se
  Kategorie von Dgl'en l"ost.
  }
% -----------------------------------------------------------------------------
\section{Ein Eindeutigkeitssatz}
% -----------------------------------------------------------------------------
\theorem:$J\subseteq\SetR$ Intervall, $x_0\in J$, 
  $a,g,v\in\SetCont(J,\SetR)$, $v$ db auf $\inner J$ mit
  $v'(x)\leq a(x)v(x)+g(x)$ f"ur $x\in\inner J$=>{
  Dann gilt f"ur $x\in J$
  \[y(x)\stack(\leq;\geq) v(x)
    e^{\int_{x_0}^x a(\xi)d\xi}+\int_{x_0}^x e^{\int_\xi^x a(\eta)d\eta}g(\xi)d\xi
     \qquad\stack(\text{f"ur $x\geq x_0$};\text{f"ur $x\leq x_0$})
    \]
  % *** FIXME stimmt so?
  }
% -----------------------------------------------------------------------------
\definition Lokale Lipschitz-Stetigkeit:{
  \index{Stetigkeit>Lokale Lipschitz-}
  \index{Lipschitz-Stetigkeit>Lokale}  
  Sei $D\subseteq\SetR\times\SetR^p$, $f:D\to\SetR^p$.
  Dann hei"st $f$ lokal Lipschitz-stetig bzgl. $y$, falls gilt:
  \begin{multline*}
    (\forall (x_0,y_0)\in D)(\exists L,\delta >0)(\forall (x,y),(x,\tilde y)\in D) \\
      (|x-x_0|,\norm{y-y_0},\norm{\tilde y-y_0}<\delta\implies
      \norm{f(x,y)-f(x,\tilde y)}\leq L\norm{y-\tilde y})
    \end{multline*}
  }
% -----------------------------------------------------------------------------
\remark:{
  Diese Definition ist unabh"angig von der gew"ahlten Norm, nicht jedoch
  die Gr"o"se von $L$ und $\delta$.
  }
% -----------------------------------------------------------------------------
\theorem:
  $D\subseteq\SetR\times\SetR^p$, $f:D\to\SetR^p$ lokal Lipschitz-stetig
  bzgl. $y$=>{
  \label{the:eindeutig}
  Seien $y_1:I_1\to\SetR^p$, $y_2:I_2\to\SetR^p$ L"osungen von
  \[y'(x)=f(x,y(x))\qquad y(x_0)=y_0
    \]
  so stimmen $y_1$ und $y_2$ auf $I_1\cap I_2$ punktweise "uberein.
  }
% -----------------------------------------------------------------------------
\remark:{
  F"ur Eindeutigkeit nach rechts ($x\geq x_0$) gen"ugt eine lokale
  Bedingung, n"amlich
  \[(y-\tilde y)(f(x,y)-f(x,\tilde y))\leq L\norm{y-\tilde y}
    \]
  }
% -----------------------------------------------------------------------------
\remark Ableitung der Euklidnorm:{
  Sei $\norm \cdot$ die Euklidnorm. Dann gilt 
  $\norm\cdot\in\SetCont^\infty(\SetR^p\setminus\{0\},\SetR)$. Sei nun 
  $u:I\to\SetR^p$ db auf $I$. Dann gilt
  \[\frac d {dx} \norm{u(x)}=\frac 1{\norm{u(x)}} u(x)u'(x)
    \]
  }
% -----------------------------------------------------------------------------
\remark:{
  Sei $D\subseteq\SetR\times\SetR$ offen, $f\in\SetCont(D,\SetR)$ und partiell
  stetig differenzierbar nach $y$. 
  Dann ist $f$ lokal Lipschitz-stetig bzgl. $y$.
  }
% -----------------------------------------------------------------------------
\section{Der Satz von Picard-Lindel"of}
% -----------------------------------------------------------------------------
\definition Lipschitz-Stetigkeit:{
   \index{Stetigkeit>Lipschitz-}
   Sei $D=[a,b]\times\SetR^p$ (man nennt $D$ dann einen \indexthis{Streifen}),
   $f:D\to\SetR^p$. Weiter existiere ein $L>0$ so, dass
   \[(\forall (x,y),(x,\tilde y)\in D)(\norm{f(x,y)-f(x,\tilde y)}\leq L\norm{y-\tilde y})
     \]
   Dann nennt man $f$ Lipschitz-stetig bzgl. $y$ auf $D$.
   }
% -----------------------------------------------------------------------------
\remark:{
  Vorsicht! Die Lipschitz-Stetigkeit bzgl. $y$ impliziert nicht die
  Stetigkeit einer Funktion.
  }
% -----------------------------------------------------------------------------
\theorem Satz von Picard-Lindel"of:
  $D=[a,b]\times\SetR^p$ ein \indexthis{Streifen}, $f\in\SetCont(D,\SetR^p)$
  Lipschitz-stetig bzgl. $y$, $(x_0,y_0)\in D$=>{
  \label{the:picard}
  Dann ist das AWP $y'(x)=f(x,y(x)),y(x_0)=y_0$ eindeutig l"osbar.
  
  Weiterhin konvergiert jede Folge ``sukzessiver Approximationen'',
  gegeben durch ein $z_0\in\SetCont([a,b],\SetR^p)$ beliebig und
  die Rekursionsformel
  \[z_{n+1}':=f(x,z_n(x))\qquad z_{n+1}(x_0)=y_0
    \]
  ($n\nnatural$) gleichm"a"sig gegen diese eine L"osung.
  }
% -----------------------------------------------------------------------------
\definition Lineare Differentialgleichung (II):{
\index{Differentialgleichung>lineare}%
\index{Differentialgleichung>homogene}%
\index{Differentialgleichung>inhomogene}%
\index{Lineare Differentialgleichung}%
\index{Homogene Differentialgleichung}%
\index{Inhomogene Differentialgleichung}%
  Sei $A\in\SetCont([a,b],\SetR^{p\times p})$, $g\in\SetCont([a,b],\SetR^p)$,
  $f:[a,b]\times \SetR^p$ definiert durch $f(x,y)=A(x)y(x)+g(x)$ und
  $(x_0,y_0)\in[a,b]\times\SetR^p$.
  
  Das zugeh"orige AWP
  \[y'(x)=A(x)y(x)+g(x)\qquad y(x_0)=y_0
    \]
  hei"st linear und die Dgl hei"st homogen, falls $g=0$, ansonsten inhomogen.
  
  Sei $\fnorm \cdot$ eine Matrixnorm auf $\SetR^{p\times p}$ mit
  $\norm{By}\leq\fnorm B\,\norm y$. Es ist $f\in\SetCont([a,b]\times\SetR^p)$,
  und da $A$ stetig auf $[a,b]$, ist auch $x\mapsto\fnorm{A(x)}$ stetig auf
  $[a,b]$, somit beschr"ankt, somit Lipschitz-stetig.
  Damit ergibt sich die eindeutige L"osbarkeit des AWP.
  }
% -----------------------------------------------------------------------------
\section{Der Existenzsatz von Peano}
% -----------------------------------------------------------------------------
\theorem Existenzsatz von Peano:
  $f\in\SetCont([a,b]\times\SetR^p,\SetR^p)$ beschr"ankt=>{
  \label{the:peano}
  \index{Peano>Existenzsatz von}
  Dann besitzt das AWP $y'(x)=f(x,y(x)),y(x_0)=y_0$ mit 
  $(x_0,y_0)\in[a,b]\times\SetR^p$ eine L"osung auf $[a,b]$.
  }
% -----------------------------------------------------------------------------
\definition Gleichgradige Stetigkeit:{
  \index{Stetigkeit>gleichgradige}
  $\cal F\subseteq\SetCont([a,b]\times\SetR^p,\SetR^p)$ hei"st 
  gleichgradig stetig $:\equiv$
  \[\epsn(\exists \delta>0)(\forall x,\tilde x\in[a,b])(\forall \phi\in\cal F)
    (|x-\tilde x|<\delta \implies \norm{\phi(x)-\phi(\tilde x)}<\epsilon)
    \]
  }
% -----------------------------------------------------------------------------
\remark:{
  Es gilt:
  \begin{itemize}
    \item Gleichgradige Stetigkeit ist eine Eigenschaft von Funktionenmengen,
      nicht von Funktionen.
    \item Ist $\cal F\subseteq\SetCont([a,b],\SetR^p)$ und ex. ein $c>0$ so,
      dass jede Funktion $\phi\in\cal F$ Lipschitz-stetig mit der Konstanten
      $c$ ist, so ist $\cal F$ gleichgradig stetig.
    \end{itemize}
  }
% -----------------------------------------------------------------------------
\theorem Satz von Ascoli-Arcel{\`a}:
  $\cal F\subseteq\SetCont([a,b],\SetR^p)$ gleichgradig stetig, ex. $k>0$ mit
  $(\forall \phi\in\cal F)(\forall x\in[a,b])(\norm{\phi(x)}\leq k)$=>{
  \label{the:ascoli}
  \index{Ascoli-Arcel{\`a}>Satz von}
  Dann besitzt jede Folge $(\phi_n)\subseteq\cal F$ eine glm. konvergente
  Teilfolge.
  }
% -----------------------------------------------------------------------------
\remark:{
  \ref{the:ascoli} ist ein Kompaktheitskriterium im Raum 
  $\SetCont([a,b],\SetR^p)$, versehen mit der Maximumsnorm.
  
  Genauer gilt: $\cal F \subseteq\SetCont([a,b],\SetR^p)$ kompakt $\equiv$
  $\cal F$ ist beschr"ankt, abgeschlossen und gleichgradig stetig.
  }
% -----------------------------------------------------------------------------
\remark Eine Absch"atzung:{
  Ist $f\in\SetCont([a,b]\times\SetR^p,\SetR^p)$ beschr"ankt durch $M$ und
  $y:[a,b]\to\SetR^p$ L"osung des AWP $y'(x)=f(x,y(x))$, so gilt 
  offensichtlich:
  \[\norm{y(x)-y_0}\leq M|x-x_0|
    \]
  }
% -----------------------------------------------------------------------------
\remark AWPe auf Rechtecken:{
  Sei $c>0$, $y_0\in\SetR$, $D:=[a,b]\times[y_0-c,y_0+c]$, 
  $f\in\SetCont(D,\SetR^p)$. Betrachte das AWP $y'(x)=f(x,y(x)), y(a)=y_0$.
  Weil $D$ kompakt und $f$ stetig ist, existiert $M:=\underset D \max f(x,y)$.
  Definiere eine neue Funktion
  \[\tilde f(x,y):=\begin{cases}
      f(x,y_0+c) & (x,y)\in[a,b]\times(y_0+c,\infty) \\
      f(x,y) & (x,y)\in[a,b]\times[y_0-c,y_0+c] \\
      f(x,y_0-c) & (x,y)\in[a,b]\times(-\infty,y_0-c)
      \end{cases}
    \]
  Wegen des Satzes von Peano ist das AWP $y'(x)=\tilde f(x,y(x)),y(a)=y_0$ 
  l"osbar, und eine L"osung von $y$ l"asst sich auf $[a,\min\{a,a+\frac c M\}]$
  auf $y'(x)=f(x,y(x))$ "ubertragen. 
  
  Eine analoge "Uberlegung l"asst sich f"ur die G"ultigkeit von 
  \ref{the:picard} anstellen.
  }
% -----------------------------------------------------------------------------
\section{Nicht fortsetzbare L"osungen}
% -----------------------------------------------------------------------------
\definition Nicht fortsetzbare L"osung:{
  \index{L"osung>nicht fortsetzbare}
  Sei $D:=\SetR^{p+1}$, $f:D\to\SetR^p$, $(x_0,y_0)\in D$. Eine L"osung
  $y:I\to\SetR^p$ des AWP $y'(x)=f(x,y(x))$ hei"st nicht fortsetzbar, wenn
  es keine L"osung $\tilde y:J\to\SetR^p$ mit $J\underset \neq \supset I$ und
  $y=\tilde y$ auf $I$ gibt.
  
  Auf naheliegende analoge Weise werden definiert: Nicht nach rechts/links 
  fortsetzbar.
  }
% -----------------------------------------------------------------------------
\theorem: $D\subseteq\SetR^{p+1}$, $f:D\to\SetR^p$, $(x_0,y_0)\in D$,
  $u:I\to\SetR^p$ L"osung von $y'(x)=f(x,y(x)),y(x_0)=y_0$=>{
  \label{the:fortsetzbar}
  Dann gibt es eine nicht fortsetzbare L"osung dieses AWP.
  }
% -----------------------------------------------------------------------------
\theorem: $D\subseteq\SetR\times\SetR^p$ offen, $f\in\SetCont(D,\SetR^p)$,
  $(x_0,y_0)\in D$=>{
  \label{the:exist-open}
  Dann besitzt das AWP $y'(x)=f(x,y(x)),y(x_0)=y_0$ eine L"osung 
  $y:I\to\SetR^p$ mit $x_0\in \inner I$.
  }
% -----------------------------------------------------------------------------
\remark:{
  Dazu folgende Bemerkungen:
  \begin{itemize}
    \item \ref{the:exist-open} ist im Grunde eine Erweiterung des
      Satzes von Peano auf beliebige offene Teilmengen des $\SetR^p$,
      wobei jetzt nicht mehr wie vormals ein Existenzintervall garantiert
      wird, sondern lediglich die Existenz einer L"osung auf irgendeinem
      Intervall gefolgert werden kann.
    \item Unter den Voraussetzungen von \ref{the:exist-open} gibt es zu
      jeder L"osung $y:I\to\SetR^p$ eine nicht fortsetzbare L"osung
      $\tilde y:J\to\SetR^p$ mit $J\supseteq I$, $\tilde y|_I=y$. 
      Da $D$ offen ist, ist $J$ offen, also von der Form 
      $(\omega_-,\omega_+)$, wobei 
      $\omega_-\in\SetR\cup\{-\infty\},\omega_+\in\SetR\cup\{+\infty\}$.
      $J$ muss offen sein, da man sonst nach \ref{the:exist-open} in
      die Richtung der Abgeschlossenheit fortsetzen k"onnte.
    \item Ist das AWP $y'(x)=f(x,y(x)),y(x_0)=y_0$ in obigem Satz 
      eindeutig l"osbar, so gibt es genau eine nicht fortsetzbare
      L"osung $y:(\omega_-,\omega_+)\to\SetR^p$. Diese L"osung bezeichnet
      man dann als \emph{die} L"osung des AWP. \index{L"osung eines AWP}
      \index{AWP>L"osung eines}\index{Anfangswertproblem>L"osung eines}
      (Analog: \emph{die} L"osung nach rechts/links).
    \item Unter den Voraussetzungen von \ref{the:exist-open} ist z.B.
      eine L"osung $y:[x_0,\omega_+)\to\SetR^p$ nach rechts fortsetzbar
      $\equiv$ es ex. keine kompakte Menge $K\subseteq D$ mit
      $\{(x,y(x))\mid x\in[x_0,\omega_+)\}\subseteq K$.
    \end{itemize}
  }
% -----------------------------------------------------------------------------
\section{Beispiele}
% -----------------------------------------------------------------------------
\remark Die logistische Gleichung:{
  Seien $a,b>0$. Durch
  \begin{equation*}
    y'(x)=ay(x)-by^2(x),y(0)=y_0\geq 0     \tag{*}
    \end{equation*}
  ist eine Differentialgleichung gegeben. Nach \ref{the:eindeutig},
  \ref{the:peano} und \ref{the:fortsetzbar} besitzt das AWP $(*)$ eine 
  eindeutig bestimmte nicht fortsetzbare L"osung:
  \begin{itemize}
    \item Bei entsprechendem $y_0$ sind L"osungen: $y=0,y=\frac a b$.
    \item Sei nun $y_0\not\in\{0,\frac a b\}$. Mit der Methode f"ur Dgl'en
      mit getrennten Ver"anderlichen erh"alt man:
      \[y(x)=\frac{ay_0}{by_0+(a-by_0)e^{-ax}}
        \]
      Falls $0<y_0<\frac a b$ ergibt sich als L"osungsintervall
      $(\omega_-,\omega_+)=(-\infty,\infty)$. Ansonsten, also mit 
      $\frac a b<y_0$ ergibt sich
      \[(\omega_-,\omega_+)=(\frac 1 a \log 1-\frac a {by_0},\infty)
        \]
    \end{itemize}
  }
% -----------------------------------------------------------------------------
\remark:{
  Sei $D:=\{(x,y)\in\SetR^2\mid x^2+y^2<1\}$, $f:D\to\SetR$ mit
  \[f(x,y):=\sin\frac 1 {1-(x^2+y^2)}
    \]
  Das AWP $y'(x)=f(x,y(x)),y(0)=0$ ist, da 
  $f$ stetig und lokal Lipschitz-stetig ist, auf $D$ eindeutig l"osbar.
  Sei $y:(\omega_-,\omega_+)\to\SetR$ L"osung des AWP. Dann gelten:
  \begin{itemize}
    \item $\omega_-=-\omega_+$. Es ergibt sich weiterhin, dass jede
      L"osung der Dgl punktsymmetrisch zum Ursrpung ist.
    \item $\omega_+\geq 1/\sqrt{2}$
    \end{itemize}
  }
% -----------------------------------------------------------------------------
\remark:{
  Betrachte
  \begin{align*}
    y_1'(x)&=y_1(x)+y_2(x)+e^{-x}y_1^2(x)&y_1(0)=0 \\
    y_2'(x)&=y_2(x)&y_2(0)=1
    \end{align*}
  Das AWP besitzt eine eindeutige L"osung, sie ist
  \[y(x)=\begin{pmatrix} e^x\tan x \\ e^x \end{pmatrix}
    \]
  auf dem Intervall $(-\frac \pi 2,\frac \pi 2)$.
  }
% -----------------------------------------------------------------------------
\remark:{
  Sei $f\in\SetCont([a,b]\times\SetR,\SetR)$, beschr"ankt und 
  $y:I\to\SetR$ eine nicht fortsetzbare L"osung des AWP's 
  $y'(x)=f(x,y(x)),y(x_0)=y_0$. Dann ist $I=[a,b]$, insbesondere
  hat also jede L"osung des AWP's eine Fortsetzung auf $[a,b]$.
  }
% -----------------------------------------------------------------------------
\section{Lineare Systeme von Dgl'en}
% -----------------------------------------------------------------------------
\definition Lineares System:{
\index{System>lineares}%
\index{Differentialgleichung>lineare}%
\index{Differentialgleichung>homogene}%
\index{Differentialgleichung>inhomogene}%
\index{Lineare Differentialgleichung}%
\index{Homogene Differentialgleichung}%
\index{Inhomogene Differentialgleichung}%
  Sei nun stets $A\in\SetCont([a,b],\SetR^{p\times p})$. Die 
  Differentialgleichung
  \begin{equation*}
    y'(x)=A(x)y(x)+g(x) \tag{L}
    \end{equation*}
  mit einem $g\in\SetCont([a,b],\SetR^p)$ hei"st lineare Dgl im $\SetR^p$,
  bzw. lineares System von $p$ Dgl'en. Die Dgl.
  \begin{equation*}
    y'(x)=A(x)y(x) \tag{H}
    \end{equation*}
  hei"st zugeh"orige homogene Gleichung, w"ahrend $(L)$ wie im Fall $p=1$
  inhomogen hei"st.
  
  Gelegentlich wird o.B.d.A. auch der Fall $I\subseteq\SetR$ statt $I=[a,b]$ 
  betrachtet.
  }
% -----------------------------------------------------------------------------
\remark L"osbarkeit von linearen Systemen:{
  Nach Definition sind alle linearen AWPe eindeutig mit einer nicht
  fortsetzbaren L"osung l"osbar. Weiterhin ist jede L"osung
  $y:[a,b]\to\SetR^p$ der Dgl von der Form $y\ident y_h+y_s$, dabei
  ist $y_h$ die L"osung des homogenen Problems $y'(x)=A(x)y(x),y(x_0)=0$
  und $y_s$ die L"osung des inhomogenen Problems
  $y'(x)=A(x)y(x),y(x_0)=y_0$.
  }
% -----------------------------------------------------------------------------
\remark Lineare Unabh"angigkeit von Funktionen:{
  Seien $y_1,\ldots,y_k\in\SetCont([a,b],\SetR)$ und $x\in[a,b]$ beliebig.
  Dann gilt
  \[y_1,\ldots,y_k \text{ linear unabh"angig}\stack(\not\implies;\Leftarrow)
    y_1(x),\ldots,y_k(x)\text{ linear unabh"angig} 
    \]
  }
% -----------------------------------------------------------------------------
\theorem:=>{
  Sei $V:=\{y\in\SetCont^1([a,b],\SetR^p)\mid y \text{ l"ost } (H)\}$.
  Dann gilt: $V$ ist ein Vektorraum mit $\dim V=p$.
  }
% -----------------------------------------------------------------------------
\subsection{Fundamentalsysteme}
% -----------------------------------------------------------------------------
\definition Fundamentalsystem:{
  Jede Basis des L"osungsraumes $V$ eines homogenen linearen Systems $(H)$
  hei"st ein Fundamentalsystem/FS von $(H)$. Ist nun
  $\{y_1,\ldots,y_p\}$ ein Fundamentalsystem von $(H)$, so fasst man diese
  Funktionen zu einer Matrixfunktion $Y:[a,b]\to\SetR^{p\times p}$ zusammen
  durch
  \[Y(x):=\begin{pmatrix}
    y_1(x)|\ldots|y_p(x)
    \end{pmatrix}
    \]
  Diese Funktion hei"st dann \indexthis{Fundamentalmatrix} von $(H)$. 
  }
% -----------------------------------------------------------------------------
\remark L"osungen und Linearit"at:{
  Gilt $Y(x_0)=I$, so ist $y(x)=Y(x)y(x_0)=Y(x)y_0$ L"osung des 
  linearen Systems $(H)$. Ist weiterhin eine L"osung $y_s:[a,b]\to\SetR^p$
  als L"osung von $(L)$ mit $y_s(x_0)=0$ bekannt, so ist die
  L"osung des AWPs $y'(x)=A(x)y(x)+g(x),y(x_0)=y_0$ die Funktion
  $y(x)=Y(x)y_0+y_s(x)$.
  
  Eine Fundamentalmatrix $Y:[a,b]\to\SetR^{p\times p}$ von $(H)$ kann auch
  aufgefasst werden als L"osung der Matrixgleichung $Y'(x)=A(x)Y(x)$
  }
% -----------------------------------------------------------------------------
\theorem: V L"osungsraum von $(H)$, $y_1,\ldots,y_k\in V$=>{
  Dann gilt
  \begin{multline*}
    y_1,\ldots,y_k\text{ linear unabh"angig} \equiv \\
    (\forall x\in[a,b])(y_1(x),\ldots,y_k(x)\text{ linear unabh"angig})
    \end{multline*}
  Insbesondere gilt: Ist $Y:[a,b]\to\SetR^{p\times p}$ ein Fundamentalsystem
  von $(H)$, so ist $\det Y(x)\neq 0$.
  Hierbei ist ``$\implies$'' die interessante Richtung, die andere ist trivial.
  }
% -----------------------------------------------------------------------------
\remark L"osung von $(L)$ durch Variation der Konstanten:{
  Ist zu einer Gleichung $(H)$ ein FS $Y:[a,b]\to\SetR^{p\times p}$ bekannt,
  so f"uhrt der Ansatz $y_s(x)=Y(x)c(x)$ (Variation der Konstanten) f"ur
  eine L"osung von $(L)$ auf
  \begin{align*}
    & y_s'(x)=\underbrace{Y'(x)}_{A(x)Y(x)}c(x)+Y(x)c'(x)\overset !=A(x)Y(x)c(x)+g(x) \\
    \equiv & Y(x)c'(x)=g(x)\equiv c'(x)=Y^{-1}(x)g(x)
    \end{align*}
  Setzt man also \footnote{hier koordinatenweise integrieren}
  \[c(x):=\int_{x_0}^x Y^{-1}(\xi)g(\xi)d\xi
    \qquad(x_0\in[a,b]\text{ fest})
    \]
  dann ist $y^*(x):=Y(x)c(x)$ die L"osung des AWP's 
  $y'(x)=A(x)Y(x)+g(x),y(x_0)=y_0$. Ist $Y:[a,b]\to\SetR^{p\times p}$
  irgendein Fundamentalsystem, so sind die L"osungen einer Gleichung der Art
  $(L)$ genau die Funktionen 
  \[y(x)=Y(x)c+Y(x)\int_{x_0}^x Y^{-1}(\xi)g(\xi)d\xi \qquad c\in\SetR^p
    \]
  }
% -----------------------------------------------------------------------------
\subsection{Die Wronski-Determinante}
% -----------------------------------------------------------------------------
\definition Wronski-Determinante:{
  \index{Determinante>Wronski-}
  Sind $y_1,\ldots,y_p$ $p$ L"osungen von $(H)$ (nicht notwendigerweise
  l.u.), so nennt man $\Phi(x):=\det(y_1|\ldots|y_p)(x)$ ihre 
  Wronski-Determinante.
  }
% -----------------------------------------------------------------------------
\remark Eigenschaften der Wronski-Determinante:{
  Da $p$ L"osungen von $(H)$ genau dann l.u. sind, wenn $y_1(x),\ldots,y_p(x)$
  f"ur alle $x\in[a,b]$ l.u. sind, gilt entweder $\Phi(x)=0$ oder 
  $\Phi(x)\neq 0$, jeweils f"ur $x\in[a,b]$.
  
  Weiterhin ist $\Phi(x)$ differenzierbar und es gilt f"ur $x\in[a,b]$
  \[\Phi'(x)=(\Spur A(x))\cdot \Phi(x)
    \]
  }
% -----------------------------------------------------------------------------
\subsection{Das Reduktionsverfahren von d'Alembert}
% -----------------------------------------------------------------------------
\lessertheorem Reduktionsverfahren von d'Alembert:$y:[a,b]\to\SetR^p$ L"osung
  von $y'(x)=A(x)y(x)$,$j\in\{1,\ldots,p\}$ beliebig gew"ahlt=>{
  Dann erh"alt man mit dem Ansatz $z(x):=h(x)y(x)+g(x)$, 
  $h$ reellwertig (unbekannt), $g\cdot e_j=0$:
  \begin{align*}
    &z'(x)=h'(x)\cdot y(x)+\underbrace{h(x)y'(x)}_{h(x)A(x)y(x)}+g'(x)
    \overset !=A(x)h(x)y(x)+A(x)g(x)\\
    \equiv &h'(x)y(x)+g'(x)=A(x)g(x)
    \end{align*}
  Ist nun $y\cdot e_j\neq 0$ auf einem Intervall $I\subseteq[a,b]$, so
  erh"alt man wegen $g\cdot e_j=0$ ein lineares Dgl.system mit $p-1$ 
  Gleichungen f"ur alle Komponentenfunktionen von $g$ au"ser der $j$-ten,
  in welchem ebendiese $j$-te Koordinate durch $h$ ersetzt werden kann.
  }
% -----------------------------------------------------------------------------
\example Reduktionsverfahren von d'Alembert:{
  Betrachte $A:(0,\infty)\to\SetR^{2\times 2}$ mit
  \[A(x)=\begin{pmatrix}
      \frac 1x & -1 \\ \frac 1 {x^2} & \frac 2 x
      \end{pmatrix}
      \text{ und }
      y_1(x):=\begin{pmatrix}
      x^2\\-x
      \end{pmatrix}
    \]
  $y_1(x)$ ist L"osung von $y'(x)=A(x)y(x)$ und z.B. durch einen Polynomansatz 
  zu ermitteln.
  Ziel: Gewinnung einer weiteren, linear unabh"angigen L"osung.
  W"ahle $j:=1$ f"ur die Reduktion. Ansatz:
  \begin{align*}
    &y_2(x)=h(x)y_1(x)+
    \underbrace{\begin{pmatrix}0\\ \eta(x)\end{pmatrix}}_{g(x):=}\\
    \underset{\text{oben}}\implies &h'(x)\begin{pmatrix}
      x^2\\-x
      \end{pmatrix}+\begin{pmatrix}0\\ \eta'(x)\end{pmatrix}=
      \begin{pmatrix}
      \frac 1x & -1 \\ \frac 1 {x^2} & \frac 2 x
      \end{pmatrix}g(x)\\
    \equiv&\left\{\begin{gathered}
       h'(x)x^2=-\eta(x)\\
       h'(x)x+\eta'(x)=\frac 2 x \eta(x)\end{gathered}\right.
       \implies h'(x)=-\frac{\eta(x)}{x^2}\\
    \implies &-\frac{\eta(x)}{x^2}'(x)x+\eta'(x)=\frac 2 x \eta(x)\\
    \equiv &\eta'(x)=\frac 1 x \eta(x) 
    \end{align*}
  Daher erkennt man: $\eta(x)=x$ ist eine L"osung dieser Dgl.
  Dann ist $h'(x)=-\frac 1 x$, also $h(x)=-\log x$.
  Damit ist ein Fundamentalsystem bestimmt:
  \[
    Y(x):=\begin{pmatrix}
      x^2 & x^2\log x \\
      -x & x+x\log x
      \end{pmatrix}
    \]
  ist Fundamentalsystem, denn $\det Y(x)\neq 0$ f"ur $x\in(0,\infty)$.
  Um nun das AWP $y'(x)=A(x)y(x),y(1)=\begin{pmatrix}1\\1\end{pmatrix}$
  zu l"osen, betrachte man das LGS
  \[Y(1)c=\begin{pmatrix}1\\1\end{pmatrix}
    \equiv c=\begin{pmatrix}1\\2\end{pmatrix}
    \]
  Die L"osung des obigen AWP ist somit
  \[y(x):=Y(x)\begin{pmatrix}1\\2\end{pmatrix}=
    \begin{pmatrix}
      x^2+2x^2\log x\\ 
      x+2x\log x
      \end{pmatrix}
    \qquad x\in(0,\infty)
    \]
  }
% -----------------------------------------------------------------------------

\documentclass[oneside,a4paper,fleqn]{article}

\usepackage[german]{babel}
\usepackage{latexsym}
\usepackage{amsmath}
\usepackage{amssymb}
\usepackage{makeidx}
\usepackage{fancyhdr}
\pagestyle{fancy}

\renewcommand{\headrulewidth}{0pt}
\renewcommand{\footrulewidth}{0.4pt}
\hoffset -1in
\voffset -1in
\textwidth 470pt
\textheight 700pt
\headwidth 470pt

% -*- LaTeX -*-
% -----------------------------------------------------------------------------
% Various helpful definitions
% -----------------------------------------------------------------------------
% (c) iXiONmedia 98/99 ak
% -----------------------------------------------------------------------------

\usepackage{ifthen}
\usepackage{color}

\nonstopmode
\parindent=0pt
\raggedbottom
\parskip=\medskipamount
\makeatletter

% Register definitions --------------------------------------------------------

\newcounter{nochap}
\newcounter{notheorem}
\newcounter{nodefinition}
\newcounter{stmtid}

\newboolean{annotate}
\newboolean{prove}

\setboolean{annotate}{true}
\setboolean{prove}{true}

% Abbreviations ---------------------------------------------------------------

\let\phi=\varphi
\let\epsilon=\varepsilon
\let\rho=\varrho
\let\theta=\vartheta

\def\stack(#1;#2){\genfrac{.}{.}{0pt}{}{\scriptstyle #1}{\scriptstyle #2}}
\def\tstack(#1;#2){$\stack(\text{#1};\text{#2})$}
\def\arcsinh{\operatorname{arcsinh}}
\def\arccosh{\operatorname{arccosh}}
\def\arccot{\operatorname{arccot}}
\def\id{\operatorname{id}}
\def\det{\operatorname{det}}
\def\grad{\operatorname{grad}}
\def\rot{\operatorname{rot}}
\def\exp{\operatorname{exp}}
\def\rank{\operatorname{rank}}
\def\Spur{\operatorname{Spur}}
\def\Kern{\operatorname{Ker}}
\def\Bild{\operatorname{Bild}}
\def\sinc{\operatorname{sinc}}
\def\DFT{\operatorname{DFT}}
\def\ord{\operatorname{ord}}
\def\ggT{\operatorname{ggT}}
\def\kgV{\operatorname{kgV}}
\def\degree#1{\operatorname{deg} #1}
\def\FT{\mathcal{F}}
\def\mat#1{\begin{pmatrix}#1\end{pmatrix}}

% Import from T. Dussa: Info 3 Skript -----------------------------------------
\def\GL{\operatorname{GL}}
\def\Aut{\operatorname{Aut}}
\def\bij{\operatorname{bij.}}
\def\rg{\operatorname{rg}}
\def\diag{\operatorname{diag}}
\def\kopf{\operatorname{kopf}}
\def\first{\operatorname{first}}
\def\follow{\operatorname{follow}}
\def\Anf{\operatorname{Anf}}
\def\SLL{\operatorname{SLL}}
\def\LL{\operatorname{LL}}
\def\SLR{\operatorname{SLR}}
\def\LR{\operatorname{LR}}
\def\LK{\operatorname{LK}}
\def\korrekt{\operatorname{korrekt}}
\def\kompakt{\operatorname{kompakt}}
\def\halt{\operatorname{halt}}
\def\total{\operatorname{total}}
\def\Quad{\operatorname{quad}}
\def\anw{\operatorname{anw}}
\def\naechste{\operatorname{naechste}}
\def\Dom{\operatorname{Dom}}

\def\injto{\hookrightarrow}
\def\surto{\twoheadrightarrow}
\def\bijto{\overset{\sim}{\longrightarrow}}
\def\ot{\gets}

\def\NP{\textit{NP}}
\def\SAT{\textit{SAT}}
% End of import from T. Dussa: Info 3 Skript ----------------------------------

\def\SetN{\mathbb N}
\def\SetZ{\mathbb Z}
\def\SetQ{\mathbb Q}
\def\SetR{\mathbb R}
\def\SetC{\mathbb C}

\def\limes#1->#2{\lim\limits_{\raise1.5pt\hbox{$\scriptscriptstyle #1\to #2$}}}
\def\limesinf#1->#2{\liminf\limits_{\raise1.5pt\hbox{$\scriptscriptstyle #1\to #2$}}}
\def\limessup#1->#2{\limsup\limits_{\raise1.5pt\hbox{$\scriptscriptstyle #1\to #2$}}}

\def\limn{\limes n->\infty}
\def\liminfn{\limesinf n->\infty}
\def\limsupn{\limessup n->\infty}
\def\sumn#1{\sum_{n=#1}^\infty}
\def\limx#1{\limes x->{#1}}

\def\fsum{{\sum\nolimits'}}
\def\norm#1{{\left\| #1\right\|}}
\def\fnorm#1{{\left|\left|\left|#1\right|\right|\right|}}
\def\product#1{{\left\langle#1\right\rangle}}
\def\iprod#1#2{{\left\langle #1,#2 \right\rangle}}
\def\rim#1{\partial{#1}}
\def\inner#1{{#1}^\circ}
\def\closure#1{\overline{#1}}
\def\qed{\hfill$\Box$}

\long\def\ifnotempty#1#2{{%
\setbox9=\hbox{#1}%
\ifdim\wd9>1pt #2\fi%
}}
\def\insertpic#1{
  \begin{center}\input #1.epic \end{center}
  }
\def\insertfig#1{
  \begin{center}\input #1.pstex_t \end{center}
  }
\def\indexthis#1{\index{#1}#1}

\def\lint{\mathchoice
    {\rlap{\hskip0.2em$\underline{\phantom{A}}$}}
    {\rlap{\hskip0.1em$\underline{\phantom{\scriptstyle A}}$}}
    {OH-SORRY.}
    {OH-SORRY.}
    \int
  }
\def\uint{\mathchoice
    {\rlap{\hskip0.5em$\overline{\phantom{A}}$}}
    {\rlap{\hskip0.2em$\overline{\phantom{\scriptstyle A}}$}}
    {OH-SORRY.}
    {OH-SORRY.}
    \int
  }




% sections and numbering ------------------------------------------------------
\def\resetcounters{
  \setcounter{notheorem}{0}
  \setcounter{nodefinition}{0}
  }
\def\para#1{
  \section{#1}
  \resetcounters
  }
\def\thepara{\thesection}




% special frames --------------------------------------------------------------
\long\def\mathtitle#1#2{
  \pagestyle{fancy}
  \def\headrulewidth{0pt}
  \cfoot{}
  {
    \vskip5cm
    \hrule height 1pt width 17 cm
    \vskip0.5mm
    \hrule height 0.25pt width 17 cm
    \vskip.25cm
    {\Huge \bfseries\hfill #1}\\
    \hrule height 0.25pt width 17 cm
    \vskip0.5mm
    \hrule height 1pt width 17 cm
    \vskip 2cm
    \hfill\begin{minipage}{8cm}
      #2
    \end{minipage}
    }
  \newpage
  }




\def\framedmsg#1{
  \hrule height 0.25pt 
  \hfil #1\hfil\break
  \hrule height 0.25pt 
  }




% mathpar ----------------------------------------------------------------
% #1    Heading
% #2	Caption
% #3    Text

\definecolor{mathgray}{gray}{0.8}
\definecolor{mathbgray}{gray}{0.9}
\newlength{\ix@temp@boxwidth}
\newlength{\ix@temp@boxwidtha}

\hbadness 11000

\long\def\mathpar#1#2#3{
\settowidth{\ix@temp@boxwidth}{\strut\sffamily #1}%
\setlength{\ix@temp@boxwidtha}{\textwidth}%
\addtolength{\ix@temp@boxwidtha}{-\ix@temp@boxwidth}%
\addtolength{\ix@temp@boxwidtha}{-12pt}%
\colorbox{mathgray}{\strut\sffamily #1}%
\colorbox{mathbgray}{\strut\makebox[\ix@temp@boxwidtha][l]{\sffamily #2}}%
{\protected@edef\@currentlabel{#1}
    \nobreak\par\nobreak #3\par\vspace{5pt}
    }
  }




% convention ------------------------------------------------------------------
% #1    Text

\long\def\convention#1{
  \mathpar{Konvention f{\"u}r Kapitel \thepara}{}{#1}
  }




% simple mathpars -------------------------------------------------------------
% #1    Name 
% #2    Text 

\long\def\trick#1:#2{
  \mathpar{Trick}{#1}{#2}
  }
\long\def\remark#1:#2{
  \mathpar{Bemerkung}{#1}{#2}
  }
\long\def\deduction#1:#2{
  \mathpar{Herleitung}{#1}{#2}
  }
\long\def\example#1:#2{
  \mathpar{Beispiel}{#1}{#2}
  }
\long\def\problem#1:#2{
  \mathpar{Problem}{#1}{#2}
  }
\long\def\auxiliary#1:#2{
  \mathpar{Hilfsmittel}{#1}{#2}
  }
\long\def\motivation#1:#2{
  \mathpar{Motivation}{#1}{#2}
  }
\long\def\proof#1:#2{
  \ifprove
    \mathpar{Beweis zu #1}{}{#2}
  \fi
  }
\long\def\annotation#1:#2{
  \ifannotate
    \mathpar{Anmerkung zu #1}{}{#2}
  \fi
  }




% algorithm -------------------------------------------------------------------
% #1    Name 
% #2    Text 

\long\def\algorithm#1:#2{
  \ifnotempty{#1}{
    \index{#1}
    }
  \mathpar{Algorithmus}{#1}{#2}
  }

% definition ------------------------------------------------------------------
% #1    Name
% #2    Text 

\long\def\definition#1:#2{
  \addtocounter{nodefinition}{1}
  \ifnotempty{#1}{
    \index{#1}
    \index{[Def~\thepara.\thenodefinition] #1}
    }
  \mathpar{Definition \thepara.\thenodefinition}{#1}{#2}
  }

\long\def\lesserdefinition#1:#2{
  \ifnotempty{#1}{
    \index{#1}
    }
  \mathpar{Definition}{#1}{#2}
  }




% theorem / lemma -------------------------------------------------------------
% #1    Name
% #2    Prerequisites
% #3    Claim

\newlength{\preconlabelwidth}
\setlength{\preconlabelwidth}{2.5cm}
\newlength{\preconwidth}
\setlength{\preconwidth}{\textwidth}
\addtolength{\preconwidth}{-\preconlabelwidth}

\def\precondition#1{
  \ifnotempty{#1}{
    \parbox[t]{\preconlabelwidth}{\emph{Voraussetzung:}}
    \begin{minipage}[t]{\preconwidth}#1\end{minipage}
  }\par\nobreak%
}

\long\def\theorem#1:#2=>#3{
  \addtocounter{notheorem}{1}
  \ifnotempty{#1}{
    \index{#1}
    \index{[Satz~\thepara.\thenotheorem] #1}
    }
  \mathpar{Satz \thepara.\thenotheorem}{#1}{
    \precondition{#2}
    #3
    }
  }

\long\def\lemma#1:#2=>#3{
  \addtocounter{notheorem}{1}
  \ifnotempty{#1}{
    \index{#1}
    \index{[Lemma ~\thepara.\thenotheorem] #1}
    }
  \mathpar{Lemma \thepara.\thenotheorem}{#1}{
    \precondition{#2}
    #3
    }
  }

\long\def\corollary#1:#2=>#3{
  \addtocounter{notheorem}{1}
  \ifnotempty{#1}{
    \index{#1}
    \index{[Korollar ~\thepara.\thenotheorem] #1}
    }
  \mathpar{Korollar \thepara.\thenotheorem}{#1}{
    \precondition{#2}
    #3
    }
  }

\long\def\lessertheorem#1:#2=>#3{
  \ifnotempty{#1}{\index{#1}}
  \mathpar{Hilfssatz}{#1}{
    \precondition{#2}
    #3
    }
  }




\makeatother


\newenvironment{blatt}[2]{
  \fancyhead{}
  \fancyfoot{}
  \fancyfoot[C]{\"Ubungsblatt #2 im Fach #1 $\cdot$ Andreas Kl\"ockner $\cdot$ 992330 }
 
  \begin{minipage}{8cm}
    {\Large\bfseries #1 }
    
    \"Ubungsblatt Nr. #2
    \end{minipage}\hfill
  \begin{minipage}{5cm}
    \begin{flushright}
      \today \\
      Andreas Kl\"ockner \\
      Matr: 992330 \\
      \end{flushright}
    \end{minipage}\par
  \rule{\textwidth}{1pt}

  }{
  }
\newcounter{aufgabe}
\newenvironment{aufgabe}{
  {\bfseries Aufgabe \arabic{aufgabe}.}\par
  \addtocounter{aufgabe}{1}
  }{\par\rule{\textwidth}{0.5pt}\par}
\newcounter{teilnr}
\newenvironment{teile}{
  \begin{list}{\alph{teilnr})}{\usecounter{teilnr}}
  }{
  \end{list}
  }
\def\dieskipich{\addtocounter{aufgabe}{1}}
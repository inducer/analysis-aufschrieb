\def\To{ERROR}
% -*- LaTeX -*-
% -----------------------------------------------------------------------------
\part{Formale Sprachen}
% -----------------------------------------------------------------------------
\section{Grammatiken und Sprachen}
% -----------------------------------------------------------------------------
\subsection{Grundbegriffe}
% -----------------------------------------------------------------------------
\definition Alphabet:{
  Ein Alphabet ist eine endliche nichtleere Menge $V$ von \indexthis{Zeichen}.
  }
% -----------------------------------------------------------------------------
\definition Wort:{
  Ein Wort oder eine \indexthis{Zeichenkette} $w$ "uber einem Alphabet $V$ ist 
  eine geordnete Menge $w=a_1\ldots a_n$ mit $a_i\in V$. Worte werden
  gemeinhin mit Kleinbuchstaben bezeichnet.
  
  Die Menge aller Zeichenketten schreibt man $V^*$, die Menge aller nichtleeren
  Zeichenketten $V^+:=V^*\setminus \{\lambda\}$.

  Ein Wort $v$ ist genau dann ein \indexthis{Teilwort} des Wortes $w$, wenn 
  es $b,c\in V^*$ gibt mit $w = bvc$.
  }
% -----------------------------------------------------------------------------
\definition Sprache:{
  Eine Sprache $L$ ist eine Teilmenge $L\subseteq V^*$.
  }
% -----------------------------------------------------------------------------
\example Sprache und Alphabet:{
  Eine Sprache "uber dem Alphabet $V := \{0,1\}$ ist z.B. 
  $L:=\{0^n\mid n\nnatural\}$.
  }
% -----------------------------------------------------------------------------
\definition Produktion:{
  Eine Produktion oder \indexthis{Ersetzungsregel} ist ein Zweitupel
  $(\alpha, \beta) \in V^* \times V^*$, auch geschrieben $\alpha \to \beta$. 
  Ein \indexthis{Ersetzungssystem} oder \indexthis{Semi-Thue-System} $(V,P)$ 
  ist ein geordnetes Paar, bestehend aus einem Alphabet $V$ und einer 
  Produktionsmenge $P$.
  }
% -----------------------------------------------------------------------------
\definition Ableitung:{
  Seien nun $x, y \in V^*$. $y$ hei"st direkt ableitbar aus $x$ 
  (Schreibweise: $x\dderivable y$) $:\equiv$
  \[(\exists (\alpha, \beta) \in P)(\exists u, v \in V^*)(x = u \alpha
    v \land y = u \beta v)
    \]
  $y$ hei"st (indirekt) ableitbar aus $x$ (Schreibweise: $x\derivable y$) 
  $:\equiv$ 
  \[(\exists x_1,\ldots,x_n\in V^*)(x\dderivable x_1 \dderivable \cdots 
    \dderivable x_n \dderivable y)
    \]
  }
% -----------------------------------------------------------------------------
\remark:{
  Das Problem festzustellen, ob $x \derivable y$ f"ur gegebene $x,y$ gilt, 
  ist im allgemeinen unentscheidbar.
  }
% -----------------------------------------------------------------------------
\subsection{Grammatiken}
% -----------------------------------------------------------------------------
\definition Grammatik:{
  Ein Viertupel $G=(V,T,S,P)$ mit den Alphabeten $V$ und $T$ sowie dem
  Startsymbol $S$ und der Produktionsmenge $P$ hei"st eine Grammatik,
  falls gilt
  \begin{stmts}
  \item $V\cap T=\emptyset$
  \item $S\in V$
  \item $|P|<\infty$ und f"ur $P$ gilt
    \[(\forall p=u\to v \in P)(u,v\in(V\cup T)^+, u\cap V\ne\emptyset) 
      \]
  \end{stmts}
  $T$ hei"st das \indexthis{Terminalalphabet}, $V$ das
  \indexthis{Variablenalphabet} oder \indexthis{Nichtterminalalphabet}. Ein
  beliebiges Element $q \in T$ hei"st \indexthis{Terminalzeichen}, ein
  beliebiges Element $r \in V$ hei"st \indexthis{Variable} oder 
  \indexthis{Nichtterminal}.

  Sei nun $G=(V,T,S,P)$ eine Grammatik. Dann ist $L(G)$ die von $G$
  erzeugte Sprache, definiert durch 
  \[L(G):=\{w\in T^*\mid  S \derivable w\}
    \]
  Worte \index{Wort} "uber Grammatiken sind Elemente aus $T^*$, w"ahrend
  f"ur eine \indexthis{Satzform} $u$ "uber $G$ ein $u\in(V\cup T)^*$ gilt.
  }
% -----------------------------------------------------------------------------
\example Grammatik:{
  Sei $V=\{S,A,B\}$ und $T=\{0,1\}$. Die Menge $P$ der Produktionen sei
  \[P:=\{S \to 0B | 1A,A\to 0 | 0S | 1AA,B \to 1 | 1S | 0BB\}
    \]
  Dann gilt
  \[L(G)=\{w \in \{0, 1\}^+\mid\text{Anzahl der Nullen gleich
    Anzahl der Einsen}\}
    \]
  }
% -----------------------------------------------------------------------------
\definition "Aquivalenz von Grammatiken:{
  \index{Grammatik>"Aquivalenz von}
  Offensichtlich kann man f"ur ein- und dieselbe Sprache mehrere Grammatiken
  angeben. Zwei Grammatiken $G_1,G_2$ hei"sen "aquivalent 
  $:\equiv$ $L(G_1)=L(G_2)$.
  }
% -----------------------------------------------------------------------------
\subsection{Die Chomsky-Hierarchie}
\label{sub:chomsky}
% -----------------------------------------------------------------------------
\definition Chomsky-Klasse:{
  Sei $G=(V,T,S,P)$ eine Grammatik. Dann hei"st $G$
  \begin{itemize}
    \item kontextsensitiv \index{Grammatik>kontextsensitive} $:\equiv$
      die Produktionen haben die Form
      \begin{itemize}
        \item $yAz \to ywz$ mit $A\in V,w\ne\lambda,y,w,z\in(V\cup T)^*$ 
        \item $S\to\lambda$, aber nur, wenn $S$ auf keiner rechten
          Seite einer Produktionsregel vorkommt.
        \end{itemize}
      Dieser Grammatiktyp zeichnet sich dadurch aus, da"s Worte niemals
      k"urzer werden (au"ser bei der trivialen Produktionsregel).
    \item kontextfrei \index{Grammatik>kontextfreie} $:\equiv$
      die Produktionen haben die Form $A\to w$ mit $A\in V,w\in(V\cup T)^*$.
    \item rechtslinear \index{Grammatik>rechtslineare} $:\equiv$
      die Produktionen haben die Form
      \begin{itemize}
        \item $A \to bC$ mit $A,C\in V,b\in T$
        \item $A \to C$ mit $A,C\in V$
        \end{itemize}
      Die Linkslinearit"at wird analog definiert.
    \end{itemize}
  Einer Grammatik wird nun auch eine Typnummer zugeordnet: $G$ ist 
  vom Typ
  \begin{itemize}
    \item 0 $:\equiv$ die Produktionen sind beliebig
    \item 1 $:\equiv$ sie ist kontextsensitiv (kts)
    \item 2 $:\equiv$ sie ist kontextfrei (ktf) 
    \item 3 $:\equiv$ sie ist rechts- oder linkslinear
    \end{itemize}
  Sei $L_i$ die Klasse aller Grammatiken des Typs $i$. Dann gilt:
  \[L_0 \supsetneqq L_1 \supsetneqq L_2 \supsetneqq L_3
    \]
  Eine Sprache hei"st vom Typ $i$ $:\equiv$ sie wird durch eine 
  Grammatik des Typs $i$ erzeugt.
  }
% -----------------------------------------------------------------------------
\subsection{Abgeschlossenheit}
\label{sub:complete}
% -----------------------------------------------------------------------------
\definition Operationen auf Sprachen:{
  Seien $L, M$ Sprachen "uber $V$. Dann definiert man
  \begin{itemize}
    \item \indexthis{Vereinigung}: $L\cup M$ auf naheliegende Art und 
      Weise.
    \item \indexthis{Schnitt}: $L\cap M$ auf naheliegende Art und 
      Weise.
    \item \indexthis{Konkatenation}: 
      $L\cdot M:=LM:=\{\alpha\beta\mid\alpha\in L,\beta\in M\}$
      Daher auch $L^0:=\{\lambda\}$, $L^i:=L^{i-1}\cdot L$.
    \item \indexthis{Sternoperation}:
      $L^*:=\{\alpha_1\cdots\alpha_n\mid \alpha_i\in L,i,n\nnatural\}$,
      deswegen auch $L^* = \bigcup_{n \geq 0} L^n$
    \item \indexthis{Komplement}:
      $L^c:=T^*\setminus L$
    \end{itemize}
  }
% -----------------------------------------------------------------------------
\theorem Abgeschlossenheit:=>{
  \label{the:complete}
  Die Mengen $L_i$ sind abgeschlossen bez"uglich der Operationen 
  $\cup, \cdot, *$.
  }
% -----------------------------------------------------------------------------
\proof \ref{the:complete}:{
  Seien $G_1=(V_1,T,S_1,P_1),G_2:=(V_2,T,S_2,P_2)$ Grammatiken
  vom Typ $i$. Setze $L_j:=L(G_j)$ f"ur $j=1,2$. O.B.d.A. gelte
  $V_1\cap V_2=\emptyset$.
  \begin{description}
    \item[Vereinigung] Setze $P:=\{S\to S_1|S_2\}\cup P_1\cup P_2$,
      $V:=V_1\cup V_2$. Dann gilt $L(G)=L_1\cup L_2$ mit $G:=(V,T,S,P)$.
    \item[Konkatenation ($i\ne 3$)] Ein Ansatz 
      der Art $P:=\{S \to S_1S_2\}\cup P_1\cup P_2$ 
      ist nicht ausreichend, da auf dem Wege "uber das identische 
      Terminalalphabet Produktionen aus $P_1$ auf Teilworte angewandt werden 
      k"onnten, die eigentlich aus $L_2$ ``stammen''.
    
      Daher legen wir zwei verschiedene Kopien $T_1, T_2$ des
      Terminalalphabets $T$ als Variablenalphabete an. Die
      Produktionen m"ussen nat"urlich ebenfalls entsprechend angepasst
      werden. So seien $Q_i$ die Produktionenmengen nach den erforderlichen
      Umbenennungen, erg"anzt um Produktionen, die aus den 
      M"ochtegern-Terminalen in $T_i$ wieder echte Terminale machen.
    
      Seien weiter $P:=\{S\to S_1S_2\}\cup Q_1\cup Q_2$ sowie 
      $V:=V_1\cup V_2\cup T_1\cup T_2\cup\{S\}$.
    
      Dann ist $G:=(V,T,S,P)$ wieder vom Typ $i$ und $L(G) = L_1L_2$.
    \item[Konkatenation ($i=3$)] Wir nehmen an, die Grammatik sei
      o.B.d.A. rechtslinear. Dann setzen wir
      $P:=P_1\cup\{A\to bS_2\mid A\to b\in P_1\}\cup P_2$.
      Dann ist $G:=(V,T,S_1,P)$ rechtslinear und $L(G)=L_1\cdot L_2$.
    \item[Sternoperation ($i\ne 3$)] Definiere 
      \[P:=P_1\cup\{S\to\lambda|S_1|S_1S_1|S_1S_2S_1,S_2\to S_2S_1S_2\}
        \]
      *** FIXME H"a?
      dabei sind $G_1=G_2$, daher $S_1=S_2$. Dann ist 
      $G:=(V_1\cup\{S\},T,S,P)$ wieder vom gleichen Typ wie $G_1=G_2$ und
      $L(G)=L_1^*$.
    \item[Sternoperation ($i = 3$)] Die Grammatik sei o.B.d.A. 
      rechtslinear. Setze
      \[P:=P_1\cup\{S_1\to\lambda\}\cup\{A\to bS_1\mid A\to b \in P_1\}
        \]
      so ist $G:=(V_1,T,S_1,P)$ rechtslinear und $L(G)=L_1^*$.\qed
    \end{description}
  }
% -----------------------------------------------------------------------------
\theorem:=>{
  Rechtslineare Sprachen sind abgeschlossen bez"uglich der Komplement-
  und Durchschnittbildung. (Beweis sp"ater)
  }
% -----------------------------------------------------------------------------
\section{Regul"are Sprachen}
% -----------------------------------------------------------------------------
\definition Regul"are Sprache:{
  Rechts- oder linkslineare Sprachen werden als regul"are
  Sprachen bezeichnet.
  }
% -----------------------------------------------------------------------------
\subsection{Finite Akzeptoren}
% -----------------------------------------------------------------------------
\definition Deterministischer finiter Akzeptor:{
\index{DFA}%
\index{Akzeptor>deterministischer finiter}%
\index{Finiter Akzeptor>deterministischer}%
\index{Deterministischer finiter Automat}%
\index{Automat>deterministischer finiter}%
\index{Finiter Automat>deterministischer}%
  Ein deterministischer finiter Akzeptor/Automat (DFA) $M$ ist ein Quintupel
  $M=(X,Q,\delta,q_0,F)$ mit dem Eingabealphabet $X$, der Zustandsmenge $Q$,
  der Zustands"ubergangsfunktion $\delta:Q\times X\to Q$, dem Anfangszustand 
  $q_0\in Q$ und der Endzustandsmenge $F$.
  
  Man definiert rekursiv eine mehrstufige "Ubergangsfunktion 
  $\delta^*:Q\times X^*\to Q$ mit $\delta^*(q,\lambda):=q$ und 
  $\delta^*(q,wx):=\delta(\delta^*(q, w),x)$.

  Die von einem Automaten $M$ akzeptierte Sprache 
  \[L(M):=\{w\in X^*\mid\delta^*(q_0, w)\in F\}
    \]
  }
% -----------------------------------------------------------------------------
\example Ein einfacher Automat:{
  \input info3_dfa_simple.epic
  Hierbei ist $Q := \{q_0, q_1\}$, $X := \{0, 1\}$ sowie $F := \{q_0\}$.
  Die Zustands"ubergangstabelle lautet
  \[\begin{array}{c|cc}
    \delta & 0 & 1 \\\hline
    q_0 & q_1 & q_1 \\
    q_1 & q_1 & q_0
    \end{array}\]
  Die von diesem Automaten akzeptierte Sprache ist $L = ((0 + 1)0^*1)^*$. 
  (Notation siehe \ref{def:regex}.)
  }
% -----------------------------------------------------------------------------
\theorem: $M=(X,Q,\delta,q_0,F)$ DFA=>{
  \label{the:dfa-linear}
  Dann ist $L(M)$ rechtslinear.
  }
% -----------------------------------------------------------------------------
\proof \ref{the:dfa-linear}:{
  Wir legen einfach die Funktionsweise eines Automaten in Form von 
  Produktionen dar. Setze 
  \begin{multline*}
    P:=\{q\to xq'\mid\delta(q, x)=q',q,q'\in Q,x\in X\}\cup  \\
    \{q\to x\mid\delta(q, x)\in F,q\in Q,x\in X\}
    \end{multline*}
  Dann ist $G:=(Q,X,q_0,P)$ rechtslinear. \qed
  }
% -----------------------------------------------------------------------------
\definition Nichtdeterministischer finiter Akzeptor:{
\index{NFA}%
\index{Akzeptor>nichtdeterministischer finiter}%
\index{Finiter Akzeptor>nichtdeterministischer}%
\index{Nichtdeterministischer finiter Automat}%
\index{Automat>nichtdeterministischer finiter}%
\index{Finiter Automat>nicht deterministischer}%
  Ein Quintupel $M=(X,Q,\delta,q_0,F)$ hei"st ein
  nichtdeterministischer finiter Akzeptor (NFA) $:\equiv$
  $X, Q, q, F$ sind wie bei einem DFA und die Zustands"ubergangsfunktion 
  ist $\delta:Q\times X\to \cal P(Q)$.
  
  Rekursiv definiert man wieder
  $\delta^*:Q\times X^*\to \cal P(Q)$ mit $\delta^*(q,\lambda):=\{q\}$ und 
  $\delta^*(q,wx):=\delta(\delta^*(q, w),x)$. (Beachte: Ergebnis von 
  $\delta^*$ in vorigem Ausdruck ist bereits eine Menge!)

  Die von $M$ akzekptiere Sprache ist
  \[L(M):= \{w\in X^*\mid\delta^*(q_0, w)\cap F\ne\emptyset\}
    \]
  }
% -----------------------------------------------------------------------------
\example Ein einfacher NFA:{
  \input info3_nfa_simple.epic
  Die zugeh"orige Grammatik ist rechtslinear:
  \begin{align*}
    q_0 &\to aq_0 | aq_1 | bq_2 | b\\
    q_1 &\to bq_2 | b\\
    q_2 &\to aq_2 | a
    \end{align*}
  }
% -----------------------------------------------------------------------------
\theorem: $L$ rechtslinear=>{
  Dann existiert ein NFA $M$ mit $L(M)=L$.
  }
% -----------------------------------------------------------------------------
\theorem:$M=(X,Q,\delta,q_0,F)$ NFA=>{
  \label{the:nfa-dfa}
  Es existiert ein DFA $M'$ mit $L(M) = L(M')$.
  }
% -----------------------------------------------------------------------------
\proof\ref{the:nfa-dfa}:{
  Man definiert schlicht
  \[M':=(X, \cal P(Q),\delta',\{q_0\},F')
    \]
  mit 
  \[\delta'(\{q_1,\ldots,q_n\},x):=\delta(q_1, x)\cup\ldots\cup\delta(q_n,x)
    \] 
  und $F':=\{p\in\cal P(Q)\mid p\cap F\ne\emptyset\}$. 
  Dieser Automat ist ein DFA und erkennt dieselbe Sprache wie $M$.\qed
  }
% -----------------------------------------------------------------------------
\example Umwandlung NFA $\to$ DFA:{
  Wir geben zun"achst einen NFA $M$ an:
  \input info3_convert_nfa.epic
  Die Zustands"ubergangstabelle f"ur diesen NFA lautet wie folgt:
  \[\begin{array}{l}
      p_0 := \{q_0\}\\
      p_1 := \{q_1, q_2\}\\
      p_2 := \{q_0, q_1, q_2\}\\
      p_3 := \emptyset
      \end{array}
      \qquad
      \begin{array}{c|ccc}
       	\delta' & p_0 & p_1 & p_2 \\\hline
       	a & p_1 & p_2 & p_2 \\
       	b & p_3 & p_1 & p_1
      \end{array}
    \]
  Ein zugeh"origer DFA $M'$ ist also
  \input info3_convert_dfa.epic
  }
% -----------------------------------------------------------------------------
\corollary: $L$ rechtslineare Sprache=>{
  Es existiert ein DFA $M$ mit $L(M)=L$.
  }
% -----------------------------------------------------------------------------
\subsection{Das Pumping-Lemma}
% -----------------------------------------------------------------------------
\lemma Pumping-Lemma:
  $M=(X,Q,\delta,q_0,F)$ DFA, $|Q|=n$=>{
  \label{the:pumping-lemma-dfa}
  Dann gilt f"ur alle $z\in L(M)$ mit $|z|>n$
  \[(\exists u,v,w\in T^*)(z=uvw,v\ne\lambda,uv^*w\in L(M))
    \]
  }
% -----------------------------------------------------------------------------
\proof \ref{the:pumping-lemma-dfa}:{
  Sei $z = a_1\ldots a_k$, $|z|=k>n$. Dann existieren $q_1,\ldots,q_n\in Q$ mit
  $\delta(q_{i-1},a_i)=q_i$ f"ur $i=1,\ldots k$. 
  
  Wegen $k>n$ existieren nun mehr Zust"ande des DFA als Zeichen im Wort, 
  es existieren also $i<j\in\{1,\ldots,k\}$ mit $q_i=q_j$. Dann erf"ullt 
  $v:=a_{i+1}\ldots a_j$ die Behauptung, denn der Zyklus $q_i\ldots q_j=q_i$ 
  akzeptiert $v$ beliebig oft.
  \qed
  }
% -----------------------------------------------------------------------------
\remark Folgerung:{
  Die Sprache $\{a^nb^n \mid n \in \mathbb{N}\}$ ist nicht rechtslinear, 
  aber kontextfrei. W"are sie rechtslinear, m"usste f"ur sie auch
  \ref{the:pumping-lemma-dfa} gelten. Offensichtlich kann dies aber
  nicht der Fall sein, da kein Teilwort $v$ von $a^nb^n$ die geforderte
  Bedingung erf"ullen kann.
  }
% -----------------------------------------------------------------------------
\theorem: $K,L$ rechtslinear=>{
  \label{the:linear-cap-compl}
  Dann sind auch $K^c$ und $K \cap L$ rechtslinear.
  }
% -----------------------------------------------------------------------------
\proof \ref{the:linear-cap-compl}:{
  Wir zeigen beide Teile der Behauptung:
  \begin{description}
    \item[Komplement:]
      Betrachte den zu $K$ geh"origen DFA
      $M:=(X,Q,\delta,q_0,F)$. Dann ist auch 
      $M^c:=(X,Q,\delta,q_0,Q\setminus F)$ ein DFA, und er akzeptiert genau
      $K^c$. Also ist auch $K^c$ rechtslinear.
    \item[Schnitt:]
      Man mache sich klar, dass gilt
      \[K\cap L=(K^c\cup L^c)^c
       	\]
      womit sich die Aussage aus dem Vorangegangenen ergibt.\qed
    \end{description}
  }
% -----------------------------------------------------------------------------
\subsection{Regul"are Ausdr"ucke}
% -----------------------------------------------------------------------------
\definition Regul"arer Ausdruck:{
  \index{Ausdruck>regul"arer}
  \label{def:regex}
  $R$ ist ein regul"arer Ausdruck "uber einem Alphabet $T$, falls
  \begin{enumerate}
    \item $R = \emptyset, \lambda, a$ mit $a \in T$
    \item $R = (R_1 + R_2)$ mit regul"aren Ausdr"ucken $R_1, R_2$ (Vereinigung)
    \item $R = (R_1 \cdot R_2)$ mit regul"aren Ausdr"ucken $R_1, R_2$ (Konkatenation)
    \item $R = (R_1)^*$ mit regul"arem Ausdruck $R_1$ (Sternoperation)
    \end{enumerate}
  Eine Sprache $L$ hei"st regul"ar, falls sie von einem regul"aren
  Ausdruck erzeugt wird.
  \index{Regul"are Sprache}
  \index{Sprache>regul"are}
  }
% -----------------------------------------------------------------------------
\example Regul"are Ausdr"ucke:{
  Z.B. $\emptyset, \lambda, a^*, (ab + bb)^*$.
  }
% -----------------------------------------------------------------------------
\remark :{
  Mit den Ergebnissen aus \ref{sub:complete} erhalten wir:
  Jede regul"are Sprache ist rechtslinear.
  }
% -----------------------------------------------------------------------------
\theorem: $M=(Q,X,\delta,q_0,F)$ ein DFA, $Q=\{q_0,\ldots,q_n\}$=>{
  \label{the:dfa-regular}
  Dann ist $L(M)$ regul"ar.
  }
% -----------------------------------------------------------------------------
\proof \ref{the:dfa-regular}:{
  Seien $i,j\in\{0,\ldots,n\}$ beliebig.
  Setze $R_{ij}:=\{w\mid\delta^*(q_i,w)=q_j\}$. Ist $w\in R_{ij}$
  und $q_i,q_{i_1},\ldots,q_{i_l},q_j$ die Abfolge der Zust"ande
  bei der Akzeption von $w$, so definiere $Q(i,j,w):=\{q_{i_1},\ldots,q_{i_l}\}$.
  Definiere dann weiterhin 
  \[R_{ij}^k:=\{w\in R_{ij}\mid Q(i,j,w)\cap\{q_k,\ldots,q_n\}=\emptyset\}
    \]
  Wir zeigen durch vollst"andige Induktion "uber $k$: $R_{ij}^{n+1}=R^{ij}$
  ist rechtslinear. F"ur den Induktionsanfang ($R_{ij}^0$) ist dies klar, 
  denn es gilt $R_{ij}^0\subseteq X$.
  
  $k \curvearrowright k + 1$: Es gilt offensichtlich
  $R_{ij}^{k+1}=R_{ij}^k+R_{ik}^k(R_{kk}^k)^*R_{kj}^k$. Weiterhin ist
  dieser Ausdruck nach Bildung und Induktionsvoraussetzung regul"ar.\qed
  }
% -----------------------------------------------------------------------------
\remark Linearit"at:{
  Da die Argumentation von \ref{the:dfa-regular} auch auf linkslineare
  Sprachen anwendbar ist, erh"alt man:
  \[\{\text{rechtslineare Sprachen}\}=\{\text{regul"are Sprachen}\}=
    \{\text{linkslineare Sprachen}\}
    \]
  }
% -----------------------------------------------------------------------------
\subsection{Eine "Aquivalenzrelation}
\label{sub:regular-equivrel}
% -----------------------------------------------------------------------------
\definition Index einer "Aquivalenzrelation:{
  \index{"Aquivalenzrelation>Index einer}
  Sei ``$\sim$'' eine "Aquivalenzrelation. Dann nennt die
  Anzahl der von $\sim$ erzeugten "Aquivalenzklassen auch den 
  Index von $\sim$, geschrieben $\#\sim$.
  }
% -----------------------------------------------------------------------------
\definition Charakteristische "Aquivalenzrelation:{
  Sei $L\subseteq T^*$ eine Sprache. Dann definiert man f"ur $x,y\in T^*$
  \[x\sim_L y:\equiv (\forall z\in T^*)(xz\in L\equiv yz\in L)
    \]
  Man kann leicht nachpr"ufen, dass diese Relation symmetrisch,
  reflexiv und transitiv, also eine "Aquivalenzrelation ist.
  
  Sei $M=(Q,X,\delta,q_0,F)$ ein DFA. Dann definiert man f"ur $x,y\in X^*$
  \[x\sim_M y:\equiv \delta^*(q_0,x)=\delta^*(q_0,y)
    \]
  Auch diese Relation ist eine "Aquivalenzrelation, ihr Index ist gerade
  $|Q|$.
  }
% -----------------------------------------------------------------------------
\theorem: $L\subseteq T^*$ Sprache=>{
  \label{the:regular-equiv}
  Dann gilt: $L$ regul"ar $\equiv$ Index von $\sim_L$ $<\infty$.
  }
% -----------------------------------------------------------------------------
\proof \ref{the:regular-equiv}:{
  \begin{description}
    \item[``$\implies$''] Sei $M=(Q,T,\delta,q_0,F)$ ein DFA
      mit $L = L(M)$. Dann gilt f"ur alle $x\in T^*$:
      $x\sim_M y\implies x\sim_L y$. In diesem Sinne ist
      $\sim_M$ eine Verfeinerung von $\sim_L$, also gilt      
      $\infty>\#\sim_M>\#\sim_L$.
    \item[``$\Leftarrow$''] Seien $[w_1], \dots, [w_n]$ die
      "Aquivalenzklassen von $\sim_L$. Wir definieren dann $M:=(Q,X,\delta,q_0,F)$ 
      mit $Q:=\{[w_1],\ldots,[w_n]\}$, $\delta([w_i],x):=[wx]$, $q_0:=[\lambda]$,
      $F:=\{[w]\mid w\in L\}$. Die Definition von $F$ ist 
      repr"asentantenunabh"angig, denn sei $w_1\sim_L w_2,w_1\in L$. Dann
      folgt nach Def. $w_1\lambda\in L\equiv w_2\lambda\in L$.
      Insgesamt erh"alt man $w\in L(M)\equiv w\in L$.\qed
    \end{description}
  }
% -----------------------------------------------------------------------------
\example:{
  Wir geben zwei Beispiele f"ur dieses Kriterium an.
  \begin{enumerate}
    \item Sei $L := \{a^nb^n \mid n \leq 1\}$. Einige "Aquivalenzklassen
      von $\sim_L$ sind dann zum Beispiel
      \begin{itemize}
      \item $[\lambda] = \{\lambda\}$
      \item $[a] = \{a, a^2b, a^3b^2, a^4b^3, \dots\}$
      \item $[a^2] = \{a^2, a^3b, a^4b^2, a^5b^3, \dots\}$
      \item $[a^3] = \{a^3, a^4b, a^5b^2, a^6b^3, \dots\}$
      \end{itemize}
      Der Index von $\sim_L$ ist offensichtlich unendlich. Folglich ist
      $L$ nicht regul"ar.
    \item Sei nun $L := \{w \in \{0, 1\}^+ \mid w \text{ endet mit
       	00}\}$. Dann sind die s"amtlichen "Aquivalenzklassen von $\sim_L$:
      \begin{itemize}
      \item $[\lambda] = \{w \in \{0, 1\}^+ \mid w \text{ endet nicht mit
       	  0}\}$
      \item $[0] = \{w \in \{0, 1\}^+ \mid w \text{ endet mit 0, aber
       	  nicht mit 00}\}$
      \item $[00] = \{w \in \{0, 1\}^+ \mid w \text{ endet mit 00}\}$
      \end{itemize}
      Folglich ist $\{0, 1\}^+ = [\lambda] \cup [0] \cup [00]$, der Index
      von $\sim_L$ ist also gleich 3, $L$ also regul"ar.
    \end{enumerate}
  }
% -----------------------------------------------------------------------------
\subsection{Der Minimalautomat}
\label{sub:miniauto}
% -----------------------------------------------------------------------------
\definition Minimalautomat:{
  Sei $L$ eine Sprache. Dann ist der Minimalautomat f"ur diese Sprache
  derjenige DFA $M=(X,Q,\delta,q_0,F)$ mit minimalem $|Q|$, f"ur den noch
  gilt $L(M)=L$.
  }
% -----------------------------------------------------------------------------
\remark Folgerungen aus Abschnitt \ref{sub:regular-equivrel}:{
  Unmittelbar kann man beobachten:
  \begin{itemize}
    \item Sei $n$ der Index der "Aquivalenzrelation $\sim_L$. Dann hat
      jeder endliche DFA $M$ mit $L=L(M)$ mindestens $n$ Zust"ande.
    \item Der "Aquivalenzklassenautomat aus dem Beweis zu
      \ref{the:regular-equiv} ist minimal.
    \item Es gibt bis auf Isomorphie nur einen Minimalautomaten f"ur $L$.
    \item Ein DFA kann durch Verschmelzen von Zust"anden in den Minimalautomaten
      "uberf"uhrt werden.
    \end{itemize}
  Zwei Zust"ande $p,q$ sind nicht verschmelzbar, falls 
  falls es ein $w\in X^*$ gibt mit $\delta^*(p,w)\in F$, aber
  $\delta^*(q,w)\not\in F$ beziehungsweise falls $\delta^*(p,w)$ und 
  $\delta^*(q,w)$ nicht verschmelzbar sind. Daraus ergibt sich sofort
  der folgende Minimierungsalgorithmus, der von der Grundsatzidee nur 
  ``Paare von nicht verschmelzbaren Zust"anden anh"auft''.
  }
% -----------------------------------------------------------------------------
\algorithm Minimierung eines DFA:{
  \given $M=(X,Q,\delta,q_0,F)$ DFA.
  
  \aim Zugeh"origer Minimalautomat $\tilde M=(X,\tilde Q,\tilde\delta,q_0,F)$
  
  \begin{proc}
    \item Entferne alle Zust"ande, die nicht von $q_0$ aus erreichbar sind.
    \item Markiere alle Paare $(p,q)\in F\times F^c\cup F^c\times F$.
    \item Wiederhole, bis keine Markierungen mehr hinzukommen:
      \begin{itemize}
        \item
       	  F"ur alle unmarkierten Zust"ande $(p,q)\in Q^2$:
	  \begin{itemize}
	    \item
       	      Ist $(\delta(p,a),\delta(q,a))$ f"ur ein $a\in X$ markiert, so
       	      markiere $(p,q)$.
	    \end{itemize}
        \end{itemize}
    \item Verschmelze alle unmarkierten Zustandspaare miteinander.
    \end{proc}
  }
% -----------------------------------------------------------------------------
\example Minimierung eines Beispielautomaten:{
  Gegeben sei der folgende Automat
  \input info3_dfa_min_before.epic
  Im ersten Schritt werden die Paare $(q_0, q_4), (q_1, q_4), (q_2,
  q_4), (q_3, q_4)$ markiert, also alle Paare, bei denen ein Zustand ein
  Finalzustand ist, der andere aber nicht.
  
  Im zweiten Schritt werden die Paare $(q_0, q_1), (q_0, q_3), (q_1,
  q_2), (q_2, q_3)$ markiert, da sie auf Zust"ande f"uhren, die als Paar
  schon markiert sind.

  Der dritte Schritt bringt keine weiteren Markierungen, also sind wir
  fertig mit Aussortieren. Wir k"onnen nun beginnen, die verbleibenden
  Zustandspaare zu verschmelzen. Wir kombinieren $(q_0, q_2)$ zu
  $q_0q_2$ und $(q_1, q_3)$ zu $q_1q_3$. Der resultierende
  Minimalautomat ist
  \input info3_dfa_min_after.epic
  Erkannt wird in beiden F"allen die Sprache 
  $L =\{w\in\{0,1\}^+\mid w\text{ enth"alt }00\}$.
  }
% -----------------------------------------------------------------------------
\subsection{Entscheidbarkeit}
\label{sub:decidability}
% -----------------------------------------------------------------------------
\definition Produktautomat:{
  Seien $M_1=(X,Q_1,\delta_1,q_0,F)$ und $M_2=(X,Q_2,\delta_2,q_0',F)$
  DFA'en. Dann hei"st der Automat
  \[M:=(X,Q_1\times Q_2,(\delta_1,\delta_2),(q_0,q_0'),F\times F')
    \]
  der Produktautomat von $M_1$ und $M_2$.
  
  Es gilt $L(M)=L(M_1)\cap L(M_2)$.
  }
% -----------------------------------------------------------------------------
\remark Algorithmische Entscheidbarkeit:{
  Algorithmisch entscheidbar sind:
  \begin{itemize}
    \item Wortproblem: $w\in L$?
    \item Leerheit: $L=\emptyset$? Ist ein Endzustand von $q_0$ aus
      erreichbar?
    \item "Aquivalenz: $L_1=L_2$? Minimalautomaten vergleichen.
    \item Schnitt: $L_1\cap L_2=\emptyset$? Betrachte hierf"ur den
      Produktautomaten.
    \item Endlichkeit: $|L|=\infty$? Hierbei gilt wegen des
      Pumping-Lemmas (mit dem dortigen $n=|Q|$ des DFA)
      \[|L| = \infty\equiv(\exists z \in L)(n<|z|\le 2n)
        \]
      Wir pr"ufen also nur noch alle Worte der L"ange $l \in (n, 2n]$ auf
      Zugeh"origkeit zu $L$.
    \end{itemize}
  }
% -----------------------------------------------------------------------------
\section{Kontextfreie Grammatiken}
% -----------------------------------------------------------------------------
\subsection{Syntaxb"aume}
% -----------------------------------------------------------------------------
\example Kontextfreie Grammatik:{
  \index{Grammatik>kontextfreie}
  \index{Ableitungsbaum}
  \index{Syntaxbaum}
  Wir betrachten eine Grammatik mit den Produktionen
  \begin{align*}
    S &\to 0A\\
    A &\to 0AA | 1
    \end{align*}
  Die kontextfreie Ableitung
  \[S\dderivable 0A\dderivable 00AA\dderivable 000AAA\derivable 000111
    \]
  hat den Syntax oder Ableitungsbaum
  \input info3_syntree_simple.epic
  }
% -----------------------------------------------------------------------------
\remark Beobachtungen am Syntaxbaum:{
  \begin{itemize}
    \item $S$ ist die Wurzel
    \item Bl"atter sind Elemente von $T \cup \{\lambda\}$
    \item Knoten sind Elemente von $V$
    \item Ist \input info3_syntree_part.epic
      ein Teilbaum, dann gilt $A \to k_1 k_2 \ldots k_n \in P$.
    \item Zu einem gegebenen Ableitungsbaum ist die Reihenfolge der
      Ableitungen i.A. nicht eindeutig.
    \item Zu einem gegebenen Wort aus $L(G)$ mit einer Grammatik $G$ ist 
      die Ableitung i.A. nicht eindeutig. Grammatiken, auf die dieses zutrifft,
      hei"sen mehrdeutig. 
      \index{Grammatik>mehrdeutige}\index{mehrdeutige Grammatik}
    \end{itemize}
  }
% -----------------------------------------------------------------------------
\example Verschiedene B"aume:{
  Die beiden letzten Sachverhalte veranschaulicht das folgende Beispiel:
  \input info3_syntree_unique.epic
  }
% -----------------------------------------------------------------------------
\definition Linksableitung:{
  Eine Links- bzw. \indexthis{Rechtsableitung} ist eine
  Ableitung, bei der die am weitesten links beziehungsweise rechts
  stehende Variable zuerst ersetzt wird.
  }
% -----------------------------------------------------------------------------
\subsection{Das Pumping-Lemma}
\label{sub:ctf-pumping}
% -----------------------------------------------------------------------------
\lemma Baumh"ohenlemma:
  $G=(V,T,S,P)$ eine kontextfreie Grammatik, 
  $m:=\max\{|n|\mid(\exists A\in V)(A\to n\in P)\}$, $z\in L(G)$,
  Ableitungsbaum von $z$ hat die H"ohe $k$=>{
  \label{lem:ctf-tree-height}
  Dann ergibt sich sofort aus dem Ableitungsbaum: $|z|\le m^k$.
  }
% -----------------------------------------------------------------------------
\theorem Pumping-Lemma (auch $uvwxy$-Theorem):
  $G=(V,T,S,P)$ kontextfreie Grammatik=>{
  \label{the:ctf-pumping}
  \index{$uvwxy$-Theorem}
  Dann existiert ein $p\natural$ so, dass f"ur alle $z\in L(G)$ mit
  $|z|\ge p$ gilt:
  \begin{multline*}
    (\exists u,v,w,x,y\in T^*)(z=uvwxy,\\
    |vwx|\le p,vx\ne\lambda,(\forall i\nnatural)(uv^iwx^iy\in L(G)))
    \end{multline*}
  }
% -----------------------------------------------------------------------------
\proof \ref{the:ctf-pumping}:{
  Setze wieder $m:=\max\{|n|\mid(\exists A\in V)(A\to n\in P)\}$, $z\in L(G)$.
  Setze $p:=m^{|V|+1}$, sei $z\in L(G)$ nun ein Wort mit $|z|\ge p$. Wegen
  \ref{lem:ctf-tree-height} kann der Ableitungsbaum von $z$ keine H"ohe 
  $h\le|V|+1$ haben, da $z$ sonst gar nicht so lang sein k"onnte.
  
  Da nun $h>|V|$, muss im Ableitungsbaum ein Nichtterminal doppelt vorkommen,
  w"ahle dasjenige Nichtterminal, welches als letztes doppelt vorkommt,
  dieses sei $A$. Betrachte folgende Abbildung:
  \begin{center}\input info3_pumping_ctf.epic\end{center}
  Beachte: $A$ kann in der ersten Ableitung $S\derivable A$ auch schon vorkommen,
  w"ahle daher $v$ und $x$ so, dass in der Ableitung $A\derivable vAx$ kein
  weiteres $A$ mehr vorkommt, weiterhin so, dass $vx\ne\lambda$ (dies kann 
  o.B.d.A. verlangt werden, da eine Ableitung $A\derivable A$ nichts zur
  L"ange beigetragen hat, d.h. unser Wort erf"ullt die Voraussetzungen des
  Satzes auch nach Entfernung dieser vergeblichen Ableitung noch)
  
  Nun gilt: $|vwx|\le p$ aufgrund der Wahl von $v,x$. 
  Offensichtlich sind nun auch $uv^iwx^iy$ mit $i\nnatural\in L(G)$.\qed
  }
% -----------------------------------------------------------------------------
\example Folgerung:{
  Die Sprache $L:=\{a^nb^nc^n \mid n > 0\}$ ist nicht kontextfrei.

  W"ahle $n>p$. Angenommen $uvwxy = a^nb^nc^n$ mit $|vwx| \leq p < n$
  sowie $vx \not= \lambda$ und $p$ beliebig. Dann folgt
  \[vwx \in a^*b^* \text{ oder } b^*c^*\]
  und somit $uv^2wx^2y \not\in L$. Daraus folgt nach dem Pumping-Lemma,
  dass $L$ nicht kontextfrei ist.
  
  $L$ ist sehr wohl kontextsensitiv, denn die
  Produktionen
  \begin{eqnarray*}
    S & \to & aBC | aSBC\\
    CB & \to & BC\\
    aB & \to & ab\\
    bB & \to & bb\\
    bC & \to & bc\\
    cC & \to & CC
  \end{eqnarray*}
  erzeugen $L$ und sind \emph{erweiternd}. Daher ist $L$ auch kontextsensitiv.
  }
% -----------------------------------------------------------------------------
\subsection{$\lambda$- und kontextfreie Sprachen}
\label{sub:ctf-lang}
% -----------------------------------------------------------------------------
\theorem:
  $G=(V,T,S,P)$ kontextfrei=>{
  \label{the:ctf-cts}
  $L(G)$ kontextsensitiv.
  }
% -----------------------------------------------------------------------------
\proof \ref{the:ctf-cts}:{
  Einziges Problem ist hier, dass bei kontextfreien Sprachen Produktionen
  der Form $A\to\lambda$ zugelassen sind, bei kontextsensitiven jedoch nicht.

  Seien $P^*:=\{A_1\to\lambda, \dots, A_k \to \lambda\}$ alle $\lambda$-Produktionen in
  $P$. Setze $P':=P\setminus P^*$.
  
  Dann m"ussen wir lediglich f"ur jede Produktion in $P'$, in der mindestens 
  eines von $A_1,\ldots,A_k$ auf einer rechten Seite auftritt, alle 
  M"oglichkeiten ``durchrattern'', die es gibt, die $A_1,\ldots,A_k$
  wegzulassen. F"ur jeden solchen ``Ratterschritt'' erzeugen wir eine 
  neue Produktion. Dann ist die so erzeugte neue Produktionenmenge $P''$
  gleichwertig mit $P$, aber $\lambda$-frei.

  Setze weiterhin 
  \[G':=\begin{cases}
       	(V, T, S, P'') & \text{falls } \lambda \not\in L(G)\\
       	(V \cup \{S'\}, T, S', P'' \cup \{S' \to \lambda | S\}) & \text{sonst}
      \end{cases}
    \]
  Dann ist $G'$ kontextsensitiv mit $L(G) = L(G')$.\qed
  }
% -----------------------------------------------------------------------------
\subsection{Abschlusseigenschaften}
\label{sub:ctf-complete}
% -----------------------------------------------------------------------------
\theorem:=>{
  \label{the:ctf-no-cap}
  $L_2$ ist \emph{nicht} abgeschlossen bez"uglich des Schnittes.
  }
% -----------------------------------------------------------------------------
\proof \ref{the:ctf-no-cap}:{
  Durch ein Gegenbeispiel. 
  $L_1:=\{a^m b^n c^n\mid m,n\natural\}$ und
  $L_2:=\{a^n b^n c^m\mid m,n\natural\}$
  sind kontextfrei, $L_1\cap L_2=\{a^n b^n c^n\mid n\natural\}$ allerdings
  nicht.\qed
  }
% -----------------------------------------------------------------------------
\remark Abgeschlossenheit gg"u. Komplementbildung:{
  Als direkte Folge von \ref{the:ctf-no-cap} und
  aufgrund der DeMorgan-Regel sind kontextfreie Sprachen bez"uglich
  des Komplementbildung i.A. nicht abgeschlossen.
  }
% -----------------------------------------------------------------------------
\subsection{Die Chomsky-Normalform}
\label{sub:chomsky-nf}
% -----------------------------------------------------------------------------
\definition Chomsky-Normalform:{
  \index{Normalform>Chomsky-}
  Eine kontextfreie Grammatik ist in \emph{Chomsky-Normalform} (CNF),
  wenn sie $\lambda$-frei ist und alle ihre Produktionen von folgender
  Form sind:
  \begin{itemize}
    \item $S \to \lambda$ oder
    \item $A \to BC$ oder
    \item $A \to a,$
    \end{itemize}
  mit $A, B, C \in V$ sowie $a \in T$.
  }
% -----------------------------------------------------------------------------
\theorem:
  $L=L(G)$ mit $G=(V,T,S,P)$ kontextfrei=>{
  \label{the:chomsky-nf}
  Dann existiert eine Grammatik $G'$ mit $L(G')=L(G)$, wobei $G'$ 
  Chomsky-Normalform besitzt.
  }
% -----------------------------------------------------------------------------
\proof Erzeugung der Chomsky-Normalform:{
  Durch Angabe eines Algorithmus, wir erzeugen $G'=(V',T,S,P')$.
  \begin{proc}
    \item Mache die Grammatik $\lambda$-frei.
    \item Erzeuge ein neues Nichtterminal $A_a\in V'$ 
      f"ur jedes Terminal $a\in T$, Erg"anze $P'$ um $A_a\to a$.
    \item Entferne alle "uberfl"ussigen Produktionen der Art
      $A\to A$ aus $P'$.
    \item Eliminiere alle Zyklen:
      F"ur jeden Zyklus $A_1\to A_2\to \cdots\to A_k\to A_1$ eliminiere
      $A_2,\ldots,A_k$, "andere deren Produktionen so, dass sie statt dessen
      von $A_1$ ausgehen.
    \item Eliminiere alle Umwege:
      Befinden sich noch Produktionen der Art $A\to B\in P'$, wobei $A,B\in V'$,
      so existieren auch Produktionen $B\to\cdots\to C\to |\alpha|$ mit $|\alpha|\ge 2$. 
      Nimm f"ur alle auf diese Art erzeugbaren $\alpha\in (V\cup T)^*$ die 
      Produktion $A\to w$ in $P'$ auf, wirf daf"ur die Einer-Produktionen
      weg.
    \item Eliminiere alle Produktionen $A\to B_1\ldots B_m$ mit $m\ge 3$:
      Nimm einfach statt ihrer die neuen Nichtterminale $C_1,\ldots,C_{m-2}$
      in $V'$ auf, erzeuge folgende Produktionen:
      \begin{align*}
        A&\to B_1C_1 \\
        C_1&\to B_2C_2\\
        &\dots\\
        C_{m-2}&\to B_{m-1}B_m \\
        \end{align*}
    \end{proc}
  Dann ist $G'$ offensichtlich in CNF.\qed
  }
% -----------------------------------------------------------------------------
\example Chomsky-Normalform einer Grammatik:{
  Wir betrachten das folgende Beispiel:
  \begin{eqnarray*}
    S & \to & AB\\
    A & \to & aA | a\\
    B & \to & bBc | bc
   \end{eqnarray*}
  Wir formen die Grammatik um und erhalten
  \begin{eqnarray*}
    S & \to & AB\\
    A' & \to & a\\
    B' & \to & b\\
    C' & \to & c\\
    A & \to & A'A | a\\
    B & \to & B'C' | B'D\\
    D & \to & BC'
    \end{eqnarray*}
  }
% -----------------------------------------------------------------------------
\subsection{Der Cocke-Kasami-Younger-Algorithmus}
% -----------------------------------------------------------------------------
\algorithm Cocke-Kasami-Younger-Algorithmus:{
\index{CKY-Algorithmus}
  \given $G=(V,T,S,P)$ in CNF, $w\in T^*$
  
  \aim $w=x_1\ldots x_n\in L(G)$?
  
  \begin{proc}
    \item Lege eine Pyramide mit den Zeilen 
      $Y[1,1]\ldots Y[1,n]$, \dots $Y[n,1]$
      an, wobei $Y[i,j]\in \cal P(V)$.
      $Y[i,j]$ wird im Laufe des Algorithmus all diejenigen Nichtterminale
      enthalten, aus denen $x_j\ldots x_{j+i-1}$ erzeugt werden kann.
    \item Initialisierung: Schreibe in $Y[1,j]$ jeweils alle Nichtterminale
      $V$, aus denen $x_j$ $(j=1,\ldots,n)$ erzeugt werden kann.
    \item F"ur $i=2,\ldots,n$:\begin{itemize}
      \item F"ur $j=1,\ldots,i$:\begin{itemize}
        \item F"ur $m=1,\ldots,i-1$:\begin{itemize}
          \item Untersuche alle Produktionen der Form $A\to BC\in P$:
            Ist $B\in Y[m,j]$ ($\implies$ $m$ lang, gleicher Offset) und
	    $C\in Y[i-m,j+m]$ ($\implies$ $i-m$ lang, um $m$ nach rechts 
	    verschobener Offset), so erg"anze $Y[i,j]$ um $A$.
	  \end{itemize}
        \end{itemize}
      \end{itemize}
      \item Ist nun $S\in Y[n,1]$, so gilt $w\in L(G)$.
    \end{proc}

  \cpx $O(n^3)$
  }
% -----------------------------------------------------------------------------
\remark:{
  Der CKY-Algorithmus ist ein Bottom-up-Algorithmus.
  
  Aus der vom CKY-Algorithmus erzeugten Pyramide kann man s"amtliche
  m"oglichen Ableitungsb"aume eines Wortes ablesen.
  }
% -----------------------------------------------------------------------------
\example:{
  Wir betrachten als Beispiel die Grammatik mit den Produktionen
  \begin{eqnarray*}
    S & \to & AB | BB\\
    A & \to & AB | CC | a\\
    B & \to & CA | BB | b\\
    C & \to & BA | AA | b
    \end{eqnarray*}
  und fragen, ob $x_1x_2x_3x_4 := aabb \in L(G)$ gilt. Dazu bauen wir
  die Pyramide auf:
  \input info3_cky_pyramid.epic
  Es ist zu sehen, da"s gilt $S \in \text{Zelle } (4, 1)$. Es folgt
  also, da"s $aabb \in L(G)$!
  
  Im Beispiel gibt es nur einen Syntaxbaum:
  \input info3_cky_syntree.epic
  }
% -----------------------------------------------------------------------------
\subsection{Entscheidbarkeit}
% -----------------------------------------------------------------------------
\remark:{
  Sei $G=(V,T,S,P)$ eine kontextfreie Grammatik in CNF. Dann sind in
  polynomieller Zeit l"osbar:
  \begin{description}
    \item[Wortproblem:] Gilt $w \in L(G)$? Diese Frage kann der
      CKY-Algorithmus beantworten.
    \item[Leerheit:] Gilt $L(G) = \emptyset$? Der folgende Algorithmus
      bestimmt alle Variablen, aus denen Worte ableitbar sind:
      \begin{proc}
        \item $U:=\emptyset\subseteq V$
        \item $U':=\{A\in V\mid (\exists t\in T)(A\to t\in P)\}\subseteq V$
        
          (alle Nichtterminale, die in einem Schritt Terminale erzeugen)
	\item Solange $U\ne U'$: \begin{itemize}
          \item $U:=U'$
	  \item $U':=\{A\in V\mid (\exists B,C\in U)(A\to BC \in P)\}$
          \end{itemize}
        \item $S\in U\equiv L(G)\ne\emptyset$
       	\end{proc}
      Die Nichtterminale in $V\setminus U$ sind nutzlos und k"onnen 
      r"uckstandsfrei entfernt werden.
    \item[Endlichkeit:] Gilt $|L(G)| = \infty$? Enthalte $V$ keine nutzlosen
      Variablen. Der folgende Algorithmus
      bestimmt alle Variablen, die nur endliche Worte $w \in T^*$
      erzeugen.
      \begin{proc}
        \item $U:=\emptyset\subseteq V$
        \item Setze
          \begin{multline*}
	    U':=\{A\in V\mid (\exists t\in T)(A\to t\in P)\}\cap\\
            \{A\in V\mid (\forall B,C\in V)(A\to BC\not\in P)\}\subseteq V
	    \end{multline*}
          (alle Nichtterminale, die in einem Schritt \emph{nur} Terminale erzeugen)
	\item Solange $U\ne U'$: \begin{itemize}
          \item $U:=U'$
	  \item $U':=U\cup \{A\in V\mid (\exists B,C\in U)(A\to BC\in P)\}$
          \end{itemize}
        \item $S\in U\equiv |L(G)|=\infty$ 
        \end{proc}
      Erl"auterung: Der Trick an diesem Algorithmus ist, dass er in $U$ 
      keine Nichtterminale aufnimmt, die unendliche W"orter erzeugen,
      also Produktionen wie $A\to BC,C\to BA$ mit $B\to b$, und zwar deswegen,
      weil der Algorithmus nur strikte ``Bottom-up''-Konstruktion zul"asst,
      ein Nichtterminal, das Dinge produziert, die noch nicht bekannt 
      (d.h. $\in U$) sind, wird nicht zugelassen (d.h. in $U$ gesteckt).
    \end{description}
  }
% -----------------------------------------------------------------------------
\example:{
  Wir wollen nun als Beispiel eine Grammatik mit
  den Produktionen
  \begin{eqnarray*}
    S &\to& AD | BC\\
    B &\to& BC\\
    D &\to& FE | b\\
    F &\to& AD\\
    A &\to& a\\
    E &\to& b | c
  \end{eqnarray*}
  
  Die Produktionen $S \to BC$ und $B \to BC$ sind nutzlos, da das
  Nichtterminal $C$ nicht in ein Terminal umgewandelt werden kann. Die
  beiden Produktionen k"onnen also gestrichen werden.
  
  \paragraph{Nutztest:} Wir wollen nun bestimmen, ob $G$ "uberhaupt eine
  Sprache erzeugt. Dazu wenden wir Algorithmus 1 an. Wir notieren die
  Ergebnisse in Tabellenform.
  
  \begin{center}
    \begin{tabular}[c]{c|c}
      Durchlauf & $U$ \\\hline
      0 & $\emptyset$\\
      1 & $\{A, D, E\}$\\
      2 & $\{S, F, A, D, E\}$\\
      3 & $\{S, F, A, D, E\}$
    \end{tabular}
  \end{center}
  
  Nach Ablauf des Algorithmus ist das Startsymbol $S$ in $U$
  enthalten. Es folgt also $L \not= \emptyset$.
  
  \paragraph{Endlichkeitstest:} Weiterhin betrachten wir den Index von
  $L$. Hierzu verwenden wir Algorithmus 2; hierbei ist zu beachten, da"s
  nutzlose Produktionen aus der Grammatik zu entfernen sind, wie wir es
  getan haben.
  
  \begin{center}
    \begin{tabular}{c|c}
      Durchlauf & $U$ \\\hline
      0 & $\emptyset$\\
      1 & $\{A, E\}$\\
      2 & $\{A, E\}$
    \end{tabular}
  \end{center}
  
  Hier ist nach Beendigung des Algorithmus das Startsymbol $S$
  nicht in $U$ enthalten. Es folgt also $|L| = \infty$.
  }
% -----------------------------------------------------------------------------
\penalty-5000
\subsection{Einfache Kellerautomaten}
% -----------------------------------------------------------------------------
\definition Einfacher Kellerautomat:{
  \index{Kellerautomat>einfacher}
  \insertpic{info3_pda}
  Seien $T$ ein Eingabe-, $\Gamma\supseteq T$ ein Kelleralphabet, 
  $S\in\Gamma$ und $\delta:T\cup\{\lambda\}\times\Gamma\to\cal P(\Gamma^*)$ 
  eine "Ubergangsfunktion, deren Ergebnismenge immer h"ochstens endlich 
  sein darf.
  
  Dann nennt man $M=(T,\Gamma,\delta,S)$ einen einfachen Kellerautomaten. 
  Ein Tupel 
  $K:=(a_i\ldots a_n,A_k\ldots A_1)\in T^*\times\Gamma^*$ 
  hei"st \indexthis{Konfiguration} des Automaten.
  
  Eine Konfiguration $K$ eines einfachen Kellerautomaten kann mehrere 
  Nachfolgekonfiguationen haben, einfache Kellerautomaten sind demzufolge
  nach Definition nichtdeterministisch. Ausgehend von dieser Konfiguration
  werden aufgrund der Zustands"ubergangsfunktion $\delta$ m"ogliche 
  Nachfolgekonfigurationen ermittelt:
  \begin{itemize}
    \item 
      $(a_{i+1}\ldots a_n,A_{k-1}\ldots A_1)$ ist m"ogliche 
      Nachfolgekonfiguration, falls $\lambda\in\delta(a_i,A_k)$
    \item 
      $(a_i\ldots a_n,\alpha A_{k-1}\ldots A_1)$ ist m"ogliche 
      Nachfolgekonfiguration, falls $\alpha\in\delta(\lambda,A_k)$ mit
      $\alpha\in\Gamma^*$.
    \end{itemize}
  F"ur die "Uberf"uhrung von Konfigurationen in andere schreibt man
  wie gehabt $K\dderivable K', K\derivable K'$. 

  Die vom Kellerautomaten akzeptierte Sprache ist
  \[L(M):=\{w\in T^*\mid (w,S)\derivable (\lambda,\lambda)\}
    \]
  }
% -----------------------------------------------------------------------------
\theorem:
  $G=(V,T,S,P)$ kontextfreie Grammatik=>{
  \label{the:ctf->spda}
  Dann existiert ein einfacher Kellerautomat $M=(T,\Gamma,\delta,S)$
  mit $L(M)=L(G)$.
  }
% -----------------------------------------------------------------------------
\proof \ref{the:ctf->spda}:{
  Betrachte die obige Definition mit $\Gamma:=V\cup T$ und die direkte Analogie
  zwischen den M"oglichkeiten f"ur die $\delta$-Funktion und den M"oglichkeiten
  f"ur Produktionen in kontextfreien Sprachen.\qed
  }
% -----------------------------------------------------------------------------
\theorem:
  $M=(T,\Gamma,\delta,S)$ einfacher Kellerautomat=>{
  \label{the:spda->ctf}
  $L(M)$ ist kontextfreie Sprache.
  }
% -----------------------------------------------------------------------------
\proof \ref{the:spda->ctf}:{
  Betrachte die obige Definition mit $V:=\Gamma\setminus T$ und die direkte 
  Analogie zwischen den M"oglichkeiten f"ur Produktionen in kontextfreien 
  Sprachen und den M"oglichkeiten f"ur die $\delta$-Funktion.\qed
  }
% -----------------------------------------------------------------------------
\remark:{
  Im Gegensatz zum CKY-Algorithmus f"uhrt ein einfacher Kellerautomat
  eine Top-Down-Analyse durch.
  }
% -----------------------------------------------------------------------------
\remark Greibach-Normalform:{
  Jede kontextfreie Grammatik kann in eine
  "aquivalente Grammatik umgeformt werden mit Produktionen der Form
  \[A\to aB_1\dots B_k,a\in T\cup\{\lambda\}, A,B_1,\dots,B_k \in V
    \]
  Ist $\lambda\not\in L(G)$, so kann $G$ auch in 
  \indexthis{Greibach-Normalform} "uberf"uhrt werden. Alle Produktionen 
  sind dann von der Form
  \[A\to aB_1\dots B_k,a\in T,A,B_1,\dots,B_k\in V
    \]
  }
% -----------------------------------------------------------------------------
\subsection{Allgemeine Kellerautomaten}
% -----------------------------------------------------------------------------
\definition Allgemeiner Kellerautomat:{
  \index{Kellerautomat>allgemeiner}
  \index{PDA}
  \index{push-down automaton}
  Sei $Q$ eine endlichen Zustandsmenge, 
  $T$ ein Eingabealphabet, 
  $\Gamma$ ein Kelleralphabet, 
  $\delta:Q\times T\cup\{\lambda\}\times\Gamma\to \cal P(Q)\times\cal P(\Gamma^*)$
  eine Zustands"ubergangsfunktion, deren Ergebnismenge h"ochstens endlich 
  sein darf, 
  $q_0\in Q$ ein Anfangszustand und 
  $S\in\Gamma$ ein Start-Kellerzeichen.

  Dann nennt man $M=(Q,T,\Gamma,\delta,q_0,S)$ einen allgemeinen Kellerautomaten. 
  Ein Tripel 
  $K:=(q,a_i\ldots a_n,A_k\ldots A_1)\in Q\times T^*\times\Gamma^*$ 
  hei"st \indexthis{Konfiguration} des Automaten.
  
  Eine Konfiguration $K$ eines einfachen Kellerautomaten kann mehrere 
  Nachfolgekonfiguationen haben, einfache Kellerautomaten sind demzufolge
  nach Definition nichtdeterministisch. Ausgehend von dieser Konfiguration
  werden aufgrund der Zustands"ubergangsfunktion $\delta$ m"ogliche 
  Nachfolgekonfigurationen ermittelt:
  \begin{itemize}
    \item 
      $(q',a_{i+1}\ldots a_n,A_{k-1}\ldots A_1)$ ist m"ogliche 
      Nachfolgekonfiguration, falls $(q',\lambda)\in\delta(q,a_i,A_k)$
    \item 
      $(q',a_i\ldots a_n,\alpha A_{k-1}\ldots A_1)$ ist m"ogliche 
      Nachfolgekonfiguration, falls $(q',\alpha)\in\delta(q,a_i,A_k)$ mit
      $\alpha\in\Gamma^*$.
    \end{itemize}
  F"ur die "Uberf"uhrung von Konfigurationen in andere schreibt man
  wie gehabt $K\dderivable K', K\derivable K'$. 

  Die vom Kellerautomaten akzeptierte Sprache ist
  \[L(M):=\{w\in T^*\mid (q_0,w,S)\derivable (q,\lambda,\lambda) (q\in Q)\}
    \]
  }
% -----------------------------------------------------------------------------
\theorem:
  $M=(Q,T,\Gamma,\delta,q_0,S)$ PDA=>{
  \label{the:pda->ctf}
  Dann ist $L(M)$ kontextfrei.
  }
% -----------------------------------------------------------------------------
\proof \ref{the:pda->ctf}:{
  Sei $V:=\{{}^pA^q \mid A \in\Gamma,p,q\in Q\}\cup \{S'\}$. 
  Die Menge der Produktionen gewinnt man so:
  \begin{multline*}
    P:=\{S'\to {}^{q_0}S^q\mid q\in Q\}\cup
    \{{}^{p}A^{q}\to 
      a{}^{p}B_1^{p_1}{}^{p_1}B_2^{p_2}\ldots {}^{p_{k}}B_i^{q}\mid\\
      p_1,\ldots,p_k\in Q,(q,B_1B_2\ldots B_i)\in\delta(p,a,A)
      \}
    \end{multline*}
  Dann ist mit $G:=(V,T,S',P)$ gerade $L(G)=L(M)$.\qed
  }
% -----------------------------------------------------------------------------
\remark:{
  Damit ergibt sich, dass PDAs mit einfachen Kellerautomaten 
  gleichwertig sind. Sie sind eine Verallgemeinerung, bringen allerdings
  keine gr"o"seren M"oglichkeiten.

  Man kann allerdings zus"atzliche Endzust"ande einf"uhren:
  \[L(M) := \{w \in T^* \mid (q_0,w,S) \derivable (q, \lambda,\lambda) 
    \text{ mit } q\in F\subseteq Q\}
    \]
  Es gibt Sprachen, die durch deterministische
  Kellerautomaten \emph{mit} $F$, aber nicht \emph{ohne} $F$ erkannt
  werden. PDA mit $F$ erkennen also nicht zwingend alle kontextfreien Sprachen.
  }
% -----------------------------------------------------------------------------
\section{Typ 1-und Typ 0-Sprachen}
% -----------------------------------------------------------------------------
\subsection{Erweiternde Grammatiken}
% -----------------------------------------------------------------------------
\definition Erweiternde Grammatik:{
  \index{Grammatik>erweiternde}
  Sei $G=(V,T,S,P)$ eine Grammatik. $G$ hei"st erweiternd $:\equiv$
  f"ur $u\to v\in P$ gilt $|u|\le|v|$. Die Produktion $S\to\lambda$ ist
  zul"assig, falls $S$ auf keiner rechten Seite einer Produktion vorkommt.
  }
% -----------------------------------------------------------------------------
\theorem:
  $L$ Sprache=>{
  \label{the:expand<->ctf}
  Es existiert eine kontextsensitive Grammatik $G=(V,T,S,P)$ mit $L(G)=L$ $\equiv$
  es existiert eine erweiternde Grammatik $G'=(V',T,S,P')$ mit $L(G')=L$
  }
% -----------------------------------------------------------------------------
\proof \ref{the:expand<->ctf}:{
  Die Richtung ``$\implies$'' ist nach Definition klar. Wir zeigen
  ``$\Leftarrow$'':
  
  Das Problem ist, dass bei einer kontextsensitiven Grammatik nur immer
  ein Nichtterminal in etwas aus $(V\cup T)^*$ wechseln darf, 
  w"ahrend bei erweiternden Grammatiken nur Aussagen "uber die L"ange 
  getroffen sind. Zun"achst ersetze alle Terminale $a\in T$ durch
  neue Nichtterminale $A_a$. Schreibe $V$ aus $V'$ entsprechend um,
  erg"anze $P:=P'\cup\{A_a\to a\mid a\in T\}$. Dies behebt mit dem Holzhammer
  die M"oglichkeit f"ur Produktionen der Art $abc\to cde$ (d.h. nur Terminale), 
  die im kts Fall ebenfalls nicht zul"assig sind.
  
  Um nun eine beliebige erweiternde Produktion
  $A_1\ldots A_m\to B_1\ldots B_n\in P$ in eine kontextsensitive Produktion
  umzuwandeln, f"uhre neue Nichtterminale $C_1,\ldots,C_{m-1}$ ein und
  ersetze die alte Produktion durch die folgenden neuen:
  \begin{align*}
    A_1\ldots A_m &\to C_1A_2\ldots A_m\\
    C_1A_2\ldots A_m &\to C_1C_2A_3\ldots A_m\\
    &\ldots\\
    C_1\ldots C_{m-1}A_m &\to C_1\ldots C_{m-1}B_mB_{m+1}\ldots B_n\\
    C_1\ldots C_{m-1}B_mB_{m+1}\ldots B_n &\to 
      C_1\ldots C_{m-2}B_{m-1}B_mB_{m+1}\ldots B_n\\
    C_1\ldots C_{m-2}B_{m-1}B_mB_{m+1}\ldots B_n &\to 
      C_1\ldots C_{m-3}B_{m-2}B_mB_{m-1}\ldots B_n\\
    &\ldots\\
    C_1B_2\ldots B_n &\to 
      B_1\ldots B_n
    \end{align*}
  Damit ist die Behauptung gezeigt.\qed
  }
% -----------------------------------------------------------------------------
\penalty-5000
\subsection{Turingmaschinen}
% -----------------------------------------------------------------------------
\definition Turingmaschine:{
  \insertpic{info3_turing}
  Sei $Q$ eine Zustandsmenge, $F\subseteq Q$ eine Endzustandsmenge, 
  $q_0\in Q$ der Anfangszustand, $\Gamma$ ein Bandalphabet,
  $X\subseteq \Gamma$ ein Eingabealphabet, $\Box\in\Gamma$ das Leer- bzw.
  Bandzeichen und $\delta:Q\times\Gamma\to Q\times\Gamma\times\{L,R,N\}$
  eine Zustands"ubergangsfunktion.
  
  Dann hei"st das Septupel $M=(Q,F,q_0,\Gamma,X,\Box,\delta)$ eine 
  Turingmaschine. Ein Tripel 
  $K:=(\alpha,q,\beta)\in \Gamma^*\times Q\times\Gamma^*$ hei"st eine 
  Konfiguration der Turingmaschine, die gleiche Konfiguration wird auch
  durch $(\Box\alpha,q,\beta\Box)$ beschrieben. Wie "ublich verwenden
  wir die Symbolik $K\dderivable K'$, $K\derivable K'$ f"ur den 
  Konfigurations"ubergang.
  
  Sei z.B. $\delta(q,x)=(p,y,L)$. Dann wird auf dem Band an die Stelle des
  $x$ ein $y$ geschrieben, in den Zustand $p$ gewechselt und der Kopf eine
  Stelle nach links gesetzt.
  
  Die von einer Turingmaschine akzeptierte Sprache ist
  \[L(M):=\{w\in X^*\mid (\lambda,q_0,w)\derivable(\alpha,q,\beta), q\in F\}
    \]
  Das Band einer Turingmaschine darf beliebig lang werden,
  die anf"angliche Eingabe muss jedoch endlich sein. Turingmaschinen gibt
  es in den Geschmacksrichtungen deterministisch bzw. nichtdeterministisch und
  unbeschr"ankt bzw. linear beschr"ankt. In letzterem Fall hei"st die Maschine
  auch \indexthis{LBA} wie ``linear bounded automaton''.
  }
% -----------------------------------------------------------------------------
\theorem:
  $G=(V,T,S,P)$ kontextsensitive Grammatik=>{
  \label{the:cts->lba}
  $L(G)$ wird durch einen LBA erkannt.
  }
% -----------------------------------------------------------------------------
\proof \ref{the:cts->lba}:{
  O.B.d.A. $\lambda\not\in L(G)$.
  Setze $\Gamma:=T\cup V$, 
  Wir programmieren die Turingmaschine so, dass sie sich 
  (nichtdeterministisch) beliebig ein
  $v$ mit $u\to v\in P$ auf dem Band ausw"ahlt, dieses durch $u$ ersetzt
  und im Falle $|u|<|v|$ die Eingabe nach rechts zusammenschiebt. Die
  Turingmaschine geht in einen Endzustand, falls am Schluss nur noch $S$ auf
  dem Band steht. Wegen der Eigenschaft $|u|\le|v|$ f"ur $u\to v\in P$ ist
  die Turingmaschine linear beschr"ankt.
  }
% -----------------------------------------------------------------------------
\remark:{
  Analog zeigt man, dass Typ 0-Sprachen von unbeschr"ankten Turingmaschinen
  erkannt werden.
  }
% -----------------------------------------------------------------------------
\theorem:
  $M=(Q,F,q_0,\Gamma,X,\Box,\delta)$ LBA=>{
  \label{the:lba->ctf}
  Dann ist $L(M)$ kontextsensitiv.
  }
% -----------------------------------------------------------------------------
\proof \ref{the:lba->ctf}:{
  Zu einem gegebenen LBA geben wir eine Grammatik $G$ an mit $L(M)=L(G)$
  Setze $T:=X$,
  \begin{multline*}
    V:=\{S,A\}\cup(\Gamma^2)\cup(Q\Box\Box\Gamma^2)\cup(\Box\Box Q\Gamma^2)\cup\\
    (Q\Gamma^2\Box\Box)\cup(\Gamma^2Q\Box\Box)\cup(\Box\Box\Gamma^2)\cup
    (\Gamma^2\Box\Box)
    \end{multline*}
  Dabei dienen die speziellen $(*\Box*)$-Nichtterminale (anhand ihrer
  L"ange von den anderen unterscheidbar) lediglich dazu, den linken und
  rechten Rand f"ur den die ``Pseudo-Lesekopfposition'' anzeigenden Zustand
  un"uberwindlich zu machen. 
  
  Wir erzeugen folgende Arten von Produktionen:
  \begin{itemize}
    \item Produktionen, um jede beliebige Anfangskonfiguration den 
      Turingmaschine zu erzeugen:
      \begin{align*}
        \{S&\to A(aa\Box\Box)\mid a\in X\}\\
        \{A&\to A(\Box\Box q a a)|A(aa)\mid q\in Q,a\in X\}
        \end{align*}
    \item Produktionen, um jeden m"oglichen $\delta$-"Ubergang zu 
      simulieren. 
      \begin{align*}
        \{(\Box\Box qxa)&\to(p\Box\Box ya),\\
        (\Box\Box zb)(qxa)&\to(\Box\Box pzb)(ya),\\
        (zb)(qxa)&\to(pzb)(ya),\\
        (zb)(qxa\Box\Box)&\to(pzb)(ya\Box\Box)\mid \delta(q,x)=(p,y,L),z\in\Gamma\} \cup \\
        \{(xaq\Box\Box)&\to(pxa\Box\Box)\mid \delta(q,\Box)=(p,y,L)\}
        \end{align*}
      wobei die rechts- und neutrall"aufigen "Uberg"ange analog gebildet 
      werden.
    \item Produktionen, um das schlussendlich akzeptierte Wort aus dem
      Backup an den jeweils zweiten Stellen zu rekonstruieren. Wir stellen
      sicher, dass nur dann ein Terminalwort entsteht, wenn ein Endzustand
      vorliegt, indem wir nur in diesem Fall die Umwandlung des Nichtterminals
      mit dem gegenw"artigen ``Lesekopf'' zulassen:
      \begin{align*}
        \{(q\Box\Box xa)&\to a,\\
        (\Box\Box qxa)&\to a,\\
        (qxa)&\to a,\\
        (qxa\Box\Box)&\to a,\\
        (xaq\Box\Box)&\to a,\\
        (xa)&\to a,\\
        (\Box\Box xa)&\to a,\\
        (xa\Box\Box)&\to a\mid q\in F,x,a\in X\}
        \end{align*}
    \end{itemize}
  Insgesamt ergibt sich dann mit $G=(V,X,S,P)$ mit obiger 
  Produktionenmenge $P$ eine kontextsensitive Grammatik mit $L(G)=L(M)$.
  \qed
  }
% -----------------------------------------------------------------------------
\remark:{
  Indem man in obigem Beweis auch das Schreiben "uber die Bandgrenzen hinaus
  zul"asst, und die entsprechenden Backup-Bandzeichen dann zu $\lambda$s 
  umwandelt, erh"alt man, dass allgemeine Turingmaschinen Sprachen
  des Typs $0$ erkennen.
  }
% -----------------------------------------------------------------------------
\subsection{Deterministische Turingmaschinen}
% -----------------------------------------------------------------------------
\theorem:
  $M=(Q,F,q_0,\Gamma,X,\Box,\delta)$ nichtdet. Turingmaschine=>{
  \label{the:tm-det-nondet}
  Dann existiert eine deterministische Turingmaschine $M'$ mit $L(M)=L(M')$.
  }
% -----------------------------------------------------------------------------
\proof \ref{the:tm-det-nondet}:{
  Indem man die deterministische Turingmaschine in einer Art Timesharing
  f"ur jede Konfiguration der nichtdeterministsichen Maschine jeweils
  Einzelschritte durchf"uhren l"asst.\qed
  }
% -----------------------------------------------------------------------------
\remark:{
  Es ist unbekannt, ob nichtdeterministische LBA in
  deterministische LBA umgewandelt werden k"onnen.
  }
% -----------------------------------------------------------------------------
\penalty-5000
\subsection{Abschlussseigenschaften}
% -----------------------------------------------------------------------------
\remark Zusammenfassung:{
  Es sind abgeschlossen unter folgenden Operationen:\\
  \begin{center}\begin{tabular}{|l|c|c|c|}
    \hline
    & $\cdot,\cup,*$ & $\cap$ & ${}^c$ \\
    \hline
    Typ 0 & \yes & \yes & \no \\
    Typ 1 & \yes & \yes & \yes \\
    Typ 2 & \yes & \no & \no \\
    Typ 3 & \yes & \yes & \yes \\
    \hline
    \end{tabular}\end{center}
  }
% -----------------------------------------------------------------------------
\subsection{Das Halteproblem}
% -----------------------------------------------------------------------------
\definition G"odelnummer:{
  Zum Beispiel durch zeichengebundene Kodierung der formalen Festlegung
  einer Turingmaschine in einem $n$-adischen System und anschlie"sende
  Darstellung der Kodierung $a_0,\ldots, a_j$ als Zahl 
  \[q:=\sum_{k=0}^j a_k n^k
    \]
  erh"alt man eine Nummer f"ur jede Turingmaschine, genannt deren 
  G"odelnummer bzgl. eines festen Kodierungssystems.
  
  Diese Zuordnung ist offensichtlich nicht zwingend surjektiv 
  (d.h. es existieren Zahlen, zu denen es keine Turingmaschine gibt) 
  und auch nicht zwingend eindeutig
  (es exisitieren mehrere Kodierungen f"ur eine Turingmaschine), aber
  injektiv.
  
  Surjektivit"at und Eindeutigkeit stellt man durch folgende zwei Konventionen
  wieder her:
  \begin{itemize}
    \item Als das Urbild einer Zahl, der auf bisherige Weise keine 
      Turingmaschine zugeordnet war, betrachten wir die triviale
      endlos laufende Turingmaschine.
    \item Man definiert sich eine gewisse Normalform f"ur die Kodierung
      von Turingmaschinen.
    \end{itemize}
  Da unsere Zuordnung mit diesen Ver"anderungen bijektiv ist, definieren
  wir $M_i$ als die Turingmaschine mit der Nummer $i$.
  }
% -----------------------------------------------------------------------------
\definition Berechenbarkeit:{
  Eine partielle Funktion $f:\SetNN\to\SetNN$ hei"st berechenbar $:\equiv$
  
  Es existiert ein $i\nnatural$ so, dass f"ur $M_i$ folgendes gilt:
  Betrachte das erste Eingabezeichen $x_i$ von $M_i$ nach dem Bandzeichen.
  Man l"asst $M_i$ mit der Eingabe
  \[\underbrace{x_ix_i\ldots x_i}_{n\text{-mal}}
    \]
  laufen, und es muss gelten:
  \begin{itemize}
    \item $M_i$ h"alt an mit genau $k$ $x_i$'s auf dem Band $\equiv$ $f(n)=k$ und
    \item $M_i$ h"alt nicht an $\equiv$ $f(n)$ ist undefiniert ($f(n)=\bot$).
    \end{itemize}
  
  Schreibweise: $f=\phi_i$.
  
  Bei Funktionen, die ``Ja/Nein''=0/1-Antworten liefern sollen, spricht
  man auch von Entscheid- oder L"osbarkeit.
  }
% -----------------------------------------------------------------------------
\definition Die Funktion $\halt$:{
  Wir definieren
  \[\halt(i):=\begin{cases}
      1 & \phi_i(i)\downarrow \\
      0 & \phi_i(i)\uparrow \\
      \end{cases}
    \]
  (Notation: $\downarrow$ = ``TM beendet mit Ausgabe'',
  $\uparrow$ = ``TM l"auft endlos'') 
  }
% -----------------------------------------------------------------------------
\theorem:=>{
  \label{the:halt-computable}
  $\halt$ ist nicht berechenbar.
  }
% -----------------------------------------------------------------------------
\proof \ref{the:halt-computable}:{
  Annahme: $\halt$ berechenbar. Definiere:
  \[\delta(i):=\begin{cases}
      \bot & \phi_i(i)\downarrow \\
      1 & \phi_i(i)\uparrow \\
      \end{cases}
    \]
  Elementar (auf der Ebene von Turingmaschinen) "uberlegt man sich,
  dass dann auch $\delta$ berechen bar ist.
  Es existiert also ein $j\nnatural$ so, dass $\phi_j\ident\delta$,
  also f"ur $\phi_j(j)$ gilt 
  \[\phi_j(j)=\begin{cases}
      \bot & \phi_j(j) \downarrow\\
      1 & \phi_j(j)\uparrow \\
      \end{cases}
    \]
  Widerspruch! \qed
  }
% -----------------------------------------------------------------------------
\subsection{Das Wortproblem}
% -----------------------------------------------------------------------------
\theorem:=>{
  \label{the:type0-nondecidable}
  Das Wortproblem ist f"ur Sprachen des Typs 0 nicht entscheidbar.
  }
% -----------------------------------------------------------------------------
\proof \ref{the:type0-nondecidable}:{
  Wir k"onnen das Halteproblem auf das Typ 0-Wortproblem reduzieren. \qed
  }
% -----------------------------------------------------------------------------
\theorem:=>{
  \label{the:type1-decidable}
  Das Wortproblem ist f"ur Sprachen des Typs 1 entscheidbar.
  }
% -----------------------------------------------------------------------------
\proof \ref{the:type1-decidable}:{
  $G=(V,T,S,P)$ sei eine Typ 1-Grammatik. Anhand des folgenden
  Algorithmus kann das Wortproblem f"ur das Wort $w$ entschieden werden:
  \begin{proc}
    \item $M:=\emptyset,M':=\{S\}$
    \item Solange $M\ne M'$ \begin{itemize}
      \item $M:=M'$
      \item $M':=M\cup \{\beta\mid\alpha\dderivable\beta,\alpha\in M,|\beta|\le|w|\}$
      \end{itemize}
    \item $w\in L(G)\equiv w\in M$.
    \end{proc}
  \qed
  }
% -----------------------------------------------------------------------------
\subsection{Das Post'sche Korrespondenzproblem}
% -----------------------------------------------------------------------------
\definition Post'sches Korrespondezproblem:{
  Ein \indexthis{Postsystem} "uber einem endlichen Alphabet $X$ ist eine Menge
  \[\left\{\begin{pmatrix}x_1\\y_1\end{pmatrix}, \dots,
    \begin{pmatrix}x_n\\y_n\end{pmatrix}\right\} \subset X^* \times
  X^*\]
  $i_1, \dots, i_k$ ist eine L"osung des Postsystems, falls gilt
  \[x_{i_1} x_{i_2} \dots x_{i_k} = y_{i_1} y_{i_2} \dots y_{i_k}\]
  Das Postsche Korrespondenzproblem, auch 
  \indexthis{Postian Correspondence Problem} oder 
  \emph{PCP}, besteht darin, eine solche
  L"osung zu finden.
  }
% -----------------------------------------------------------------------------
\example Zwei Beispiel-Postsysteme:{
  Das Postsystem
  \[\left\{\begin{pmatrix}1\\01\end{pmatrix},
    \begin{pmatrix}\lambda\\11\end{pmatrix},
    \begin{pmatrix}0101\\1\end{pmatrix},
    \begin{pmatrix}111\\\lambda\end{pmatrix}\right\} \subset X^* \times
  X^*\]
  hat eine L"osung $x_2 x_1 x_1 x_3 x_2 x_4 = 110101111 = y_2 y_1 y_1 y_3
  y_2 y_4$.

  Die k"urzeste L"osung f"ur das Postsystem
  \[\left\{\begin{pmatrix}001\\0\end{pmatrix},
    \begin{pmatrix}01\\011\end{pmatrix},
    \begin{pmatrix}01\\101\end{pmatrix},
    \begin{pmatrix}10\\001\end{pmatrix}\right\} \subset X^* \times
  X^*\]
  besteht aus 66 Indizes.
  }
% -----------------------------------------------------------------------------
\theorem:
  $G=(V,T,S,P)$ Grammatik=>{
  \label{the:derivable<->pcp}
  Dann existiert f"ur jedes Wort $w\in L(G)$ ein Postsystem $K(w)$ mit
  \[K(w)\text{ l"osbar}\equiv w\in L(G)\text{ l"osbar}
    \]
  }
% -----------------------------------------------------------------------------
\proof \ref{the:derivable<->pcp}:{
  Wir konstruieren $K$ "uber $X := V \cup V' \cup T
  \cup T' \cup \{*, *', [, ]\}$, wobei $y'$ eine (unterscheidbare) 
  Kopie von $y$ bezeichnet. $K$ bestehe aus
  \[\left\{
      \begin{pmatrix}[S*\\{}[\end{pmatrix},
      \begin{pmatrix}]\\{} *'w]\end{pmatrix},
      \begin{pmatrix}]\\{} *w']\end{pmatrix},
      \begin{pmatrix}a\\a'\end{pmatrix},
      \begin{pmatrix}a'\\a\end{pmatrix},
      \begin{pmatrix}v\\u'\end{pmatrix},
      \begin{pmatrix}v'\\u\end{pmatrix} \mid a \in X, u \to v \in P 
      \right\}
    \]
  Jede L"osung hat zwingend folgende Gestalt:
  \[\left(\begin{array}{c|c|c|c|c|c|c|c}
      [S* & w_1' & *' & w_2  &\dots & * & w' & ]\\{}
      [   & S    & *  & w_1' & \dots & *' & w_n & (* w'|*' w)]
      \end{array}\right)
    \]
  d.h. man kann aus ihr eine Ableitungskette
  \[S\dderivable w_1'\dderivable w_2 \dderivable \cdots \dderivable w
    \] 
  ablesen. \qed
  }
% -----------------------------------------------------------------------------
\remark Berechenbarkeit von L"osungen des PCP:{
  W"are das Postsche Korrespondenzproblem l"osbar, so w"are auch das
  Wortproblem l"osbar. Das Wortproblem ist nicht l"osbar, also ist auch
  das PCP nicht entscheidbar.
  }
% -----------------------------------------------------------------------------
\subsection{Entscheidbarkeit bei Grammatiken}
% -----------------------------------------------------------------------------
\theorem:
  $G=(V,T,S,P)$ eine Typ-2-Grammatik=>{
  \label{the:ctf-unique}
  Es ist nicht entscheidbar, ob $G$ nur eindeutige Ableitungen hat.
  }
% -----------------------------------------------------------------------------
\proof \ref{the:ctf-unique}:{
  Wir beweisen die Aussage durch Reduktion des PCP auf das Eindeutigkeitsproblem.
  Sei nun
  \[\left\{\begin{pmatrix}x_1\\y_1\end{pmatrix}, \dots,
    \begin{pmatrix}x_n\\y_n\end{pmatrix}\right\}\]
  ein Postsystem "uber $K$. Wir betrachten weiter die kontextfreie
  Grammatik
  \begin{eqnarray*}
    S &\to& A | B\\
    A &\to& x_1 A 1 | x_2 A 2 | \dots | x_n A n | \lambda\\
    B &\to& y_1 B 1 | y_2 B 2 | \dots | y_n B n | \lambda
    \end{eqnarray*}
  Die erzeugten Worte sind von der Form
  \[x_{i_1} \dots x_{i_k} i_k \dots i_1 \text{ oder } y_{i_1} \dots
  y_{i_k} i_k \dots i_1\]
  $G$ hat mehr als eine Ableitung f"ur ein einziges Wort genau dann, wenn
  das PCP l"osbar ist. Das PCP ist aber nicht entscheidbar, also ist auch
  das Eindeutigkeitsproblem entscheidbar.\qed
  }
% -----------------------------------------------------------------------------
\theorem:=>{
  \label{the:ctf-cut-nondecidable}
  Das Schnittproblem ist f"ur kontextfreie Sprachen unentscheidbar.
  }
% -----------------------------------------------------------------------------
\proof \ref{the:ctf-cut-nondecidable}:{
  Wir beweisen die Aussage durch Reduktion des PCP auf das Schnittproblem.
  Sei nun
  \[\left\{\begin{pmatrix}x_1\\y_1\end{pmatrix}, \dots,
    \begin{pmatrix}x_n\\y_n\end{pmatrix}\right\}\]
  ein Postsystem "uber $K$.
  Wir betrachten weiter die kontextfreie Grammatik
  \begin{eqnarray*}
    S &\to& A|B\\
    A &\to& x_1 A 1 | x_2 A 2 | \dots | x_n A n | x_1 1 | x_2 2 | \dots
    x_n n\\
    B &\to& y_1 B 1 | y_2 B 2 | \dots | y_n B n | y_1 1 | y_2 2 | \dots
    y_n n
  \end{eqnarray*}
  Dann sind 
  \[L_1 := \{\text{aus $A$ ableitbare Worte}\}, L_2 := \{\text{aus $B$
    ableitbare Worte}\}
    \]
  kontextfreie Sprachen, und Elemente aus $L_1 \cap L_2$ sind eine L"osung des
  PCP. W"are das Schnittproblem entscheidbar, dann auch das
  PCP. Dies ist nicht der Fall, also ist das Schnittproblem nicht
  entscheidbar.\qed
  }
% -----------------------------------------------------------------------------
\remark Kontextfreie Sprachen mit kontextfreien Komplementen:{
  Bei kontextfreien Sprachen mit kontextfreien Komplementen ist das 
  Schnittproblem ebensowenig entscheidbar, da die Grammatik in obigem Beweis
  eine Sprache mit kontextfreiem Komplement erzeugt.
  }
% -----------------------------------------------------------------------------
\remark Folgerung:{
  Das Schnittproblem ist f"ur Sprachen des Typs $i\le 2$ nicht entscheidbar.
  }
% -----------------------------------------------------------------------------
\theorem: $G=(V,T,S,P)$ kontextsensitiv=>{
  \label{the:cts-emptiness-nondecidable}
  $L(G)=\emptyset$ ist nicht entscheidbar.
  }
% -----------------------------------------------------------------------------
\proof \ref{the:cts-emptiness-nondecidable}:{
  Wir reduzieren das Schnittproblem f"ur kontextsensitive Grammatiken
  auf die Leerheit einer kontextsensitiven Grammatik und 
  nehmen ohne Beweis an, dass der Schnitt zweier
  kontextsensitiven Sprachen wieder kontextsensitiv ist. Dann gilt
  \[L(G) = \emptyset \equiv L(G) \cap L(G) = \emptyset
    \]
  Das Schnittproblem ist nicht entscheidbar, also ist auch die Leerheit
  nicht entscheidbar.\qed
  }
% -----------------------------------------------------------------------------
\theorem: $G_1=(V_1,T,S_1,P_1)$,$G_2=(V_2,T,S_2,P_2)$ kontextfrei=>{
  \label{the:ctf-equiv-nondecidable}
  $L(G_1)=L(G_2)$ ist nicht entscheidbar.
  }
% -----------------------------------------------------------------------------
\proof \ref{the:ctf-equiv-nondecidable}:{
  Seien $L_1,L_2$ kontextfreie Sprachen mit kontextfreiem Komplement.
  (deren Schnittproblem ist, wie oben gezeigt, nicht entscheidbar)
  W"are das Gleichheitsproblem entscheidbar, so k"onnte man
  den Schnitt aber wegen $L_1\cap L_2=\emptyset\equiv L_1\cup L_2^c=L_2^c$
  entscheiden. Also ist die Gleichheit f"ur ktf Sprachen mit ktf Komplementen
  nicht entscheidbar, deswegen erst recht nicht f"ur allgemeine ktf
  Sprachen.
  }
% -----------------------------------------------------------------------------
\section{Fixpunkttheorie}
% -----------------------------------------------------------------------------
\definition Halbordnung:{
  \index{Ordnung>partielle}
  Eine Relation ``$\le$'' auf einer Menge $M$ 
  hei"st Halb- oder \indexthis{Teilordnung} bzw.
  \indexthis{partielle Ordnung} $:\equiv$
  \begin{stmts}
    \item $(\forall x\in M)(x\le x)$ (Reflexivit"at)
    \item $(\forall x,y,z\in M)(x\le y,y\le z\implies x\le z)$ (Transitivit"at)
    \item $(\forall x,y,z\in M)(x\le y,y\le x\implies x=y)$ (Antisymmetrie)
    \end{stmts}
  $(M,\le)$ hei"st dann eine halbgeordnete Menge.
  }
% -----------------------------------------------------------------------------
\definition Kleinstes Element:{
  \index{Element>kleinstes}
  Sei $(M,\le)$ eine halbgeordnete Menge. Dann hei"st $k\in M$ kleinstes
  Element von $M$ $:\equiv$ $k\le x$ f"ur alle $x\in M$.
  }
% -----------------------------------------------------------------------------
\definition Obere Schranke:{
  \index{Schranke>obere}
  Sei $(M,\le)$ eine halbgeordnete Menge. Dann hei"st $o\in M$ obere Schranke
  von $T\subseteq M$ $:\equiv$ $x\le o$ f"ur alle $x\in T$.
  }
% -----------------------------------------------------------------------------
\definition Supremum:{
  $s\in M$ hei"st Supremum von $T\subseteq M$ $:\equiv$ $s$ ist kleinste
  obere Schranke von $T$.
  }
% -----------------------------------------------------------------------------
\remark Eindeutigkeit:{
  Suprema sind eindeutig.
  }
% -----------------------------------------------------------------------------
\definition Kette:{
  Sei $(M,\le)$ eine halbgeordnete Menge. Eine Folge $(x_i)_{i=1}^\infty\subseteq M$
  hei"st Kette $:\equiv$ $x_i\le x_{i+1}$ f"ur $i\ge 1$.
  }
% -----------------------------------------------------------------------------
\definition Vollst"andige halbgeordnete Menge:{
  \index{Menge>vollst"andige halbgeordnete}
  Sei $(M,\le)$ eine halbgeordnete Menge. $M$ hei"st vollst"andig $:\equiv$
  \begin{stmts}
    \item Es existiert ein kleinstes Element in $M$.
    \item Jede Kette in $M$ hat ein Supremum in $M$.
    \end{stmts}
  }
% -----------------------------------------------------------------------------
\definition Stetige Funktion:{
  \index{Funktion>stetige}
  Seien $(A,\le),(B,\le)$ vollst"andige halbgeordnete Mengen. Dann
  hei"st $f: A \to B$ stetig, $\equiv$ f"ur jede Kette 
  $(x_i)_{i=1}^\infty\subseteq A$ gilt
  \[\sup_{i=1}^\infty f(x_i) = f(\sup_{i=1}^\infty x_i)
    \]
  }
% -----------------------------------------------------------------------------
\remark:{
  Im analytischen Sinne sind unsere stetigen Funktionen monoton wachsend
  und linksseitig stetig auf abgeschlossenen Intervallen.
  }
% -----------------------------------------------------------------------------
\definition Monotone Funktion:{
  \index{Funktion>monotone}
  Seien $(A,\le),(B,\le)$ halbgeordnete Mengen. Dann
  hei"st $f: A \to B$ monoton, $\equiv$ f"ur alle $x,y\in A$ mit
  $x\le_A y$ gilt auch $f(x)\le_B f(y)$
  }
% -----------------------------------------------------------------------------
\lemma:
  $(A, \leq),(B,\le)$ vollst"andige halbgeordnete Mengen,
  $f:A\to B$ stetig=>{
  \label{the:cont-mono}
  Dann ist $f$ monoton.
  }
% -----------------------------------------------------------------------------
\proof \ref{the:cont-mono}:{
  Sei $a\le_A b$ in $A$. Betrachte $x_1:=a$, $x_2:=x_3\ldots:=b$.
  
  $f$ ist stetig, woraus folgt $\sup_n f(x_n)=f(\sup_n x_n)=f(b)$
  Also gilt $f(a)=f(x_1)\le f(b)$.\qed
  }
% -----------------------------------------------------------------------------
\definition Fixpunkt:{
  Sei $(A,\leq)$ eine vollst"andige halbgeordnete Menge und $f:A\to A$ 
  eine Abbildung. $a \in A$ hei"st Fixpunkt von $f$ $:\equiv$ $f(a)=a$.
  }
% -----------------------------------------------------------------------------
\theorem Kleene'scher Fixpunktsatz:
  $(A,\le)$ vollst"andige halbgeordnete Menge, $f:A\to A$ stetig=>{
  \label{the:kleene-fix}
  Sei $m$ das kleinste Element von $A$. Dann ist
  $\sup_{n\natural} f^n(m)$ der kleinste Fixpunkt von $f$. 
  Insbesondere existiert mindestens ein Fixpunkt.
  }
% -----------------------------------------------------------------------------
\proof \ref{the:kleene-fix}:{
  Wir zeigen zun"achst, dass $\sup f^n(m)$ existiert:
  \[m\leq f(m)\implies f(m)\leq f^2(m)\implies\ldots\implies f^n(m)\le f^{n+1}(m)
    \]
  Also ist $(f^n(m))_{n\natural}$ eine Kette. Es folgt direkt die
  Existenz des Supremums.
  
  Weiter zeigen wir: Das Supremum von $f^n(m)$ ist ein Fixpunkt:
  \[f(\sup(f^n(m))) = \sup(f^{n + 1}(m)) = \sup(f^n(m))
    \]
  Und schlie"slich zeigen wir, dass $\sup f^n(m)$ kleinster Fixpunkt ist,
  angenommen $c\in A$ sei ein kleinerer Fixpunkt:
  \[m\le c\implies f(m)\le f(c)=c \implies\ldots\implies f^n(m) \le f(c)=c
    \]
  \qed
  }
% -----------------------------------------------------------------------------
\subsection{Der Fixpunktsatz}
% --------------------------------------------------------------------------
\remark Kurzschreibweise von Wortmengen:{
  Analog zu den regul"aren Ausdr"ucken schreiben wir z.B.
  \[a\{b,c\}d:=\{abd,acd\}
    \]
  Beachte:
  \[a\emptyset d:=\emptyset
    \]
  }
% -----------------------------------------------------------------------------
\definition Funktion einer kontextfreien Grammatik:{
  Sei $G=(V,T,A_1,P)$ mit $V=\{A_1,\ldots,A_n\}$ und
  \[P=\{A_i\to a_{i1},\ldots,a_{ik_i}\mid i=1,\ldots,n, a_{ij}\in (T\cup V)^*\}
    \]
  eine kontextfreie Grammatik.
  
  Definiere
  \[g_i:\cal P(T^*)^n\to\cal P(T^*)
    \]
  mit $g_i(M_1,\ldots,M_n):=\overline{a_{i1}}\cup\ldots\cup\overline{a_{ik_i}}$.
  Ist z.B. $a_{ik}=\alpha A_j\beta$, so ist $\overline{a_{ik}}:=\alpha M_j \beta$.
  Beachte: $a_{ik}$ ist eine Zeichenkette, $\overline{a_{ik}}$ ist eine Menge.
  
  Dann hei"st $g:=(g_1,\ldots,g_n)$ die Funktion der kontextfreien 
  Grammatik $G$.
  }
% -----------------------------------------------------------------------------
\convention{
  Ab hier betrachten wir als Teilordnung die Ordnung ``$\subseteq$'' auf
  Mengen $M\in\cal P(T^*)$ "uber einem Alphabet $T$:
  $M\le N:\equiv M\subseteq N$.
  
  F"ur Tupel $(M_1,\ldots,M_n)\in\cal P(T^*)^n$ von solchen Mengen gelte 
  dabei die Ordnung
  \[(M_1,\ldots,M_k)\le(N_1,\ldots,N_k):\equiv (\forall i\in\{1,\ldots,k\})(M_i\le N_i)
    \]
  }
% -----------------------------------------------------------------------------
\lemma:=>{
  \label{lem:fun-cont-comp}
  Eine zusammengesetzte Funktion ist stetig $\equiv$ ihre 
  Komponentenfunktionen sind stetig.
  }
% -----------------------------------------------------------------------------
\lemma:=>{
  \label{lem:fun-ctf-cont}
  Die Funktion einer kontextfreien Grammatik ist stetig.
  }
% -----------------------------------------------------------------------------
\proof \ref{lem:fun-ctf-cont}:{
  Wegen \ref{lem:fun-cont-comp} muss nur noch die Stetigkeit einer
  Komponentenfunktion $g_i$ gezeigt werden. 
  Sei $(M^{(k)})_{k=1}^\infty=(M_1^{(k)},\ldots,M_n^{(k)})_{k=1}^\infty$ eine 
  Kette.
  Zu zeigen ist $g_i(\sup_k M^{(k)})=\sup_k g_i(M^{(k)})$.
  Mit den Bezeichnungen aus der Definition von $g$ gilt:
  \begin{align*}
    w\in g_i(\sup_k M^{(k)})
    \equiv &w\in g_i(\sup_k(M_1^{(k)},\ldots,M_n^{(k)}))\\
    \equiv &w\in g_i(\sup_k M_1^{(k)},\ldots,\sup_k M_n^{(k)})\\
    \equiv &(\exists j\in\{1,\ldots,k_i\})(w\in\overline{a_{ij}}
      \overset{\text{z.B.}}= \alpha(\sup_k M^{(k)}_l)\beta) \\
    \equiv &(\exists k\natural)(\exists j\in\{1,\ldots,k_i\})(w\in\overline{a_{ij}}
      \overset{\text{z.B.}}= \alpha(M^{(k)}_l)\beta)\\
    \equiv &(\exists k\natural)(w\in g_1(M^{(k)}))
    \equiv w\in \sup_k g_1(M^{(k)})
    \end{align*}
  \qed
  }
% -----------------------------------------------------------------------------
\theorem:
  $G=(V,T,S,P)$ eine kontextfreie Grammatik, $g$ die Funktion dazu=>{
  \label{the:fix}
  Sei dann
  \[L_i:=\{w\in T^*\mid A_i\derivable w,A_i\in V\}\qquad i=1,ldots,n
    \]
  also z.B. $L_1=L(G)$, falls $A_1=S$. Dann ist $(L_1,\ldots,L_n)$ der
  kleinste Fixpunkt von $g$.
  }
% -----------------------------------------------------------------------------
\proof \ref{the:fix}:{
  Offenbar ist $\emptyset\in\cal P(T^*)$ das Minimum von $\cal P(T^*)$ 
  bez"uglich der Teilmengenordnung. Weiter ist 
  \[\sup_k g(\emptyset,\ldots,\emptyset)=(L_1,\ldots,L_n)
    \]
  und damit ergibt sich nach \ref{the:kleene-fix} sofort die Behauptung.\qed
  }
% -----------------------------------------------------------------------------
\subsection{Gleichungssysteme}
% -----------------------------------------------------------------------------
\theorem:
  $A,B\in\cal P(T^*)$, $T$ Alphabet=>{
  \label{the:les}
  F"ur das Gleichungssystem $X=AX+B$ ist $A^*B$ die kleinste L"osung.
  }
% -----------------------------------------------------------------------------
\proof \ref{the:les}:{
  Wir betrachten einfach $g(X)=AX+B$. Der kleinste Fixpunkt ist
  \[\sup_k g^k(\emptyset)=B+AB+A^2B+A^3B+\cdots=A^*B
    \]
  Hallelujah!\qed
  }
% -----------------------------------------------------------------------------
\example:{
  Sei $G := (\{A_1, A_2\}, \{a, b\}, A_1, P)$ mit
  \[P := A_1 \to aA_1 | bA_2 | a, A_2 \to aA_1 | aA_2 | b
    \]
  Wir haben also gegeben
  \begin{eqnarray*}
    x_1 &=& ax_1 + bx_2 + a\\
    x_2 &=& ax_1 + ax_2 + b
  \end{eqnarray*}
  mit $x_1, x_2 \in \mathcal{P}((a+b)^*)$. Es folgt nach Satz
  \ref{the:les}
  \[x_1 = a^*(bx_2 + a),\]
  woraus weiter folgt
  \[x_2 = a^+(bx_2 + a) + ax_2 + b = a^+bx_2 + ax_2 + aa^+b = (a^+b +
  a)x_2 + aa^+ + b\]
  Erneutes Anwenden von Satz \ref{the:les} ergibt
  \[x_2 = (a^+b + a)^*(aa^+ + b)\]
  Einsetzen f"uhrt auf
  \[x_1 = a^*(b(a^+b + a)^*(aa^+ + b) + a) = a^*b(a^+b + a)^*(aa^+ + b)
  + a^+\]
  Also folgt
  \[L(G) = a^*b(a^+b + a)^*(aa^+ + b) + a^+\]
  }
% -----------------------------------------------------------------------------
\section{Syntaxanalyse}
% -----------------------------------------------------------------------------

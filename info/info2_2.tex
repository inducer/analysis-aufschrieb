% -----------------------------------------------------------------------
\para{Fourier-Transformation}
% -----------------------------------------------------------------------
\subsection{Fourier-Reihe f"ur periodische kontinuierliche Signale}
% -----------------------------------------------------------------------
\definition Kontinuierliches Signal:{
  \index{Signal>kontinuierliches}
  Ein kontinuierliches Signal sei hier eine Funktion $f:\SetR\to\SetR$, 
  dargestellt als $f(x):=y$, wobei $x,y\real$.
}
% -----------------------------------------------------------------------
\definition Periodisches Signal:{
  \index{Signal>periodisches}
  Gilt f"ur alle $x\real$ und ein $T\real^+$
  \[
    f(x)=f(x+T)
  \]
  so hei"st $f$ periodisch mit der \indexthis{Periode} $T$.
  (kurz: $T$-periodisch) Existiert ein kleinstes $T_0\real^+$, so dass
  $f$ $T_0$-periodisch ist, so nennt man $T_0$ die Grundperiode von $f$.
}
% -----------------------------------------------------------------------
\definition Kreisfrequenz:{
  Den einer Periode $T$ zugeordneten Wert
  \[
    \omega=\frac{2\pi}T
  \]
  nennt man die Kreisfrequenz. Die der Grundperiode $T_0$ eines Signals
  zugeordnete Kreisfrequenz hei"st Grundkreisfrequenz $\omega_0$.
}
% -----------------------------------------------------------------------
\definition Trigonometrische Reihe:{
  \index{Reihe>trigonometrische}
  Seien $a_k,b_k\in\SetR$ ($k\nnatural$). Dann hei"st eine Funktionenreihe
  der Gestalt
  \[
    f(x)=\frac{a_0}2+\sumn 1 a_n \cos(\omega_0 nx) + b_n \sin(\omega_0 nx)
  \]
  eine trigonometrische Reihe oder \indexthis{Fourierreihe} mit Periode $T_0$.
}
% -----------------------------------------------------------------------
\theorem:
  $f$ trigonometrische Reihe
  =>
{
  Sei
  \[
    f(x)=\frac{a_0}2+\sumn 1 a_n \cos(\omega_0 nx) + b_n \sin(\omega_0 nx)
  \]
  mit $a_n,b_n\real$. Dann legt man f"ur $k=1\ldots n$ fest
  \begin{align*}
    c_k&:=\frac{a_k-ib_k}2 \\
    c_0&:=\frac{a_0}2 \\
    c_{-k}&:=\frac{a_k+ib_k}2
  \end{align*}
  Dann gilt
  \begin{align*}
    f(x)&=\sum_{n=-\infty}^\infty c_n e^{i\omega_0 nx} \\
        &=\underbrace{
          \sum_{n=1}^\infty \frac{a_n-ib_n}2 (\cos(-\omega_0 nx)+i\sin(-\omega_0 nx)) +
          }_{\text{``negative H"alfte''}}
          \frac{a_0}2 + 
          \underbrace{
          \sum_{n=1}^\infty \frac{a_n+ib_n}2 (\cos(\omega_0 nx)+i\sin(\omega_0 nx))
          }_{\text{``positive H"alfte''}}\\
        &=\frac{a_0}2 + 
          \sum_{n=1}^\infty 
            \frac{a_n-ib_n}2 (\cos(\omega_0 nx)-i\sin(\omega_0 nx)) +
            \underbrace{\frac{a_n+ib_n}2 (\cos(\omega_0 nx)+i\sin(\omega_0 nx))}
            _{\text{konjugiert komplex zum 1. Summanden}} \\
        &=\frac{a_0}2 + 
          \sum_{n=1}^\infty 2\Re(\frac{a_n+ib_n}2(\cos(\omega_0 nx)-i\sin(\omega_0 nx)) \\
        &=\frac{a_0}2 + 
          \sum_{n=1}^\infty a_n \cos(\omega_0 nx)+b_n\sin(\omega_0 nx)
  \end{align*}
}
% -----------------------------------------------------------------------
\definition Integral "uber beliebiges Intervall:{
  Der Ausdruck
  \[
    \int_T f(x) dx = a
  \]
  mit $f:\SetR\to\SetR$, $T,a\real$ bedeutet
  \[
    (\forall s\real)(\int_s^{s+T} f(x) dx = a)
  \]
}
% -----------------------------------------------------------------------
\theorem Orthogonalit"atsrelationen:
  $m,n\natural$
  =>
{
  Dann gilt:
  \begin{stmts}
    \item 
      \[
        \int_{2\pi} \sin(mx) \cdot \sin (nx) dx=
        \begin{cases}
          0 & \text{f"ur $m\ne n$} \\
          \pi & \text{f"ur $m=n>0$} \\
          0 & \text{f"ur $m=n=0$} \\
        \end{cases}
      \]
    \item 
      \[
        \int_{2\pi} \sin(mx) \cdot \cos (nx) dx=0
      \]
    \item 
      \[
        \int_{2\pi} \cos(mx) \cdot \cos (nx) dx=
        \begin{cases}
          0 & \text{f"ur $m\ne n$} \\
          \pi & \text{f"ur $m=n>0$} \\
          2\pi & \text{f"ur $m=n=0$} \\
        \end{cases}
      \]
    \item Oder, deutlich k"urzer und eleganter:
      \[
        \int_{2\pi} e^{inx}\cdot e^{-imx} dx=
        \begin{cases}
          0 & \text{f"ur $m\ne n$} \\
          2\pi & \text{f"ur $m=n$}
        \end{cases}
      \]
    \item Und noch allgemeiner:
      \[
        \int_{T_0} e^{i\omega_0 (n-m) x} dx=
        \begin{cases}
          0 & \text{f"ur $m\ne n$} \\
          T_0 & \text{f"ur $m=n$}
        \end{cases}
      \]
  \end{stmts}
}
% -----------------------------------------------------------------------
\theorem:
  $f:(-T_0/2,T_0/2]\to\SetR$ integrierbar und durch 
  eine konvergente trigonometrische Reihe darstellbar
  =>
{
  \label{the:contfourseries}
  Sei
  \[
    g(x):=\sum_{n=-\infty}^\infty c_n e^{i\omega_0 nx}
  \]
  mit
  \[
    c_n:=\frac 1 {T_0} \int_{T_0} f(x) e^{-i\omega_0 nx} dx 
  \]
  Dann gilt $f(x)=g(x)$.
}
% -----------------------------------------------------------------------
\theorem Dirichletsche Bedingungen:
  $f:\SetR\to\SetR$ $T_0$-periodisch
  =>
{
  Gelten f"ur $f$ die folgenden drei Bedinungen
  \begin{stmts}
    \item $f$ ist absolut integrierbar:
      \[
        \int_{T_0} |f(x)| dx < \infty
      \]
    \item $f$ ist auf ganz $\SetR$ von beschr"ankter Variation
    \item $f$ hat auf jedem endlichen Intervall h"ochstens endlich viele
      Unstetigkeitsstellen.
  \end{stmts}
  so ist $f$ als Fourier-Reihe darstellbar.
}
% -----------------------------------------------------------------------
\remark:{
  Insbesondere erf"ullen st"uckweise stetig differenzierbare Funktionen
  die Dirichletschen Bedinungen.
}
% -----------------------------------------------------------------------
\subsection{Fourier-Transformation f"ur aperiodische kontinuierliche Signale}
% -----------------------------------------------------------------------
\definition Fourier-Transformation:{
  Sei $f$ ein aperiodisches Signal, f"ur das gilt 
  $(\forall |x|>T_1)(f(x)=0)$. Sei $\tilde f(x)$ die periodische Fortsetzung
  von $f$ mit Periode $T_0>T_1$.
  Betrachte \ref{the:contfourseries}. Setzt man
  \[
    \FT(\tilde f)(n\omega_0)=T_0 c_n=\int_{T_0} \tilde f(x) e^{-i\omega_0 nx} dx 
  \]
  so gilt
  \[
    \tilde f(x)=\frac 1 {2\pi} \sum_{n=-\infty}^\infty \FT(\tilde f)(n\omega_0) e^{i\omega_0 nx} \omega_0
  \]
  L"asst man nun $T_0\to\infty$, also $\omega_0\to 0$ 
  (und somit auch ``$\tilde f\to f$'') gehen und 
  verallgemeinert man $\FT(f)(n\omega_0)$ f"ur beliebige $\omega$ 
  (``$\omega:=n\omega_0$''), erh"alt man die Analysegleichung 
  \[
    \FT(f)(\omega)=\int_{-\infty}^\infty f(x) e^{-i\omega x} dx 
  \]
  und die Synthesegleichung 
  \[
    \FT^{-1}(\FT(f))(x)=f(x)=
      \frac 1 {2\pi} \int_{-\infty}^\infty \FT(f)(\omega) e^{i\omega x} d\omega
  \]
  der Fouriertransformation. Hierbei nennt sich $\FT(f):\SetR\to\SetR$ die 
  Fourier-Transfomierte bzw. das \indexthis{Spektrum} von $f$.
}
% -----------------------------------------------------------------------
\remark Weitere Spektren:{
  Weiterhin unterscheidet man folgende Spektren:
  \begin{itemize}
    \item \indexthis{Amplitudenspektrum} $|\FT(f)|$
    \item \indexthis{Phasenspektrum} 
      $\phi(\omega):=\arctan \frac {\Im(\FT(f))}{\Re(\FT(f))}$
    \item \indexthis{Leistungsspektrum} $|\FT(f)|^2$
  \end{itemize}
}
% -----------------------------------------------------------------------
\theorem Dirichletsche Bedingungen:
  $f:\SetR\to\SetR$
  =>
{
  Gelten f"ur $f$ die folgenden drei Bedinungen
  \begin{stmts}
    \item $f$ ist absolut integrierbar:
      \[
        \int_{-\infty}^\infty |f(x)| dx < \infty
      \]
    \item $f$ ist auf ganz $\SetR$ von beschr"ankter Variation
    \item $f$ hat auf jedem endlichen Intervall h"ochstens endlich viele
      Unstetigkeitsstellen.
  \end{stmts}
  so hat $f$ eine Fourier-Transformierte und es gilt $\FT^{-1}\circ\FT(f)=f$.
}
% -----------------------------------------------------------------------
\definition Dirac-Distribution:{
  Die Zuordnung $\delta:\SetR\to\SetR$ mit
  \[
    \delta(x):=\begin{cases}
      0 & \text{f"ur $|x|>\epsilon$} \\
      \frac 1 {2\epsilon} & \text {f"ur $|x|\le 2\epsilon$} 
    \end{cases}
  \]
  und $\epsilon\to 0+$ ist eine m"ogliche Ann"aherung der Dirac-Distribution.
}
% -----------------------------------------------------------------------
\definition Einheitssprung:{
  Das kontinuierliche Signal
  \[
    \sigma(x):=\begin{cases}
      0 & \text{f"ur $x<0$} \\
      1 & \text{f"ur $x\ge 0$} \\
    \end{cases}
  \]
  nennt man das Einheitssprung-Signal.
}
% -----------------------------------------------------------------------
\remark:{
  N"aherungsweise gelten:
  \begin{stmts}
    \item \[ \frac {d\sigma}{dx}=\delta(x) \]
    \item \[ \int_{-\infty}^\infty \delta(x) dx = 1 \]
    \item \[ \int_{-\infty}^\infty f(x)\delta(x) dx = f(0) \]
  \end{stmts}
  Aus (3) folgt, dass die Fouriertransformierte des Impulses 
  konstant $1$ ist.
}
% -----------------------------------------------------------------------
\example:{
  Es gelten:
  \begin{align*}
    \FT(1)(\omega) &= 2\pi \delta(\omega) \\
    \FT(\delta(x))(\omega) &= 1 \\
    \FT(\sigma(x))(\omega) &= \pi \delta(\omega)-i\frac 1 \omega \\    
    \FT(e^{i\omega_0 x})(\omega) &= 2\pi\delta(\omega-\omega_0) \\
    \FT(\sin(\omega_0 x))(\omega) &= -i\pi [\delta(\omega-\omega_0)+\delta(\omega+\omega_0)] \\
    \FT(\cos(\omega_0 x))(\omega) &= \pi [\delta(\omega-\omega_0)-\delta(\omega+\omega_0)] \\
    \FT(\sum_{k=-\infty}^\infty \delta(x-kT_0))(\omega) &=
      \omega_0 \sum_{k=-\infty}^\infty \delta(\omega-k\omega_0) \\
  \end{align*}
}
% -----------------------------------------------------------------------
\remark:{
  Sei $f$ durch eine trigonometrische Reihe darstellbar. Dann folgt mit Hilfe
  des Beispiels, dass $\FT(f)$ aus einer Summe von Impulsen besteht, 
  deren Intensit"aten den Fourier-Koeffizienten entsprechen.
}
% -----------------------------------------------------------------------
\definition Faltung:{
  Seien $f,g:\SetR\to\SetR$ Funktionen. Dann nennt man
  \[
    (f*g)(t):=\int_{-\infty}^\infty f(t-\tau)g(\tau) d\tau
  \]
  die Faltung von $f$ und $g$.
}
% -----------------------------------------------------------------------
\theorem Eigenschaften der Fourier-Transformation:
  $f,g$ kontinuierliche Signale, $a,b\complex$, $t_0\real$=>
{
  Es gelten
  \begin{stmts}
    \item Linearit"at:
      \[
        \FT(af+bg)=a\FT(f)+b\FT(g)
      \]
    \item Differentiation:
      \[
        \FT(f^{(n)})=(i\omega)^n \FT(f)
      \]
    \item Originalverschiebung:
      \[
        \FT(f(x-x_0))(\omega)=e^{-i\omega x_0} \FT(f)(\omega)
      \]
    \item Dehnung:
      \[
        \FT(f(ax))(\omega)=\frac 1 {|a|} \FT(f)(\frac \omega a)
      \]
    \item Faltung:
      \[
        \FT(f*g)=\FT(f)\cdot \FT(g)
      \]
    \item Konjugierte Symmetrie:
      \[
        \FT(f)(-\omega)=\FT^*(f)(\omega)
      \]
  \end{stmts}
  wobei $x^*$ die zu $x$ konjugiert komplexe Zahl darstellt.
}
% -----------------------------------------------------------------------
\remark:{
  Aus der konjugierten Symmetrie ergibt sich (durch Substitution im Integral), 
  dass gerade Signale ($f(x)=f(-x)$) eine rein reelle und gerade 
  Fouriertransformierte haben, w"ahrend reelle ungerade Signale eine rein
  imagin"are und ungerade Fouriertransformierte haben.
}
% -----------------------------------------------------------------------
\theorem Reziprozit"at der Fouriertransformation:
  $f$ kontinuierliches Signal, $g:=\FT(f)$
  =>
{
  Falls $g$ existiert, gilt
  \[
    \FT(g)(\omega)=2\pi f(-\omega)
  \]
}
% -----------------------------------------------------------------------
\subsection{Fourier-Reihe f"ur periodische diskrete Signale}
% -----------------------------------------------------------------------
\definition Diskretes Signal:{
  \index{Signal>diskretes}
  Ein diskretes Signal sei hier eine Funktion $f:\SetZ\to\SetR$, 
  dargestellt als $f[x]:=y$, wobei $x\integer$ und $y\real$.
}
% -----------------------------------------------------------------------
\definition Periodisches Signal:{
  \index{Signal>periodisches}
  Gilt f"ur alle $x\integer$ und ein $N\natural$
  \[
    f[x]=f[x+N]
  \]
  so hei"st $f$ periodisch mit der \indexthis{Periode} $N$.
  (kurz: $N$-periodisch) Existiert ein kleinstes $N_0\natural$, so dass
  $f$ $N_0$-periodisch ist, so nennt man $N_0$ die Grundperiode von $f$.
}
% -----------------------------------------------------------------------
\definition Kreisfrequenz:{
  Den einer Periode $N$ zugeordneten Wert
  \[
    \omega=\frac{2\pi}N
  \]
  nennt man die Kreisfrequenz. Die der Grundperiode $N_0$ eines Signals
  zugeordnete Kreisfrequenz hei"st Grundkreisfrequenz $\omega_0$.
}
% -----------------------------------------------------------------------
\theorem Periodizit"at von diskreten komplex exponentiellen Signalen:
  $f$ diskretes komplex exponentielles Signal=>
{
  \label{the:discperiodicity}
  Sei $f[x]:=e^{i \omega x}$. Dann ist $f$ genau dann periodisch, wenn
  es $m,N\natural$ gibt mit $\frac{\omega}{2\pi}=\frac m N$. Die 
  Grundperiode ist dann $N_0=N=m\frac{2\pi}{\omega}$. Dass solche
  $m,N$ existieren sei ab jetzt immer vorausgesetzt, wenn $\omega_0$ 
  erscheint.
}
% -----------------------------------------------------------------------
\definition Summe "uber beliebigen Bereich:{
  Der Ausdruck
  \[
    \sum_{n=<N>} f[n] = a
  \]
  mit $f:\SetZ\to\SetR$, $N\natural,a\real$ bedeutet
  \[
    (\forall s\integer)(\sum_{n=s}^{s+N} f[n] = a)
  \]
}
% -----------------------------------------------------------------------
\definition Trigonometrische Reihe:{
  \index{Reihe>trigonometrische}
  Seien $c_n\real$ ($n\integer$). Dann hei"st eine Funktionenreihe der Gestalt
  \[
    f[x]=\sum_{n=<N_0>} c_n e^{i\omega_0 nx}
  \]
  eine trigonometrische Reihe oder \indexthis{Fourierreihe} mit Periode $N_0$.
}
% -----------------------------------------------------------------------
\remark:{
  Da gilt
  \[
    e^{i\omega_0 n}=e^{i\omega_0 (n+N_0)}
  \]
  gen"ugen f"ur eine $N_0$-periodische Fourierreihe $N_0$ Koeffizienten.
}
% -----------------------------------------------------------------------
\theorem:
  $k\natural$
  =>
{
  Dann gilt:
  \[
    \sum_{n=<N_0>} e^{i\omega_0 nk} =
    \begin{cases}
      N_0 & \text{f"ur $k=zN_0$ ($z\integer$)} \\
      0 & \text{sonst} 
    \end{cases}
  \]
}
% -----------------------------------------------------------------------
\theorem:
  $f:\SetZ\to\SetR$ diskretes $N_0$-periodisches Signal 
  =>
{
  \label{the:discfourseries}
  Sei
  \[
    g[x]:=\sum_{n=<N_0>} c_n e^{i\omega_0 nx}
  \]
  mit
  \[
    c_n:=\frac 1 {N_0} \sum_{x=<N_0>} f[x] e^{-i\omega_0 nx}
  \]
  Dann gilt $f[x]=g[x]$.
}
% -----------------------------------------------------------------------
\remark:{
  Diese Zerlegung existiert f"ur jedes $N_0$-periodische Signal.
}
% -----------------------------------------------------------------------
\subsection{Fourier-Transformation f"ur aperiodische diskrete Signale}
% -----------------------------------------------------------------------
\definition Fourier-Transformation:{
  Sei $f$ ein aperiodisches Signal, f"ur das gilt 
  $(\forall |x|>N_1)(f[x]=0)$. Sei $\tilde f$ die periodische Fortsetzung
  von $f$ mit Periode $N_0>N_1$.
  Betrachte \ref{the:discfourseries}. Setzt man
  \[
    \FT(\tilde f)(n\omega_0)=N_0 c_n=\sum_{x=<N_0>} \tilde f[x] e^{-i\omega_0 nx}
  \]
  so gilt
  \[
    \tilde f[x]=\frac 1 {2\pi} \sum_{n=<N_0>} \FT(\tilde f)(n\omega_0) e^{i\omega_0 nx} \omega_0
  \]
  L"asst man nun $N_0\to\infty$, also $\omega_0\to 0$ 
  (und somit auch ``$\tilde f\to f$'') gehen und 
  verallgemeinert man $\FT(f)(n\omega_0)$ f"ur beliebige $\omega$ 
  (``$\omega:=n\omega_0$''), erh"alt man die Analysegleichung 
  \[
    \FT(f)(\omega)=\sum_{x=-\infty}^\infty f[x] e^{-i\omega x} dx 
  \]
  und die Synthesegleichung 
  \[
    \FT^{-1}(\FT(f))[x]=f[x]=
      \frac 1 {2\pi} \int_{2\pi} \FT(f)(\omega) e^{i\omega x} d\omega
  \]
  der Fouriertransformation. Hierbei nennt sich $\FT(f):\SetR\to\SetR$ die 
  Fourier-Transfomierte bzw. das \indexthis{Spektrum} von $f$.
}
% -----------------------------------------------------------------------
\remark:{
  Hier wirkt sich die Reziprozit"at deutlich aus: Bei der 
  kontinuierlichen Fourierreihe war das Spektrum diskret, jedoch
  das Signal kontinuierlich, hier ist das Spektrum kontinuierlich,
  jedoch das Signal diskret.
}
% -----------------------------------------------------------------------
\remark:{
  Die Eigenschaften der diskreten Fouriertransformation ergeben sich analog
  zur kontinuierlichen.
}
% -----------------------------------------------------------------------
\subsection{Erg"anzungen}
% -----------------------------------------------------------------------
\definition DFT:{
  Sei $f$ ein aperiodisches diskretes Signal, f"ur das gilt 
  $(\forall x\not\in\{0,\ldots,N_1-1\})(f[x]=0)$. 
  Sei $\tilde f$ die periodische Fortsetzung von $f$ mit Periode $N_0>N_1$ so,
  dass $f[x]=\tilde f[x]$ f"ur alle $x\in\{0,\ldots,N_0-1\}$.
  Betrachte \ref{the:discfourseries}. Setzt man
  \[
    \DFT(f)[n]=c_n=\frac 1 {N_0} \sum_{x=0}^{N_0-1} \tilde f[x] e^{-i\omega_0 nx}
  \]
  so gilt
  \[
    f[x]\approxeq\tilde f[x]=\sum_{x=0}^{N_0-1} \DFT(f)[n] e^{i\omega_0 nx}
  \]
  Dann nennt man $\DFT(f)$ die diskrete Fouriertransformierte von $f$.
}
% -----------------------------------------------------------------------
\remark:{
  Die diskrete Fouriertransformation ist \emph{nicht} gleich der
  Fouriertransformation eines diskreten Signals. (dieser aber in gewisser
  Weise "ahnlich)
}
% -----------------------------------------------------------------------
\remark:{
  Da gilt 
  \[
    e^{i\omega_0 nx}=
      (\underbrace{e^{i 2\pi/N_0}}_{\text{$N_0$-te Einheitswurzel}})^{xn}
  \]
  l"asst sich die DFT mit Hilfe der DFT-Matrix auf einen
  Schlag berechnen. (Beachte formale "Ubereinstimmung der DFT-Synthesegleichung
  mit der Matrixmultiplikation mit der DFT, ebenso der DFT-Analyse mit der
  Inversen der DFT-Matrix)
}
% -----------------------------------------------------------------------
\definition Bandbegrenztheit:{
  Sei $f:\SetR\to\SetR$, eine Funktion, $F:\SetR\to\SetC$ ihre 
  Fouriertransformierte. Dann hei"st $f$ bandbegrenzt mit der 
  Grenzkreisfrequenz $\omega_g$ genau dann, wenn gilt
  \[
    f(t)=\frac 1 {2\pi} \int_{-\omega_g}^{\omega_g} F(\omega) d\omega
  \]
}
% -----------------------------------------------------------------------
\remark:{
  F"ur eine mit $\omega_g$ bandbegrenzte Funktion folgt 
  $\omega>\omega_g\implies F(\omega)=0$.
}
% -----------------------------------------------------------------------
\theorem Abtasttheorem:
  $f:\SetR\to\SetR$ $\omega$-periodisch, bandbegrenzt mit $\omega_g$
  =>
{
  Sei $n:=\frac {\omega_g}\omega$. 
  W"ahle ein $t_s$ mit $|t_s|<\frac \pi {\omega_g}$ und ein $t_0\in\SetR$
  beliebig.
  Dann ist $f$ durch die Funktionswerte $f(t_0-nt_s),\ldots,f(t_0+nt_s)$
  eindeutig bestimmt.
}
% -----------------------------------------------------------------------
\remark:{
  Betrachtet man das Abtasten/Samplen als Faltung mit einer
  Impulssequenz der Frequenz $\omega_0$, so ergibt sich als 
  Fouriertransformierte eine Reihe von Spektren im Abstand $\omega_0$. 
  R"ucken diese nun so dicht aneinander, dass sie sich "uberlappen 
  (ist also der Abstand der Spektren $\omega_0$ kleiner als deren 
  jeweilige Ausdehnung $2\omega_g$),so "uberlappen sich diese (Aliasing) 
  und das urspr"ungliche Signal ist nicht mehr rekonstruierbar.
}
% -----------------------------------------------------------------------
\para{Information und Kodierung}
% -----------------------------------------------------------------------
\definition Alphabet:{
  Ein Alphabet $A$ ist eine geordnete endliche Menge von Symbolen.
  \[
    A=(a_1,a_2,\ldots,a_n)
  \]
}
% -----------------------------------------------------------------------
\definition Verteilung eines Alphabets:{
  Als Verteilung $p_A$ eines Alphabets $A$ bezeichnet man einen Vektor 
  von normierten Wahrscheinlichkeiten, der f"ur jedes Zeichen 
  angibt, mit welcher Wahrscheinlichkeit es auftritt.
  \[
    p_A=(p(a_1),p(a_2),\ldots,p(a_n))
  \]
  ``Normiert'' bedeutet, dass gelten muss
  \[
    1 = \sum_{k=1}^n p(a_i)
  \]
}
% -----------------------------------------------------------------------
\definition "Ubertragungseinrichtung:{
  Zu einer "Ubertragungseinrichtung geh"oren ein \indexthis{Sender}
  mit einem Senderalphabet $S=(s_1,s_2,\ldots,s_n)$ 
  und einer Verteilung $p_S=(p(s_1),p(s_2),\ldots,p(s_n))$, 
  ein \indexthis{Empf"anger} mit einem 
  mit einem Empf"angeralphabet $E=(e_1,e_2,\ldots,e_m)$ 
  und einer Verteilung $p_E=(p(e_1),p(e_2),\ldots,p(e_m))$
  und einem "Ubertragungskanal, der definiert wird durch eine Matrix
  von bedingten Wahrscheinlichkeiten $p_K=((p(e_i|s_j)))$.
}
% -----------------------------------------------------------------------
\remark:{
  Bei dieser Definition einer "Ubertragungseinrichtung ist die Verteilung
  des Empf"angers durch Senderverteilung und Kanalmatrix in folgender 
  Weise vollst"andig bestimt:
  \[
    p(e_i)=\sum_{j=1}^n p(e_i|s_j) \cdot p(s_j)
  \]
}
% -----------------------------------------------------------------------
\definition Information:{
  Gesucht ist als $I(p)$ der mittlere Informationswert eines Zeichens aus einem
  Alphabet $A$ mit der Verteilung $p_A$, bzw. als $I(p_S,p_E)$ der mittlere 
  Informationswert eines Zeichens aus dem Empf"angeralphabet $E$ mit der 
  m"oglicherweise "uber die Kanalmatrix $p_K$ von der Senderverteilung $p_S$ 
  abh"angigen Verteilung $p_E$. An beide stellt man folgende Forderungen:
  \begin{stmts}
    \item \textbf{Stetigkeit:} $I(p_A)$ und $I(p_S,p_E)$ sind f"ur die 
      Wahrscheinlichkeitswerte $\in[0,1]$ stetig.
    \item \textbf{Positivit"at:} $I(p_A)\ge 0$, $I(p_S,p_E)\ge 0$
    \item \textbf{Sicherheit:} Man nennt eine Verteilung degeneriert 
      $:\equiv$ sie ist eine Permutation von $(1,0,\ldots,0)$.
      Ist $p$ degeneriert, gilt $I(p_A)=0$, sind $p_S$ und $p_E$ beide
      degeneriert, so gilt $I(p_S,p_E)=0$.
    \item \textbf{Unsicherheit:} Maximum der Information genau dann, wenn
      $p$ bzw. $p_S,p_E$ Gleichverteilungen sind.
    \item \textbf{Additivit"at:} Sind $p_S$ und $p_E$ unabh"angige 
      Verteilungen, dann gilt $I(p_S,p_E)=I(p_S)+I(p_E)$. (bezogen auf obiges
      Modell m"usste gelten $p(e_i|s_j)=p(e_i)$)
  \end{stmts}
  Die beiden folgenden Terme erf"ullen die obigen Forderungen:
  \begin{align*}
    I(p)&=\sum_{i=1}^n p(a_i) \log_2 \frac 1 {p(a_i)} \\
    I(p_S,p_E)&=\sum_{i=1}^n \sum_{j=1}^n p(e_i|s_j)\cdot p(s_j) 
      \log_2 \frac 1 {p(e_i|s_j)\cdot p(s_j) } 
  \end{align*}
  Hierbei muss man zur Wahrung der Stetigkeit $0 \log_2 0:=0$ festlegen.
  Der Zahlenwert $I(p)$ wird als Information, \indexthis{Unsicherheit} oder
  \indexthis{Entropie} bezeichnet. Die Information $I(p)$ wird in bit 
  angegeben.
}
% -----------------------------------------------------------------------
\definition bit:{
  Die Information $I(p)$ zweier sich gegenseitig ausschliessender Ereignisse,
  die beide gleich wahrscheinlich sind, ist $1$ bit.
  (Ein bit darf nicht mit dem technischen Ausdruck Bit verwechselt. Ersteres
  ist das Z"ahlma"s der Information, w"ahrend letzteres einer Bin"arziffer
  entspricht.)
}
% -----------------------------------------------------------------------
\definition Mittlere Wortl"ange:{
  \index{Wortl"ange>mittlere}
  Sei $N_i$ die Anzahl Bits, die zur Kodierung des Zeichens $a_i$ aus
  dem Alphabet $A$ mit der Verteilung $p_A$ verwendet wird. Dann hei"st
  \[
    L(p_A):=\sum_{i=1}^n p_A(a_i) N_i
  \]
  die mittlere Wortl"ange von $A$ bez"uglich $p_A$ und den Zeichenl"angen $N_i$
}
% -----------------------------------------------------------------------
\theorem Kodierungstheorem von Shannon:
  $A$ Alphabet mit Verteilung $p_A$, $N_i$ Zeichenl"angen
  =>
{
  \index{Shannon>Kodierungstheorem von}
  Dann gilt
  \[
    I(p_A)\,\mathrm{Bit} \le L(p_A)
  \]
}
% -----------------------------------------------------------------------
\definition Datei:{
  Sei $A$ ein Alphabet. Dann ist eine Datei $(f_i):\{0,\ldots,m\}\to A$ eine 
  endliche Folge $f_i$ von Elementen von $A$.
}
% -----------------------------------------------------------------------
\definition Arten von Codes:{
  Ein Code hei"st
  \begin{itemize}
    \item \textit{pr"afixfrei} bzw. \textit{Pr"afixcode}, wenn keines 
      seiner Symbole Pr"afix eines anderen ist
    \item \textit{eindeutig} bzw. nicht \textit{ambig}, wenn es zu seiner 
      Dekodierung nur eine M"oglichkeit gibt
    \item \textit{sofort dekodierbar} wenn zu seiner Dekodierung kein Lookahead
      erforderlich ist
  \end{itemize}
  \index{Pr"afixcode}
  \index{Code>pr"afixfreier}
  \index{Code>eindeutiger}
  \index{Code>sofort dekodierbarer}
}
% -----------------------------------------------------------------------
\algorithm Erzeugung des Huffman-Codes:{
  \given Alphabet $A=(a_1,\ldots,a_n)$, Datei $(f_i)_{i=0}^m$
  
  \aim Huffman-Baum $T$, der einen optimalen pr"afixfreien bin"aren 
    Code f"ur die gegebene Datei repr"asentiert
    
  \begin{proc}
    \item Berechne f"ur jedes Zeichen $a_i$ die absolute H"aufigkeit $n(a_i)$
      in $(f_i)$
    \item Baue einen Heap $H$ auf, der die Zeichen von $A$ nach ihrer
      absoluten H"aufigkeit $n(a_i)$ ordnet (geringste zuoberst)
    \item Lege einen leeren Baum $T$ an
    \item Solange $H$ nicht leer ist
      \begin{itemize}
        \item Ist nur noch ein Zeichen $c$ in $H$ enthalten, so nimm 
          den $c$ entsprechenden Knoten in $T$ als Wurzel und beende
        \item Entferne die zwei Zeichen $c$ und $d$ mit geringster H"aufigkeit
          aus dem Heap
        \item Erzeuge ein Pseudozeichen $z$, das in $T$ Elternknoten von $c$
          und $d$ wird. Setze $n(z):=n(c)+n(d)$. F"uge $z$ mit $n(z)$ in
          $H$ ein.
      \end{itemize}
  \end{proc}
}
% -----------------------------------------------------------------------
\definition Kolmogorov-Komplexit"at:{
  Ordnet einer Datenmenge die L"ange des k"urzesten Programms, das diese
  Datenmenge generiert, als Komplexit"at zu.
}
% -----------------------------------------------------------------------
\algorithm RLE-Kompression:{
  Kodiert Wiederholungen durch Bl"ocke der Art (Anzahl,Datenelement).
}
% -----------------------------------------------------------------------
\definition Parit"at:{
  Das Parit"atsverfahren besteht im Anf"ugen eines zus"atzlichen Bits
  an eine Bitsequenz, so dass eine bestimmte Eigenschaft der
  Gesamtsequenz sichergestellt ist. Man unterscheidet
  \begin{itemize}
    \item Das Anf"ugen eines Bits so, dass die Anzahl der Einsen 
      stets gerade ist (even parity)
    \item Das Anf"ugen eines Bits so, dass die Anzahl der Einsen 
      stets ungerade ist (odd parity)
  \end{itemize}
}
% -----------------------------------------------------------------------
\definition Rechteck-Code:{
  Man denke sich eine Sequenz von $m\cdot n$ Bits ($m,n\natural$) in
  einer $m\times n$-Matrix angeordnet. Bildet man nun "uber jede Spalte
  und jede Zeile eine Parit"at, so kann man bei einem fehlerhaften Bit
  anhand der beiden falschen Parit"aten das fehlerhafte Bit erkennen
  und ``umdrehen''.
  
  Analog: \indexthis{Dreieck-Code}, \indexthis{kubische Codes}
}
% -----------------------------------------------------------------------
\remark:{
  Hat man f"ur zwei Bitsequenzen die Parit"aten $p_1$ und $p_2$ ermittelt,
  so l"asst sich mit Hilfe der Operation $p_1\text{ xor } p_2$ die
  Parit"at der gesamten Sequenz ermitteln.
}
% -----------------------------------------------------------------------
\definition Pr"ufsumme:{
  F"ur eine Datei $(f_i)_{i=1}^n$ gibt es verschiedene Varianten, Pr"ufsummen zu 
  ermitteln:
  \begin{itemize}
    \item Einfach: \[\sum_{i=1}^n f_i\]
    \item Gewichtet: \[\sum_{i=1}^n i\cdot f_i\]
  \end{itemize}
}
% -----------------------------------------------------------------------
\definition Verlustbehaftete Kompression:{
  \index{Kompression>verlustbehaftete}
  Man spricht von verlustbehafteter Kompression, wenn aus den komprimierten
  Daten die Originaldaten nicht vollst"andig identisch rekonstruiert werden
  k"onnen.
}
% -----------------------------------------------------------------------
\algorithm $k$-Mittelwerte Algorithmus:{
  \given Ein Vektorraum $\VR V$ mit Abstandsfunktion 
    $d(x,y):\VR V\times\VR V\to\SetR$, $v_1,\ldots,v_n\in\VR V$, $k\natural$
    
  \aim Ein Codebuch $C=\{\mu_1,\ldots,\mu_k\}$, so dass eine Abbildung 
    $\phi:V\to\{0,\ldots,k\}$ gefunden werden kann, f"ur die
    \[
      \sum_{i=1}^n d(\mu_{\phi(v_i)},v_i)
    \]
    einem lokalen Minimum nahekommt.
    
  \begin{proc}
    \item Initialisiere $\mu_i$ mit beliebigen $v_j$. 
      ($i\in\{1,\ldots k\}$,$j\in\{1,\ldots n\}$)
    \item Solange Quantisierungsfehler zu gro"s
      \begin{itemize}
        \item Z"ahle je Codebuchvektor $\mu_i$ die auf diesen Vektor 
          abgebildeten Mustervektoren, setze deren Anzahl
          \[
            n[i]:=|\{j\mid \phi(v_j)=i\}|
          \]
        \item Lege einen neuen Satz von Codebuchvektoren $\mu_1',\ldots,\mu_k'$
          nach folgendem Verfahren fest:
          \[
            \mu_i':=\frac 1 {n[i]} 
              \sum_j \begin{cases} v_j & \phi(j)=i \\ 0 & \text{sonst}\end{cases}
          \]
        \item Nimm das Codebuch $(\mu_i')$ als neues Codebuch $(\mu_i)$
      \end{itemize}
  \end{proc}
}
% -----------------------------------------------------------------------
\remark:{
  Die Menge aller jeweils einem Vektor aus dem Codebuch zugeordneten 
  Vektoren nennt man Voronoi-Region.
}
% -----------------------------------------------------------------------
\para{Problemklassen}
% -----------------------------------------------------------------------
\definition Entscheidungsproblem:{
  Ein Entscheidungsproblem ist ein Problem, dessen L"osung f"ur alle
  m"oglichen Eingaben in einer Ja-Nein-Antwort besteht.
}
% -----------------------------------------------------------------------
\remark:{
  Sei $U$ die Menge aller m"oglichen Eingaben eines Entscheidungsproblems,
  und sei $L\subseteq U$ die Menge aller Eingaben, f"ur die die Antwort
  ``ja'' lautet. Dann nennt man $L$ die dem Problem entsprechende Sprache.
  Folglich kann jedes Entscheidungsproblem als ein Spracherkennungsproblem 
  aufgefasst werden. Die Begriffe ``Problem'' und ``Sprache'' k"onnen 
  weiterhin als gleichwertig aufgefasst werden.
}
% -----------------------------------------------------------------------
\definition Polynomialer Algorithmus:{
  \index{Algorithmus>polynomialer}
  Ein Algorithmus hei"st polynomial genau dann, wenn 
  seine asymptotische Laufzeit $O(T(n))$ durch ein Polynom dargestellt 
  werden kann.
  
  U.a. erf"ullen polynomiale Algorithmen die Beziehung
  \[
    T(cn)=O(T(n))\quad c\in\SetR^+
  \]
  Die Begriffe effizient bzw. machbar (tractable) sind in Bezug auf
  Algorithmen synonym zu polynomial.
}
% -----------------------------------------------------------------------
\remark:{
  Man spricht auch von einem ``polynomialen Problem'' und meint damit
  die Polynomialit"at des Algorithmus, der das Problem in den asymptotisch
  wenigsten Schritten l"ost.
}
% -----------------------------------------------------------------------
\remark:{
  Man kann zeigen, dass die meisten ``sinnvollen'' Rechnerarchitekturen
  einander in polynomialer Zeit emulieren k"onnen. Damit ist jeder 
  polynomiale Algorithmus auf jeder ``sinnvollen'' Maschine polynomial.
}
% -----------------------------------------------------------------------
\definition Reduzierbarkeit:{
  \index{Reduzierbarkeit}
  Seien $L_1\subseteq U_1$ und $L_2\subseteq U_2$ zwei Sprachen.
  Dann nennt man $L_1$ reduzierbar auf $L_2$, falls es einen
  Algorithmus $f:U_1\to U_2$ gibt, so dass gilt $f(L_1)=L_2$.
}
% -----------------------------------------------------------------------
\definition Polynomiale Reduzierbarkeit:{
  \index{Reduzierbarkeit>polynomiale}
  Seien $L_1\subseteq U_1$ und $L_2\subseteq U_2$ zwei Sprachen.
  Dann nennt man $L_1$ polynomial reduzierbar auf $L_2$, falls es einen
  polynomialen Algorithmus $f:U_1\to U_2$ gibt, so dass gilt $f(L_1)=L_2$.
}
% -----------------------------------------------------------------------
\definition Polynomiale "Aquivalenz:{
  \index{"Aquivalenz>polynomiale}
  Seien $L_1\subseteq U_1$ und $L_2\subseteq U_2$ zwei Sprachen.
  Dann nennt man $L_1$ polynomial "aquivalent zu $L_2$, falls
  $L_1$ und $L_2$ gegenseitig aufeinander polynomial reduzierbar sind.
}
% -----------------------------------------------------------------------
\theorem:
  $L_1, L_2$ Sprachen, $L_2$ polynomial erkennbar, $L_1$ auf $L_2$
  polynomial reduzierbar
  =>
{
  Dann existiert auch ein polynomialer Algorithmus f"ur die Erkennung
  von $L_1$.
}
% -----------------------------------------------------------------------
\remark:{
  Polynomiale Reduzierbarkeit und damit auch polynomiale "Aquivalenz sind
  transitive Relationen.
  
  Hat man f"ur ein Problem ein Laufzeitminimum bewiesen, und kann dieses
  Problem auf ein anderes reduziert werden, so hat auch das letztere
  das f"ur das erste bewiesene Laufzeitminimum. (Voraussetzung: 
  Aufwand des Algorithmus asymptotisch gr"o"ser als Aufwand der
  Reduktion)
}
% -----------------------------------------------------------------------
\problem Lineare Optimierung:{
  \index{Optimierung>lineare}
  Gegeben ist eine Zielfunktion $f:\SetR^n\to\SetR$ mit folgender Gestalt
  \[
    f(x):=\vec a\cdot \vec x
  \]
  wobei $\vec a=(a_1,\ldots,a_n)\in\SetR^n$ und $\vec x=(x_1,\ldots,x_n)\in U$.
  Der Wertebereich $U$, in dem sich die Werte von $x$ bewegen d"urfen,
  wird eingeschr"ankt durch Bedingungen der folgenden drei Arten
  \begin{stmts}
    \item $\vec\alpha_i \cdot \vec x \le \beta_i$, 
      wobei $\vec\alpha_i\in\SetR^n,\beta_i\real$
    \item $\vec\gamma_i \cdot\vec x = \delta_i$, 
      wobei $\vec\gamma_i\in\SetR^n,\delta_i\real$
    \item $x_i\ge 0$
  \end{stmts}
  F"ur die lineare Optimierung gibt es polynomiale Algorithmen. Weiterhin
  ist die ganzzahlige lineare Optimierung ($\vec x\in\SetN^n$) ziemlich 
  schwierig.
}
% -----------------------------------------------------------------------
\problem Problem des Philantropen:{
  $n_S$ Spender wollen an $n_E$ Empf"anger spenden, es existieren jedoch
  Beschr"ankungen der folgenden Arten:
  \begin{itemize}
    \item Ein Spender hat eine Spendeh"ochstsumme
    \item Ein Empf"anger hat eine Empfangsh"ochstsumme
    \item Ein Spender will nur einen H"ochstbetrag an einen bestimmten Spender
      richten
  \end{itemize}
  Welches ist die Verteilung, die die meisten Spendengelder flie"sen l"asst?
}
% -----------------------------------------------------------------------
\problem Zuweisungsproblem:{
  $n_S$ Spender wollen an $n_E$ Empf"anger spenden, es existieren jedoch
  Beschr"ankungen der folgenden Arten:
  \begin{itemize}
    \item Ein Spender spendet an h"ochstens einen Empf"anger
    \item Ein Spender hat eine Spendeh"ochstsumme
    \item Ein Empf"anger hat eine Empfangsh"ochstsumme
    \item Ein Spender will nur einen H"ochstbetrag an einen bestimmten Spender
      richten
  \end{itemize}
  Welches ist die Verteilung, die die meisten Spendengelder flie"sen l"asst?
}
% -----------------------------------------------------------------------
\example:{
  Die folgenden Probleme sind polynomial aufeinander reduzierbar 
  (``$\rightarrow$''):
  \begin{itemize}
    \item System verschiedener Repr"asentanten 
      $\rightarrow$ bipartites Matching
      (Jede Menge und jedes Element in der Gesamtvereinigung bekommt einen
      Knoten)
    \item Editierdistanz $\rightarrow$ K"urzester-Pfad-Problem
    \item Suche nach Dreiecken in Graphen $\rightarrow$ Multiplikation
      boolescher Matrizen
    \item Netzwerkflussproblem $\rightarrow$ Lineare Optimierung
    \item Statisches Routing $\rightarrow$ Lineare Optimierung
    \item Problem des Philantropen $\rightarrow$ Lineare Optimierung
    \item Zuweisungsproblem $\rightarrow$ Lineare Optimierung
    \item Sortieren $\leftrightarrow$ Erzeugung eines kreuzungsfreien Polygons
    \item Multiplikation symmetrischer Matrizen $\leftrightarrow$
      Multiplikation zweier beliebiger Matrizen
    \item Quadrieren einer Matrix $\leftrightarrow$
      Multiplikation zweier beliebiger Matrizen
  \end{itemize}
  Diese Aufz"ahlung umfasst alle Beispiele der Vorlesung.
}
% -----------------------------------------------------------------------
\definition Nichtdeterministischer Algorithmus:{
  \index{Algorithmus>nichtdeterministischer}
  Ein nichtdeterministischer Algorithmus, der zus"atzlich zu den
  verschiedenen ``"ublichen'' Operationen eine zus"atzliche 
  Verzweigungsoperation gestattet, deren beide Zweige gewisserma"sen
  ``gleichzeitig'' ausgef"uhrt werden. Die Ausf"uhrung eines 
  nichtdeterministischen Problems resultiert also in einer Masse von
  Einzelergebnissen, entsprechend den f"ur verschiedene Verzweigungen
  ausgew"ahlten Wegen.
  
  Als Ergebnis eines nichtdeterministischen Spracherkennungsalgorithmus
  gilt die Disjunktion aller Ergebnisse. 
  (mindestens einmal wahr $\implies$ wahr, alle falsch $\implies$ falsch)
}
% -----------------------------------------------------------------------
\definition Problemklasse P:{
  Man sagt, ein Problem sei in P $:\equiv$ es gibt f"ur das Problem einen 
  deterministischen Algorithmus mit polynomialer Laufzeit. 
}
% -----------------------------------------------------------------------
\definition Problemklasse NP:{
  Man sagt, ein Problem sei in NP $:\equiv$ es gibt f"ur das Problem einen 
  nichtdeterministischen Algorithmus mit polynomialer Laufzeit.
}
% -----------------------------------------------------------------------
\remark:{
  Offensichtlich gilt P $\subseteq$ NP.
}
% -----------------------------------------------------------------------
\definition $K$-hart:{
  Sei $K$ eine Problemklasse und $P$ ein Problem. Dann hei"st $P$ $K$-hart,
  falls jedes Problem aus $K$ polynomial auf $P$ reduzierbar ist.
}
% -----------------------------------------------------------------------
\definition $K$-vollst"andig:{
  Sei $K$ eine Problemklasse und $P$ ein $K$-hartes Problem. Dann hei"st 
  $P$ $K$-vollst"andig, falls $P$ selbst in $K$ liegt.
}
% -----------------------------------------------------------------------
\theorem:
  $L_1,L_2$ Sprachen, $L_2$ NP-vollst"andig, $L_2$ auf $L_1$ reduzierbar,
  $L_1$ in NP
  =>
{
  Dann ist auch $L_1$ NP-vollst"andig.
}
% -----------------------------------------------------------------------
\problem SAT:{
  \index{Satisfiability}
  Finde zu einer gegebenen aussagenlogischen Formel in konjunktiver Normalform 
  eine Variablenbelegung, so dass die Auswertung der Formel den Wert ``Wahr''
  ergibt.
}
% -----------------------------------------------------------------------
\theorem Satz von Cook:=>
{
  SAT ist NP-vollst"andig, denn (a) jeder von einer Turingmaschine 
  entscheidbare Algorithmus l"asst sich auf SAT zur"uckf"uhren 
  (b) SAT ist in NP.
}
% -----------------------------------------------------------------------
\problem Cliquenproblem:{
  Gegeben ist ein Graph $G$, gesucht ist eine Clique mit $k$ Elementen in $G$.
  
  Das Cliquenproblem ist NP-vollst"andig, da SAT darauf reduzierbar ist
  und es selbst in NP liegt.
  (Man erzeugt einen Graphen mit einer ``Spalte'' f"ur jede Klausel mit 
  allen Literalen dieser Klausel, erzeugt dann zwischen je zwei Variablen, 
  die nicht in der gleichen Spalte stehen, und nicht zueinander negiert
  sind, eine Kante. Dann ist jede Clique eine m"ogliche Belegung und jede
  m"ogliche Belegung eine Clique.)
}
% -----------------------------------------------------------------------
\problem 3SAT:{
  \given Eine aussagenlogische Formel in konjunktiver Normalform, bei
  der pro konjunktiv verbundenem Einzelterm nur drei Variablen vorkommen
  
  \aim Eine Belegung der vorkommenden Variablen, so dass die Auswertung
  der Formel den Wert ``Wahr'' ergibt.
  
  SAT ist auf 3SAT reduzierbar, indem man nach folgendem Schema
  neue Variablen einf"uhrt ($k>3$):
  \begin{multline*}
    (x_1 \lor x_2 \lor \ldots x_k) \equiv \\
    (x_1 \lor x_2 \lor y_1) \land (x_3\lor \neg y_1 \lor y_2) \land
      \ldots \land (x_{k-1} \lor x_k \lor \neg y_{k-3})
  \end{multline*}
  Da 3SAT selbst in NP liegt und SAT darauf reduzierbar ist, ist 3SAT
  NP-vollst"andig.
}
% -----------------------------------------------------------------------
\problem 3COL:{
  \given Ein Graph $G=(V,E)$
  
  \aim Eine Abbildung $f:V\to\{A,B,C\}$, so dass keine zwei benachbarten
    Knoten die gleiche ``Farbe'' ($A,B,C$) haben.
    
  3SAT ist auf 3COL reduzierbar, indem man nach folgendem Bild die Formel
  in einen Graphen verwandelt:

  \input info2_3col.epic
  
  Zun"achst baut man eine Konstruktion der Art von (*), um die drei 
  verf"ugbaren Farben an den drei Knoten benennen zu k"onnen. Nennen wir sie
  hier F,T und A (wie False, True und A). Dann stellen Konstruktionen wie (**)
  sicher, dass jeweils $x$ und $\overline x$ mit F oder T bzw. umgekehrt
  gef"arbt werden, indem $x$ und $\overline x$ mit A aus (*) verbunden werden.
  
  Zuletzt verunm"oglicht die Konstruktion (***) f"ur jede Klausel die Belegung 
  aller drei Variablen einer Klausel mit F. W"are dies der Fall, w"are die 
  einzige m"ogliche Belegung f"ur die mit T verbundenen Knoten in (***) die 
  Belegung A, womit die F"arbung des darunterliegenden Dreiecks unm"oglich 
  w"urde.
  
  3COL liegt offensichtlich in NP, und da 3SAT polynomial darauf reduziert 
  werden kann, ist 3COL NP-vollst"andig.
}
% -----------------------------------------------------------------------
\example Weitere NP-vollst"andige Probleme:{
  Als weitere NP-vollst"andige Probleme sind bekannt:
  \begin{itemize}
    \item Das Problem des Handlungsreisenden
  \end{itemize}
}
% -----------------------------------------------------------------------
\remark:{
  Zum tats"achlichen L"osung von NP-vollst"andigen Problemen empfiehlt sich
  eine der folgenden Vorgehensweisen:
  \begin{itemize}
    \item Algorithmus, der nur f"ur bestimmte Eingaben schnell/gut/"uberhaupt
      funktioniert 
    \item \textbf{\indexthis{Backtracking}} Arbeite solange vorw"arts, bis 
      es nicht mehr weitergeht, geh' dann nur einen Schritt zur"uck 
      (verwirf nicht gleich alles), versuche dort anders zu gehen usw.
      
      Eigenschaften: In der Regel drastische Aufwandsreduktion, aber nicht
        garantiert. Es wird immer eine exakte L"osung gefunden.
    \item \textbf{\indexthis{Branch and Bound}} Backtracking, das schon 
      aufh"ort, eine Variante zu verfolgen, sobald eine bestimmte 
      Grenze/Bedingung (``boundary'') erreicht ist.
      
      Eigenschaften wie Backtracking.
    \item N"aherungsl"osungen (Beispiel: Handlungsreisender, zun"achst
      Spannbaum aufstellen, dann mittels DFS Abk"urzungen suchen)
    \item Probabilistische Algorithmen (z.B. genetische Algorithmen)
    \item Speziall"osungen f"ur Einzelf"alle
  \end{itemize}
}

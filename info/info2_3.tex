% -----------------------------------------------------------------------
\para{Parallelit"at}
% -----------------------------------------------------------------------
\definition Zeitaufwand mit mehreren Prozessoren:{
  Mit $T(n,p)$ ($p,n\natural)$ bezeichnet man den Aufwand zur L"osung eines 
  gegebenen Problems mit der Eingabegr"o"se $n$, wenn $p$ Prozessoren zur
  Verf"ugung stehen. (F"ur unterschiedliche $p$ kommen gegebenfalls 
  unterschiedliche Algorithmen zum Einsatz.)
}
% -----------------------------------------------------------------------
\definition Speedup:{
  Den Speedup $S(n,p)$ (die Beschleunigung) durch die Verwendung von $p$
  gegen"uber einem Prozessor definiert man durch folgende Gleichung:
  \[
    S(n,p):=\frac{T(n,1)}{T(n,p)}
  \]
  Den Speedup nennt man optimal \index{Speedup>optimaler} $:\equiv$ $S(n,p)=p$.
}
% -----------------------------------------------------------------------
\definition Effizienz:{
  Die Effizienz $E(n,p)$ eines parallelen Algorithmus gibt die durchschnittliche
  Auslastung eines Prozessors an:
  \[
    E(n,p):=\frac {S(n,p)} p = \frac{T(n,1)}{pT(n,p)}
  \]
  Die Effizienz nennt man optimal \index{Effizienz>optimale} $:\equiv$ $E(n,p)=1$.
}
% -----------------------------------------------------------------------
\remark:{
  In der Regel gilt
  \[
    T(n,p)=X \equiv T(n,p/k)\approx k X
  \]
}
% -----------------------------------------------------------------------
\definition Parallelrechnerarchitekturen:{
  Man unterteilt Parallelrechner nach verschieden Kriterien in Klassen, 
  darunter:
  \begin{description}
    \item[SISD] Eine Anweisung wird auf ein Datenwort angewendet.
    \item[MISD] Mehrere Anweisungen werden parallel auf ein Datenwort angewendet.
    \item[SIMD] Eine Anweisung wird auf parallel auf mehrere Datenworte angewendet.
    \item[MIMD] Mehrere Anweisungen werden parallel auf mehrere Datenworte angewendet.
  \end{description}
  Die bekanntesten Parallelarchitekturen sind:
  \begin{itemize}
    \item Mehrkopf-Turing-Maschine
    \item Schaltkreis
    \item systolischer/zellul"arer Automat (Transputer, Connection Machine)
    \item Rechnernetzwerk
    \item PRAM (parallel random access machine), shared memory machine
  \end{itemize}
  Hierunter bilden die PRAMs die wichtigste/h"aufigste Klasse. Hier 
  unterscheidet man noch nach den F"ahigkeiten der 
  Speicherzugriffsarchitektur:
  \begin{description}
    \item[EREW] exclusive read exclusive write 
    \item[CREW] concurrent read exclusive write 
    \item[ERCW] exclusive read concurrent write 
    \item[CRCW] concurrent read concurrent write 
  \end{description}
  Im Falle von concurrent write wird ein Vereinheitlichung bez"uglich
  des Schreibergebnisses ben"otigt, z.B.:
  \begin{itemize}
    \item Es z"ahlt das erste geschriebene Wort
    \item Es z"ahlt das letzte geschriebene Wort
    \item Es z"ahlt ein zuf"alliges geschriebenes Wort
    \item Es z"ahlt das Wort vom Prozessor mit der kleinsten/gr"o"sten Nummer
    \item Alle geschriebenen Worte m"ussen gleich sein
  \end{itemize}
}
% -----------------------------------------------------------------------
\subsection{Algorithmen f"ur shared memory machines}
% -----------------------------------------------------------------------
\algorithm Parallele Addition:{
  \given Zwei bin"are Zahlen mit $n$ Ziffern, $a=a_{n-1}\ldots a_0$,
     $b=b_{n-1}\ldots b_0$,$u\in\{0,1\}$, $n=2^m$, $m\natural$,
     eine EREW-Maschine mit $O(n)$ Prozessoren
  
  \aim $c=a+b+u=c_n\ldots c_0$
  
  \begin{proc}
    \item Basisfall: $n\le\text{Maschinenwortl"ange}$ Berechne Ergebnis 
      unmittelbar.
    \item Sei $k:=n/2$. Berechne parallel rekursiv
      \begin{align*}
        l&=a_{n-1}\ldots a_k+b_{n-1}\ldots b_k \\
        l'&=a_{n-1}\ldots a_k+b_{n-1}\ldots b_k+1 \\
        r&=a_{k-1}\ldots a_0+b_{k-1}\ldots b_0+u 
      \end{align*}
    \item Setze $v:=$ "Ubertrag bei $r$.
    \item W"ahle Ergebnis $r$
      \[
        r:=\begin{cases}
          lr & v=0 \\
          l'r & v=1 \\
        \end{cases}
     \]
  \end{proc}
  
  \cpx $O(\log n)$ 
}
% -----------------------------------------------------------------------
\definition Statischer Algorithmus:{
  \index{Algorithmus>statischer}
  Ein paralleler Algorithmus hei"st statisch genau dann, wenn die
  Verteilung der Rechenschritte auf die Prozessoren im Voraus 
  festgelegt ist.
}
% -----------------------------------------------------------------------
\theorem Satz von Brent:
  $A$ statischer EREW-Algorithmus, $T(n,p)=O(t)$, $s$ gesamte Anzahl der 
  Rechenschritte
  =>
{
  \label{the:brent}
  Dann existiert auch ein EREW-Algorithmus $B$ mit einem Aufwand $T(n,s/t)=O(t)$.
}
% -----------------------------------------------------------------------
\remark:{
  \ref{the:brent} l"asst sich folgenderma"sen interpretieren:
  Wenn der parallele Algorithmus nicht mehr Schritte als der sequentielle macht,
  dann ist die optimale Effizienz erreichbar.
}
% -----------------------------------------------------------------------
\algorithm Maximumsuche I:{
  \given: $a_1,\ldots,a_n$ aus einer angeordneten Menge, $n\natural$, 
    CRCW-Maschine mit $n(n-1)/2$ Prozessoren, $n=2^m$, $m\natural$
  
  \aim $\max\{a_1,\ldots,a_n\}$
  
  \begin{proc}
    \item Belege Variablen $v_1,\ldots,v_n:=1$, $m$
    \item Prozessor $(i,j)$: 
      \begin{itemize}
        \item Sei o.B.d.A $a_i<a_j$ $\rightarrow$ $v_i:=0$. 
        \item Ist $v_j=1$, so setze $\max:=a_j$.
      \end{itemize}
    \item Ist $v_j=1$, so schreibe $m:=j$
    \item Index des Maximums findet sich in $m$
  \end{proc}
  
  \cpx $O(1)$, $E(n,n(n-1)/2)=O(1/n)$
}
% -----------------------------------------------------------------------
\algorithm Maximumsuche II:{
  \given: $a_1,\ldots,a_n$, $n\natural$, CRCW-Maschine mit $n$ Prozessoren,
    $n=2^m$, $m\natural$
  
  \aim $\max\{a_1,\ldots,a_n\}$
  
  \begin{proc}
    \item Setze $g:=2$ als Gr"o"se der zu bearbeitenden Gruppen, 
      setze die Zahl der Kandidaten $k:=n$
    \item Solange $k\ne 1$
      \begin{itemize}
        \item Finde mit Maximumsuche I das Maximum in jeder Gruppe der 
          Gr"o"se $g$.
        \item Nun sind noch $k/2$ Kandidaten "ubrig. Setze also $k:=k/2$.
          Setze $g:=g^2$.
          Dann ist 
          \[
            g\cdot\left(\frac k g\right)^2 <n
          \]
          Also gen"ugt die gegebene Anzahl der Prozessoren noch immer, um
          in $O(1)$ die Maxima jeder Gruppe zu finden.
      \end{itemize}
  \end{proc}
  
  \cpx $O(\log\log n)$, $E(n,n)=O(1/\log\log n)$
}
% -----------------------------------------------------------------------
\algorithm Parallel Prefix:{
  \given $a_1,\ldots,a_n$, $\circ$ ein beliebiger assoziativer Operator,
    eine EREW-Maschine mit $n$ Prozessoren, $n=2^m$, $m\natural$
  
  \aim $a_1$,$a_1\circ a_2$,$\ldots$,$a_1\circ\cdots\circ a_n$
  
  \begin{proc}
    \item $n=2$ $\rightarrow$ trivial, ansonsten
    \item Setze $a_{2i}:=a_{2i-1}\circ a_{2i}$, $i=1\ldots n/2$
    \item Benutze Parallel Prefix mit $\{a_{2i}\mid i=1\ldots n/2 \}$
    \item Setze $a_{2i-1}:=a_{2i-2}\circ a_{2i-1}$, $i=2\ldots n/2$
  \end{proc}
  
  \cpx $O(\log n)$, $E(n,n)=O(1/\log n)$. Gesamtzahl der Schritte ist 
    $O(n)$. Nach dem Satz von Brent kann nun ein Algorithmus gefunden 
    werden, der ebenfalls in $O(\log n)$ Schritten und mit nur $O(n/\log n)$ 
    Prozessoren zum Ziel kommt.
}
% -----------------------------------------------------------------------
\algorithm Rang von Listenelementen:{
  \given $a_1,\ldots,a_n$ verkettete Liste ($a_i$ sei Index des n"achsten
    Elements), eine EREW-Maschine mit $n$ Prozessoren
    
  \aim $r_1,\ldots,r_n$, wobei $r_i=$Abstand vom Listenende
  
  \begin{proc}
    \item Belege $n_1,\ldots,n_n$ als Index des am weitesten entfernten
      bekannten Nachfolgers
    \item Prozessor $i$ mit $i=1,\ldots,n$:
      \begin{itemize}
        \item $n_i:=a_i$, $r_i:=0$
        \item Ist $n_i=NIL$, setze $r_i:=1$.
        \item Setze $d:=1$ als Verdopplungsz"ahler
        \item Solange $r_i=0$
          \begin{itemize}
            \item Ist $r_{n_i}=0$, so setze $n_i:=n_{n_i}$, $d:=2\cdot d$.
              Ansonsten setze $r_i:=r_{n_i}+d$.
          \end{itemize}
      \end{itemize}
  \end{proc}
  
  \cpx $O(\log n)$, $E(n,n)=O(1/\log n)$. Die von diesem Algorithmus 
    verwandte Technik nennt sich die \indexthis{Verdoppelungsmethode}.
}
% -----------------------------------------------------------------------
\trick Verfahren des Euler'schen Weges:{
  Man kann den obigen Rang-Algorithmus leicht f"ur B"aume modifizieren, 
  indem man eine DFS-artig verzeigerte Liste von Knoten bereith"alt.
}
% -----------------------------------------------------------------------
\subsection{Algorithmen f"ur Netzwerke}
% -----------------------------------------------------------------------
\definition Netzwerk:{
  Ein Rechnernetzwerk wird dargestellt als ein Graph $G=(V,E)$, wobei
  die Knoten Prozessoren/Recheneinheiten und die Kanten 
  Kommunikationsverbindungen darstellen.
}
% -----------------------------------------------------------------------
\definition Durchmesser eines Netzwerks:{
  Der Durchmesser eines Netzwerks ist das Maximum "uber die L"ange aller
  k"urzesten Verbindungen zwischen zwei Knoten.
}
% -----------------------------------------------------------------------
\remark Beliebte Netzwerk-Topologien:{
  Folgende Topologien finden breitere Anwendung:
  \begin{itemize}
    \item Kette, Kreis
    \item Raster (gro"ser Durchmesser), Torus
    \item Baum (geringer Durchmesser, daf"ur mit Flaschenhals)
    \item Hyperw"urfel
  \end{itemize}
}
% -----------------------------------------------------------------------
\algorithm Odd-even transposition sort:{
  \given $a_1,\ldots,a_n$ aus einer geordneten Menge, Kette von $n$ Prozessoren
  
  \aim Sortierte Folge
  
  \begin{proc}
    \item Verteile $a_i$ an Prozessor $i$ als $p_i$ ($i=1,\ldots,n$)
    \item Mache $n$-mal mit Prozessor $i$ ($i=1,\ldots,n$)
      \begin{itemize}
        \item Vergleiche $p_i$ mit $p_{i-1}$, wenn falsche Reihenfolge, tausche
        \item Vergleiche $p_i$ mit $p_{i+1}$, wenn falsche Reihenfolge, tausche
      \end{itemize}
  \end{proc}
  
  \cpx $O(n)$ 
}
% -----------------------------------------------------------------------
\algorithm Paralleler Mergesort:{
  Zu kompliziert, um nur verbal beschrieben zu werden. Grundidee:
  Wenn man zwei aufsteigend sortierte Felder $a,b$ in gerade $e_a,e_b$
  und ungerade indizierte H"alfte $o_a,o_b$ zerlegt, dann $e_a$ und $e_b$ 
  zu $e$ sowie $o_a$ $o_b$ zu $o$ ``merget'', erh"alt man einen 
  Gesamt-``Merge'' dadurch, dass man jeweils indizierte Elemente aus 
  $o_{i+1}$ und $e_i$ miteinander vergleicht.
  
  \cpx $O(\log^2 n)$ bei $O(n \log^2 n)$ Prozessoren $\implies$ 
    $E(\log^2 n,n\log^2 n)=O(1/\log^3 n)$
}
% -----------------------------------------------------------------------
\algorithm $k$-kleinstes Element:{
  \given $a_1,\ldots,a_n$ aus einer angeordneten Menge, $n$ Prozessoren
    mit allen Verbindungen, die f"ur einen vollst"andigen bin"aren Baum 
    erforderlich sind, $k\in\{1,\ldots,n\}$
    
  \aim $k$-kleinstes Element
  
  \begin{proc}
    \item Ordne jedem Prozessor $p_i$ ein Datenelement $a_i$ fest zu
      ($i=1,\ldots,n$)
    \item W"ahle durch zuf"allige Aufw"artspropagation ein Pivotelement $p$,
      ber"ucksichtige dabei Chancenverschiebung durch eliminierte Elemente
    \item Propagiere das gew"ahlte Pivotelement abw"arts
    \item Propagiere die Anzahl der Elemente, die kleiner sind, aufw"arts
    \item Propagiere den Rang $r$ des Pivot abw"arts
    \item Ist $r<k$, eliminiere alle Elemente $a_i\le p$, ansonsten alle $a_i>p$.
  \end{proc}
  
  \cpx $O(\log n)$. Indem man in jedem Schritt eine neue Pivotwahl initiiert,
    kann man die Laufzeit auf $O(\log^2 n/\log \log n)$ senken.
}
% -----------------------------------------------------------------------
\algorithm Matrixmultiplikation:{
  \given $A=((a_{ij}))$, $B=((b_{ij}))$, $A,B\in\SetR^{n\times n}$, ein
    Gitter-Netzwerk mit Wrap-Around (Torus)
  
  \aim $C=AB=((c_{ij}))$
  
  \begin{proc}
    \item Belege f"ur jeden Prozessor $P_{ij}$ ein $a'_{ij}:=a_{i,i+j}$ und
      ein $b'_{ij}:=b_{i+j,j}$ sowie $c_{ij}=0$. ($i=1,\ldots,n$)
    \item Mache $n$-mal:
      \begin{itemize}
        \item Setze f"ur jeden Prozessor $P_{ij}$:
          $c_{ij}:=c_{ij}+a'_{i,j}\cdot b'_{ij}$ ($i=1,\ldots,n$)
        \item Schiebe die $((a'_{ij}))$ eine Spalte nach links
        \item Schiebe die $((b'_{ij}))$ eine Zeile nach oben
      \end{itemize}
  \end{proc}
  
  \cpx $O(n)$, $E(n,n^2)=O(1)$
}
% -----------------------------------------------------------------------
\remark Matrix-Vektor-Produkt:{
  Analog kann mit einer Zeile von $n$-Prozessoren ein Matrix-Vektor-Produkt
  implementiert werden.
}
% -----------------------------------------------------------------------
\algorithm Minimale Editierdistanz:{
  \index{Editierdistanz>Minimale}
  \given Zwei Zeichenketten $a_1,\ldots,a_n/b_1,\ldots,b_n$, $c_i,c_s,c_d$
    Kosten der Editierschritte \textit{insert}, \textit{substitute},
    \textit{delete}, eine Prozessorenkette mit $2n-1$ Prozessoren
    
  \aim Minimale Kostensumme einer Editiersequenz, die $(a_i)$ in $(b_i)$ umformt
  
  \begin{proc}
    \item Bereite zum rechtsseitigen Einschieben vor:
      $\leftarrow$ $a_1,c_d,a_2,2c_d,\ldots,a_n$
    \item Bereite zum linksseitigen Einschieben vor: 
      $b_n,\ldots,2c_i,b_2,c_i,b_1$ $\rightarrow$
    \item Setze $c_i:=0$ und $d_i:=0$ f"ur jeden Prozessor ($i=1,\ldots,2n-1$)
    \item F"ur jeden Prozessor $P_i$ f"uhre $4n$-mal folgenden Schritt durch:
      \begin{itemize}
        \item Kommen von links und rechts $a_j$,$b_{n+i-j}$, vergleiche diese,
	  merke $d_i=0$ (gleich) bzw. $d_i=c_s$ (ungleich) und gib $a_j$ nach
	  links, $b_{n+i-j}$ nach rechts weiter
	\item Kommen von links Editierkosten $f,g$ , so setze 
	  $c_i:=\min\{d_i,f,g\}$ und gib $c_i$ nach links und rechts weiter
      \end{itemize}
    \item Die Zahl, die auf beiden Seiten nach dem letzten Zeichen die Kette
      verl"asst, gibt die minimalen Editierkosten an.      
  \end{proc}
}

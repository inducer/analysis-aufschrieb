\documentclass[10pt,a4paper]{article}
\usepackage[german]{babel}
\usepackage{amsmath}
\usepackage{amssymb}

% -----------------------------------------------------------------------------
% old compatibility crap
% -----------------------------------------------------------------------------

\input macros

\let\after=\circ
\let\ident=\equiv
\let\equiv=\Leftrightarrow

\def\ffa{\text{ f.f.a. }}
\def\epsn{(\forall \epsilon>0)}
\def\tocond#1{\buildrel #1 \over \longrightarrow}
\def\xnull{{x_0}}
\def\cont{\in\SetCont(D)}
\def\div{\operatorname{div}}

\def\VRL{\mathcal{L}}
\def\LSet#1{{\rm #1}}
\def\SetCont{\LSet C}
\def\SetBV{\LSet{BV}}
\def\SetQuad#1{R_{#1}}
\def\SetLMeas{\cal L}
\def\SetFLMeas{\cal L}
\def\SetLInt{\LSet L}
\def\SetRInt{\LSet R}
\let\MBSet=\mathbb
\def\SetNN{\SetN_0}

\def\natural{\in\SetN}
\def\nnatural{\in\SetNN}
\def\integer{\in\SetZ}
\def\rational{\in\SetQ}
\def\real{\in\SetR}
\def\complex{\in\SetC}
\def\defined{\in D}

\let\othersum=\sum
\let\otherint=\int
\def\int{\otherint\limits}
\def\sum{\othersum\limits}

\def\VR#1{#1}
\def\nullvec{o}
\def\transpose#1{{#1}^T}
\def\unitgroup#1{#1^\times}
\def\ntuple#1{#1_1,\ldots,#1_n}
\def\lcoset#1#2{\lower2pt\hbox{$#2$}\backslash #1}
\def\rcoset#1#2{#1/\lower2pt\hbox{$#2$}}
\def\combover#1#2{\begin{pmatrix}#1\\#2\end{pmatrix}}
\def\conjug#1{\overline{#1}}
\def\cal#1{\mathcal{#1}}
\def\divides{\big|}
\def\bigcupd{\biguplus}
\def\normdiv{\trianglelefteq}
\def\dderivable{\Rightarrow}
\def\derivable{\Rightarrow^*}
\def\yes{$\surd$}
\def\no{$\times$}

% statements ------------------------------------------------------------------
\newenvironment{stmts}{
  \begin{list}{(\thestmtid)}{\usecounter{stmtid}}
  }{
  \end{list}
  }
\newenvironment{displaytext}{
  \begin{quote}
  }{
  \end{quote}
  }


\begin{document}
\def\op#1{{\itshape #1:}}

{\huge\bfseries\center Info I-Zusammenfassung}




% -----------------------------------------------------------------------
\para{Grundlagen}
% -----------------------------------------------------------------------

\definition Algorithmus:{
  Ein Algorithmus wird gekennzeichnet durch die folgenden Eigenschaften:
  \begin{itemize}
    \item Schrittweise Ausf"uhrung
    \item Determiniertheit
    \item Eindeutigkeit
    \item Jeder Schritt bewirkt eine "Anderung
    \item Endlichkeit jedes einzelnen Schrittes, aber nicht unbedingt des
      Algorithmus
    \item Endlichkeit der Beschreibung
  \end{itemize}
}




\definition Von-Neumann-Rechner:{
  Ein \textit{Von-Neumann-Rechner} ist ein speicherprogrammierter Rechner.
}




% -----------------------------------------------------------------------
\para{Grammatiken}
% -----------------------------------------------------------------------

\definition Alphabet:{
  Ein Alphabet $V$ ist eine endliche Menge von Zeichen.
}




\definition Zeichenfolge:{
  Eine Zeichenfolge $f$ ist eine endliche, geordnete Menge von Zeichen aus dem 
  Alphabet. $\epsilon$ bezeichnet man die leere Zeichenkette.
  
  \textit{Schreibweise}: $a^n:=aa^{n-1},a^0:=\epsilon,a\epsilon=a$
  
  $x\in V, a\text{ Zeichenfolge}: |xa|:=|a|+1, |\epsilon|:=0$
  
  $V^*$ bezeichnet die Menge der Zeichenfolgen "uber $V$, 
  einschlie"slich $\epsilon$.
}




\definition Grammatik:{
  Eine \textit{Grammatik} ist ein Quadrupel von Mengen $G=(V_T,V_N,P,S)$. 
  Dabei nennt man $V_T$ die Menge der terminalen Symbole, $V_N$ die
  Menge der nicht-terminalen Symbole, $P$ die Menge der Produktionen und $S$
  das Startsymbol.
  
  Es muss gelten: $V_T \cap V_N=\emptyset$, $S\in V_N$. Weiterhin bezeichnet
  man mit $V=V_T\cup V_N$ das Vokabular der Grammatik. 
  
  F"ur alle Elemente $p\in P$ gilt: $p=l\to r$ mit zwei Zeichenfolgen $l$ 
  und $r$, f"ur die gilt: $l=a\alpha b$ mit $a,b\in V^*,\alpha\in V_N$,
  $r\in V^*$.
}




\definition Ableitung:{
  Eine Zeichenfolge $\beta$ ist eine \textit{direkte Ableitung} einer
  Zeichenkette $\alpha$ ($\alpha\to\beta$), wenn $\alpha=w_1lw_2$,
  $\beta=w_1rw_2$ und $l\to r$ eine der Produktionen der Grammatik ist.
  
  Eine \textit{Ableitung} ist dann eine beliebige (endliche) Aneinaderreihung
  von direkten Ableitungen.
}




\definition Satzform/Phrase:{
  Eine Zeichenfolge $\alpha$ ist eine \textit{Satzform/Phrase} einer 
  Grammatik, wenn $\alpha$ eine Ableitung des Startsymbols ist.
}




\definition Sprache:{
  Die Sprache $L(G)$ einer Grammatik $G$ besteht aus der Menge der Phrasen
  der Grammatik, die nur aus Terminalsymbolen bestehen.
}




\definition Chomsky-Klasse:{
  Eine Grammatik geh"ort einer der folgenden \textit{Chomsky-Klassen}
  an, falls die entsprechende Bedingung in der Tabelle zutrifft.
  
  \begin{tabular}{|l|c|l|}
    \hline
    \textbf{Chomsky-Klasse} & \textbf{Produktion} & \textbf{Bedingung} \\
    \hline
    0 (``unbeschr"ankt'') & $l\to r$ & $l,r\in V^*$  \\
    1 (``beschr"ankt'') & $l\to r$ & $l,r\in V^*,1\le|l|\le|r|$ \\
    1 (``kontextsensitiv'') & $ulv\to urv$ & $l\in V_N,u,r,v\in V^*,r\ne\epsilon$ \\
    2 (``kontextfrei'') & $l\to r$ & $l\in V_N,r\in V^*$ \\
    3 (``linkslinear'',``regul"ar'') & 
      $\stack(l\to r;l\to ur)$ & $l,u\in V_N,r\in V_T^*$ \\
    3 (``rechtslinear'',``regul"ar'') & 
      $\stack(l\to r;l\to ru)$ & $l,u\in V_N,r\in V_T^*$ \\
    \hline
  \end{tabular}
  
  Bei $C_1$-Grammatiken ist zus"atzlich $S\to\epsilon$ eine erlaubte 
  Produktion.(Das Startsymbol $S$ darf dann allerdings in keiner Expansion 
  einer Produktion vorkommen.)
  
  Bei $C_2$-Grammatiken ist es m"oglich, Produktionen der Form $l\to\epsilon$
  zu eliminieren.
  
  Es gilt: $C_3\subseteq C_2\subseteq C_1\subseteq C_0$. Zur Zerteilung 
  von $C_2$- und $C_3$-Sprachen gibt es effiziente Algorithmen.
}

% -----------------------------------------------------------------------
\para{Boole'sche Algebra}
% -----------------------------------------------------------------------

\definition Boole'sche Algebra:{
  Eine Menge $B$ mit zwei Verkn"upfungen $\land$ und $\lor$ hei"st 
  \textit{Boole'sche Algebra}, falls die folgenden Axiome gelten:
  \begin{stmts}
    \item $(\forall a,b \in B)(b \lor a = a\lor b)$ (\emph{Kommutativit"at $\lor$})
    \item $(\forall a,b \in B)(b \land a = a\land b)$ (\emph{Kommutativit"at $\land$})
    \item $(\forall a,b,c \in B)(a \lor (b \lor c)=(a \lor b)\lor c)$ (\emph{Assoziativit"at $\lor$})
    \item $(\forall a,b,c \in B)(a \land (b \land c)=(a \land b)\land c)$ (\emph{Assoziativit"at $\land$})
    \item $(\forall a,b \in B)(a\land(a \lor b)=a)$ (\emph{Verschmelzung $\lor$})
    \item $(\forall a,b \in B)(a\lor(a \land b)=a)$ (\emph{Verschmelzung $\land$})
    \item $(\forall a,b,c \in B)(a\lor(b\land c)= (a\lor b) \land (a\lor c))$ (\emph{Distributivit"at $\lor$})
    \item $(\forall a,b,c \in B)(a\land(b\lor c)= (a\land b) \lor (a\land c))$ (\emph{Distributivit"at $\land$})
    \item $(\forall a \in B)(a\lor 0=a)$ (\emph{Neutralelement $\lor$})
    \item $(\forall a \in B)(a\land 1=a)$ (\emph{Neutralelement $\land$})
    \item $(\forall a \in B)(a\lor (\neg a)=a)$ (\emph{Inverses $\lor$})
    \item $(\forall a \in B)(a\land (\neg a)=0)$ (\emph{Inverses $\land$})
    \end{stmts}
}




\definition Implikation/"Aquivalenz/Antivalenz:{
  \begin{stmts}
    \item $a\implies b :\equiv \neg a \lor b$ (\emph{Implikation})
    \item $a\equiv b   :\equiv (\neg a \land \neg b)\lor(a  \land b)$ (\emph{"Aquivalenz})
    \item $a\equiv b   :\equiv (a \land \neg b)\lor(\neg a  \land b)$ (\emph{Antivalenz})
    \end{stmts}
 }




\theorem Weitere Regeln:=>{
  \begin{stmts}
    \item $a=\neg\neg a$ (\emph{Doppelte Negierung})
    \item $a\land a=a$ (\emph{Idempotenz $\land$})
    \item $a\lor a=a$ (\emph{Idempotenz $\lor$})
    \item $\neg(a\lor b)=\neg a \land \neg b$ (\emph{DeMorgan'sche Regel $\lor$})
    \item $\neg(a\land b)=\neg a \lor \neg b$ (\emph{DeMorgan'sche Regel $\land$})
    \item $(a\implies b)\land (b\implies c) \implies a\implies c$ (\emph{Transitivit"at})
    \end{stmts}
}




\definition Tautologie:{
   Eine \textit{Tautologie} ist eine Aussage, die immer zutrifft.
}




% -----------------------------------------------------------------------
\para{Induktion}
% -----------------------------------------------------------------------

Verschiedene Induktionsverfahren (z.z. Aussage: $A(n) \forall n$)
\begin{itemize}
  \item Zeige: $A(1)$, $A(n)\implies A(n+1)$
  \item Zeige: $A(1)$, $A(n-1)\implies A(n)$
  \item Zeige: $A(1)$, $(\forall k\natural)(k<n \implies A(k))\implies A(n)$ 
     (``starke'' Induktion)
  \item Zeige: $A(1),A(2)$, $A(n-2)\implies A(n)$
  \item Zeige: $A(1)$, $A(\frac n 2)\implies A(n)$ (Aussage gilt dann aber 
    nur f"ur $n=2^k$)
  \item Sei $M_A\subseteq\SetN$ unendlich. Zeige $(\forall n\in M_A)(A(n))$ und
    $A(n)\implies A(n-1)$
\end{itemize}

Tricks bei Induktionsbeweisen:
\begin{itemize}
  \item Aussage zun"achst nicht "uber Ausdruck selbst, sondern nur "uber 
    sein Wachstum
  \item Keine Eile! Versuche evtl. verschachtelte Induktionen. Teile eine 
    schwierige Behauptung in mehrere einfache Teile!
  \item Hole alles aus der Induktionsvoraussetzung heraus! 
    (Benutze sie evtl. mehrfach in versch. Situationen)
  \item Induktion "uber mehrere Parameter: Wichtig ist eine geschickte Wahl
    des Induktionsparameters (vielleicht auch eine Kombination?)
  \item Benutze evtl. starke Induktion, zeige nur, da"s die Behauptung
    aus irgendeinem der vorhergehenden F"alle folgt.
  \item Zwei F"alle: Lohnt es sich, beide gemeinsam zu betrachten, so
    da"s sie beide aufeinander aufbauen?
  \item Kann die Induktionsvoraussetzung verst"arkt werden (so da"s ein
    st"arkerer Beweis entsteht, der dennoch einfacher ist)? 
  \item Gilt der Schlu"s f"ur {\itshape wirklich} alle $n$?
  \item Falls anwendbar: Beziehe dich vielleicht nicht auf die ``ersten'', sondern
    auf die ``letzten'' Elemente der Behauptung.
\end{itemize}

\definition Schleifeninvariante:{
  Eine Bedingung, die zu Schleifenbeginn, nach jeder Ausf"uhrung der Schleife
  und an deren Ende wahr ist, nennt sich {\itshape Schleifeninvariante}.
}




% -----------------------------------------------------------------------
\para{O-Kalk"ul}
% -----------------------------------------------------------------------

\definition O-Kalk"ul:{
  Seien $f(n),g(n):\SetN\to\SetN$ zwei Funktionen. Dann gilt:
  \begin{align*}
    f(n)=O(g(n)) &:\equiv 
      (\exists n_0,c\natural)(\forall n>n_0)(\frac{f(n)}{g(n)}\le c) \\
    f(n)=\Omega(g(n)) &:\equiv 
      (\exists n_0,c\natural)(\forall n>n_0)(\frac{f(n)}{g(n)}\ge c) \\
    f(n)=\Theta(g(n)) &:\equiv 
      f(n)=O(g(n)) \land f(n)=\Omega(g(n)) \\
    f(n)=o(g(n)) &:\equiv 
      \limn \frac{f(n)}{g(n)}=0 \\
    f(n)=\omega(g(n)) &:\equiv 
      \limn \frac{g(n)}{f(n)}=0 \equiv g(n)=o(f(n)) \\
  \end{align*}
}

\theorem: $c>0,a>1,f(n) \nearrow$=>{\begin{align*}
  f^c(n)&=O(a^{f(n)}) \\
  f^c(n)&=o(a^{f(n)})
\end{align*}}

\theorem: $f(n)=O(s_f(n)),g(n)=O(s_g(n))$=>{\begin{align*}
  f(n)+g(n)&=O(s_f(n)+s_g(n)) \\
  f(n)\cdot g(n)&=O(s_f(n)\cdot s_g(n)) 
\end{align*}}




% -----------------------------------------------------------------------
\para{ Rekurrenzrelationen }
% -----------------------------------------------------------------------

Tricks f"ur Rekurrenzen:
\begin{itemize}
\item Sei $T(g(n))=E(T(n),n)$ eine gegebene Rekurrenz. Wir wollen zeigen,
  da"s $T(n)\le f(n)$. Dann mu"s gezeigt werden:
  \[  f(g(n))\ge E(f(n),n) \]
\end{itemize}

Tricks zum Finden von geschlossenen Ausdr"ucken:
\begin{itemize}
  \item Anfang hinschreiben (ausmultipliziert, nicht ausmultipliziert, 
    faktorisiert, ...)
  \item Summenindex verschieben, voneinander abziehen, dann z.B. Differenz wie
    $2T(n)-T(n)$ betrachten.
  \item Nimm vielleicht zun"achst an, da"s $n$ eine bestimmte Form hat:
    z.B. $n=2^k$
  \item Rekurrenzen: Gibt es eine charakteristische Gleichung? 
    (in der Art von $a_{n+1}^2=a_n+c$)?
  \item Wachstum betrachten: Was sagt $T(n+1)-T(n)$?
  \item ``Full-history''? Eliminiere die Geschichte mittels geschickter 
    Subtraktion.
\end{itemize}

\example ``Teile-und-Herrsche''-Relationen:{
  Ein rekursiver Algorithmus zerteile bei der Bearbeitung eines Problems 
  der Gr"o"se $n$ das Problem in $a$ Teilprobleme der Gr"o"se $\frac 1 b$.
  Die Kombination der L"osungen der Teilprobleme zu der des Gesamtproblems
  ben"otige $c\cdot n^k$. Damit gilt:
  \[T(n)=aT\left(\frac n b\right)+c\cdot n^k\]
  Als Absch"atzung hierf"ur kann verwendet werden:
  \[
    T(n)=
      \begin{cases}
        O(n^{\log_b a)} & \text{falls $a>b^k$} \\
        O(n^k \log n) & \text{falls $a=b^k$} \\
        O(n^k) & \text{falls $a<b^k$} \\
      \end{cases}
  \]
}




% -----------------------------------------------------------------------
\para{Datenstrukturen}
% -----------------------------------------------------------------------

\definition Abstrakter Datentyp:{
  Template auf einen Typ, gekoppelt mit bestimmten Anforderungen, z.B. 
  Wohlordnung
}




\definition Multiset:{
  Eine ungeordnete Struktur, die mehrere Vorkommen eines einzelnen
  Elements darstellen kann.
}




\definition Heap:{
  Baum, in dem Eltern-Elemente stets gr"o"ser als ihre Kind-Elemente sind.

  \op{Einf"ugen} Element irgendwo am Ende einf"ugen und dann so lange tauschen,
  bis Heap-Eigenschaft wiederhergestellt ist.

  \op{L"�schen der Wurzel} Wurzel weg, dann irgendein Blatt an der Wurzel 
  einf"ugen und dann so lange tauschen, bis Heap-Eigenschaft 
  wiederhergestellt ist.
}  




\definition Dictionary:{
  Eine abstrakte Datenstruktur, die effizient einf"ugen, l"oschen und suchen 
  kann.
}




\definition Binary Search Tree/BST:{
  \op{Suchen/Einf"ugen} Klar
  
  \op{L"oschen} (nur schwierig, fallls 2 Kinder vorhanden, in diesem Fall) 
    Vorg"anger/Nachfolger suchen, tauschen (Baum nicht mehr konsistent), 
    Element hat jetzt nur max. ein Kind (Def. Vorg"anger/Nachfolger!),
    L"oschen, Baum wieder konsistent.
}




\definition AVL Tree/Balanced BST:{
  Eine Seite eines (Unter-)Baumes ist h"ochstens 1 Element h"oher als die
  andere.
  
  \op{Einf"ugen} Eine (einfache oder doppelte) ``Rotation'' gen"ugt, um die
  AVL-Eigenschaft beizubehalten. Zentraler Punkt: ``Kritischer Knoten'':
  Wurzel des kleinsten Baumes, der aus dem Gleichgewicht ger"at. Hier findet
  die Rotation statt.
  
  \op{L"oschen} Offensichtlich ein bi"schen komplizierter. :-)
}  

\definition Hashing:{
  BLAH.
  
  Sei $p$ eine Primzahl, $m$ die Gr"o"se der Hashtabelle, $a,b$ zwei 
  Zufallszahlen, $x$ der Schl"ussel als Integer. Dann gelten gemeinhin 
  als ``gute'' Hash-Funktionen:
  
  \begin{align*}
    f_1(x)&=x \bmod m \\
    f_2(x)&=(x \bmod p) \bmod m \\
    f_3(x)&=((ax+b) \bmod p) \bmod m
  \end{align*}
}

\remark Repr"asentation von  Graphen:{
  Ein Graph $G=(V,E)$ kann mindestens auf diese Arten und Weisen 
  datentechnisch repr"asentiert werden:
  \begin{itemize}
    \item ``Nachbarschaftsmatrix''
    \item Zeiger-Netz
    \item $|V|+|E|$-Array (nicht-dynamisch) mit Knoten zuerst, die Indizes
      in Array speichern, ab denen benachbarte Knoten indiziert werden.
  \end{itemize}
}




% -----------------------------------------------------------------------
\para{Strategien f"ur gute Algorithmen}
% -----------------------------------------------------------------------

\begin{itemize}
  \item Versuche, das Problem zu reduzieren. Wende dazu folgende 
    Techniken an:
    \begin{itemize}
      \item Vielleicht lassen sich Teile des Problems einfach weglassen,
        ohne die Problemstellung zu "andern.
	
	Beispiel: Prominentenproblem (eine Person l"asst sich mit nur einer 
	Frage eliminieren), Diskrete Bijektion (auf Surjektivit"at spielen,
	nicht auf Injektivit"at)
      \item Teile \& Herrsche$^{\text{TM}}$ (mehrere gleiche kleinere Probleme)
        
	Beispiel: Skyline-Problem
      \item Suche nach mehrere verschiedene, aber kleinere Teilproblemen, und
        wende diese Verfahren an
    \end{itemize}
  \item Verst"arkung der Induktionshypothese (Mach' einfach mehr, als du
    eigentlich machen musst)
    
    Beispiel: Maximale zusammenh"angende Unterfolge (zus"atzlich noch 
    gr"o"stes Suffix liefern), Gleichgewichtsfaktoren in BSTs (zus"atzlich
    noch H"ohe berechnen)
  \item Dynamisches Programmieren 
    Existiert zur L"osung des Problems eine Funktion $F(x,y)$ mit einigen 
    (wenigen) Parametern, so l"asst sich ein mittels Rekursion exponentiell
    l"osbares Problem auch in $O(n^k)$ Schritten l"osen. Es muss dabei 
    eine Abh"angigkeit des Wertes $F(x,y)$ von $F(x',y'), x'<x \lor y'<y$
    geben.
    
    Bsp.: Rucksackproblem. Zugeh"orige Funktion $F(R)$ ergibt, ob
    ein Rucksack der Gr"o"se $R$ mit bisherigen Elementen schon einmal
    erfolgreich bepackt wurde. Sei $g$ die Gr"o"se des n"achsten
    einzupackenden Elementes. Dann ist $F(R+g)=F(R)$ f"ur alle $R$.
\end{itemize}




% -----------------------------------------------------------------------
\para{Algorithmen f"ur Folgen und Mengen}
% -----------------------------------------------------------------------

\definition Prinzip der Bin"arsuche:{
  Gegeben sei eine sortierte (!) Folge oder Sequenz.
  
  a) \textit{Suche eines bestimmten Elements in einer bekannten Anzahl Elemente}:
  Man eliminiert jeweils die H"alfte des Suchraums, indem man das Element
  ``in der Mitte'' mit dem gesuchten vergleicht.
  
  b) \textit{Suche eines bestimmten Elements in einer unbekannten Anzahl Elemente}:
  Man verdoppelt solange die angenommene Gr"o"se des Suchraums, bis ein Element
  an dessen hinteren Ende liegt, das gr"o"ser als das gesuchte ist.
  
  Dieses Prinzip l"asst sich auch auf allgemeine ``suche ein $i$, ab dem\ldots'' oder
  ``suche ein $i$, bis zu dem\ldots''-Probleme anwenden.
  Vergleiche auch: Intervallhalbierungsverfahren zur Bestimmung von Nullstellen
  
  Zur Verfeinerung des Verfahrens kann man mit Hilfe einer Art 
  ``Tangentenbestimmung'' einen g"unstigeren Teilungsort als immer die H"alfte
  bestimmen (vgl. Newton-Verfahren). [``Interpolationssuche'']
}

Bekannte Sortieralgorithmen:
\begin{description}
  \item[Bucket sort] F"ur jeden m"oglichen Schl"ussel einen Bucket bereitstellen,
    einsortieren, zusammensammeln, fertig: $O(n)$
  \item[Radix sort] Eine Art ``rekursiver Bucket sort''. Man stellt jeweils
    nur eine feste Anzahl Buckets bereit, sortiert ein, sortiert dann den 
    Inhalt eines Buckets wieder rekursiv in eine feste Anzahl ein usw. $O(n)$
  \begin{description}
    \item[Straight radix sort] Radix sort, bei dem die ``unwichtigsten'' 
      (am differenzierendsten) Teile des Schl"ussels zuerst
      betrachtet werden.
    \item[Radix-exchange sort] Radix sort, bei dem die ``wichtigsten'' 
      (am wenigsten differenzierenden) Teile des Schl"ussels zuerst
      betrachtet werden.
    \item[Lexicographic sort] (alphabetisches Sortieren mittels radix-exchange 
      sort)~?
  \end{description}
  \item[Insertion sort] Elemente nacheinander an entsprechender Stelle 
    einsortieren: $O(n^2)$
  \item[Selection sort] Gr"o"stes Element ausw"ahlen, an letzte Stelle schieben,
    n"achstkleineres ausw"ahlen, an vorletzte Stelle... usw. $O(n^2)$
  \item[Mergesort] Teile Sequenz in zwei H"alften, sortiere rekursiv (bis zum
    Basisfall: 2 Elemente), kombiniere (``merge'') dann die zwei sortierten 
    H"alften zusammen. (nicht in-place) $O(n\log n)$
  \item[Quicksort] Pivot w"ahlen, Sequenz anhand dieses Pivots in zwei
    H"alften teilen, rekursiv wieder das gleiche f"ur beide H"alften. 
    In der Regel $O(n\log n)$, im schlimmsten Fall $O(n^2)$.
  \item[Heapsort] Baue Heap nach einem der folgenden Verfahren auf:
    \begin{description}
      \item[top-down] Betrachte die Sequenz bis einschlie"slich $i-1$ als Heap, 
	f"uge das $i$-te Element ein (gem"a"s Heap-Einf"uge-Algorithmus)
      \item[bottom-up] ``Unterste Zeile'' des Baumes (bis einschl. 
        $\lfloor n/2\rfloor$) besteht bereits aus (Einer-)Heaps. Jetzt kann
	jedes potentielle neue (Eltern-)Element so weit im Baum nach unten
	wandern, bis das Heap-Kriterium wieder erf"ullt ist.
    \end{description}
    Jetzt kann mit Hilfe des Heap-L"osche-Algorithmus immer das
    maximale Element aus dem Heap entnommen werden. $\Theta(n\log n)$
\end{description}

Man kann zeigen, dass kein vergleichsbasierter Sortieralgorithmus in weniger
als $O(n\log n)$ Schritten zum Ziel kommen kann. (Entscheidender Trick:
Entscheidungsbaum)

\definition Median:{
  Der Median einer Menge $M$ reeller Zahlen ist diejenige Zahl $m$, f"ur die 
  zwei Mengen $M_<:=\{x\in M\mid x<m\}$ und $M_>:=\{x\in M\mid x>m\}$ und
  eine bijektive Abbildung zwischen diesen Mengen existieren.
}

\problem $k$-kleinstes Element finden:{
  Partitionieren wie Quicksort. Dann ist bekannt, in welcher H"alfte 
  das gesuchte Element liegt.
}

\definition Huffman encoding:{
  F"uge alle Zeichen nach ihrer umgekehrten H"aufigkeit in einen Heap ein. 
  Entnimm die zwei seltensten. F"uge statt ihrer ein neues ``Pseudo''-Zeichen
  mit der Summe ihrer H"aufigkeiten in den Heap ein.
}

\problem Effizientes Durchsuchen von Text: {
  \begin{description}
    \item[KMP] Berechne, wie weit der gesuchte Text weitergeschoben werden kann,
      wenn ein Match bei Position $i$ fehlgeschlagen ist.
    \item[Boyer-Moore] Betrachte letzten Buchstaben. Berechne, wie weit 
      verschoben werden kann, so da"s dieser zum ersten Mal richtig im 
      gesuchten String zu liegen kommt.
  \end{description}
}

\problem Finden einer minimalen Anzahl an Bearbeitungsschritten: {
  \textit{L"osung} Dynamisches Programmieren.
  Zeichenfolge $A$ soll in Zeichenfolge $B$ umgebr"oselt werden. 
  $F(i,j):=$ Anzahl der Schritte, die es braucht,um $A[1..i]$ in $B[1..j]$
  umzubr"oseln. Abh"angigkeit je nach zul"assigen Bearbeitungsschritten.
}




\definition Stochastische Algorithmen:{
  Einen \textit{Monte-Carlo-Algorithmus} nennt man einen Algorithmus, der
  mit einer gewissen (hohen) Wahrscheinlichkeit ein richtiges
  Ergebnis liefert. (Beispiel: Element in der oberen H"alfte)
  
  Einen \textit{Las-Vegas-Algorithmus} nennt man einen Algorithmus, der
  zwar mit Sicherheit ein richtiges Ergebnis liefert, dessen Laufzeit aber
  unbestimmt ist. 
}




Ein Zufallsgenerator: Sei $r(1),b,t$ gegeben. 
Dann ist $r(i)=(r(i-1)\cdot b +1)\bmod t$ eine ``ordentliche'' Zufallszahl,
wenn $t>>10^6,b\approx \frac t {10}$.




\problem Finden einer absoluten Mehrheit:{
  \textit{Beobachtung}: Die Entfernung zweier ungleicher Stimmen ist 
  mehrheitsneutral.
} 




\problem Finden der l"angsten aufsteigenden Teilfolge:{
  Speichere alle bisherigen besten Teilfolgen und update sie entsprechend
  dem gerade bearbeiteten Element
}




% -----------------------------------------------------------------------
\para{Algorithmen f"ur Graphen}
% -----------------------------------------------------------------------

\definition Graph:{
  Ein {\itshape Graph} $G=(V,E)$ besteht aus einer Menge $V$ von Knoten 
  (vertices) und einer Menge $E$ von Kanten (edges). Es gibt 
  {\itshape gerichtete} und {\itshape ungerichtete} Graphen. Bei ersteren
  besteht die Menge $V$ aus geordneten, bei letzterem aus ungeordneten Paaren.
  
  Der {\itshape Grad} eines Knotens $v\in V$ ist die Anzahl an Kanten, die an 
  $v$ enden. (analog: Einw"arts-Grad/Ausw"arts-Grad)
  
  Ein {\itshape Pfad} ist eine Folge von Knoten $v_1,v_2,...,v_n$, f"ur die
  eine Folge von Kanten $(v_1,v_2),(v_2,v_3),...,(v_{n-1},v_n)$ existiert.
  Ein Pfad hei"st \textit{einfach}, falls kein Knoten in ihm zweimal vorkommt.
  
  Ein Graph hei"st \textit{verbunden}, wenn zwischen allen Knoten Pfade
  existieren. Ein Graph hei"st \textit{zweifach/$n$-fach verbunden} 
  (biconnected/$n$-connected), wenn zwischen je zwei Knoten mindestens 
  zwei/$n$ knotendisjunkte Pfade existieren.

  Ein Knoten $v_2$ hei"st {\itshape erreichbar} von $v_1$, falls ein Pfad 
  zwischen $v_1$ und $v_2$ existiert.
  
  Sei $U\subseteq V$ eine Menge von Knoten. Dann hei"st $H=(U,F)$ der
  {\itshape von $U$ induzierte Subgraph}, wenn $F$ alle zu $U$ geh"origen 
  Kanten beinhaltet.
  
  Eine {\itshape unabh"angige Menge $S$ in $G$} ist eine Menge von Knoten,
  die paarweise nicht durch Kanten verbunden werden.
  
  Ein {\itshape Kreis} in $G$ ist ein Pfad $P$ mit $|P|>1$ und dem gleichen
  Anfangs- und Endpunkt, in dem kein Knoten doppelt vorkommt.
  (circuit: Kreis, cycle: einfacher Kreis)
  
  Ein Graph hei"st \textit{Wald}, falls er keine Kreise enth"alt. 
  Ein verbundener Wald hei"st \textit{Baum}.
  Ist ein Knoten in ihm besonders ausgezeichnet, so hei"st der Graph ein 
  \textit{Baum mit Wurzel}.
  
  Ein \textit{aufspannender Wald} eines Graphen ist ein Wald, der alle
  Knoten des Graphen enth"alt. Analog: aufspannender Baum (``spanning tree'')
  
  Ein \textit{bipartiter} Graph ist ein Graph, dessen Knoten in zwei Mengen
  aufgeteilt sind, so da"s jeder Knoten mit einem Knoten aus der jeweils
  anderen Menge verbunden ist.
  
  Ein Graph hei"st \textit{eulersch}, wenn er verbunden ist und seine Knoten
  von geradem Grad sind.
  
  {\itshape Fl"ache}: Keine Definition gefunden.
}




\theorem: $G=(V,E)$ ein Graph, $F$ die Menge der Fl"achen von $G$=>
{\[|V|+|F|=|E|+2\]}




\definition Tiefensuche (DFS{:} Depth first search):{
  Gehe zu einem Knoten, markiere ihn, klappere rekursiv alle benachbarten 
  Knoten ab, vermeide dabei bereits markierte Knoten.
  
  Versieht man jeden Knoten in der Reihenfolge der Abarbeitung bei der
  DFS mit einer Nummer, so nennt man diese Nummer die \textit{DFS-Nummer}.
  
  Mittels DFS l"asst sich aus einem ungerichteten Graphen in einem
  Anlauf ein DFS-Baum erzeugen, der folgenden Eigenschaft hat: Entweder 
  ist eine Kante Teil des Baumes, oder sie ist eine Querkante (``cross edge''), 
  d.h. sie verbindet im DFS-Baum einen Knoten mit einem seiner Vorg"anger.
  
  Ebenso lassen sich mittels DFS aus gerichteten Graphen DFS-B"aume erzeugen.
  (daf"ur sind gegebenenfalls im Gegensatz zu ungerichteten Graphen mehrere
  Anl"aufe n"otig. Es kann ja schlie"slich mehrere ``H"ugel'' im Graphen
  geben, von denen man hinunter, aber nicht mehr hinauf kommt.)
  Hier gibt es f"ur eine Kante die folgenden M"oglichkeiten:
  \begin{itemize}
    \item im Baum enthalten
    \item Querkante
    \item Vorw"artskante
    \item R"uckw"artskante
  \end{itemize}
  Haupteigenschaft von gerichteten DFS-B"aumen ist, dass der DFS-Baum
  bez"uglich der DFS-Numerierung ein Heap ist. (Eltern-Knoten hat 
  kleinere Zahl als Kind-Knoten)
}




\definition Breitensuche (BFS{:} Breadth first search):{
  Beginne bei einem Knoten. Markiere ihn, markiere alle seine Kinder und setze 
  Verweise auf sie in einen FIFO. Nimm den n"achsten so zu bearbeitenden Knoten
  aus dem FIFO.

  Versieht man jeden Knoten in der Reihenfolge der Abarbeitung bei der
  BFS mit einer Nummer, so nennt man diese Nummer die \textit{BFS-Nummer}.
}




\problem Alle k"urzesten Pfade von einem Ausgangspunkt:{
  (Kurz ist hier im Sinne von Gewichten an einem gewichteten Graphen zu
  verstehen) Gesucht wird jeweils die L"ange des k"urzesten Pfades von einem 
  Ausgangspunkt zu allen Knoten eines Graphen.
  
  \textit{L"osung}: Jedem Knoten wird eine gegenw"artige Pfadl"ange zugeordnet. 
  (anfangs $\infty$) Dann werden die Knoten nacheinander bearbeitet und f"ur 
  dem bearbeiteten Knoten benachbarte Knoten "uberpr"uft, ob der Weg "uber 
  den bearbeiteten Knoten k"urzer als dessen gegenw"artiger ist.
}




\problem Aufspannender Baum mit minimalen Kosten:{
  (Kosten sind hier wieder die Summe der Gewichte eines gewichteten Graphen)
   
  \textit{L"osung}: Einziger Unterschied zum ``alle k"urzesten Pfade''-Problem
  besteht darin, dass anfangs eine Kante und kein Knoten ausgew"ahlt wird.
}




\problem Alle k"urzesten Pfade zwischen allen Knoten:{
  (Kurz ist hier im Sinne von Gewichten an einem gewichteten Graphen zu
  verstehen)
  
  \textit{L"osung}: Durchlaufe alle Knoten und ersetze Gewichte ($\infty$
  falls Kante nicht existent) zwischen ihnen durch Summe der Gewichte der
  Wege "uber den gerade durchlaufenen Knoten, falls diese kleiner als
  das bestehende Gewicht ist.
}




\definition Transitive H"ulle (transitive closure):{
  Die transitive H"ulle $H=(V,F)$ eines Graphen $G=(V,E)$ besteht aus
  den gleichen Knoten und genau denjenigen Kanten zwischen zwei Knoten $v_1$
  und $v_2$, f"ur die ein Pfad in $G$ von $v_1$ nach $v_2$ existiert.
}




\theorem Satz von Menger:$G=(V,E)$ ungerichtet, verbunden. $u,v\in V, 
  (u,v)\not\in E$=>{
    Die minimale Anzahl an zu entfernenden Knoten, die ben"otigt wird,
    um den Graph bez"uglich $u$ und $v$ zu trennen, ist gleich der Maximalzahl 
    an knotendisjunkten Pfaden zwischen $u$ und $v$.
  }




\theorem Satz von Whitney:=>{
    Ein ungerichteter Graph ist $k$-verbunden $\equiv$ $k$ 
    Knotenentfernungen erforderlich sind, um den Graph zu trennen.
  }
  



\definition Artikulationspunkt:{
  Ein \textit{Artikulationspunkt} eines Graphen ist ein Knoten, dessen
  Entfernung die Trennung des Graphen bewirkt.
}




\definition $n$-fach verbundene Komponente:{
  Eine $n$-fach verbundene Komponente eines Graphen ist die maximale 
  Untermenge an Kanten, die einen $n$-fach verbundenen Graphen induzieren.
}




\theorem:=>{
  Zwei Kanten $e$ und $f$ geh"oren genau dann zur selben zweifach verbundenen
  Komponente, wenn es einen einfachen Kreis gibt, der durch diese zwei 
  Kanten f"uhrt.
}




\theorem:=>{
  Jede Kante geh"ort zu genau einer zweifach verbundenen Komponente.
}




\problem Finden von zweifach verbundenen Komponenten in Graphen:{
  \textit{L"osung}: Baue mittels DFS einen Baum auf, der f"ur je eine
  zweifach verbundene Komponente einen Knoten enth"alt. Merke dir dazu
  die niedrigste DFS-Nummer, bis zu der ein gewisser Baumabschnitt wieder
  hinaufreicht. ("uber R"uckw"artskanten, die ja von DFS bekanntlich ohnehin
  eliminiert werden) Der durch diese DFS-Nummer markierte Punkt ist ein
  Artikulationspunkt.
}




\definition Stark verbundene Komponente:{
  Eine stark verbundene Komponente ist eine maximale Untermenge eines 
  gerichteten Graphen, so dass f"ur je zwei Knoten $v_1$ und $v_2$ Pfade
  von $v_1$ nach $v_2$ und umgekehrt von $v_2$ nach $v_1$ existieren.
}




\theorem:=>{
  Zwei Knoten geh"oren zur selben stark verbundenen Komponente $\equiv$
  Es existiert ein Kreis, der beide Knoten enth"alt.
}




\end{document}
